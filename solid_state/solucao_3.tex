\documentclass[a4paper,10pt]{article}
\usepackage[brazilian]{babel}
\usepackage[left=2.5cm,right=2.5cm,top=3cm,bottom=2.5cm]{geometry}
\usepackage{mathtools}
\usepackage{amsthm}
\usepackage{amsmath}
%\usepackage{nccmath}
\usepackage{amssymb}
\usepackage{amsfonts}
\usepackage{physics}
%\usepackage{dsfont}
%\usepackage{mathrsfs}

\usepackage{titling}
\usepackage{indentfirst}

\usepackage{bm}
\usepackage[dvipsnames]{xcolor}
\usepackage{cancel}

\usepackage{xurl}
\usepackage[colorlinks=true]{hyperref}

\usepackage{float}
\usepackage{graphicx}
%\usepackage{tikz}
\usepackage{caption}
\usepackage{subcaption}

%%%%%%%%%%%%%%%%%%%%%%%%%%%%%%%%%%%%%%%%%%%%%%%%%%%

\newcommand{\eps}{\epsilon}
\newcommand{\vphi}{\varphi}
\newcommand{\cte}{\text{cte}}

\newcommand{\N}{\mathbb{N}}
\newcommand{\Z}{\mathbb{Z}}
\newcommand{\Q}{\mathbb{Q}}
\newcommand{\R}{\vb{R}}
\newcommand{\C}{\mathbb{C}}
\renewcommand{\S}{\hat{S}}
%\renewcommand{\H}{\s{H}}

\renewcommand{\a}{\vb{a}}
\newcommand{\nn}{\hat{n}}
\renewcommand{\d}{\dagger}
\newcommand{\up}{\uparrow}
\newcommand{\down}{\downarrow}

\newcommand{\0}{\vb{0}}
%\newcommand{\1}{\mathds{1}}
\newcommand{\E}{\vb{E}}
\newcommand{\B}{\vb{B}}
\renewcommand{\v}{\vb{v}}
\renewcommand{\r}{\vb{r}}
\renewcommand{\k}{\vb{k}}
\newcommand{\p}{\vb{p}}
\newcommand{\q}{\vb{q}}
\newcommand{\F}{\vb{F}}

\newcommand{\s}{\sigma}
%\newcommand{\prodint}[2]{\left\langle #1 , #2 \right\rangle}
\newcommand{\cc}[1]{\overline{#1}}
\newcommand{\Eval}[3]{\eval{\left( #1 \right)}_{#2}^{#3}}

\newcommand{\unit}[1]{\; \mathrm{#1}}

\newcommand{\n}{\medskip}
\newcommand{\e}{\quad \mathrm{e} \quad}
\newcommand{\ou}{\quad \mathrm{ou} \quad}
\newcommand{\virg}{\, , \;}
\newcommand{\ptodo}{\forall \,}
\renewcommand{\implies}{\; \Rightarrow \;}
%\newcommand{\eqname}[1]{\tag*{#1}} % Tag equation with name

\setlength{\droptitle}{-7em}

\theoremstyle{plain}
\newtheorem{theorem}{Teorema}[section]
%\newtheorem{defi}[theorem]{Definição}
\newtheorem{lemma}[theorem]{Lema}
%\newtheorem{corol}[theorem]{Corolário}
%\newtheorem{prop}[theorem]{Proposição}
%\newtheorem{example}{Exemplo}
%
%\newtheorem{inneraxiom}{Axioma}
%\newenvironment{axioma}[1]
%  {\renewcommand\theinneraxiom{#1}\inneraxiom}
%  {\endinneraxiom}
%
%\newtheorem{innerpostulado}{Postulado}
%\newenvironment{postulado}[1]
%  {\renewcommand\theinnerpostulado{#1}\innerpostulado}
%  {\endinnerpostulado}
%
%\newtheorem{innerexercise}{Exercício}
%\newenvironment{exercise}[1]
%  {\renewcommand\theinnerexercise{#1}\innerexercise}
%  {\endinnerexercise}
%
%\newtheorem{innerthm}{Teorema}
%\newenvironment{teorema}[1]
%  {\renewcommand\theinnerthm{#1}\innerthm}
%  {\endinnerthm}
%
\newtheorem{innerlema}{Lema}
\newenvironment{lema}[1]
  {\renewcommand\theinnerlema{#1}\innerlema}
  {\endinnerlema}
%
%\theoremstyle{remark}
%\newtheorem*{hint}{Dica}
%\newtheorem*{notation}{Notação}
%\newtheorem*{obs}{Observação}


\usepackage[shortlabels]{enumitem}

\title{\Huge{\textbf{Lista 2 - Estado Sólido 1}}}
\author{Mateus Marques}

\begin{document}

\maketitle

\section{}

(a) Falso!

\n

Não é verdade que a aproximação do elétron independente leva em consideração o potencial externo de forma completa. Essa aproximação é baseada numa ideia de campo médio, onde as interações elétron-elétron geram um potencial efetivo que age sobre cada elétron. Dessa maneira, a aproximação de elétron independente ignora os efeitos de muitos-corpos com dependência $(\r_1 - \r_2)$, $(\r_1 - \r_3)$, $(\r_1 - \r_4)$,  $(\r_1 - \r_5)$, $(\r_2 - \r_3)$, $(\r_2 - \r_4)$, etc.

\n\n

(b) Falso!

\n

Existem muitas aproximações envolvidas no tratamento de Hartree-Fock, \textbf{UMA DELAS É...} (*) que não leva em conta a correlação eletrônica exata.essa

\n\n

(c) \textbf{ACHO QUE É} Verdadeiro!

\n

\textbf{Como vimos em aula,} temos que o último autovalor corresponde ao potencial de ionização do material.

\pagebreak

\section{}

(a) A questão do gap ser direto ou indireto.

\textbf{DAR UMA VERIFICADA ABAIXO.}

\begin{itemize}
\item Si: indireto.
\item Ge: direto.
\item GaP: indireto.
\item GaAs: direto.
\item ZnSe: direto.
\end{itemize}

\n\n

(b) Portadores leves ou pesados, mas todos são do tipo ``buraco''.

\begin{itemize}
\item Si: leve.
\item Ge: leve.
\item GaP: leve (pesado na real CHATGPT).
\item GaAs: pesado.
\item ZnSe: pesado.
\end{itemize}

\n\n

(c) Portadores leves ou pesados, do tipo elétron.

\begin{itemize}
\item Si: pesado (leve CHATGPT).
\item Ge: leve.
\item GaP: pesado.
\item GaAs: leve.
\item ZnSe: leve.
\end{itemize}

\n\n

(d) A principal diferença entre a estrutura de bandas do ZnSe para os demais é que a banda de condução não é degenerada, igual aos demais que sempre possuem degenerescência no trecho $\Gamma \to X$. Essa separação da degenerescência (``splitting'') provavelmente ocorre devido à uma interação spin-órbita, que é relevante para o material ZnSe.

\pagebreak

\section{}

(a) \textbf{CHATGPTOU:}

A Densidade de Estados (DOS, do inglês Density of States) e a Densidade Conjunta de Estados (JDOS, do inglês Joint Density of States) são conceitos fundamentais na física do estado sólido, particularmente na teoria de bandas em sólidos. Vamos entender cada um desses conceitos:

1. **Densidade de Estados (DOS):**

   - **Definição:** A DOS representa o número de estados eletrônicos disponíveis em um sistema em função da energia. Em outras palavras, ela descreve como os estados eletrônicos estão distribuídos ao longo da escala de energia.

   - **Fórmula:** Se \( D(E) \) é a DOS, então \( D(E) \, dE \) é o número de estados em um intervalo de energia \( dE \) em torno da energia \( E \).

   - **Aplicações/Medidas:**

     - **Caracterização de Materiais:** A DOS é usada para compreender a estrutura eletrônica de materiais, ajudando a prever propriedades como condutividade elétrica, comportamento magnético e óptico.

     - **Semicondutores e Metais:** A forma da DOS influencia se um material é um isolante, um semicondutor ou um metal.

     - **Transições de Fase:** Mudanças na DOS podem indicar transições de fase em materiais.


2. **Densidade Conjunta de Estados (JDOS):**

   - **Definição:** A JDOS é uma extensão da DOS e leva em consideração a interação entre diferentes espécies de estados, como estados eletrônicos e estados vibracionais.

   - **Fórmula:** A JDOS considera a densidade de estados em relação a duas ou mais variáveis, como energia e momento.

   - **Aplicações/Medidas:**

     - **Acoplamento Eletrônico-Vibracional:** Em materiais que exibem forte acoplamento entre elétrons e fonons (vibrações cristalinas), a JDOS é essencial para entender como as mudanças na estrutura eletrônica afetam os modos vibracionais e vice-versa.

     - **Processos de Excitação:** Na caracterização de processos de excitação, como transições eletrônicas e absorção de fótons, a JDOS é útil para entender como diferentes tipos de estados contribuem para esses processos.

     - **Dinâmica Quântica:** Para estudar a dinâmica temporal de sistemas, onde a evolução dos estados eletrônicos está acoplada a processos vibracionais, a JDOS fornece uma perspectiva mais completa.

Em resumo, a DOS e a JDOS são ferramentas cruciais para compreender a estrutura eletrônica e as propriedades físicas dos materiais sólidos, sendo aplicáveis em diversas áreas da física do estado sólido e da química de materiais.

\n

(b) \textbf{Fazer gráfico da DOS e JDOS para gap direto (GaAs).}

\n

(c) \textbf{Fazer gráfico da DOS e JDOS para gap indireto (Si).}

\n

(d) Para aplicações ópticas que dependem da absorção de energia da luz, geralmente, semicondutores de gap direto, como o GaAs, são mais adequados. Isso ocorre porque a transição eletrônica de um estado na banda de valência para um estado na banda de condução pode ocorrer com uma quantidade mínima de perda de energia, o que é benéfico para eficientes processos de absorção.


No GaAs (arsenieto de gálio), o gap entre a banda de valência e a banda de condução é direto, o que significa que o momento e a energia são conservados simultaneamente durante a transição eletrônica. Esse tipo de material é altamente eficiente na absorção de fótons, especialmente na região do infravermelho próximo e médio.


Por outro lado, no silício (Si), que é um semicondutor de gap indireto, a transição eletrônica entre a banda de valência e a banda de condução envolve uma mudança significativa de momento. Isso torna a absorção de fótons menos eficiente e requer uma quantidade maior de energia para excitar elétrons da banda de valência para a banda de condução.


Portanto, para dispositivos optoeletrônicos, como fotodetectores, LEDs (Light Emitting Diodes) e lasers, onde a absorção de luz desempenha um papel crucial, semicondutores de gap direto, como o GaAs, são preferíveis devido à sua eficiência na absorção de energia óptica.

\pagebreak

\section{}

(a) \textbf{CHATGPTOU:}

A expressão simplificada do potencial químico (\(\mu\)) para um semicondutor intrínseco envolve algumas aproximações que simplificam o tratamento matemático, tornando a análise mais acessível. Vamos considerar o contexto de temperatura ambiente para discutir essas aproximações:

1. **Aproximação de Semicondutor Intrínseco:**

   A expressão simplificada para o potencial químico em um semicondutor intrínseco é derivada sob a suposição de que a concentração de portadores de carga (elétrons e lacunas) é predominantemente governada pela temperatura e pela energia térmica, e não por dopantes externos.

2. **Aproximação de Doping Nulo:**

   A expressão assume que o semicondutor é intrinsecamente puro, ou seja, não há dopantes externos presentes para contribuir significativamente para a geração de portadores de carga. Isso simplifica a análise, pois a concentração intrínseca de portadores em um semicondutor puro é predominantemente devido a processos térmicos.

3. **Baixa Densidade de Portadores:**

   A aproximação é válida para baixas densidades de portadores. Em um semicondutor intrínseco a temperaturas moderadas, a densidade de portadores é relativamente baixa, e os elétrons na banda de condução e as lacunas na banda de valência são gerados principalmente por processos térmicos.

4. **Baixa Variação de Temperatura:**

   A expressão é mais precisa em uma faixa limitada de temperatura, geralmente em torno da temperatura ambiente. Variações significativas de temperatura podem introduzir termos adicionais na expressão do potencial químico.

A expressão simplificada do potencial químico em um semicondutor intrínseco a temperatura ambiente (\(T\)) é muitas vezes dada por:

\[ \mu = E_i + \frac{k_B T}{2} \ln\left(\frac{m_p^*}{m_n^*}\right) \]

Onde:

- \(E_i\) é a energia do nível intrínseco de Fermi.

- \(k_B\) é a constante de Boltzmann.

- \(T\) é a temperatura absoluta.

- \(m_p^*\) e \(m_n^*\) são as massas efetivas dos buracos e elétrons, respectivamente.

Essa expressão simplificada assume que a densidade de portadores é dominada por portadores intrínsecos gerados termicamente e que a contribuição externa de dopagem é negligenciável.

É importante notar que, em situações mais complexas, como em semicondutores dopados ou em condições de alta densidade de portadores, a expressão do potencial químico pode ser mais elaborada e requer considerações adicionais.

\n\n

(b) A expressão simplificada do potencial químico para um semicondutor intrínseco, como mencionada anteriormente, é uma formulação geral que pode ser aplicada a semicondutores, independentemente de serem de gap direto ou gap indireto.

\n\n

(c) Fazer cálculos para o Si e GaAs.



\end{document}
