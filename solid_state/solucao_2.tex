\documentclass[a4paper,10pt]{article}
%\usepackage{mathtools}
\usepackage{amsthm}     % for definitions and theorems
\usepackage[many]{tcolorbox}    % boxes around definitions and theorems
%\usepackage{amsmath}
%\usepackage{nccmath}
\usepackage{amssymb}    % \ltimes, semi-direct product
%\usepackage{etoolbox}   % for start of Chapter
%\usepackage{amsfonts}
\usepackage{physics}    % for all Physics related
\usepackage{dsfont}     % for the identity matrix symbol \1
%\usepackage{mathrsfs}
\usepackage[notextcomp]{stix}   % font package and some symbols like filled square
%\usepackage{MnSymbol}   % symbols font package

\usepackage{titling}
\usepackage{indentfirst}

\usepackage{bm}
\usepackage[dvipsnames]{xcolor}
\usepackage{cancel}
\usepackage{enumitem}

\usepackage{xurl}
%\usepackage[colorlinks=true]{hyperref} % links have colors
\usepackage{hyperref}  % no colors

\usepackage{float}
\usepackage{graphicx}
\usepackage{subcaption}
%\usepackage{tikz}

\usepackage{ctable}     % tabelas
\renewcommand{\P}{\phantom{+}}  % empty space to indent things
\usepackage{multirow}
\usepackage{tabulary}

%%%%%%%%%%%%%%%%%%%%%%%%%%%%%%%%%%%%%%%%%%%%%%%%%%%

\newcommand{\eps}{\epsilon}
\newcommand{\vphi}{\varphi}
\newcommand{\cte}{\text{cte}}

\newcommand{\N}{{\mathbb{N}}}
\newcommand{\Z}{{\mathbb{Z}}}
%\newcommand{\Q}{{\mathbb{Q}}}
\newcommand{\C}{{\mathbb{C}}}
\renewcommand{\S}{{\hat{S}}}
%\renewcommand{\H}{\s{H}}

\renewcommand{\a}{{\vb{a}}}
\renewcommand{\b}{{\vb{b}}}
\renewcommand{\d}{{\dagger}}
\newcommand{\up}{{\uparrow}}
\newcommand{\down}{{\downarrow}}
\newcommand{\hc}{{\text{h.c.}}}

\newcommand{\ihat}{\bm{\hat{\imath}}}
\newcommand{\jhat}{\bm{\hat{\jmath}}}
\newcommand{\khat}{\bm{\hat{k}}}

\newcommand{\0}{{\vb{0}}}
\newcommand{\1}{\mathds{1}}
\newcommand{\E}{{\vb{E}}}
\newcommand{\B}{{\vb{B}}}
\renewcommand{\u}{{\vb{u}}}
\renewcommand{\v}{{\vb{v}}}
\renewcommand{\r}{{\vb{r}}}
\newcommand{\R}{{\vb{R}}}
\newcommand{\Q}{{\vb{Q}}}
\newcommand{\G}{{\vb{G}}}
\newcommand{\g}{{\vb{g}}}
\renewcommand{\k}{{\vb{k}}}
\newcommand{\K}{{\vb{K}}}
\newcommand{\p}{{\vb{p}}}
\newcommand{\q}{{\vb{q}}}
\newcommand{\F}{{\vb{F}}}
\renewcommand{\t}{{\vb{t}}}
\newcommand{\vtau}{{\bm{\tau}}}
\newcommand{\vdelta}{{\bm{\delta}}}

% COLORED SYMMETRY ELEMENTS
\newcommand{\Ct}{{\textcolor{Cyan}{C_3}}}
\newcommand{\Ctn}[1]{{\textcolor{Cyan}{C_3^{\textcolor{black}{#1}}}}}
\newcommand{\Cs}{{\textcolor{ForestGreen}{C_6}}}
\newcommand{\Csn}[1]{{\textcolor{ForestGreen}{C_6^{\textcolor{black}{#1}}}}}
\newcommand{\sd}{{\textcolor{RoyalBlue}{\sigma_d}}}
\newcommand{\sdn}[1]{{\textcolor{RoyalBlue}{\sigma_d^{\textcolor{black}{#1}}}}}
\newcommand{\sdp}{{\textcolor{RoyalBlue}{\sigma_d'}}}
\newcommand{\sdpp}{{\textcolor{RoyalBlue}{\sigma_d''}}}
\newcommand{\sv}{{\textcolor{Orange}{\sigma_v}}}
\newcommand{\svn}[1]{{\textcolor{Orange}{\sigma_v^{\textcolor{black}{#1}}}}}
\newcommand{\svp}{{\textcolor{Orange}{\sigma_v'}}}
\newcommand{\svpp}{{\textcolor{Orange}{\sigma_v''}}}

\newcommand{\GL}{{\text{GL}}}
\newcommand{\U}{{\text{U}}}

\newcommand{\s}{\sigma}
%\newcommand{\prodint}[2]{\left\langle #1 , #2 \right\rangle}
\newcommand{\cc}[1]{\overline{#1}}
\newcommand{\Eval}[3]{\eval{\left( #1 \right)}_{#2}^{#3}}
\newcommand{\sg}[2]{\{ #1 \mid #2 \}}
\renewcommand{\AA}{{\mathring{\text{A}}}}
\newcommand{\I}{{\mathbb{I}}}
\newcommand{\bP}{{\mathbb{P}}}
\newcommand{\bQ}{{\mathbb{Q}}}

\newcommand{\unit}[1]{\; \mathrm{#1}}

\newcommand{\n}{\medskip}
\newcommand{\e}{\quad \mathrm{and} \quad}
\newcommand{\ou}{\quad \mathrm{or} \quad}
\newcommand{\virg}{\, , \;}
\newcommand{\ptodo}{\forall \,}
\renewcommand{\implies}{\; \Rightarrow \;}
%\newcommand{\eqname}[1]{\tag*{#1}} % Tag equation with name

%\setlength{\droptitle}{-7em}   % título um pouco mais em cima na página
%\makeatletter
%\patchcmd{\chapter}{\if@openright\cleardoublepage\else\clearpage\fi}{}{}{}  % start 'Chapter' at the same page. needs package etoolbox
%\makeatother

%% Theorems, definitions, proofs
\theoremstyle{definition}

%%% defining my own colors %%%
\definecolor{my-blue}{HTML}{f2f4ff}
\definecolor{my-green}{HTML}{f5fcf6}    % a little better: green!5!white
\definecolor{my-cyan}{HTML}{f2fffe}
\definecolor{my-yellow}{HTML}{fffbed}
\definecolor{my-green2}{HTML}{efffdb}

%%% alternative colors %%%
\definecolor{my-pink}{HTML}{fff2f7}
\definecolor{my-teal}{HTML}{ebfffc}

\newtheorem{definition}{Definition}[section]
\tcolorboxenvironment{definition}{
  colback=my-blue,
  %colback=blue!5!white,
  boxrule=0.1pt,
  boxsep=1pt,
  left=2pt,right=2pt,top=2pt,bottom=2pt,
  oversize=2pt,
  sharp corners,
  before skip=\topsep,
  after skip=\topsep,
}

\newtheorem{theorem}{Theorem}[section]
\tcolorboxenvironment{theorem}{
  colback=my-yellow,
  %colback=yellow!22!white!95!black,
  boxrule=0.1pt,
  boxsep=1pt,
  left=2pt,right=2pt,top=2pt,bottom=2pt,
  oversize=2pt,
  sharp corners,
  before skip=\topsep,
  after skip=\topsep,
}

\newtheorem{corollary}{Corollary}[section]
\tcolorboxenvironment{corollary}{
  colback=my-green2,
  boxrule=0.1pt,
  boxsep=1pt,
  left=2pt,right=2pt,top=2pt,bottom=2pt,
  oversize=2pt,
  sharp corners,
  before skip=\topsep,
  after skip=\topsep,
}

\newtheorem{lemma}{Lemma}[section]
\tcolorboxenvironment{lemma}{
  colback=my-cyan,
  boxrule=0.1pt,
  boxsep=1pt,
  left=2pt,right=2pt,top=2pt,bottom=2pt,
  oversize=2pt,
  sharp corners,
  before skip=\topsep,
  after skip=\topsep,
}

\newtheorem{example}{Example}[section]
\tcolorboxenvironment{example}{
  %colback=my-green,
  colback=green!5!white,
  boxrule=0.1pt,
  boxsep=1pt,
  left=2pt,right=2pt,top=2pt,bottom=2pt,
  oversize=2pt,
  sharp corners,
  before skip=\topsep,
  after skip=\topsep,
}


\title{\Huge{\textbf{Lista 1 - Estado Sólido 1}}}
\author{Mateus Marques}

\begin{document}

\maketitle

\section{}

(a)
\begin{itemize}
\item $\R$ é um vetor que pertence à rede de Bravais, ou seja, $\R = n_1 \a_1 + n_2 \a_2 + n_3 \a_3$, com $n_1,n_2,n_3$
inteiros e $\a_1,\a_2,\a_3$ os vetores primitivos da rede de Bravais.

\item $\G$ é vetor da rede recíproca, que satisfaz $e^{i \G \vdot \R} = 1$, $\ptodo \R \in$ rede de Bravais.

\item $V_{\G}$ são as componentes da transformada discreta de Fourier do potencial periódico
$$
V(\r) = \sum_{\G} V_{\G} e^{i \G \vdot \R}.
$$

\item $\k$ é um vetor qualquer do espaço recíproco. Mas devido à periodicidade da rede recíproca, podemos tomar $\k \in$ Zona de Brillouin.

\end{itemize}


(b) Um ``mesmo ponto'' de diferentes células unitárias diferem-se apenas por vetores $\R \in$ Rede de Bravais. Assim, dadas duas células unitárias $\Omega$ e $\Omega'$, o ``mesmo ponto'' na célula $C'$ é da forma $\r' = \r + \R$. Temos então
$$
\psi_\k(\r') = \psi_\k(\r+\R) = e^{i \k \vdot (\r+\R)} \underbrace{u_\k(\r+\R)}_{=u_\k(\r)} =
e^{i \k \vdot \R} \qty[e^{i\k\vdot\r} u_\k(\r)] = e^{i\k\vdot\R} \psi_\k(\r).
$$

Portanto
$$
\abs{\psi_\k(\r')}^2 = \abs{\psi_\k(\r)}^2,
$$
ou seja, a probabilidade de encontrar um elétron (no estado $\k$) no ponto $\r'$ é a mesma de encontrar no ponto $\r$.


\pagebreak

\section{}

(a)
\begin{itemize}
\item Falso para o espaço recíproco!

Para o espaço direto é indiferente a escolha da origem, pois sempre podemos redefinir os vetores da base, mantendo os vetores primitivos da rede de Bravais.

Para o espaço recíproco importa a escolha da origem devido ao ponto $\Gamma$ (que é o de mais alta simetria) e também porque a energia $\eps(\k)$ depende de $\k$. Por exemplo, no caso de elétrons livres $\eps(\k) = \hbar^2 \k^2 / 2m$. Se deslocarmos a origem do espaço recíproco por um vetor $\vb{Q}$, isso faz com que os elétrons certo momento $\k$ mudem de energia $\eps(\k) = \hbar^2 \k^2 / 2m$ para $\eps(\k+\vb{Q}) = \hbar^2 (\k+\vb{Q})^2 / 2m$.

\item Falso!

Existem infinitas escolhas diferentes de células primitivas tanto para uma dada rede de Bravais. Assim, tanto para o espaço direto quanto para o espaço recíproco a escolha de montagem de células primitivas \textbf{não é única}.

\begin{figure}[H]
\centering
\includegraphics[width=\linewidth]{fig/primitive_cells.png}
\caption{Diferentes possíveis escolhas de célula primitiva para uma única rede de Bravais. Figura 4.10 do Ashcroft \& Mermin.}
\label{fig:primitive_cells}
\end{figure}


\item Verdade!

Lendo o Ashcroft \& Mermin (Capítulo 8), aprendemos que \textit{o número de vetores de onda $\k$ permitidos em uma célula primitiva da rede recíproca é igual ao número de sítios no cristal.} Lembre-se também que cada estado $\k$ acomoda 2 elétrons devido ao spin.

A Zona de Brillouin é a célula Wigner-Seitz da rede recíproca, portanto ela se trata de uma célula primitiva da rede recíproca e acomoda $2 N$ elétrons. Se o cristal tem $N$ sítios e 2 elétrons por célula unitária, então o sistema possui um total de $2N$ elétrons, que é exatamente o número de elétrons que a Zona de Brillouin acomoda. Finalmente, concluimos que nesse caso todos os pontos da primeira Zona de Brillouin estarão preenchidos.

\item Falso!

Na página seguinte mostrarei o cálculo da DOS nos casos 1D, 2D e 3D que eu escrevi numa lista da disciplina de Estado Sólido 2, que cursei semestre passado com o Prof. Eric Andrade. No cálculo eu utilizo a convenção $\hbar = 1$.

\pagebreak

A dispersão de elétrons livres é $E(\k) = \frac{k^2}{2m}$, com a substituição $\sum_{\k} \to \frac{V_d}{(2\pi)^d} \int \dd[d]{k}$,
$$
\rho_d(\eps) = \sum_{\k, \s} \delta(\eps - E(\k)) =
\frac{2V_d}{(2\pi)^d} \int \delta\qty(\eps - \frac{k^2}{2m}) \dd[d]{k} =
\frac{2V_d}{(2\pi)^d} \int \dd{\Omega_d} \int_0^\infty \delta\qty(\eps - \frac{k^2}{2m}) k^{d-1} \dd{k},
$$
onde $\Theta_d = \int \dd{\Omega_d}$ é o ângulo sólido de acordo com a dimensão $d$, sendo $\Theta_1 = 2, \Theta_2 = 2\pi$ e $\Theta_3 = 4\pi$.

Vale a seguinte propriedade da função $\delta$ de Dirac:
$$
\int_{D} f(x) \delta(g(x)) \dd{x} =
\sum_{\substack{g(a)=0 \\ a \in D}} \frac{1}{\abs{g'(a)}} \int_{-\infty}^{\infty} f(x) \delta(x-a) \dd{x}=
\sum_{\substack{g(a)=0 \\ a \in D}} \frac{1}{\abs{g'(a)}} \, f(a).
$$
onde a soma é sobre todos os zeros $a \in D$ da função $g(x)$ dentro do domínio de integração $D$.

Definindo $g(k) = \eps - \frac{k^2}{2m}$, temos $\abs{g'(k)} = \abs{k} / m$ e sua única raiz no domínio $D = (0, +\infty)$ é $k = \sqrt{2m \eps}$. Portanto
$$
\rho_d(\eps) = \frac{2V_d}{(2\pi)^d} \Theta_d \, \frac{m}{\sqrt{2m\eps}}
\qty(\sqrt{2m \eps})^{d-1} \implies
\boxed{ \rho_d(\eps) = \frac{2 m V_d \Theta_d}{(2\pi)^d} \, (2m\eps)^{\frac{d}{2}-1}. }
$$

Substituindo explicitamente as dimensões $d = 1, 2$ e $3$:
\begin{itemize}
\item $d = 1$, $V_1 = L$, $\Theta_1 = 2$:
$$
\boxed{\rho_1(\eps) = \frac{2 m L}{\pi} \, \frac{1}{\sqrt{2m\eps}}.}
$$
\item $d = 2$, $V_2 = A$, $\Theta_2 = 2\pi$:
$$
\boxed{\rho_2(\eps) = \frac{m A}{\pi}.}
$$
\item $d = 3$, $V_3 = V$, $\Theta_3 = 4\pi$:
$$
\boxed{\rho_3(\eps) = \frac{m V}{\pi^2} \sqrt{2m\eps}.}
$$
\end{itemize}

\begin{figure}[H]
\centering
\includegraphics[width=0.79\textwidth]{fig/dos-k2.png}
\caption{Densidade de estados $\rho_d(\eps)$ para as dimensões $d = 1, 2$ e $3$. Parâmetros: $m = L = A = V = 1$.}
\label{fig:dosk2}
\end{figure}

\end{itemize}



\pagebreak

\section{}

\pagebreak

\section{}

\end{document}
