\documentclass[a4paper,10pt]{article}
%\usepackage{mathtools}
\usepackage{amsthm}     % for definitions and theorems
\usepackage[many]{tcolorbox}    % boxes around definitions and theorems
%\usepackage{amsmath}
%\usepackage{nccmath}
\usepackage{amssymb}    % \ltimes, semi-direct product
%\usepackage{etoolbox}   % for start of Chapter
%\usepackage{amsfonts}
\usepackage{physics}    % for all Physics related
\usepackage{dsfont}     % for the identity matrix symbol \1
%\usepackage{mathrsfs}
\usepackage[notextcomp]{stix}   % font package and some symbols like filled square
%\usepackage{MnSymbol}   % symbols font package

\usepackage{titling}
\usepackage{indentfirst}

\usepackage{bm}
\usepackage[dvipsnames]{xcolor}
\usepackage{cancel}
\usepackage{enumitem}

\usepackage{xurl}
%\usepackage[colorlinks=true]{hyperref} % links have colors
\usepackage{hyperref}  % no colors

\usepackage{float}
\usepackage{graphicx}
\usepackage{subcaption}
%\usepackage{tikz}

\usepackage{ctable}     % tabelas
\renewcommand{\P}{\phantom{+}}  % empty space to indent things
\usepackage{multirow}
\usepackage{tabulary}

%%%%%%%%%%%%%%%%%%%%%%%%%%%%%%%%%%%%%%%%%%%%%%%%%%%

\newcommand{\eps}{\epsilon}
\newcommand{\vphi}{\varphi}
\newcommand{\cte}{\text{cte}}

\newcommand{\N}{{\mathbb{N}}}
\newcommand{\Z}{{\mathbb{Z}}}
%\newcommand{\Q}{{\mathbb{Q}}}
\newcommand{\C}{{\mathbb{C}}}
\renewcommand{\S}{{\hat{S}}}
%\renewcommand{\H}{\s{H}}

\renewcommand{\a}{{\vb{a}}}
\renewcommand{\b}{{\vb{b}}}
\renewcommand{\d}{{\dagger}}
\newcommand{\up}{{\uparrow}}
\newcommand{\down}{{\downarrow}}
\newcommand{\hc}{{\text{h.c.}}}

\newcommand{\ihat}{\bm{\hat{\imath}}}
\newcommand{\jhat}{\bm{\hat{\jmath}}}
\newcommand{\khat}{\bm{\hat{k}}}

\newcommand{\0}{{\vb{0}}}
\newcommand{\1}{\mathds{1}}
\newcommand{\E}{{\vb{E}}}
\newcommand{\B}{{\vb{B}}}
\renewcommand{\u}{{\vb{u}}}
\renewcommand{\v}{{\vb{v}}}
\renewcommand{\r}{{\vb{r}}}
\newcommand{\R}{{\vb{R}}}
\newcommand{\Q}{{\vb{Q}}}
\newcommand{\G}{{\vb{G}}}
\newcommand{\g}{{\vb{g}}}
\renewcommand{\k}{{\vb{k}}}
\newcommand{\K}{{\vb{K}}}
\newcommand{\p}{{\vb{p}}}
\newcommand{\q}{{\vb{q}}}
\newcommand{\F}{{\vb{F}}}
\renewcommand{\t}{{\vb{t}}}
\newcommand{\vtau}{{\bm{\tau}}}
\newcommand{\vdelta}{{\bm{\delta}}}

% COLORED SYMMETRY ELEMENTS
\newcommand{\Ct}{{\textcolor{Cyan}{C_3}}}
\newcommand{\Ctn}[1]{{\textcolor{Cyan}{C_3^{\textcolor{black}{#1}}}}}
\newcommand{\Cs}{{\textcolor{ForestGreen}{C_6}}}
\newcommand{\Csn}[1]{{\textcolor{ForestGreen}{C_6^{\textcolor{black}{#1}}}}}
\newcommand{\sd}{{\textcolor{RoyalBlue}{\sigma_d}}}
\newcommand{\sdn}[1]{{\textcolor{RoyalBlue}{\sigma_d^{\textcolor{black}{#1}}}}}
\newcommand{\sdp}{{\textcolor{RoyalBlue}{\sigma_d'}}}
\newcommand{\sdpp}{{\textcolor{RoyalBlue}{\sigma_d''}}}
\newcommand{\sv}{{\textcolor{Orange}{\sigma_v}}}
\newcommand{\svn}[1]{{\textcolor{Orange}{\sigma_v^{\textcolor{black}{#1}}}}}
\newcommand{\svp}{{\textcolor{Orange}{\sigma_v'}}}
\newcommand{\svpp}{{\textcolor{Orange}{\sigma_v''}}}

\newcommand{\GL}{{\text{GL}}}
\newcommand{\U}{{\text{U}}}

\newcommand{\s}{\sigma}
%\newcommand{\prodint}[2]{\left\langle #1 , #2 \right\rangle}
\newcommand{\cc}[1]{\overline{#1}}
\newcommand{\Eval}[3]{\eval{\left( #1 \right)}_{#2}^{#3}}
\newcommand{\sg}[2]{\{ #1 \mid #2 \}}
\renewcommand{\AA}{{\mathring{\text{A}}}}
\newcommand{\I}{{\mathbb{I}}}
\newcommand{\bP}{{\mathbb{P}}}
\newcommand{\bQ}{{\mathbb{Q}}}

\newcommand{\unit}[1]{\; \mathrm{#1}}

\newcommand{\n}{\medskip}
\newcommand{\e}{\quad \mathrm{and} \quad}
\newcommand{\ou}{\quad \mathrm{or} \quad}
\newcommand{\virg}{\, , \;}
\newcommand{\ptodo}{\forall \,}
\renewcommand{\implies}{\; \Rightarrow \;}
%\newcommand{\eqname}[1]{\tag*{#1}} % Tag equation with name

%\setlength{\droptitle}{-7em}   % título um pouco mais em cima na página
%\makeatletter
%\patchcmd{\chapter}{\if@openright\cleardoublepage\else\clearpage\fi}{}{}{}  % start 'Chapter' at the same page. needs package etoolbox
%\makeatother

%% Theorems, definitions, proofs
\theoremstyle{definition}

%%% defining my own colors %%%
\definecolor{my-blue}{HTML}{f2f4ff}
\definecolor{my-green}{HTML}{f5fcf6}    % a little better: green!5!white
\definecolor{my-cyan}{HTML}{f2fffe}
\definecolor{my-yellow}{HTML}{fffbed}
\definecolor{my-green2}{HTML}{efffdb}

%%% alternative colors %%%
\definecolor{my-pink}{HTML}{fff2f7}
\definecolor{my-teal}{HTML}{ebfffc}

\newtheorem{definition}{Definition}[section]
\tcolorboxenvironment{definition}{
  colback=my-blue,
  %colback=blue!5!white,
  boxrule=0.1pt,
  boxsep=1pt,
  left=2pt,right=2pt,top=2pt,bottom=2pt,
  oversize=2pt,
  sharp corners,
  before skip=\topsep,
  after skip=\topsep,
}

\newtheorem{theorem}{Theorem}[section]
\tcolorboxenvironment{theorem}{
  colback=my-yellow,
  %colback=yellow!22!white!95!black,
  boxrule=0.1pt,
  boxsep=1pt,
  left=2pt,right=2pt,top=2pt,bottom=2pt,
  oversize=2pt,
  sharp corners,
  before skip=\topsep,
  after skip=\topsep,
}

\newtheorem{corollary}{Corollary}[section]
\tcolorboxenvironment{corollary}{
  colback=my-green2,
  boxrule=0.1pt,
  boxsep=1pt,
  left=2pt,right=2pt,top=2pt,bottom=2pt,
  oversize=2pt,
  sharp corners,
  before skip=\topsep,
  after skip=\topsep,
}

\newtheorem{lemma}{Lemma}[section]
\tcolorboxenvironment{lemma}{
  colback=my-cyan,
  boxrule=0.1pt,
  boxsep=1pt,
  left=2pt,right=2pt,top=2pt,bottom=2pt,
  oversize=2pt,
  sharp corners,
  before skip=\topsep,
  after skip=\topsep,
}

\newtheorem{example}{Example}[section]
\tcolorboxenvironment{example}{
  %colback=my-green,
  colback=green!5!white,
  boxrule=0.1pt,
  boxsep=1pt,
  left=2pt,right=2pt,top=2pt,bottom=2pt,
  oversize=2pt,
  sharp corners,
  before skip=\topsep,
  after skip=\topsep,
}


\renewcommand{\p}{\phantom{+}}
\newcommand{\mchi}{\chi^{\Gamma^\pi_{p_z}}}
\renewcommand{\c}[1]{\textcolor{red}{#1}}

\title{\Huge{\textbf{Teoria de grupos - Exercício 3}}}
\author{Mateus Marques}

\begin{document}

\maketitle

\section*{Enunciado}

\begin{itemize}
\item encontrar o grupo de simetria da molécula (descrever ou desenhar todos os elementos de simetria)

\item dados os orbitais $p_z$ dos seis átomos de carbono que formam os orbitais moleculares $\pi$, encontrar a representação $\Gamma_{p_z}^\pi$ \implies
traços das matrizes dos operadores de transformação dos orbitais $p_z$ $(\vphi_1, \vphi_2, \vphi_3, \vphi_4, \vphi_5, \vphi_6)$.

\item encontrar com que representações irredutíveis do grupo da molécula os orbitais $p_z$ se transformam

\item obter os orbitais simetrizados (e normalizados) da molécula \implies aplicar dos operadores de projeção para obter as funções projetadas nos subespaços das representações irredutíveis.
\end{itemize}


\section*{A ligação $\pi$ da molécula de benzeno C$_6$H$_6$}

A molécula de benzeno tem geometria de um hexágono. Assim, ela possui um eixo $C_6$ (o eixo $z$), 6 eixos $C_2$ (no plano da molécula, 3 passando pelos átomos de carbono e 3 passando pelo meio das arestas do hexágono) e, além disso, ela também possui um plano de reflexão $\sigma_h$. No total, isso totaliza $12 \cdot 2 = 24$ elementos. Note que essas características correspondem unicamente ao grupo $D_{6h} = D_6 \otimes C_{1h}$.

\begin{figure}[H]
\centering
\includegraphics[width=1.0\linewidth]{fig/D6h.png}
\caption{Elementos de simetria do grupo $D_{6h}$, correspondente à molécula de benzeno C$_6$H$_6$.}
\label{fig:D6h}
\end{figure}

Agora descobriremos a representação $\Gamma^\pi_{p_z}$. Note que a base tem dimensão $6$, já que temos 6 orbitais, representados pelo vetor $(\vphi_1, \vphi_2, \vphi_3, \vphi_4, \vphi_5, \vphi_6)$. Avaliaremos os caracteres dessa representação $\Gamma^\pi_{p_z}$ para seus elementos, seguindo os desenhos da Figura \ref{fig:D6h}:
\begin{enumerate}
\item $E$ $\implies$ matriz identidade de dimensão 6, portanto $\mchi(E) = 6$.
\item $C_6$ (eixo $z$) $\implies$ troca os 6 orbitais de lugar, portanto $\mchi(C_6) = 0$.
\item $C_3$ (eixo $z$) $\implies$ troca os 6 orbitais de lugar, portanto $\mchi(C_3) = 0$.
\item $C_2$ (eixo $z$) $\implies$ troca os 6 orbitais de lugar, portanto $\mchi(C_2) = 0$.
\item $C_2'$ $\implies$ mantém 2 dos 6 orbitais inalterados e inverte a polarização, portanto $\mchi(C_2') = -2$.
\item $C_2''$ $\implies$ troca os 6 orbitais de lugar, portanto $\mchi(C_2'') = 0$.
\item $i$ $\implies$ troca os 6 orbitais de lugar, portanto $\mchi(i) = 0$.
\item $S_3$ $\implies$ troca os 6 orbitais de lugar, portanto $\mchi(S_3) = 0$.
\item $S_6$ $\implies$ troca os 6 orbitais de lugar, portanto $\mchi(S_6) = 0$.
\item $\sigma_h$ $\implies$ mantém os 6 orbitais, mas inverte a polarização de todos, portanto $\mchi(\sigma_h) = -6$.
\item $\sigma_d$ $\implies$ troca os 6 orbitais de lugar, portanto $\mchi(\sigma_d) = 0$.
\item $\sigma_v$ $\implies$ mantém 2 dos 6 orbitais e mantém a polarização, portanto $\mchi(\sigma_v) = 2$.
\end{enumerate}

Retirando a tabela de caracter do site \url{http://symmetry.jacobs-university.de/cgi-bin/group.cgi?group=606&option=4}.

\begin{table}[H]
\caption{Tabela de caracteres para o grupo $D_{6h}$.}
\centering

\begin{tabular} { |c|c c c c c c c c c c c c | }
\hline
$D_{6h}$ & $E$ & $2 C_6$ & $2 C_3$ & $C_2$ & $3 C_2'$ & $3 C_2''$ & $i$ & $2 S_3$ & $2 S_6$ & $\sigma_h$ & $3 \sigma_d$ & $3 \sigma_v$ \\
\hline
$A_{1g}$ & $\p1$ & $\p1$ & $\p1$ & $\p1$ & $\p1$ & $\p1$ & $\p1$ & $\p1$ & $\p1$ & $\p1$ & $\p1$ & $\p1$ \\
$A_{2g}$ & $\p1$ & $\p1$ & $\p1$ & $\p1$ & $ -1$ & $ -1$ & $\p1$ & $\p1$ & $\p1$ & $\p1$ & $ -1$ & $ -1$ \\
$B_{1g}$ & $\p1$ & $ -1$ & $\p1$ & $ -1$ & $\p1$ & $ -1$ & $\p1$ & $ -1$ & $\p1$ & $ -1$ & $\p1$ & $ -1$ \\
$\c{\boxed{B_{2g}}}$ & $\c{\p1}$ & $\c{ -1}$ & $\c{\p1}$ & $\c{ -1}$ & $\c{ -1}$ & $\c{\p1}$ & $\c{\p1}$ & $\c{ -1}$ & $\c{\p1}$ & $\c{ -1}$ & $\c{ -1}$ & $\c{\p1}$ \\
$\c{\boxed{E_{1g}}}$ & $\c{\p2}$ & $\c{\p1}$ & $\c{ -1}$ & $\c{ -2}$ & $\c{\p0}$ & $\c{\p0}$ & $\c{\p2}$ & $\c{\p1}$ & $\c{ -1}$ & $\c{ -2}$ & $\c{\p0}$ & $\c{\p0}$ \\
$E_{2g}$ & $\p2$ & $ -1$ & $ -1$ & $\p2$ & $\p0$ & $\p0$ & $\p2$ & $ -1$ & $ -1$ & $\p2$ & $\p0$ & $\p0$ \\
$A_{1u}$ & $\p1$ & $\p1$ & $\p1$ & $\p1$ & $\p1$ & $\p1$ & $ -1$ & $ -1$ & $ -1$ & $ -1$ & $ -1$ & $ -1$ \\
$\c{\boxed{A_{2u}}}$ & $\c{\p1}$ & $\c{\p1}$ & $\c{\p1}$ & $\c{\p1}$ & $\c{ -1}$ & $\c{ -1}$ & $\c{ -1}$ & $\c{ -1}$ & $\c{ -1}$ & $\c{ -1}$ & $\c{\p1}$ & $\c{\p1}$ \\
$B_{1u}$ & $\p1$ & $ -1$ & $\p1$ & $ -1$ & $\p1$ & $ -1$ & $ -1$ & $\p1$ & $ -1$ & $\p1$ & $ -1$ & $\p1$ \\
$B_{2u}$ & $\p1$ & $ -1$ & $\p1$ & $ -1$ & $ -1$ & $\p1$ & $ -1$ & $\p1$ & $ -1$ & $\p1$ & $\p1$ & $ -1$ \\
$E_{1u}$ & $\p2$ & $\p1$ & $ -1$ & $ -2$ & $\p0$ & $\p0$ & $ -2$ & $ -1$ & $\p1$ & $\p2$ & $\p0$ & $\p0$ \\
$\c{\boxed{E_{2u}}}$ & $\c{\p2}$ & $\c{ -1}$ & $\c{ -1}$ & $\c{\p2}$ & $\c{\p0}$ & $\c{\p0}$ & $\c{ -2}$ & $\c{\p1}$ & $\c{\p1}$ & $\c{ -2}$ & $\c{\p0}$ & $\c{\p0}$ \\
\hline
\hline
$\Gamma_{p_z}^\pi$ & $\p6$ & $\p0$ & $\p0$ & $\p0$ & $ -2$ & $\p0$ & $\p0$ & $\p0$ & $\p0$ & $ -6$ & $\p0$ & $\p2$ \\
\hline
\end{tabular}

\label{tab:mult_D3h}
\end{table}

Temos portanto que
$$
\Gamma^\pi_{p_z} = B_{2g} \oplus E_{1g} \oplus A_{2u} \oplus E_{2u}.
$$

Agora basta aplicar os operadores de projeção para obter os orbitais simetrizados.

B2g
$$
\frac{\sqrt{6} \varphi_{1}}{6} - \frac{\sqrt{6} \varphi_{2}}{6} + \frac{\sqrt{6} \varphi_{3}}{6} - \frac{\sqrt{6} \varphi_{4}}{6} + \frac{\sqrt{6} \varphi_{5}}{6} - \frac{\sqrt{6} \varphi_{6}}{6}
$$

E1g
$$
\frac{\sqrt{3} \varphi_{1}}{3} + \frac{\sqrt{3} \varphi_{2}}{6} - \frac{\sqrt{3} \varphi_{3}}{6} - \frac{\sqrt{3} \varphi_{4}}{3} - \frac{\sqrt{3} \varphi_{5}}{6} + \frac{\sqrt{3} \varphi_{6}}{6}
$$
$$
\frac{\varphi_{2}}{2} + \frac{\varphi_{3}}{2} - \frac{\varphi_{5}}{2} - \frac{\varphi_{6}}{2}
$$

A2u
$$
\frac{\sqrt{6} \varphi_{1}}{6} + \frac{\sqrt{6} \varphi_{2}}{6} + \frac{\sqrt{6} \varphi_{3}}{6} + \frac{\sqrt{6} \varphi_{4}}{6} + \frac{\sqrt{6} \varphi_{5}}{6} + \frac{\sqrt{6} \varphi_{6}}{6}
$$

E2u
$$
\frac{\sqrt{3} \varphi_{1}}{3} - \frac{\sqrt{3} \varphi_{2}}{6} - \frac{\sqrt{3} \varphi_{3}}{6} + \frac{\sqrt{3} \varphi_{4}}{3} - \frac{\sqrt{3} \varphi_{5}}{6} - \frac{\sqrt{3} \varphi_{6}}{6}
$$
$$
\frac{\varphi_{2}}{2} - \frac{\varphi_{3}}{2} + \frac{\varphi_{5}}{2} - \frac{\varphi_{6}}{2}
$$



\end{document}
