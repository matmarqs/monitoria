\documentclass[a4paper,10pt]{article}
%\usepackage{mathtools}
\usepackage{amsthm}     % for definitions and theorems
\usepackage[many]{tcolorbox}    % boxes around definitions and theorems
%\usepackage{amsmath}
%\usepackage{nccmath}
\usepackage{amssymb}    % \ltimes, semi-direct product
%\usepackage{etoolbox}   % for start of Chapter
%\usepackage{amsfonts}
\usepackage{physics}    % for all Physics related
\usepackage{dsfont}     % for the identity matrix symbol \1
%\usepackage{mathrsfs}
\usepackage[notextcomp]{stix}   % font package and some symbols like filled square
%\usepackage{MnSymbol}   % symbols font package

\usepackage{titling}
\usepackage{indentfirst}

\usepackage{bm}
\usepackage[dvipsnames]{xcolor}
\usepackage{cancel}
\usepackage{enumitem}

\usepackage{xurl}
%\usepackage[colorlinks=true]{hyperref} % links have colors
\usepackage{hyperref}  % no colors

\usepackage{float}
\usepackage{graphicx}
\usepackage{subcaption}
%\usepackage{tikz}

\usepackage{ctable}     % tabelas
\renewcommand{\P}{\phantom{+}}  % empty space to indent things
\usepackage{multirow}
\usepackage{tabulary}

%%%%%%%%%%%%%%%%%%%%%%%%%%%%%%%%%%%%%%%%%%%%%%%%%%%

\newcommand{\eps}{\epsilon}
\newcommand{\vphi}{\varphi}
\newcommand{\cte}{\text{cte}}

\newcommand{\N}{{\mathbb{N}}}
\newcommand{\Z}{{\mathbb{Z}}}
%\newcommand{\Q}{{\mathbb{Q}}}
\newcommand{\C}{{\mathbb{C}}}
\renewcommand{\S}{{\hat{S}}}
%\renewcommand{\H}{\s{H}}

\renewcommand{\a}{{\vb{a}}}
\renewcommand{\b}{{\vb{b}}}
\renewcommand{\d}{{\dagger}}
\newcommand{\up}{{\uparrow}}
\newcommand{\down}{{\downarrow}}
\newcommand{\hc}{{\text{h.c.}}}

\newcommand{\ihat}{\bm{\hat{\imath}}}
\newcommand{\jhat}{\bm{\hat{\jmath}}}
\newcommand{\khat}{\bm{\hat{k}}}

\newcommand{\0}{{\vb{0}}}
\newcommand{\1}{\mathds{1}}
\newcommand{\E}{{\vb{E}}}
\newcommand{\B}{{\vb{B}}}
\renewcommand{\u}{{\vb{u}}}
\renewcommand{\v}{{\vb{v}}}
\renewcommand{\r}{{\vb{r}}}
\newcommand{\R}{{\vb{R}}}
\newcommand{\Q}{{\vb{Q}}}
\newcommand{\G}{{\vb{G}}}
\newcommand{\g}{{\vb{g}}}
\renewcommand{\k}{{\vb{k}}}
\newcommand{\K}{{\vb{K}}}
\newcommand{\p}{{\vb{p}}}
\newcommand{\q}{{\vb{q}}}
\newcommand{\F}{{\vb{F}}}
\renewcommand{\t}{{\vb{t}}}
\newcommand{\vtau}{{\bm{\tau}}}
\newcommand{\vdelta}{{\bm{\delta}}}

% COLORED SYMMETRY ELEMENTS
\newcommand{\Ct}{{\textcolor{Cyan}{C_3}}}
\newcommand{\Ctn}[1]{{\textcolor{Cyan}{C_3^{\textcolor{black}{#1}}}}}
\newcommand{\Cs}{{\textcolor{ForestGreen}{C_6}}}
\newcommand{\Csn}[1]{{\textcolor{ForestGreen}{C_6^{\textcolor{black}{#1}}}}}
\newcommand{\sd}{{\textcolor{RoyalBlue}{\sigma_d}}}
\newcommand{\sdn}[1]{{\textcolor{RoyalBlue}{\sigma_d^{\textcolor{black}{#1}}}}}
\newcommand{\sdp}{{\textcolor{RoyalBlue}{\sigma_d'}}}
\newcommand{\sdpp}{{\textcolor{RoyalBlue}{\sigma_d''}}}
\newcommand{\sv}{{\textcolor{Orange}{\sigma_v}}}
\newcommand{\svn}[1]{{\textcolor{Orange}{\sigma_v^{\textcolor{black}{#1}}}}}
\newcommand{\svp}{{\textcolor{Orange}{\sigma_v'}}}
\newcommand{\svpp}{{\textcolor{Orange}{\sigma_v''}}}

\newcommand{\GL}{{\text{GL}}}
\newcommand{\U}{{\text{U}}}

\newcommand{\s}{\sigma}
%\newcommand{\prodint}[2]{\left\langle #1 , #2 \right\rangle}
\newcommand{\cc}[1]{\overline{#1}}
\newcommand{\Eval}[3]{\eval{\left( #1 \right)}_{#2}^{#3}}
\newcommand{\sg}[2]{\{ #1 \mid #2 \}}
\renewcommand{\AA}{{\mathring{\text{A}}}}
\newcommand{\I}{{\mathbb{I}}}
\newcommand{\bP}{{\mathbb{P}}}
\newcommand{\bQ}{{\mathbb{Q}}}

\newcommand{\unit}[1]{\; \mathrm{#1}}

\newcommand{\n}{\medskip}
\newcommand{\e}{\quad \mathrm{and} \quad}
\newcommand{\ou}{\quad \mathrm{or} \quad}
\newcommand{\virg}{\, , \;}
\newcommand{\ptodo}{\forall \,}
\renewcommand{\implies}{\; \Rightarrow \;}
%\newcommand{\eqname}[1]{\tag*{#1}} % Tag equation with name

%\setlength{\droptitle}{-7em}   % título um pouco mais em cima na página
%\makeatletter
%\patchcmd{\chapter}{\if@openright\cleardoublepage\else\clearpage\fi}{}{}{}  % start 'Chapter' at the same page. needs package etoolbox
%\makeatother

%% Theorems, definitions, proofs
\theoremstyle{definition}

%%% defining my own colors %%%
\definecolor{my-blue}{HTML}{f2f4ff}
\definecolor{my-green}{HTML}{f5fcf6}    % a little better: green!5!white
\definecolor{my-cyan}{HTML}{f2fffe}
\definecolor{my-yellow}{HTML}{fffbed}
\definecolor{my-green2}{HTML}{efffdb}

%%% alternative colors %%%
\definecolor{my-pink}{HTML}{fff2f7}
\definecolor{my-teal}{HTML}{ebfffc}

\newtheorem{definition}{Definition}[section]
\tcolorboxenvironment{definition}{
  colback=my-blue,
  %colback=blue!5!white,
  boxrule=0.1pt,
  boxsep=1pt,
  left=2pt,right=2pt,top=2pt,bottom=2pt,
  oversize=2pt,
  sharp corners,
  before skip=\topsep,
  after skip=\topsep,
}

\newtheorem{theorem}{Theorem}[section]
\tcolorboxenvironment{theorem}{
  colback=my-yellow,
  %colback=yellow!22!white!95!black,
  boxrule=0.1pt,
  boxsep=1pt,
  left=2pt,right=2pt,top=2pt,bottom=2pt,
  oversize=2pt,
  sharp corners,
  before skip=\topsep,
  after skip=\topsep,
}

\newtheorem{corollary}{Corollary}[section]
\tcolorboxenvironment{corollary}{
  colback=my-green2,
  boxrule=0.1pt,
  boxsep=1pt,
  left=2pt,right=2pt,top=2pt,bottom=2pt,
  oversize=2pt,
  sharp corners,
  before skip=\topsep,
  after skip=\topsep,
}

\newtheorem{lemma}{Lemma}[section]
\tcolorboxenvironment{lemma}{
  colback=my-cyan,
  boxrule=0.1pt,
  boxsep=1pt,
  left=2pt,right=2pt,top=2pt,bottom=2pt,
  oversize=2pt,
  sharp corners,
  before skip=\topsep,
  after skip=\topsep,
}

\newtheorem{example}{Example}[section]
\tcolorboxenvironment{example}{
  %colback=my-green,
  colback=green!5!white,
  boxrule=0.1pt,
  boxsep=1pt,
  left=2pt,right=2pt,top=2pt,bottom=2pt,
  oversize=2pt,
  sharp corners,
  before skip=\topsep,
  after skip=\topsep,
}


\title{\Huge{\textbf{Teoria de grupos - Exercício 1}}}
\author{Mateus Marques}

\begin{document}

\maketitle

\section*{Grupo da molécula de amônia NH$_3$}

\begin{figure}[H]
\centering
\includegraphics[width=\linewidth]{fig/C3v.png}
\caption{Grupo $C_{3\text{v}}$ associado à molécula de amônia.}
\label{fig:C3v}
\end{figure}

Nos slides da Aula 3 (Figura \ref{fig:C3v}), a professora deu como exemplo o grupo $C_{3\text{v}}$ da amônia. Este grupo contém 6 elementos, sendo eles
\begin{itemize}
\item a identidade $E$;
 \item as três reflexões $\sigma_{\text{v}1}$, $\sigma_{\text{v}2}$ e $\sigma_{\text{v}3}$ representadas na Figura \ref{fig:C3v};
\item as rotações $C_3$ (por $120^\circ)$ e $C_3^2$ (por $240^\circ$) pelo eixo que passa pelo átomo de nitrogênio N e o baricentro do triângulo definido pelos três átomos de hidrogênio H.
\end{itemize}

\begin{figure}[H]
\centering
\includegraphics[width=\linewidth]{fig/triangulo.png}
\caption{Grupo do triângulo visto na Aula 2.}
\label{fig:triangulo}
\end{figure}

Nos slides da Aula 2 (Figura \ref{fig:triangulo}) a professora passou o grupo do triângulo. Note bem que se compararmos as Figuras \ref{fig:C3v} e \ref{fig:triangulo}, conseguimos estabelecer diretamente um isomorfismo entre o grupo do triângulo e o grupo $C_{3\text{v}}$. De fato, colocando os dois desenhos lado a lado na Figura \ref{fig:comp} vemos imediatamente a identificação:
\begin{equation} \label{eq:identificacao}
E \leftrightarrow E, \quad
\sigma_{\text{v}1} \leftrightarrow B, \quad
\sigma_{\text{v}2} \leftrightarrow A, \quad
\sigma_{\text{v}3} \leftrightarrow C, \quad
C_3 \leftrightarrow D, \quad
C_3^2 \leftrightarrow F.
\end{equation}

Devido a esse isomorfismo, todas as propriedades algébricas entre os grupos seram iguais, em particular a tabela de multiplicação e a partição do grupo em classes.

\begin{figure}[H]
\centering
\includegraphics[width=0.6\linewidth]{fig/comp.png}
\caption{Identificação imediata entre os elementos do grupo do triângulo e o grupo $C_{3\text{v}}$.}
\label{fig:comp}
\end{figure}


\subsection*{1. Ordem}

O grupo se escreve $C_{3\text{v}} = \{E, \sigma_{\text{v}1}, \sigma_{\text{v}2}, \sigma_{\text{v}3}, C_3, C_3^2\}$ e possui ordem $6$.

\subsection*{2. Tabela de Multiplicação}

Pelo isomorfismo, a tabela de multiplicação do grupo $C_{3\text{v}}$ é idêntica ao do grupo do triângulo (mostrada na Figura \ref{fig:triangulo}), dado que façamos a identificação \ref{eq:identificacao}. Portanto ela é:

\begin{table}[ht]
\caption{Tabela de multiplicação do grupo $C_{3\text{v}}$, construída pela tabela de multiplicação do grupo do triângulo (Figura \ref{fig:triangulo}) e o isomorfismo da equação \ref{eq:identificacao}.}
\centering
\begin{tabular}{ |c|c c c c c c| }
\hline
\phantom{x}          & $E$                  & $\sigma_{\text{v}1}$   & $\sigma_{\text{v}2}$ & $\sigma_{\text{v}3}$ & $C_3$                & $C_3^2$ \\
\hline
$E$                  & $E$                  & $\sigma_{\text{v}1}$   & $\sigma_{\text{v}2}$ & $\sigma_{\text{v}3}$ & $C_3$                & $C_3^2$ \\
$\sigma_{\text{v}1}$ & $\sigma_{\text{v}1}$ & $E$                    & $C_3$                & $C_3^2$              & $\sigma_{\text{v}1}$ & $\sigma_{\text{v}3}$ \\
$\sigma_{\text{v}2}$ & $\sigma_{\text{v}2}$ & $C_3^2$                & $E$                  & $C_3$                & $\sigma_{\text{v}3}$ & $\sigma_{\text{v}2}$ \\
$\sigma_{\text{v}3}$ & $\sigma_{\text{v}3}$ & $C_3$                  & $C_3^2$              & $E$                  & $\sigma_{\text{v}2}$ & $\sigma_{\text{v}1}$ \\
$C_3$                & $C_3$                & $\sigma_{\text{v}3}$   & $\sigma_{\text{v}2}$ & $\sigma_{\text{v}1}$ & $C_3^2$              & $E$ \\
$C_3^2$              & $C_3^2$              & $\sigma_{\text{v}1}$   & $\sigma_{\text{v}3}$ & $\sigma_{\text{v}2}$ & $E$                  & $C_3$ \\
\hline
\end{tabular}
\label{tab:mult}
\end{table}

\subsection*{3. Cíclico?}

Como vemos na tabela de multiplicação \ref{tab:mult}, ela não é simétrica e portanto o grupo não é abeliano. Assim, ele também não é cíclico (não existe um único elemento gerador que gera todo o grupo).

\subsection*{4. Abeliano?}

A tabela de multiplicação \ref{tab:mult} não é simétrica, logo o grupo não é abeliano.

\subsection*{5. Classes}

As classes do grupo $C_{3\text{v}}$ são idênticas às classes do grupo do triângulo, ao realizarmos a identificação na equação \ref{eq:identificacao}. Portanto, olhando na Figura \ref{fig:triangulo} vemos que as três classes são $\{E\}$, $\{\sigma_{\text{v}1}, \sigma_{\text{v}2}, \sigma_{\text{v}3}\}$ e $\{C_3, C_3^2\}$.

\end{document}
