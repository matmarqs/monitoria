\documentclass[a4paper,10pt]{article}
\usepackage[brazilian]{babel}
\usepackage[left=2.5cm,right=2.5cm,top=3cm,bottom=2.5cm]{geometry}
\usepackage{mathtools}
\usepackage{amsthm}
\usepackage{amsmath}
%\usepackage{nccmath}
\usepackage{amssymb}
\usepackage{amsfonts}
\usepackage{physics}
%\usepackage{dsfont}
%\usepackage{mathrsfs}

\usepackage{titling}
\usepackage{indentfirst}

\usepackage{bm}
\usepackage[dvipsnames]{xcolor}
\usepackage{cancel}

\usepackage{xurl}
\usepackage[colorlinks=true]{hyperref}

\usepackage{float}
\usepackage{graphicx}
%\usepackage{tikz}
\usepackage{caption}
\usepackage{subcaption}

%%%%%%%%%%%%%%%%%%%%%%%%%%%%%%%%%%%%%%%%%%%%%%%%%%%

\newcommand{\eps}{\epsilon}
\newcommand{\vphi}{\varphi}
\newcommand{\cte}{\text{cte}}

\newcommand{\N}{\mathbb{N}}
\newcommand{\Z}{\mathbb{Z}}
\newcommand{\Q}{\mathbb{Q}}
\newcommand{\R}{\vb{R}}
\newcommand{\C}{\mathbb{C}}
\renewcommand{\S}{\hat{S}}
%\renewcommand{\H}{\s{H}}

\renewcommand{\a}{\vb{a}}
\newcommand{\nn}{\hat{n}}
\renewcommand{\d}{\dagger}
\newcommand{\up}{\uparrow}
\newcommand{\down}{\downarrow}

\newcommand{\0}{\vb{0}}
%\newcommand{\1}{\mathds{1}}
\newcommand{\E}{\vb{E}}
\newcommand{\B}{\vb{B}}
\renewcommand{\v}{\vb{v}}
\renewcommand{\r}{\vb{r}}
\renewcommand{\k}{\vb{k}}
\newcommand{\p}{\vb{p}}
\newcommand{\q}{\vb{q}}
\newcommand{\F}{\vb{F}}

\newcommand{\s}{\sigma}
%\newcommand{\prodint}[2]{\left\langle #1 , #2 \right\rangle}
\newcommand{\cc}[1]{\overline{#1}}
\newcommand{\Eval}[3]{\eval{\left( #1 \right)}_{#2}^{#3}}

\newcommand{\unit}[1]{\; \mathrm{#1}}

\newcommand{\n}{\medskip}
\newcommand{\e}{\quad \mathrm{e} \quad}
\newcommand{\ou}{\quad \mathrm{ou} \quad}
\newcommand{\virg}{\, , \;}
\newcommand{\ptodo}{\forall \,}
\renewcommand{\implies}{\; \Rightarrow \;}
%\newcommand{\eqname}[1]{\tag*{#1}} % Tag equation with name

\setlength{\droptitle}{-7em}

\theoremstyle{plain}
\newtheorem{theorem}{Teorema}[section]
%\newtheorem{defi}[theorem]{Definição}
\newtheorem{lemma}[theorem]{Lema}
%\newtheorem{corol}[theorem]{Corolário}
%\newtheorem{prop}[theorem]{Proposição}
%\newtheorem{example}{Exemplo}
%
%\newtheorem{inneraxiom}{Axioma}
%\newenvironment{axioma}[1]
%  {\renewcommand\theinneraxiom{#1}\inneraxiom}
%  {\endinneraxiom}
%
%\newtheorem{innerpostulado}{Postulado}
%\newenvironment{postulado}[1]
%  {\renewcommand\theinnerpostulado{#1}\innerpostulado}
%  {\endinnerpostulado}
%
%\newtheorem{innerexercise}{Exercício}
%\newenvironment{exercise}[1]
%  {\renewcommand\theinnerexercise{#1}\innerexercise}
%  {\endinnerexercise}
%
%\newtheorem{innerthm}{Teorema}
%\newenvironment{teorema}[1]
%  {\renewcommand\theinnerthm{#1}\innerthm}
%  {\endinnerthm}
%
\newtheorem{innerlema}{Lema}
\newenvironment{lema}[1]
  {\renewcommand\theinnerlema{#1}\innerlema}
  {\endinnerlema}
%
%\theoremstyle{remark}
%\newtheorem*{hint}{Dica}
%\newtheorem*{notation}{Notação}
%\newtheorem*{obs}{Observação}


\title{\Huge{\textbf{Lista 2 - Mecânica Estatística}}}
\author{Mateus Marques}

\begin{document}

\maketitle

\section*{2) Oscilador harmônico no ensemble canônico}

(a) Para contar todos os microestados com energia variável no cálculo da função de partição, integramos todas as variáveis $q_i, p_i$ sobre todos os valores possíveis:
$$
Z_N = \int_{-\infty}^{\infty}\int_{-\infty}^{\infty} \frac{\dd{p_1} \dd{q_1}}{2\pi\hbar} \cdots \int_{-\infty}^{\infty}\int_{-\infty}^{\infty} \frac{\dd{p_N} \dd{q_N}}{2\pi\hbar} e^{-\beta H(\q, \p)} =
$$
$$
= \qty{\frac{1}{2\pi\hbar} \int_{-\infty}^{\infty} e^{-\beta \frac{p^2}{2m}} \dd{p} \int_{-\infty}^{\infty} e^{-\beta \frac{m\omega^2q^2}{2} }\dd{q}}^N =
\qty[ \frac{1}{2\pi\hbar} \sqrt{\frac{2m\pi}{\beta}} \sqrt{\frac{2\pi}{m\omega^2\beta}} ]^N =
\qty(\frac{1}{\beta \hbar \omega})^N = \qty(\frac{k_B T}{\hbar \omega})^N.
$$

\begin{itemize}
\item A energia livre é dada por
$$
F = - k_B T \log Z = -\frac{1}{\beta} N \log(\frac{1}{\beta \hbar \omega}) = \frac{N}{\beta} \log(\beta\hbar\omega) =
-N k_B T \log(\frac{k_B T}{\hbar \omega}).
$$

\item A energia interna é
$$
E = \frac{\tr(H e^{-\beta H})}{\tr(e^{-\beta H})} = - \frac{1}{Z} \pdv{Z}{\beta} = - \pdv{\log Z}{\beta} =
N\pdv{\log(\beta\hbar\omega)}{\beta} = \frac{N}{\beta} = N k_B T,
$$
que corresponde ao que obtemos na lista 1 para o ensemble microcanônico.

\item Da definição termodinâmica de energia livre $F = E - TS$, temos que a entropia é
$$
S = \frac{E-F}{T} = \frac{N k_B T + N k_B T \log(\frac{k_B T}{\hbar\omega})}{T} = N k_B \qty[1 + \log(\frac{k_B T}{\hbar\omega})],
$$
que também é a mesma entropia obtida na lista 1.

\item Pressão:
$$
P = -\pdvc{F}{V}{T,N} = 0,
$$
porque o sistema não depende do volume, como foi discutido na lista 1.

\item Potencial químico:
$$
\mu = \pdvc{F}{N}{T,V} = - k_B T \log(\frac{k_B T}{\hbar\omega}),
$$
que também está de acordo com o que obtivemos no ensemble microcanônico.
\end{itemize}

(b) A função de partição no caso quântico é
$$
Z_N = \sum_{n_1 = 0}^{\infty} \sum_{n_2 = 0}^{\infty} \cdots \sum_{n_N = 0}^{\infty} e^{-\beta \sum_{j=1}^{N} E_n^j} =
\qty(\sum_{n=0}^{\infty} e^{-\beta E_n})^N =
$$
$$
= \qty(e^{-\beta \hbar \omega / 2} \sum_{n=0}^{\infty} e^{-\beta n \hbar\omega})^N =
\qty(\frac{e^{-\beta \hbar \omega / 2}}{1 - e^{-\beta\hbar\omega}})^N =
\qty[\frac{1}{2\sinh(\frac{\beta\hbar\omega}{2})}]^N.
$$

A energia livre é dada por
\begin{equation} \label{eq:free_energy-1qu}
F = - \frac{1}{\beta} \log Z = \frac{N}{\beta} \log\qty[2 \sinh(\frac{\beta\hbar\omega}{2})] =
N k_B T \log\qty[2 \sinh(\frac{\hbar\omega}{2 k_B T})].
\end{equation}

A entropia pode ser calculada por
$$
S = -\pdvc{F}{T}{V,N} = - N k_B \log\qty[2 \sinh(\frac{\hbar\omega}{2 k_B T})] -
N k_B T \frac{2 \cosh(\frac{\hbar\omega}{2 k_B T})}{2 \sinh(\frac{\hbar\omega}{2 k_B T})} \,
\qty(-\frac{\hbar\omega}{2 k_B T^2}) \implies
$$
$$
S = - N k_B \log\qty[2 \sinh(\frac{\hbar\omega}{2 k_B T})] + \frac{N\hbar\omega}{2 T} \coth(\frac{\hbar\omega}{2 k_B T}).
$$
Notando que na expressão acima a fórmula \ref{eq:free_energy-1qu} da energia livre reaparece no primeiro termo, temos que
$$
S = - \frac{F}{T} + \frac{N\hbar\omega}{2 T} \coth(\frac{\hbar\omega}{2 k_B T}) \implies
E = T \qty(S + \frac{F}{T}) = \frac{N\hbar\omega}{2} \coth(\frac{\hbar\omega}{2 k_B T}).
$$

O calor específico $C_V$ é então
$$
C_V = \pdvc{E}{T}{V,N} =
- \frac{N\hbar\omega}{2} \frac{1}{\sinh[2](\frac{\hbar\omega}{2 k_B T})} \qty(-\frac{\hbar\omega}{2 k_B T^2}) =
\frac{N (\hbar\omega)^2}{4 k_B T^2} \cdot \frac{1}{\sinh[2](\frac{\hbar\omega}{2 k_B T})}.
$$

\n

Como $\hat{H} = \sum_{i} \hbar\omega(\hat{N}_i + \frac{1}{2})$, temos que a ocupação média $\ev{n} = \frac{1}{N} \sum_{i} \ev{\hat{N}_i}$ satisfaz
$$
E = \ev{\hat{H}} = N \hbar\omega\qty(\ev{n} + \frac{1}{2}),
$$
portanto
$$
\ev{n} = \frac{1}{2} \qty[\coth(\frac{\hbar\omega}{2 k_B T}) - 1].
$$
Reescrevendo $\coth x - 1 = \frac{e^x + e^{-x}}{e^x - e^{-x}} - 1 = \frac{e^{2x} + 1}{e^{2x} - 1} - \frac{e^{2x} - 1}{e^{2x} - 1} = \frac{2}{e^{2x} - 1}$, temos
$$
\ev{n} = \frac{1}{e^{\frac{\hbar\omega}{k_B T}} - 1},
$$
que é a distribuição de Bose-Einstein.

\n

\textbf{FAZERRRRRRR}

Fazer esboço das curvas $C_V/N \times T$ e $S/N \times T$. Discutir os limites $T \to 0$ e $T \to \infty$. Nós claramente recuperaremos o limite clássico quando $\hbar \to 0$.



\pagebreak

\section*{4) Distribuição de Maxwell}

\pagebreak

\section*{6) Magnetismo de gases}


\end{document}
