\documentclass[a4paper,10pt]{article}

%\usepackage{mathtools}
\usepackage{amsthm}     % for definitions and theorems
\usepackage[many]{tcolorbox}    % boxes around definitions and theorems
%\usepackage{amsmath}
%\usepackage{nccmath}
\usepackage{amssymb}    % \ltimes, semi-direct product
%\usepackage{etoolbox}   % for start of Chapter
%\usepackage{amsfonts}
\usepackage{physics}    % for all Physics related
\usepackage{dsfont}     % for the identity matrix symbol \1
%\usepackage{mathrsfs}
\usepackage[notextcomp]{stix}   % font package and some symbols like filled square
%\usepackage{MnSymbol}   % symbols font package

\usepackage{titling}
\usepackage{indentfirst}

\usepackage{bm}
\usepackage[dvipsnames]{xcolor}
\usepackage{cancel}
\usepackage{enumitem}

\usepackage{xurl}
%\usepackage[colorlinks=true]{hyperref} % links have colors
\usepackage{hyperref}  % no colors

\usepackage{float}
\usepackage{graphicx}
\usepackage{subcaption}
%\usepackage{tikz}

\usepackage{ctable}     % tabelas
\renewcommand{\P}{\phantom{+}}  % empty space to indent things
\usepackage{multirow}
\usepackage{tabulary}

%%%%%%%%%%%%%%%%%%%%%%%%%%%%%%%%%%%%%%%%%%%%%%%%%%%

\newcommand{\eps}{\epsilon}
\newcommand{\vphi}{\varphi}
\newcommand{\cte}{\text{cte}}

\newcommand{\N}{{\mathbb{N}}}
\newcommand{\Z}{{\mathbb{Z}}}
%\newcommand{\Q}{{\mathbb{Q}}}
\newcommand{\C}{{\mathbb{C}}}
\renewcommand{\S}{{\hat{S}}}
%\renewcommand{\H}{\s{H}}

\renewcommand{\a}{{\vb{a}}}
\renewcommand{\b}{{\vb{b}}}
\renewcommand{\d}{{\dagger}}
\newcommand{\up}{{\uparrow}}
\newcommand{\down}{{\downarrow}}
\newcommand{\hc}{{\text{h.c.}}}

\newcommand{\ihat}{\bm{\hat{\imath}}}
\newcommand{\jhat}{\bm{\hat{\jmath}}}
\newcommand{\khat}{\bm{\hat{k}}}

\newcommand{\0}{{\vb{0}}}
\newcommand{\1}{\mathds{1}}
\newcommand{\E}{{\vb{E}}}
\newcommand{\B}{{\vb{B}}}
\renewcommand{\u}{{\vb{u}}}
\renewcommand{\v}{{\vb{v}}}
\renewcommand{\r}{{\vb{r}}}
\newcommand{\R}{{\vb{R}}}
\newcommand{\Q}{{\vb{Q}}}
\newcommand{\G}{{\vb{G}}}
\newcommand{\g}{{\vb{g}}}
\renewcommand{\k}{{\vb{k}}}
\newcommand{\K}{{\vb{K}}}
\newcommand{\p}{{\vb{p}}}
\newcommand{\q}{{\vb{q}}}
\newcommand{\F}{{\vb{F}}}
\renewcommand{\t}{{\vb{t}}}
\newcommand{\vtau}{{\bm{\tau}}}
\newcommand{\vdelta}{{\bm{\delta}}}

% COLORED SYMMETRY ELEMENTS
\newcommand{\Ct}{{\textcolor{Cyan}{C_3}}}
\newcommand{\Ctn}[1]{{\textcolor{Cyan}{C_3^{\textcolor{black}{#1}}}}}
\newcommand{\Cs}{{\textcolor{ForestGreen}{C_6}}}
\newcommand{\Csn}[1]{{\textcolor{ForestGreen}{C_6^{\textcolor{black}{#1}}}}}
\newcommand{\sd}{{\textcolor{RoyalBlue}{\sigma_d}}}
\newcommand{\sdn}[1]{{\textcolor{RoyalBlue}{\sigma_d^{\textcolor{black}{#1}}}}}
\newcommand{\sdp}{{\textcolor{RoyalBlue}{\sigma_d'}}}
\newcommand{\sdpp}{{\textcolor{RoyalBlue}{\sigma_d''}}}
\newcommand{\sv}{{\textcolor{Orange}{\sigma_v}}}
\newcommand{\svn}[1]{{\textcolor{Orange}{\sigma_v^{\textcolor{black}{#1}}}}}
\newcommand{\svp}{{\textcolor{Orange}{\sigma_v'}}}
\newcommand{\svpp}{{\textcolor{Orange}{\sigma_v''}}}

\newcommand{\GL}{{\text{GL}}}
\newcommand{\U}{{\text{U}}}

\newcommand{\s}{\sigma}
%\newcommand{\prodint}[2]{\left\langle #1 , #2 \right\rangle}
\newcommand{\cc}[1]{\overline{#1}}
\newcommand{\Eval}[3]{\eval{\left( #1 \right)}_{#2}^{#3}}
\newcommand{\sg}[2]{\{ #1 \mid #2 \}}
\renewcommand{\AA}{{\mathring{\text{A}}}}
\newcommand{\I}{{\mathbb{I}}}
\newcommand{\bP}{{\mathbb{P}}}
\newcommand{\bQ}{{\mathbb{Q}}}

\newcommand{\unit}[1]{\; \mathrm{#1}}

\newcommand{\n}{\medskip}
\newcommand{\e}{\quad \mathrm{and} \quad}
\newcommand{\ou}{\quad \mathrm{or} \quad}
\newcommand{\virg}{\, , \;}
\newcommand{\ptodo}{\forall \,}
\renewcommand{\implies}{\; \Rightarrow \;}
%\newcommand{\eqname}[1]{\tag*{#1}} % Tag equation with name

%\setlength{\droptitle}{-7em}   % título um pouco mais em cima na página
%\makeatletter
%\patchcmd{\chapter}{\if@openright\cleardoublepage\else\clearpage\fi}{}{}{}  % start 'Chapter' at the same page. needs package etoolbox
%\makeatother

%% Theorems, definitions, proofs
\theoremstyle{definition}

%%% defining my own colors %%%
\definecolor{my-blue}{HTML}{f2f4ff}
\definecolor{my-green}{HTML}{f5fcf6}    % a little better: green!5!white
\definecolor{my-cyan}{HTML}{f2fffe}
\definecolor{my-yellow}{HTML}{fffbed}
\definecolor{my-green2}{HTML}{efffdb}

%%% alternative colors %%%
\definecolor{my-pink}{HTML}{fff2f7}
\definecolor{my-teal}{HTML}{ebfffc}

\newtheorem{definition}{Definition}[section]
\tcolorboxenvironment{definition}{
  colback=my-blue,
  %colback=blue!5!white,
  boxrule=0.1pt,
  boxsep=1pt,
  left=2pt,right=2pt,top=2pt,bottom=2pt,
  oversize=2pt,
  sharp corners,
  before skip=\topsep,
  after skip=\topsep,
}

\newtheorem{theorem}{Theorem}[section]
\tcolorboxenvironment{theorem}{
  colback=my-yellow,
  %colback=yellow!22!white!95!black,
  boxrule=0.1pt,
  boxsep=1pt,
  left=2pt,right=2pt,top=2pt,bottom=2pt,
  oversize=2pt,
  sharp corners,
  before skip=\topsep,
  after skip=\topsep,
}

\newtheorem{corollary}{Corollary}[section]
\tcolorboxenvironment{corollary}{
  colback=my-green2,
  boxrule=0.1pt,
  boxsep=1pt,
  left=2pt,right=2pt,top=2pt,bottom=2pt,
  oversize=2pt,
  sharp corners,
  before skip=\topsep,
  after skip=\topsep,
}

\newtheorem{lemma}{Lemma}[section]
\tcolorboxenvironment{lemma}{
  colback=my-cyan,
  boxrule=0.1pt,
  boxsep=1pt,
  left=2pt,right=2pt,top=2pt,bottom=2pt,
  oversize=2pt,
  sharp corners,
  before skip=\topsep,
  after skip=\topsep,
}

\newtheorem{example}{Example}[section]
\tcolorboxenvironment{example}{
  %colback=my-green,
  colback=green!5!white,
  boxrule=0.1pt,
  boxsep=1pt,
  left=2pt,right=2pt,top=2pt,bottom=2pt,
  oversize=2pt,
  sharp corners,
  before skip=\topsep,
  after skip=\topsep,
}


\newcommand{\vac}{\ket{vac}}
\newcommand{\hh}{\tilde{h}}
\newcommand{\hhh}{\bm{\tilde{h}}}
\newcommand{\vecs}{(\s^x, \s^y, \s^z)}
\renewcommand{\ss}{\bm{\sigma}}


\title{\Huge{\textbf{Prova final - Mecânica Estatística}}}
\author{Mateus Marques}

\begin{document}

\maketitle

\section*{1) Modelo de Ising em um campo transverso}

(a) Primeiramente, observe que:
\begin{itemize}
\item Pelas relações de comutação de férmions, temos $[n_k, n_\ell] = 0$ para quaisquer sítios $k, \ell$.

\n

\item Em particular, a exponencial da soma $e^{\pm i\pi \sum_{k<j} n_k}$ pode ser separada no produto $\prod_{k<j} e^{\pm i\pi n_k}$, pois todos os operadores $n_k$ comutam entre si.

\n

\item Como $n_k^2 = n_k$ pelo princípio de Pauli, temos também $n_k^p = n_k$ para qualquer natural $p\geq 1$. Assim:
$$
e^{\pm i\pi n_k} = 1 + \sum_{p=1}^{\infty} \frac{(\pm i\pi n_k)^p}{p!} = 1+ \qty( \sum_{p=1}^{\infty} \frac{(\pm i\pi)^p}{p!} ) n_k = 1 + (e^{\pm i \pi} - 1) n_k = 1 - 2 n_k = - \s_j^z,
$$
$$
e^{\pm i\pi \sum_{k<j} n_k} = \prod_{k<j} e^{\pm i\pi n_k} = \prod_{k<j} (-\s_k^z).
$$

\end{itemize}

Temos então que
$$
\sigma_j^x = \frac{\qty(\sigma_j^+ + \sigma_j^-)}{2} =
e^{i\pi \sum_{k<j} n_k} f_j^\d + e^{-i\pi \sum_{k<j} n_k} f_j =
\qty[\prod_{k<j} (-\s_k^z)] \qty(f_j^\d + f_j).
$$

\n

Substituindo $\s_j^x$ acima e $\s_j^z = 2 n_j - 1$ na hamiltoniana:
$$
H = -J \sum_{j} \s_j^x \s_{j+1}^x + h \sum_{j} \s_j^z =
-J \sum_{j} \qty[\prod_{k<j} \big(\s_k^z\big)^2] (f_j+f_j^\d)(1-2n_j)(f_{j+1} + f_{j+1}^\d) + h \sum_{j} (2n_j - 1).
$$

Veja que $\big(\s_k^z\big)^2 = 1$ e que
$$
(f_j+f_j^\d)(1-2n_j) = (f_j+f_j^\d)(1-2 f_j^\d f_j) =
f_j+f_j^\d - 2 f_j f_j^\d f_j =
f_j+f_j^\d - 2 (1 - f_j^\d f_j) f_j =
f_j^\d - f_j.
$$

\n

Assim, temos
$$
H =
-J \sum_{j} \qty[ (f_j^\d - f_j) (f_{j+1} + f_{j+1}^\d) - g (2 f_j^\d f_j - 1)] \implies
$$

\begin{equation} \label{eq:hamil_transvising}
\boxed{ H =
-J \sum_{i} \qty[ f_i^\d f_{i+1}^\d + f_{i+1} f_i + f_i^\d f_{i+1} + f_{i+1}^\d f_i - 2 g f_i^\d f_i + g]. }
\end{equation}

\n\n

(b) Tomando agora as transformadas $f_j^\d = \frac{1}{\sqrt{N}} \sum_{k} e^{-ikja} f_k^\d$, $f_j = \frac{1}{\sqrt{N}} \sum_{k'} e^{ik'ja} f_{k'}$ e lembrando da relação $\sum_{j} e^{-i(k-k')ja} = N \delta_{k,k'}$ para a transformada discreta de Fourier da função $\delta$, temos
$$
H =
-J \sum_{i} \qty[ \Big( f_{i+1} f_i + f_{i+1}^\d f_i + \hc \Big) - 2 g f_i^\d f_i + g] =
$$
$$
H =
-\frac{J}{N} \sum_{j} \sum_{k,k'} \qty{ \qty[ e^{ik'a} e^{i(k+k')ja} f_{k'} f_k + e^{ik'a} e^{-i(k-k')ja} f_k^\d f_{k'} + \hc ] - 2g \, e^{-i(k-k')ja} f_k^\d f_{k'} + g \, \delta_{k,k'}} =
$$
$$
=
-J \sum_{k,k'} \qty{ \qty[ e^{ik'a} \delta_{-k,k'} f_{k'} f_k + e^{ik'a} \delta_{k,k'} f_k^\d f_{k'} + \hc ] - 2g \, \delta_{k,k'} f_k^\d f_{k'} + g \, \delta_{k,k'}} =
$$
$$
=
-J \sum_{k} \qty{ \qty[ e^{-ika} f_{-k} f_k + e^{ika} f_k^\d f_{k} + \hc ]
- 2 g f_k^\d f_k + g } =
$$
$$
=
-J \sum_{k} \qty{ \qty[ e^{-ika} f_{-k} f_k + \boxed{e^{ika} f_k^\d f_{k}} + e^{ika} f_{k}^\d f_{-k}^\d + \boxed{e^{-ika} f_k^\d f_{k}} ]
\boxed{- 2 g f_k^\d f_k} + g } =
$$
$$
=
J \sum_{k} \qty{ 2 \qty[g - \cos(ka)] f_k^\d f_{k} - \qty( e^{-ika} f_{-k} f_k + e^{ika} f_{k}^\d f_{-k}^\d ) - g }.
$$

O termo acima em parênteses que possui exponenciais complexas pode ser reescrito como (fazendo a média com outra somatória em $-k$)
$$
\sum_{k} \qty( e^{-ika} f_{-k} f_k + e^{ika} f_{k}^\d f_{-k}^\d ) =
\frac{1}{2} \sum_{k} \qty( e^{-ika} f_{-k} f_k + e^{ika} f_{k}^\d f_{-k}^\d ) \; + \;
\frac{1}{2} \sum_{k} \qty( e^{ika} f_{k} f_{-k} + e^{-ika} f_{-k}^\d f_{k}^\d ) =
$$
$$
=
\frac{1}{2} \sum_{k} \qty( e^{-ika} f_{-k} f_k - e^{ika} f_{-k}^\d f_{k}^\d ) \; + \;
\frac{1}{2} \sum_{k} \qty( -e^{ika} f_{-k} f_k + e^{-ika} f_{-k}^\d f_{k}^\d ) =
$$
$$
= \sum_{k} \qty{
\qty(\frac{e^{-ika} - e^{ika} }{2}) f_{-k} f_k + \qty(\frac{e^{-ika} - e^{ika} }{2}) f_{-k}^\d f_k^\d }
= \sum_{k} \qty[ -i \sin(ka) \qty(
f_{-k}^\d f_k^\d + f_{-k} f_k ) ].
$$

\n

Portanto, a hamiltoniana fica
$$
\boxed{ H =
J \sum_{k} \qty{ 2 \qty[g - \cos(ka)] f_k^\d f_{k} + i \, \sin(ka) \qty( f_{-k}^\d f_k^\d + f_{-k} f_{k} ) - g }. }
$$

\n\n

(c) Desenvolvendo a multiplicação de matriz pedida, temos
$$
\begin{pmatrix}
f_k^\d & f_{-k}
\end{pmatrix}
\begin{pmatrix}
g-\cos(ka) & -i\sin(ka) \\
i\sin(ka) & -[g-\cos(ka)]
\end{pmatrix}
\begin{pmatrix}
f_k \\ f_{-k}^\d
\end{pmatrix}
=
\begin{pmatrix}
f_k^\d & f_{-k}
\end{pmatrix}
\begin{pmatrix}
[g-\cos(ka)] f_k - i \sin(ka) f_{-k}^\d \\ -[g-\cos(ka)] f_{-k}^\d + i \sin(ka) f_k
\end{pmatrix}
=
$$
$$
=
[g-\cos(ka)] f_k^\d f_k - i\sin(ka) \underbrace{f_k^\d f_{-k}}_{=-f_{-k}^\d f_k^\d}
- [g-\cos(ka)] \underbrace{f_{-k} f_{-k}^\d}_{=1-f_{-k}^\d f_{-k}} + i \sin(ka) f_{-k} f_k =
$$
$$
=
[g-\cos(ka)] (f_k^\d f_k + \boxed{f_{-k}^\d f_{-k}}) - g + \boxed{\cos(ka)} + i\sin(ka) (f_{-k}^\d f_{-k}^\d + f_{-k} f_k).
$$

Considerando os dois termos acima destacados em uma caixa, quando se realiza a somatória em $k$ eles se simplificam:
$$
\sum_{k} f_{-k}^\d f_{-k} = \sum_{k} f_{k}^\d f_k, \quad (\text{trocando a somatória por} \, -k),
$$
$$
\sum_{k} \cos(ka) = \frac{1}{2} \sum_{k} e^{ika} + \frac{1}{2} \sum_{k} e^{-ika} = 0,
$$
pois $\sum_{k} e^{\pm ika(j-j')} = N \delta_{j,j'} = N \delta_{(j-j', 0)} \implies \sum_{k} e^{\pm ika} = \delta_{1,0} = 0$.

\n

Dessa maneira, esses cálculos mostram que a hamiltoniana se reescreve como
$$
J \sum_{k}
\begin{pmatrix}
f_k^\d & f_{-k}
\end{pmatrix}
\begin{pmatrix}
g-\cos(ka) & -i\sin(ka) \\
i\sin(ka) & -g+\cos(ka)
\end{pmatrix}
\begin{pmatrix}
f_k \\ f_{-k}^\d
\end{pmatrix}
=
$$
$$
= J \sum_{k} \qty{ 2[g-\cos(ka)] f_k^\d f_k + i\sin(ka) \Big(f_{-k}^\d f_{-k}^\d + f_{-k} f_k\Big) - g } =
H.
$$

\n

Usando a transformação de Bogoliubov explicitada
$$
f_k = \cos(\frac{\theta_k}{2}) \gamma_k + i \sin(\frac{\theta_k}{2}) \gamma_{-k}^\d \implies
\begin{pmatrix}
f_k \\ f_{-k}^\d
\end{pmatrix}
=
\begin{pmatrix}
\cos(\frac{\theta_k}{2}) & i \sin(\frac{\theta_k}{2}) \\
i \sin(\frac{\theta_k}{2}) & \cos(\frac{\theta_k}{2})
\end{pmatrix}
\begin{pmatrix}
\gamma_k \\ \gamma_{-k}^\d
\end{pmatrix} ,
$$
com $\theta_{-k} = -\theta_k$, temos
$$
\begin{pmatrix}
f_k^\d & f_{-k}
\end{pmatrix}
\begin{pmatrix}
g-\cos(ka) & -i\sin(ka) \\
i\sin(ka) & -[g-\cos(ka)]
\end{pmatrix}
\begin{pmatrix}
f_k \\ f_{-k}^\d
\end{pmatrix}
=
$$
$$
=
\begin{pmatrix}
\gamma_k^\d & \gamma_{-k}
\end{pmatrix}
\begin{pmatrix}
\cos(\frac{\theta_k}{2}) & -i \sin(\frac{\theta_k}{2}) \\
-i \sin(\frac{\theta_k}{2}) & \cos(\frac{\theta_k}{2})
\end{pmatrix}
\begin{pmatrix}
g-\cos(ka) & -i\sin(ka) \\
i\sin(ka) & -[g-\cos(ka)]
\end{pmatrix}
\begin{pmatrix}
\cos(\frac{\theta_k}{2}) & i \sin(\frac{\theta_k}{2}) \\
i \sin(\frac{\theta_k}{2}) & \cos(\frac{\theta_k}{2})
\end{pmatrix}
\begin{pmatrix}
\gamma_k \\ \gamma_{-k}^\d
\end{pmatrix}.
$$

Portanto, basta que efetuemos as multiplicações de matrizes:
$$
\begin{pmatrix}
g-\cos(ka) & -i\sin(ka) \\
i\sin(ka) & -[g-\cos(ka)]
\end{pmatrix}
\begin{pmatrix}
\cos(\frac{\theta_k}{2}) & i \sin(\frac{\theta_k}{2}) \\
i \sin(\frac{\theta_k}{2}) & \cos(\frac{\theta_k}{2})
\end{pmatrix}
=
$$
$$
\begin{pmatrix}
[g-\cos(ka)] \cos(\frac{\theta_k}{2}) + \sin(ka) \sin(\frac{\theta_k}{2}) &
[g-\cos(ka)] i \sin(\frac{\theta_k}{2}) - i \sin(ka) \cos(\frac{\theta_k}{2}) \\
-[g-\cos(ka)] i \sin(\frac{\theta_k}{2}) + i \sin(ka) \cos(\frac{\theta_k}{2}) &
-[g-\cos(ka)] \cos(\frac{\theta_k}{2}) - \sin(ka) \sin(\frac{\theta_k}{2})
\end{pmatrix}.
$$
$$
\begin{pmatrix}
\cos(\frac{\theta_k}{2}) & -i \sin(\frac{\theta_k}{2}) \\
-i \sin(\frac{\theta_k}{2}) & \cos(\frac{\theta_k}{2})
\end{pmatrix}
\times
$$
$$
\times
\begin{pmatrix}
[g-\cos(ka)] \cos(\frac{\theta_k}{2}) + \sin(ka) \sin(\frac{\theta_k}{2}) &
[g-\cos(ka)] i \sin(\frac{\theta_k}{2}) - i \sin(ka) \cos(\frac{\theta_k}{2}) \\
-[g-\cos(ka)] i \sin(\frac{\theta_k}{2}) + i \sin(ka) \cos(\frac{\theta_k}{2}) &
-[g-\cos(ka)] \cos(\frac{\theta_k}{2}) - \sin(ka) \sin(\frac{\theta_k}{2})
\end{pmatrix} =
$$
$$
\tiny{
\begin{pmatrix}
[g-\cos(ka)] [\cos[2](\frac{\theta_k}{2}) - \sin[2](\frac{\theta_k}{2})] +
2 \sin(ka) \sin(\frac{\theta_k}{2}) \cos(\frac{\theta_k}{2}) &
i [g-\cos(ka)] 2 \sin(\frac{\theta_k}{2}) \cos(\frac{\theta_k}{2}) - i \sin(ka) [\cos[2](\frac{\theta_k}{2}) - \sin[2](\frac{\theta_k}{2})] \\
- i [g-\cos(ka)] 2 \sin(\frac{\theta_k}{2}) \cos(\frac{\theta_k}{2}) + i \sin(ka) [\cos[2](\frac{\theta_k}{2}) - \sin[2](\frac{\theta_k}{2})] &
- [g-\cos(ka)] [\cos[2](\frac{\theta_k}{2}) - \sin[2](\frac{\theta_k}{2})] -
2 \sin(ka) \sin(\frac{\theta_k}{2}) \cos(\frac{\theta_k}{2})
\end{pmatrix} }
$$
$$
=
\begin{pmatrix}
\qty{[g-\cos(ka)] \cos\theta_k + \sin(ka) \sin\theta_k } &
 i \qty{[g-\cos(ka)] \sin\theta_k - \sin(ka) \cos\theta_k} \\
-i \qty{[g-\cos(ka)] \sin\theta_k - \sin(ka) \cos\theta_k} &
-\qty{[g-\cos(ka)] \cos\theta_k + \sin(ka) \sin\theta_k}
\end{pmatrix}.
$$

Olhando para a matriz acima, vemos que ela se torna diagonal quando
$$
[g-\cos(ka)] \sin\theta_k - \sin(ka) \cos\theta_k = 0 \implies
\boxed{\tan\theta_k = \frac{\sin(ka)}{g-\cos(ka)}.}
$$

Usando as relações $\cos\theta_k = \frac{1}{\sqrt{1+\tan^2\theta_k}}$ e
$\sin\theta_k = \frac{\tan\theta_k}{\sqrt{1+\tan^2\theta_k}}$, é direto mostrar que o termo da diagonal se escreve como
$$
[g-\cos(ka)] \cos\theta_k + \sin(ka) \sin\theta_k =
\sqrt{[g-\cos(ka)]^2 + \sin[2](ka)} = \lambda_k.
$$

Isso mostra então que
$$
H = J \sum_{k}
\begin{pmatrix}
\gamma_k^\d & \gamma_{-k}
\end{pmatrix}
\begin{pmatrix}
\lambda_k & 0 \\
0 & -\lambda_k
\end{pmatrix}
\begin{pmatrix}
\gamma_k \\ \gamma_{-k}^\d
\end{pmatrix} =
J \sum_{k} (\lambda_k \gamma_k^\d \gamma_k -
\lambda_k \underbrace{\gamma_{-k} \gamma_{-k}^\d}_{= 1 - \gamma_{-k}^\d \gamma_{-k}}) =
\sum_{k} \underbrace{2 J \lambda_k}_{=E_k} \qty(\gamma_k^\d \gamma_k - \frac{1}{2}).
$$

Assim, identificamos que o autovalor positivo de $M(k)$ é
$$
\boxed{ E_k = 2J \lambda_k = 2J \sqrt{1+g^2-2g\cos(ka)}. }
$$

\n

Na Figura \ref{fig:transv_ising2} fazemos os esboços de $E_k/2J \times ka$ para diferentes valores de $g$. Percebemos que para $g = 1$ o gap da dispersão fecha, e aparece um cone de Dirac ($E_k \propto \abs{k}$ para $k \approx 0$). Isso indica que a transição de fase quântica acontece em $g_c = 1$, devido ao fechamento do gap.
Note que o gap é dado pelo mínimo de $E_k$ que ocorre quando $k = 0$:
$$
\Delta = E_{k=0} = 2J \sqrt{g^2 - 2g + 1} = 2J \sqrt{(g-1)^2} = 2J \abs{g-1}.
$$

\n

\begin{figure}[H]
\centering
\includegraphics[width=0.6\linewidth]{fig/transv_ising2.png}
\caption{Dispersão $E_k$ para diferentes valores de $g$. No ponto $g_c = 1$ temos a transição de fase quântica, pois o gap fecha (a curva $E_k$ toca o zero).}
\label{fig:transv_ising2}
\end{figure}


Em nossas vimos que o comprimento de correlação é definido a partir do propagador $G(\k)$ que obtivemos considerando flutuações espaciais na teoria de Landau:
$$
G(\k) = \frac{k_B T / Ja^2}{k^2 + \xi^{-2}}.
$$

Veja que esse propagador é análogo a de uma partícula relativística, onde $G \propto \frac{1}{k^2 + m^2}$, de maneira que $\xi^{-1}$ faz o papel da massa $m$ da partícula. Por outro lado, se expandirmos $\cos(ka) \approx 1 - \frac{(ka)^2}{2}$ na dispersão $E_k$, nós obtemos
$$
E_k \approx 2J \sqrt{1 + g^2 - 2g \qty[1-\frac{(ka)^2}{2}]} =
\sqrt{2g(ka)^2 + []}
$$
<++>


\n\n\n\n\n\n
$$
H =
\frac{1}{2}
\sum_{k}
\begin{pmatrix}
f_k^\d & f_{-k}
\end{pmatrix}
\begin{pmatrix}
2(h - J \cos k) & -2iJ \sin k \\
2iJ \sin k & -2(h - J\cos k)
\end{pmatrix}
\begin{pmatrix}
f_k \\ f_{-k}^\d
\end{pmatrix}
+ \cte.
$$

É imediato que os autovalores são $\omega(k) = \pm 2\sqrt{J^2 + h^2 - 2hJ \cos k}$. A transformação de Bogoliubov que diagonaliza a hamiltoniana é
$$
\begin{pmatrix}
f_k \\ f_{-k}^\d
\end{pmatrix}
=
\begin{pmatrix}
 \cos \theta_k & i\sin \theta_k  \\
i\sin \theta_k &  \cos \theta_k
\end{pmatrix}
\begin{pmatrix}
a_k \\ a_{-k}^\d
\end{pmatrix},
$$
e fazendo as contas, descobre-se que $\tan(2\theta_k) = \frac{J \sin k}{h - J \cos k}$.

\begin{figure}[H]
\centering
\includegraphics[width=0.5\textwidth]{fig/transv_ising.png}
\caption{Dispersão $\omega(k) = 2\sqrt{J^2 + h^2 - 2hJ \cos k}$ para várias razões $h/J$. Vemos que para $h/J = 1$ a curva forma um cone (fecha o gap).}
\label{fig:transv_ising}
\end{figure}

Olhando para a Figura \ref{fig:transv_ising}, podemos identificar que a transição de fase quântica ocorre quando $h/J = 1$, que é o valor em que a dispersão forma um cone em torno de $k = 0$ (fechamento do gap).

\n

A magnetização é dada por
$$
m_z = \frac{1}{N} \sum_{j} \ev{\s_j^z} = 2 \frac{1}{N}\sum_{k} \ev{f_k^\d f_k} - 1.
$$

Pela transformação de Bogoliubov tem-se
$$
f_k^\d f_k = \cos[2](\theta_k) a_k^\d a_k + \sin[2](\theta_k) a_{-k}a_{-k}^\d
+ i \sin\theta_k \cos\theta_k (a_{k}^\d a_{-k}^\d - a_{-k} a_{-k})
$$

Para $T = 0$ temos que $\ev{X} = \ev{X}{0}$ e que $a_k \ket{0} = 0$. Portanto $\ev{f_k^\d f_k} = \sin[2](\theta_k)$. Usando que $2 \sin[2](x) = \frac{\tan[2](2x)}{1 + \tan[2](2x)}$, obtemos que
$$
m_z + 1 = \int_{-\pi}^{\pi} \frac{J^2 \sin[2](k)}{[h-J\cos(k)]^2 + J^2\sin[2](k)} \dd{k}.
$$
Pedindo para o Mathematica resolver a integral acima, ele retorna
$$
m_z + 1 =
\begin{cases}
\; \pi \frac{J^2}{h^2} , \quad h/J \geq 1, \\
\; \pi , \quad \quad \, h/J < 1. \\
\end{cases}
$$

Acho que o resultado acima não faz sentido né... Mas não sei como consertar.

\n

(d) O hamiltoniano de Kitaev é
$$
H = -t \sum_{i} (f_{i+1}^\d f_i + \hc) - \mu \sum_{i} f_i^\d f_i
+ \Delta \sum_{i} (f_{i+1}^\d f_i^\d + \hc).
$$

É fácil ver que a hamiltoniana do Ising transverso \ref{eq:hamil_transvising} é um caso particular do Kitaev para $\Delta = -J$, $t = J$ e $\mu = -2h$. Como podemos ver em \url{https://topocondmat.org/w1_topointro/1D.html}, para o modelo de Kitaev, a fase trivial acontece para $\abs{\mu} < 2t$ e a topológica para $\abs{\mu} > 2t$. Essas duas fases correspondem à paramagnética e ferromagnética do Ising transversal, respectivamente.

\n\n

\section*{2) Mapa clássico quântico}




%%-----
%% Referências bibliográficas
%%-----
\addcontentsline{toc}{chapter}{\bibname}
%\bibliographystyle{abntex2-num}
\bibliography{citations}
\bibliographystyle{ieeetr}


\end{document}
