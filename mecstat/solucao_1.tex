\documentclass[a4paper,10pt]{article}
\usepackage[brazilian]{babel}
\usepackage[left=2.5cm,right=2.5cm,top=3cm,bottom=2.5cm]{geometry}
\usepackage{mathtools}
\usepackage{amsthm}
\usepackage{amsmath}
%\usepackage{nccmath}
\usepackage{amssymb}
\usepackage{amsfonts}
\usepackage{physics}
%\usepackage{dsfont}
%\usepackage{mathrsfs}

\usepackage{titling}
\usepackage{indentfirst}

\usepackage{bm}
\usepackage[dvipsnames]{xcolor}
\usepackage{cancel}

\usepackage{xurl}
\usepackage[colorlinks=true]{hyperref}

\usepackage{float}
\usepackage{graphicx}
%\usepackage{tikz}
\usepackage{caption}
\usepackage{subcaption}

%%%%%%%%%%%%%%%%%%%%%%%%%%%%%%%%%%%%%%%%%%%%%%%%%%%

\newcommand{\eps}{\epsilon}
\newcommand{\vphi}{\varphi}
\newcommand{\cte}{\text{cte}}

\newcommand{\N}{\mathbb{N}}
\newcommand{\Z}{\mathbb{Z}}
\newcommand{\Q}{\mathbb{Q}}
\newcommand{\R}{\vb{R}}
\newcommand{\C}{\mathbb{C}}
\renewcommand{\S}{\hat{S}}
%\renewcommand{\H}{\s{H}}

\renewcommand{\a}{\vb{a}}
\newcommand{\nn}{\hat{n}}
\renewcommand{\d}{\dagger}
\newcommand{\up}{\uparrow}
\newcommand{\down}{\downarrow}

\newcommand{\0}{\vb{0}}
%\newcommand{\1}{\mathds{1}}
\newcommand{\E}{\vb{E}}
\newcommand{\B}{\vb{B}}
\renewcommand{\v}{\vb{v}}
\renewcommand{\r}{\vb{r}}
\renewcommand{\k}{\vb{k}}
\newcommand{\p}{\vb{p}}
\newcommand{\q}{\vb{q}}
\newcommand{\F}{\vb{F}}

\newcommand{\s}{\sigma}
%\newcommand{\prodint}[2]{\left\langle #1 , #2 \right\rangle}
\newcommand{\cc}[1]{\overline{#1}}
\newcommand{\Eval}[3]{\eval{\left( #1 \right)}_{#2}^{#3}}

\newcommand{\unit}[1]{\; \mathrm{#1}}

\newcommand{\n}{\medskip}
\newcommand{\e}{\quad \mathrm{e} \quad}
\newcommand{\ou}{\quad \mathrm{ou} \quad}
\newcommand{\virg}{\, , \;}
\newcommand{\ptodo}{\forall \,}
\renewcommand{\implies}{\; \Rightarrow \;}
%\newcommand{\eqname}[1]{\tag*{#1}} % Tag equation with name

\setlength{\droptitle}{-7em}

\theoremstyle{plain}
\newtheorem{theorem}{Teorema}[section]
%\newtheorem{defi}[theorem]{Definição}
\newtheorem{lemma}[theorem]{Lema}
%\newtheorem{corol}[theorem]{Corolário}
%\newtheorem{prop}[theorem]{Proposição}
%\newtheorem{example}{Exemplo}
%
%\newtheorem{inneraxiom}{Axioma}
%\newenvironment{axioma}[1]
%  {\renewcommand\theinneraxiom{#1}\inneraxiom}
%  {\endinneraxiom}
%
%\newtheorem{innerpostulado}{Postulado}
%\newenvironment{postulado}[1]
%  {\renewcommand\theinnerpostulado{#1}\innerpostulado}
%  {\endinnerpostulado}
%
%\newtheorem{innerexercise}{Exercício}
%\newenvironment{exercise}[1]
%  {\renewcommand\theinnerexercise{#1}\innerexercise}
%  {\endinnerexercise}
%
%\newtheorem{innerthm}{Teorema}
%\newenvironment{teorema}[1]
%  {\renewcommand\theinnerthm{#1}\innerthm}
%  {\endinnerthm}
%
\newtheorem{innerlema}{Lema}
\newenvironment{lema}[1]
  {\renewcommand\theinnerlema{#1}\innerlema}
  {\endinnerlema}
%
%\theoremstyle{remark}
%\newtheorem*{hint}{Dica}
%\newtheorem*{notation}{Notação}
%\newtheorem*{obs}{Observação}


\title{\Huge{\textbf{Lista 1 - Mecânica Estatística}}}
\author{Mateus Marques}

\begin{document}

\maketitle

\section*{2) Oscilador harmônico clássico no ensemble microcanônico}

(a) Temos que
$$
\Gamma(E) = \int \frac{\dd{p_1} \dd{q_1}}{2\pi\hbar} \cdots \int \frac{\dd{p_N} \dd{q_N}}{2\pi\hbar} 1,
$$
onde a integração é sobre o volume em que $H \leq E$, sendo
$$
H = \sum_{i=1}^{N}
\qty(
\frac{p_i^2}{2m} + \frac{m\omega^2 q_i^2}{2}
).
$$

Se mudarmos as variáveis $P_i^2 = \frac{p_i^2}{2m}$ e $Q_i^2 = \frac{m \omega^2 q_i^2}{2}$, o vínculo pode ser reescrito como
$$
\sum_{i=1}^{N} (P_i^2 + Q_i^2) \leq (\sqrt{E})^2,
$$
que define uma bola $2N-$dimensional de raio $R = \sqrt{E}$ nas variáveis $P_i, Q_i$. Assim
$$
\Gamma(E) = \int \frac{\dd{P_1} \dd{Q_1}}{\pi\hbar \omega} \cdots \int \frac{\dd{P_N} \dd{Q_N}}{\pi\hbar \omega} =
\frac{1}{(\pi \hbar \omega)^N} V_{2N}(\sqrt{E}),
$$
onde $V_n(R) = \frac{\pi^{n/2}}{(n/2)!} R^n$ (na minha notação $x!$ é a função Gamma) é o volume da bola $n-$dimensional de raio $R$ (\url{https://en.wikipedia.org/wiki/Volume_of_an_n-ball}).

Portanto
$$
\Gamma(E) =
\frac{1}{(\pi \hbar \omega)^N} \, \frac{\pi^N}{N!} \, E^N = \frac{1}{N!} \, \qty(\frac{E}{\hbar\omega})^N.
$$

A aproximação de Stirling que derivamos na Questão 1 é
$$
N! = \sqrt{2\pi N} \, N^N e^{-N}.
$$

Temos que
$$
\log \Gamma(E) = N \log(E) - N \log(\hbar\omega) - \log N! \implies
$$
$$
\frac{D(E)}{\Gamma(E)} = \dv{\log \Gamma(E)}{E} = \frac{N}{E} \implies
$$
$$
\boxed{ D(E) \simeq \sqrt{\frac{N}{2\pi E^2}} \, \qty(\frac{E e}{N \hbar\omega})^N .}
$$

\n\n

(b) Falta calcular a entropia $S(E, N, V)$ para $N \to \infty$, temperatura $1/T = \qty(\pdv{S}{E})_{V,N}$, pressão $P = -T \qty(\pdv{S}{V})_{E,N}$ e potencial químico $\mu = T \qty(\pdv{S}{N})_{E, V}$. O Eric falou pra discutir que $P = 0$ porque a entropia não depende do volume e também porque o potencial é harmônico, ou seja, a pressão não deforma o volume do sistema.

\textbf{FALTA ITEM (b)}


\pagebreak

\section*{3) Transformada de Legendre}

(a) Sendo $E(S)$ tal que o sinal de sua curvatura não mude, sem perda de generalidade $\dv[2]{E}{S} > 0$, temos que sua derivada $T = \dv{E}{S}$ é uma função crescente com $S$. Em particular, $T(S)$ é inversível. Assim, dado $T$, é possível achar os valores de $S(T)$ e $E(S(T))$ correspondentes resolvendo $\dv{E}{S} = T$ para $S$. Pela definição de $F$ como o intercepto da tangente da curva $E(S)$ com o eixo $E$, é imediato a relação $E = F + \dv{E}{S} \, S$. Isso define $F(T) = E(S(T)) - T S(T)$ inequivocamente como função apenas de $T$.

\n\n

(b) Pela própria definição de $T = \dv{E}{S}$, é imediato que $\dd{E}(S) = T(S) \dd{S}$. Agora, para a energia livre aplicamos regras de derivadas:
$$
\dv{F}{T} = \xcancel{\dv{E}{S} \dv{S}{T}} - S(T) - \xcancel{T \dv{S}{T}} = -S(T) \implies \boxed{\dd{F}(T) = -S(T) \dd{T}.}
$$

\n\n

(c) Como $E(S,V)$ tem os sinais de curvatura definidos tanto para $S$ quanto para $V$, definindo $\displaystyle{ T = \qty(\pdv{E}{S})_V }$ e $\displaystyle{P = - \qty(\pdv{E}{V})_S}$, podemos resolver a primeira equação para $S$ em função de $T, V$ e a segunda para $V$ em função de $S, P$. Isso define as funções $S(T,V)$ e $V(S,P)$.

\begin{itemize}
\item Mantendo $V$ constante, define-se $F(T,V)$ como o intercepto da tangente de $E(S(T,V), V)$ com o eixo $E$, o que nos dá $\boxed{ F(T,V) = E(S(T,V), V) - T S(T,V) }$.

\item Mantendo $S$ constante, define-se $H(S,P)$ como o intercepto da tangente de $E(S, V(S,P))$ com o eixo $E$. Assim $\displaystyle{E = H + V \qty(\pdv{E}{V})_S}$ e temos que $\boxed{ H(S,P) = E(S, V(S,P)) + P V(S,P) }$.
\end{itemize}

\n

As diferenciais completas de $F(T,V)$ e $H(S, P)$ são dadas por
$$
\dd{F} = \qty(\pdv{F}{T})_{V} \dd{T} + \qty(\pdv{F}{V})_{T} \dd{V},
$$
$$
\dd{H} = \qty(\pdv{H}{S})_{P} \dd{S} + \qty(\pdv{H}{P})_{S} \dd{P}.
$$

\n

Calculando as derivadas:
$$
\qty(\pdv{F}{T})_{V} = \cancel{\qty(\pdv{E}{S})_V \qty(\pdv{S}{T})_V} - S - \cancel{T \qty(\pdv{S}{T})_V} = - S.
$$
$$
\qty(\pdv{F}{V})_{T} = \cancel{\qty(\pdv{E}{S})_V \qty(\pdv{S}{V})_T} + \cancelto{\boxed{-P}}{\qty(\pdv{E}{V})_S} - \cancel{T \qty(\pdv{S}{V})_T} = -P.
$$
$$
\qty(\pdv{H}{S})_{P} = \cancelto{\boxed{T}}{\qty(\pdv{E}{S})_V} + \cancel{\qty(\pdv{E}{V})_S \qty(\pdv{V}{S})_P} + \cancel{P \qty(\pdv{V}{S})_P} = T.
$$
$$
\qty(\pdv{H}{P})_{S} = \cancel{\qty(\pdv{E}{V})_S \qty(\pdv{V}{P})_S} + V + \cancel{P \qty(\pdv{V}{P})_S} = V.
$$

Isso estabelece que
$$
\boxed{ \dd{F} = -S \dd{T} - P \dd{V}, }
$$
$$
\boxed{ \dd{H} = T \dd{S} + V \dd{P}. }
$$


\pagebreak

\section*{5) Gás ideal}

Neste exercício utilizarei as duas relações de Maxwell
\begin{equation} \label{eq:maxwell_1}
\pdvc{S}{V}{T} = \pdvc{P}{T}{V} = - \pdv{F}{T}{V},
\end{equation}
\begin{equation} \label{eq:maxwell_2}
\pdvc{S}{P}{T} = - \pdvc{V}{T}{P} = - \pdv{G}{T}{P},
\end{equation}
e também
\begin{equation} \label{eq:ptvp}
\pdvc{P}{T}{V} \pdvc{T}{V}{P} \pdvc{V}{P}{T} = -1,
\end{equation}
que foram demonstradas no item (4) dessa lista.

\n

(a)
\begin{itemize}
\item Considerando a entropia como uma função de $T$ e $P$, da forma $S(T, V(T, P))$, temos pela regra da cadeia que
$$
\pdvc{S}{T}{P} = \pdvc{S}{T}{V} + \pdvc{S}{V}{T} \pdvc{V}{T}{P}.
$$

Multiplicando a equação acima por $T$, pelas definições de $C_V$ e $C_P$ temos
$$
C_P - C_V = T \pdvc{S}{V}{T} \pdvc{V}{T}{P}.
$$
Usando então as relações \ref{eq:maxwell_1} e \ref{eq:ptvp} obtemos
\begin{equation} \label{eq:cpcv}
C_P - C_V = T \pdvc{P}{T}{V} \pdvc{V}{T}{P} = - \pdvc{V}{T}{P}^2 \pdvc{P}{V}{T} > 0,
\end{equation}
que é maior que zero pois $\pdvc{P}{V}{T} < 0$, i.e., se o volume aumenta a pressão diminui.

\n

\item Considerando a energia interna $E$ como função de $T$ e $V$, da forma $E(S(T,V), V)$, usamos a relação \ref{eq:maxwell_1} e obtemos
$$
\pdvc{E}{V}{T} = \pdvc{E}{S}{V} \pdvc{S}{V}{T} + \pdvc{E}{V}{S} = T \pdvc{P}{T}{V} - P.
$$

\n

\item Considerando agora $E$ como função de $T$ e $P$, da forma $E(S(T,P), V(T,P))$, usamos a relação \ref{eq:maxwell_2} e temos
$$
\pdvc{E}{P}{T} = \pdvc{E}{S}{V} \pdvc{S}{P}{T} + \pdvc{E}{V}{S} \pdvc{V}{P}{T} = - T \pdvc{V}{T}{P} - P \pdvc{V}{P}{T}.
$$

\n

\item Usando as relações de Maxwell \ref{eq:maxwell_1}, \ref{eq:maxwell_2} e trocando a ordem das derivadas parciais, segue
$$
\pdvc{C_V}{V}{T} = \pdv{V} \qty(T \pdvc{S}{T}{V})_T = T \pdv{T} \qty(\pdvc{S}{V}{T})_V = T \qty(\pdv[2]{P}{T})_V.
$$
$$
\pdvc{C_P}{P}{T} = \pdv{P} \qty(T \pdvc{S}{T}{P})_T = T \pdv{T} \qty(\pdvc{S}{P}{T})_P = -T \qty(\pdv[2]{V}{T})_P.
$$
\end{itemize}

\n\n\n\n

(b) Por hipótese temos $PV = N k_B T$ e $C_V = \alpha N k_B$. Usando a equação \ref{eq:cpcv} temos
$$
C_P - C_V = T \qty(\frac{N k_B}{P}) \qty(\frac{N k_B}{V}) = N k_B \implies \boxed{ C_P = (\alpha + 1) N k_B. }
$$

Para a entropia, calculemos sua diferencial
$$
\dd{S} = \pdvc{S}{T}{V} \dd{T} + \pdvc{S}{V}{T} \dd{V}.
$$

Pela definição de $C_V$:
$$
\pdvc{S}{T}{V} = \frac{C_V}{T} = \frac{\alpha N k_B}{T}.
$$

Pela relação de Maxwell \ref{eq:maxwell_1} e usando a equação de estado:
$$
\pdvc{S}{V}{T} = \pdvc{P}{T}{V} = \frac{N k_B}{V}.
$$

Integrando a diferencial:
$$
S = \int \dd{S} = \int \qty(\frac{\alpha N k_B}{T} \dd{T} + \frac{N k_B}{V} \dd{V})
\implies \boxed{ S = N k_B \log V + \alpha N k_B \log T + \cte. }
$$

\n\n

Em um processo adiabático a entropia $S(T,V) = S(T_0, V_0)$ é constante, portanto
$$
N k_B \log V + \alpha N k_B \log T = N k_B \log V_0 + \alpha N k_B \log T_0 \implies
\log(V T^\alpha) = \log(V_0 T_0^\alpha)
$$
$$
\implies V T^{\alpha} = \cte.
$$

Substituindo $T = PV / N k_B$ também temos
$$
V \qty(\frac{PV}{N k_B})^{\alpha} = \cte. \implies P^\alpha V^{\alpha + 1} = \cte. \implies P V^{\gamma} = \cte,
$$
onde $\gamma = \frac{\alpha + 1}{\alpha} = C_P / C_V$.

\end{document}
