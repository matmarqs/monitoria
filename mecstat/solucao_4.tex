\documentclass[a4paper,10pt]{article}

\usepackage[brazilian]{babel}
\usepackage[left=2.5cm,right=2.5cm,top=3cm,bottom=2.5cm]{geometry}
\usepackage{mathtools}
\usepackage{amsthm}
\usepackage{amsmath}
%\usepackage{nccmath}
\usepackage{amssymb}
\usepackage{amsfonts}
\usepackage{physics}
%\usepackage{dsfont}
%\usepackage{mathrsfs}

\usepackage{titling}
\usepackage{indentfirst}

\usepackage{bm}
\usepackage[dvipsnames]{xcolor}
\usepackage{cancel}

\usepackage{xurl}
\usepackage[colorlinks=true]{hyperref}

\usepackage{float}
\usepackage{graphicx}
%\usepackage{tikz}
\usepackage{caption}
\usepackage{subcaption}

%%%%%%%%%%%%%%%%%%%%%%%%%%%%%%%%%%%%%%%%%%%%%%%%%%%

\newcommand{\eps}{\epsilon}
\newcommand{\vphi}{\varphi}
\newcommand{\cte}{\text{cte}}

\newcommand{\N}{\mathbb{N}}
\newcommand{\Z}{\mathbb{Z}}
\newcommand{\Q}{\mathbb{Q}}
\newcommand{\R}{\vb{R}}
\newcommand{\C}{\mathbb{C}}
\renewcommand{\S}{\hat{S}}
%\renewcommand{\H}{\s{H}}

\renewcommand{\a}{\vb{a}}
\newcommand{\nn}{\hat{n}}
\renewcommand{\d}{\dagger}
\newcommand{\up}{\uparrow}
\newcommand{\down}{\downarrow}

\newcommand{\0}{\vb{0}}
%\newcommand{\1}{\mathds{1}}
\newcommand{\E}{\vb{E}}
\newcommand{\B}{\vb{B}}
\renewcommand{\v}{\vb{v}}
\renewcommand{\r}{\vb{r}}
\renewcommand{\k}{\vb{k}}
\newcommand{\p}{\vb{p}}
\newcommand{\q}{\vb{q}}
\newcommand{\F}{\vb{F}}

\newcommand{\s}{\sigma}
%\newcommand{\prodint}[2]{\left\langle #1 , #2 \right\rangle}
\newcommand{\cc}[1]{\overline{#1}}
\newcommand{\Eval}[3]{\eval{\left( #1 \right)}_{#2}^{#3}}

\newcommand{\unit}[1]{\; \mathrm{#1}}

\newcommand{\n}{\medskip}
\newcommand{\e}{\quad \mathrm{e} \quad}
\newcommand{\ou}{\quad \mathrm{ou} \quad}
\newcommand{\virg}{\, , \;}
\newcommand{\ptodo}{\forall \,}
\renewcommand{\implies}{\; \Rightarrow \;}
%\newcommand{\eqname}[1]{\tag*{#1}} % Tag equation with name

\setlength{\droptitle}{-7em}

\theoremstyle{plain}
\newtheorem{theorem}{Teorema}[section]
%\newtheorem{defi}[theorem]{Definição}
\newtheorem{lemma}[theorem]{Lema}
%\newtheorem{corol}[theorem]{Corolário}
%\newtheorem{prop}[theorem]{Proposição}
%\newtheorem{example}{Exemplo}
%
%\newtheorem{inneraxiom}{Axioma}
%\newenvironment{axioma}[1]
%  {\renewcommand\theinneraxiom{#1}\inneraxiom}
%  {\endinneraxiom}
%
%\newtheorem{innerpostulado}{Postulado}
%\newenvironment{postulado}[1]
%  {\renewcommand\theinnerpostulado{#1}\innerpostulado}
%  {\endinnerpostulado}
%
%\newtheorem{innerexercise}{Exercício}
%\newenvironment{exercise}[1]
%  {\renewcommand\theinnerexercise{#1}\innerexercise}
%  {\endinnerexercise}
%
%\newtheorem{innerthm}{Teorema}
%\newenvironment{teorema}[1]
%  {\renewcommand\theinnerthm{#1}\innerthm}
%  {\endinnerthm}
%
\newtheorem{innerlema}{Lema}
\newenvironment{lema}[1]
  {\renewcommand\theinnerlema{#1}\innerlema}
  {\endinnerlema}
%
%\theoremstyle{remark}
%\newtheorem*{hint}{Dica}
%\newtheorem*{notation}{Notação}
%\newtheorem*{obs}{Observação}


\title{\Huge{\textbf{Lista 3 - Mecânica Estatística}}}
\author{Mateus Marques}

\begin{document}

\maketitle

\section*{1) Gases não ideais}

(a) Primeiramente, como $\displaystyle{N = \ev{\hat{N}} = \frac{\tr[\hat{N} e^{-\beta(\hat{H} - \mu \hat{N})}]}{Z} = \frac{1}{\beta Z} \pdv{Z}{\mu}}$, temos que
$$
\pdvc{N}{\mu}{T,V} = - \frac{\tr[\hat{N} e^{-\beta(\hat{H} - \mu \hat{N})}]}{Z^2} \pdv{Z}{\mu} +
\beta \frac{\tr[\hat{N}^2 e^{-\beta(\hat{H} - \mu \hat{N})}]}{Z} =
$$
$$
= \beta \qty{ \ev{\hat{N}^2} - \ev{\hat{N}}^2 }.
$$


Lembrando que $v = V/N$, temos
$$
\kappa = - \frac{1}{v} \pdvc{v}{P}{T} = - \frac{N}{V} \pdvc{(V/N)}{P}{T,V} =
- \frac{N}{\cancel{V}} \, \cancel{V} \pdvc{(1/N)}{P}{T, V} = -N \qty(-\frac{1}{N^2}) \pdvc{N}{P}{T,V}
$$
$$
= \frac{1}{N} \pdvc{N}{\mu}{T,V} \pdvc{\mu}{P}{T}
$$

Agora de $G(P,T,N) = N \mu(P,T)$, temos
$$
\dd{G} = -S \dd{T} + V \dd{P} + \cancel{\mu \dd{N}} = \cancel{\mu \dd{N}} + N \dd{\mu} \implies
\boxed{ -S \dd{T} + V \dd{P} - N \dd{\mu} = 0. }
$$
Para um processo isotérmico:
$$
V \dd{P} = N \dd{\mu} \implies \pdvc{\mu}{P}{T} = \frac{V}{N}.
$$

Logo
$$
\kappa = \frac{V}{N^2} \pdvc{N}{\mu}{T,V} = \frac{\beta V}{N^2} \, \Big(\ev{\hat{N}}^2 - \ev{\hat{N}^2}\Big).
$$

\n\n\n

(b) Obter $B_2(T)$ e os parâmetros $a$ e $b$ da equação de Van der Waals.

Fatorando a energia livre na parte do gás ideal $F_0$ e na parte de interação, temos
$$
F = - k_B T \qty[
\frac{1}{N!} \int \prod_{i=1}^N \frac{\dd[3]{\r_i} \dd[3]{\p_i}}{(2\pi \hbar)^3} e^{-\beta E(r,p)}
]
$$
<++>


(c) Repetir cálculos das notas de aula para o ponto crítico.


\pagebreak

\section*{2) Modelo de Ising de alcance infinito}



\pagebreak

\section*{4) Pontos multicríticos}



\pagebreak

\section*{5) Parâmetros de ordem acoplados}

%%-----
%% Referências bibliográficas
%%-----
\addcontentsline{toc}{chapter}{\bibname}
%\bibliographystyle{abntex2-num}
\bibliography{citations}
\bibliographystyle{ieeetr}


\end{document}
