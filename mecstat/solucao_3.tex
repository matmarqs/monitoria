\documentclass[a4paper,10pt]{article}

\usepackage[brazilian]{babel}
\usepackage[left=2.5cm,right=2.5cm,top=3cm,bottom=2.5cm]{geometry}
\usepackage{mathtools}
\usepackage{amsthm}
\usepackage{amsmath}
%\usepackage{nccmath}
\usepackage{amssymb}
\usepackage{amsfonts}
\usepackage{physics}
%\usepackage{dsfont}
%\usepackage{mathrsfs}

\usepackage{titling}
\usepackage{indentfirst}

\usepackage{bm}
\usepackage[dvipsnames]{xcolor}
\usepackage{cancel}

\usepackage{xurl}
\usepackage[colorlinks=true]{hyperref}

\usepackage{float}
\usepackage{graphicx}
%\usepackage{tikz}
\usepackage{caption}
\usepackage{subcaption}

%%%%%%%%%%%%%%%%%%%%%%%%%%%%%%%%%%%%%%%%%%%%%%%%%%%

\newcommand{\eps}{\epsilon}
\newcommand{\vphi}{\varphi}
\newcommand{\cte}{\text{cte}}

\newcommand{\N}{\mathbb{N}}
\newcommand{\Z}{\mathbb{Z}}
\newcommand{\Q}{\mathbb{Q}}
\newcommand{\R}{\vb{R}}
\newcommand{\C}{\mathbb{C}}
\renewcommand{\S}{\hat{S}}
%\renewcommand{\H}{\s{H}}

\renewcommand{\a}{\vb{a}}
\newcommand{\nn}{\hat{n}}
\renewcommand{\d}{\dagger}
\newcommand{\up}{\uparrow}
\newcommand{\down}{\downarrow}

\newcommand{\0}{\vb{0}}
%\newcommand{\1}{\mathds{1}}
\newcommand{\E}{\vb{E}}
\newcommand{\B}{\vb{B}}
\renewcommand{\v}{\vb{v}}
\renewcommand{\r}{\vb{r}}
\renewcommand{\k}{\vb{k}}
\newcommand{\p}{\vb{p}}
\newcommand{\q}{\vb{q}}
\newcommand{\F}{\vb{F}}

\newcommand{\s}{\sigma}
%\newcommand{\prodint}[2]{\left\langle #1 , #2 \right\rangle}
\newcommand{\cc}[1]{\overline{#1}}
\newcommand{\Eval}[3]{\eval{\left( #1 \right)}_{#2}^{#3}}

\newcommand{\unit}[1]{\; \mathrm{#1}}

\newcommand{\n}{\medskip}
\newcommand{\e}{\quad \mathrm{e} \quad}
\newcommand{\ou}{\quad \mathrm{ou} \quad}
\newcommand{\virg}{\, , \;}
\newcommand{\ptodo}{\forall \,}
\renewcommand{\implies}{\; \Rightarrow \;}
%\newcommand{\eqname}[1]{\tag*{#1}} % Tag equation with name

\setlength{\droptitle}{-7em}

\theoremstyle{plain}
\newtheorem{theorem}{Teorema}[section]
%\newtheorem{defi}[theorem]{Definição}
\newtheorem{lemma}[theorem]{Lema}
%\newtheorem{corol}[theorem]{Corolário}
%\newtheorem{prop}[theorem]{Proposição}
%\newtheorem{example}{Exemplo}
%
%\newtheorem{inneraxiom}{Axioma}
%\newenvironment{axioma}[1]
%  {\renewcommand\theinneraxiom{#1}\inneraxiom}
%  {\endinneraxiom}
%
%\newtheorem{innerpostulado}{Postulado}
%\newenvironment{postulado}[1]
%  {\renewcommand\theinnerpostulado{#1}\innerpostulado}
%  {\endinnerpostulado}
%
%\newtheorem{innerexercise}{Exercício}
%\newenvironment{exercise}[1]
%  {\renewcommand\theinnerexercise{#1}\innerexercise}
%  {\endinnerexercise}
%
%\newtheorem{innerthm}{Teorema}
%\newenvironment{teorema}[1]
%  {\renewcommand\theinnerthm{#1}\innerthm}
%  {\endinnerthm}
%
\newtheorem{innerlema}{Lema}
\newenvironment{lema}[1]
  {\renewcommand\theinnerlema{#1}\innerlema}
  {\endinnerlema}
%
%\theoremstyle{remark}
%\newtheorem*{hint}{Dica}
%\newtheorem*{notation}{Notação}
%\newtheorem*{obs}{Observação}


\title{\Huge{\textbf{Lista 3 - Mecânica Estatística}}}
\author{Mateus Marques}

\begin{document}

\maketitle

\section*{2) Entropia para bósons e férmions}

Primeiramente, definamos o parâmetro $\zeta = \pm 1$, sendo $\zeta = +1$ referindo à bósons e $\zeta = -1$ a férmions. Queremos mostrar que
\begin{equation} \label{eq:entropia_ruim}
S = \zeta k_B \sum_{j} \Big[
(1 + \zeta n_j) \log(1 + \zeta n_j) - \zeta n_j \log n_j
\Big], \quad n_j = \frac{1}{e^{\beta(\eps_j-\mu)} - \zeta}.
\end{equation}
Desenvolvendo essa expressão (usando que $\zeta^2 = 1$), temos
$$
S = k_B \sum_{j} \qty[
\zeta \log(1 + \zeta n_j) +
n_j \log(\frac{1}{n_j} + \zeta)
].
$$
Substituindo a expressão de $n_j$ explícita:
$$
\zeta \log(1 + \zeta n_j) =
\zeta \log\qty[\frac{(e^{\beta(\eps_j-\mu)}-\cancel{\zeta})+\cancel{\zeta}}{e^{\beta(\eps_j-\mu)}-\zeta}] =
\zeta \qty{ \beta(\eps_j-\mu) + \log\qty[\frac{1}{e^{\beta(\eps_j-\mu)}-\zeta}] }.
$$
$$
n_j \log(\frac{1}{n_j} + \zeta) = \frac{1}{e^{\beta(\eps_j-\mu)}-\zeta}
\log[(e^{\beta(\eps_j-\mu)}-\cancel{\zeta})+\cancel{\zeta}] =
\frac{\beta (\eps_j-\mu)}{e^{\beta(\eps_j-\mu)}-\zeta}.
$$

A entropia então se simplifica para
$$
S = k_B \sum_{j}
\qty{
\zeta\log\qty[\frac{1}{e^{\beta(\eps_j-\mu)}-\zeta}] +
\beta(\eps_j-\mu) \qty[\frac{\zeta e^{\beta(\eps_j-\mu)}-\cancel{\zeta^2}+\cancel{1}}{e^{\beta(\eps_j-\mu)}-\zeta}] =
}
$$
\begin{equation} \label{eq:entropia_boa}
\boxed{ S = \zeta k_B \sum_{j}
\qty[
\log n_j +
\beta(\eps_j-\mu) e^{\beta(\eps_j-\mu)} n_j
]. }
\end{equation}

Dessa forma, temos que a equação \ref{eq:entropia_boa} é equivalente à equação \ref{eq:entropia_ruim}. Mostrarei que a entropia dos gases quânticos se escreve como na forma da equação \ref{eq:entropia_boa}.

\n

Como vimos em aula, no ensemble grande canônico a função de partição é
$$
Z = \prod_j Z_j, \quad Z_j = [1 \mp e^{-\beta(\eps_j-\mu)}]^{\mp 1}= [1 - \zeta e^{-\beta(\eps_j-\mu)}]^{-\zeta},
$$
onde $Z_j$ é a função de partição para o estado $j$. Perceba que $Z_j$ se reescreve como
$$
Z_j = \qty[\frac{1}{ 1 - \zeta e^{-\beta(\eps_j-\mu)} }]^{\zeta} =
\qty[\frac{e^{\beta(\eps_j-\mu)}}{ e^{\beta(\eps_j-\mu)} - \zeta }]^{\zeta} =
\Big[
e^{\beta(\eps_j-\mu)} n_j
\Big]^{\zeta}.
$$

Portanto, a entropia $S = -\pdvc{F}{T}{V,\mu}$, com $F = -k_B T \log Z$, é dada por
$$
S = \pdv{T} \qty(k_B T \sum_{j} \zeta \Big[\beta(\eps_j-\mu) + \log n_j\Big])_{V,\mu} =
$$
$$
= \zeta k_B \sum_{j} \Big[\cancel{\beta(\eps_j-\mu)} + \log n_j\Big] +
\zeta k_B \sum_{j} \qty{-\cancel{\beta (\eps_j - \mu)} + \beta(\eps_j-\mu) n_j e^{\beta(\eps_j-\mu)}} =
$$
$$
= \zeta k_B \sum_{j} \qty[\log n_j + \beta (\eps_j - \mu) n_j e^{\beta(\eps_j-\mu)}],
$$
que é exatamente a equação \ref{eq:entropia_boa}. Na dedução acima eu utilizei (mas não explicitei) as derivadas $T \pdv{\beta}{T} = -\beta$ e $T \pdv{\log n_j}{T} = \beta (\eps_j-\mu) n_j e^{\beta(\eps_j-\mu)}$. Isso termina o exercício.

\pagebreak

\section*{3) Radiação de corpo negro}

<++>

\pagebreak

\section*{5) Condensado de Bose-Einstein: duas dimensões e armadilha parabólica}

<++>

\pagebreak

\section*{6) Paramagnetismo de Pauli}

<++>

\pagebreak




\end{document}
