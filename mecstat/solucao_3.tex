\documentclass[a4paper,10pt]{article}

%\usepackage{mathtools}
\usepackage{amsthm}     % for definitions and theorems
\usepackage[many]{tcolorbox}    % boxes around definitions and theorems
%\usepackage{amsmath}
%\usepackage{nccmath}
\usepackage{amssymb}    % \ltimes, semi-direct product
%\usepackage{etoolbox}   % for start of Chapter
%\usepackage{amsfonts}
\usepackage{physics}    % for all Physics related
\usepackage{dsfont}     % for the identity matrix symbol \1
%\usepackage{mathrsfs}
\usepackage[notextcomp]{stix}   % font package and some symbols like filled square
%\usepackage{MnSymbol}   % symbols font package

\usepackage{titling}
\usepackage{indentfirst}

\usepackage{bm}
\usepackage[dvipsnames]{xcolor}
\usepackage{cancel}
\usepackage{enumitem}

\usepackage{xurl}
%\usepackage[colorlinks=true]{hyperref} % links have colors
\usepackage{hyperref}  % no colors

\usepackage{float}
\usepackage{graphicx}
\usepackage{subcaption}
%\usepackage{tikz}

\usepackage{ctable}     % tabelas
\renewcommand{\P}{\phantom{+}}  % empty space to indent things
\usepackage{multirow}
\usepackage{tabulary}

%%%%%%%%%%%%%%%%%%%%%%%%%%%%%%%%%%%%%%%%%%%%%%%%%%%

\newcommand{\eps}{\epsilon}
\newcommand{\vphi}{\varphi}
\newcommand{\cte}{\text{cte}}

\newcommand{\N}{{\mathbb{N}}}
\newcommand{\Z}{{\mathbb{Z}}}
%\newcommand{\Q}{{\mathbb{Q}}}
\newcommand{\C}{{\mathbb{C}}}
\renewcommand{\S}{{\hat{S}}}
%\renewcommand{\H}{\s{H}}

\renewcommand{\a}{{\vb{a}}}
\renewcommand{\b}{{\vb{b}}}
\renewcommand{\d}{{\dagger}}
\newcommand{\up}{{\uparrow}}
\newcommand{\down}{{\downarrow}}
\newcommand{\hc}{{\text{h.c.}}}

\newcommand{\ihat}{\bm{\hat{\imath}}}
\newcommand{\jhat}{\bm{\hat{\jmath}}}
\newcommand{\khat}{\bm{\hat{k}}}

\newcommand{\0}{{\vb{0}}}
\newcommand{\1}{\mathds{1}}
\newcommand{\E}{{\vb{E}}}
\newcommand{\B}{{\vb{B}}}
\renewcommand{\u}{{\vb{u}}}
\renewcommand{\v}{{\vb{v}}}
\renewcommand{\r}{{\vb{r}}}
\newcommand{\R}{{\vb{R}}}
\newcommand{\Q}{{\vb{Q}}}
\newcommand{\G}{{\vb{G}}}
\newcommand{\g}{{\vb{g}}}
\renewcommand{\k}{{\vb{k}}}
\newcommand{\K}{{\vb{K}}}
\newcommand{\p}{{\vb{p}}}
\newcommand{\q}{{\vb{q}}}
\newcommand{\F}{{\vb{F}}}
\renewcommand{\t}{{\vb{t}}}
\newcommand{\vtau}{{\bm{\tau}}}
\newcommand{\vdelta}{{\bm{\delta}}}

% COLORED SYMMETRY ELEMENTS
\newcommand{\Ct}{{\textcolor{Cyan}{C_3}}}
\newcommand{\Ctn}[1]{{\textcolor{Cyan}{C_3^{\textcolor{black}{#1}}}}}
\newcommand{\Cs}{{\textcolor{ForestGreen}{C_6}}}
\newcommand{\Csn}[1]{{\textcolor{ForestGreen}{C_6^{\textcolor{black}{#1}}}}}
\newcommand{\sd}{{\textcolor{RoyalBlue}{\sigma_d}}}
\newcommand{\sdn}[1]{{\textcolor{RoyalBlue}{\sigma_d^{\textcolor{black}{#1}}}}}
\newcommand{\sdp}{{\textcolor{RoyalBlue}{\sigma_d'}}}
\newcommand{\sdpp}{{\textcolor{RoyalBlue}{\sigma_d''}}}
\newcommand{\sv}{{\textcolor{Orange}{\sigma_v}}}
\newcommand{\svn}[1]{{\textcolor{Orange}{\sigma_v^{\textcolor{black}{#1}}}}}
\newcommand{\svp}{{\textcolor{Orange}{\sigma_v'}}}
\newcommand{\svpp}{{\textcolor{Orange}{\sigma_v''}}}

\newcommand{\GL}{{\text{GL}}}
\newcommand{\U}{{\text{U}}}

\newcommand{\s}{\sigma}
%\newcommand{\prodint}[2]{\left\langle #1 , #2 \right\rangle}
\newcommand{\cc}[1]{\overline{#1}}
\newcommand{\Eval}[3]{\eval{\left( #1 \right)}_{#2}^{#3}}
\newcommand{\sg}[2]{\{ #1 \mid #2 \}}
\renewcommand{\AA}{{\mathring{\text{A}}}}
\newcommand{\I}{{\mathbb{I}}}
\newcommand{\bP}{{\mathbb{P}}}
\newcommand{\bQ}{{\mathbb{Q}}}

\newcommand{\unit}[1]{\; \mathrm{#1}}

\newcommand{\n}{\medskip}
\newcommand{\e}{\quad \mathrm{and} \quad}
\newcommand{\ou}{\quad \mathrm{or} \quad}
\newcommand{\virg}{\, , \;}
\newcommand{\ptodo}{\forall \,}
\renewcommand{\implies}{\; \Rightarrow \;}
%\newcommand{\eqname}[1]{\tag*{#1}} % Tag equation with name

%\setlength{\droptitle}{-7em}   % título um pouco mais em cima na página
%\makeatletter
%\patchcmd{\chapter}{\if@openright\cleardoublepage\else\clearpage\fi}{}{}{}  % start 'Chapter' at the same page. needs package etoolbox
%\makeatother

%% Theorems, definitions, proofs
\theoremstyle{definition}

%%% defining my own colors %%%
\definecolor{my-blue}{HTML}{f2f4ff}
\definecolor{my-green}{HTML}{f5fcf6}    % a little better: green!5!white
\definecolor{my-cyan}{HTML}{f2fffe}
\definecolor{my-yellow}{HTML}{fffbed}
\definecolor{my-green2}{HTML}{efffdb}

%%% alternative colors %%%
\definecolor{my-pink}{HTML}{fff2f7}
\definecolor{my-teal}{HTML}{ebfffc}

\newtheorem{definition}{Definition}[section]
\tcolorboxenvironment{definition}{
  colback=my-blue,
  %colback=blue!5!white,
  boxrule=0.1pt,
  boxsep=1pt,
  left=2pt,right=2pt,top=2pt,bottom=2pt,
  oversize=2pt,
  sharp corners,
  before skip=\topsep,
  after skip=\topsep,
}

\newtheorem{theorem}{Theorem}[section]
\tcolorboxenvironment{theorem}{
  colback=my-yellow,
  %colback=yellow!22!white!95!black,
  boxrule=0.1pt,
  boxsep=1pt,
  left=2pt,right=2pt,top=2pt,bottom=2pt,
  oversize=2pt,
  sharp corners,
  before skip=\topsep,
  after skip=\topsep,
}

\newtheorem{corollary}{Corollary}[section]
\tcolorboxenvironment{corollary}{
  colback=my-green2,
  boxrule=0.1pt,
  boxsep=1pt,
  left=2pt,right=2pt,top=2pt,bottom=2pt,
  oversize=2pt,
  sharp corners,
  before skip=\topsep,
  after skip=\topsep,
}

\newtheorem{lemma}{Lemma}[section]
\tcolorboxenvironment{lemma}{
  colback=my-cyan,
  boxrule=0.1pt,
  boxsep=1pt,
  left=2pt,right=2pt,top=2pt,bottom=2pt,
  oversize=2pt,
  sharp corners,
  before skip=\topsep,
  after skip=\topsep,
}

\newtheorem{example}{Example}[section]
\tcolorboxenvironment{example}{
  %colback=my-green,
  colback=green!5!white,
  boxrule=0.1pt,
  boxsep=1pt,
  left=2pt,right=2pt,top=2pt,bottom=2pt,
  oversize=2pt,
  sharp corners,
  before skip=\topsep,
  after skip=\topsep,
}


\title{\Huge{\textbf{Lista 3 - Mecânica Estatística}}}
\author{Mateus Marques}

\begin{document}

\maketitle

\section*{2) Entropia para bósons e férmions}

Primeiramente, definamos o parâmetro $\zeta = \pm 1$, sendo $\zeta = +1$ referindo à bósons e $\zeta = -1$ a férmions. Queremos mostrar que
\begin{equation} \label{eq:entropia_ruim}
S = \zeta k_B \sum_{j} \Big[
(1 + \zeta n_j) \log(1 + \zeta n_j) - \zeta n_j \log n_j
\Big], \quad n_j = \frac{1}{e^{\beta(\eps_j-\mu)} - \zeta}.
\end{equation}
Desenvolvendo essa expressão (usando que $\zeta^2 = 1$), temos
$$
S = k_B \sum_{j} \qty[
\zeta \log(1 + \zeta n_j) +
n_j \log(\frac{1}{n_j} + \zeta)
].
$$
Substituindo a expressão de $n_j$ explícita:
$$
\zeta \log(1 + \zeta n_j) =
\zeta \log\qty[\frac{(e^{\beta(\eps_j-\mu)}-\cancel{\zeta})+\cancel{\zeta}}{e^{\beta(\eps_j-\mu)}-\zeta}] =
\zeta \qty{ \beta(\eps_j-\mu) + \log\qty[\frac{1}{e^{\beta(\eps_j-\mu)}-\zeta}] }.
$$
$$
n_j \log(\frac{1}{n_j} + \zeta) = \frac{1}{e^{\beta(\eps_j-\mu)}-\zeta}
\log[(e^{\beta(\eps_j-\mu)}-\cancel{\zeta})+\cancel{\zeta}] =
\frac{\beta (\eps_j-\mu)}{e^{\beta(\eps_j-\mu)}-\zeta}.
$$

A entropia então se simplifica para
$$
S = k_B \sum_{j}
\qty{
\zeta\log\qty[\frac{1}{e^{\beta(\eps_j-\mu)}-\zeta}] +
\beta(\eps_j-\mu) \qty[\frac{\zeta e^{\beta(\eps_j-\mu)}-\cancel{\zeta^2}+\cancel{1}}{e^{\beta(\eps_j-\mu)}-\zeta}] =
}
$$
\begin{equation} \label{eq:entropia_boa}
\boxed{ S = \zeta k_B \sum_{j}
\qty[
\log n_j +
\beta(\eps_j-\mu) e^{\beta(\eps_j-\mu)} n_j
]. }
\end{equation}

Dessa forma, temos que a equação \ref{eq:entropia_boa} é equivalente à equação \ref{eq:entropia_ruim}. Mostrarei que a entropia dos gases quânticos se escreve como na forma da equação \ref{eq:entropia_boa}.

\n

Como vimos em aula, no ensemble grande canônico a função de partição é
$$
Z = \prod_j Z_j, \quad Z_j = [1 \mp e^{-\beta(\eps_j-\mu)}]^{\mp 1}= [1 - \zeta e^{-\beta(\eps_j-\mu)}]^{-\zeta},
$$
onde $Z_j$ é a função de partição para o estado $j$. Perceba que $Z_j$ se reescreve como
$$
Z_j = \qty[\frac{1}{ 1 - \zeta e^{-\beta(\eps_j-\mu)} }]^{\zeta} =
\qty[\frac{e^{\beta(\eps_j-\mu)}}{ e^{\beta(\eps_j-\mu)} - \zeta }]^{\zeta} =
\Big[
e^{\beta(\eps_j-\mu)} n_j
\Big]^{\zeta}.
$$

Portanto, a entropia $S = -\pdvc{F}{T}{V,\mu}$, com $F = -k_B T \log Z$, é dada por
$$
S = \pdv{T} \qty(k_B T \sum_{j} \zeta \Big[\beta(\eps_j-\mu) + \log n_j\Big])_{V,\mu} =
$$
$$
= \zeta k_B \sum_{j} \Big[\cancel{\beta(\eps_j-\mu)} + \log n_j\Big] +
\zeta k_B \sum_{j} \qty{-\cancel{\beta (\eps_j - \mu)} + \beta(\eps_j-\mu) n_j e^{\beta(\eps_j-\mu)}} =
$$
$$
= \zeta k_B \sum_{j} \qty[\log n_j + \beta (\eps_j - \mu) n_j e^{\beta(\eps_j-\mu)}],
$$
que é exatamente a equação \ref{eq:entropia_boa}. Na dedução acima eu utilizei (mas não explicitei) as derivadas $T \pdv{\beta}{T} = -\beta$ e $T \pdv{\log n_j}{T} = \beta (\eps_j-\mu) n_j e^{\beta(\eps_j-\mu)}$. Isso termina o exercício.

\pagebreak

\section*{3) Radiação de corpo negro}

Um estado de partícula única é caracterizada pelo vetor de onda $\k$ e pela polarização $p$ do fóton. Sendo $E(\k) = \hbar c \abs{\k}$ a relação de dispersão, a DOS (por volume) desse sistema é
$$
D(\eps) = \frac{1}{V} \sum_{\k,p} \delta(\eps-E(\k)) = \frac{1}{V} \, \frac{2V}{(2\pi)^3} \int \dd[3]{\k} \delta(\eps-E(\k)) =
\frac{1}{\pi^2} \int \dd{k} k^2 \delta(\eps-\underbrace{\hbar c k}_{x})
$$
$$
D(\eps) = \frac{1}{\pi^2 (\hbar c)^3} \int \dd{x} x^2 \delta(\eps-x)
= \frac{\eps^2}{\pi^2 (\hbar c)^3}.
$$

Sendo $n(\eps)$ a distribuição de Bose-Einstein, como $E/V = \int \eps D(\eps) n(\eps) \dd{\eps} = \int \hbar \omega D(\hbar\omega)\, n(\hbar\omega) \hbar \dd{\omega} = \int u(\omega) \dd{\omega}$, temos que a densidade de energia dos fótons com frequência entre $\omega$ e $\omega + \dd{\omega}$ é
$$
u(\omega) = \hbar^2 \omega D(\hbar\omega) n(\hbar\omega) = \frac{\hbar^2 \omega \, \hbar^2 \omega^2}{\pi^2 \hbar^3 c^3} \cdot
\frac{1}{e^{\beta\hbar\omega} - 1} \implies
\boxed{ u(\omega) =
\frac{\hbar}{\pi^2 c^3} \cdot \frac{\omega^3}{e^{\beta\hbar\omega} - 1}. }
$$

A partir de $\abs{u(\omega) \dd{\omega}} = \abs{u(\lambda) \dd{\lambda}}$ e usando $h = 2\pi \hbar$, temos que
$$
u(\lambda) = u(\omega) \abs{\dv{\omega}{\lambda}} = u\qty(\frac{2\pi c}{\lambda}) \cdot \frac{2\pi c}{\lambda^2} \implies
\boxed{ u(\lambda) =
\frac{8 \pi h c}{\lambda^5} \cdot \frac{1}{e^{\beta hc / \lambda} - 1}. }
$$

\n

(b) Com a mudança de variável $x = \beta \hbar\omega$, temos
$$
\frac{E}{V} = \int_0^\infty u(\omega) \dd{\omega} = \frac{\hbar}{\pi^2 c^3} \,
\qty(\frac{1}{\beta\hbar})^4 \int_0^\infty \frac{x^3}{e^{x} - 1} \dd{x}.
$$

Pedindo para o Mathematica calcular a integral acima, ele retorna $\int_0^\infty \frac{x^3}{e^{x} - 1} \dd{x} = \frac{\pi^4}{15}$. Portanto
$$
\frac{E}{V} = \frac{k_B^4 T^4}{\pi^2 \hbar^3 c^3} \cdot \frac{\pi^4}{15} =
\frac{4}{c} \, \underbrace{\qty(\frac{\pi^2 k_B^4}{60 \hbar^3 c^2})}_{\sigma} \, T^4 = \frac{4}{c} \sigma T^4.
$$

Temos $C_V = \pdvc{E}{T}{V,N} = \frac{16}{c} V \sigma T^3 \propto T^3$. O número de fótons (por volume) é dado por ($x = \beta \eps$)
$$
\frac{N}{V} = \int D(\eps) n(\eps) \dd{\eps} =
\frac{1}{\pi^2 (\hbar c)^3} \int_0^{\infty} \frac{\eps^2}{e^{\beta \eps}-1} \dd{\eps} =
\frac{1}{\pi^2 (\hbar c)^3} \cdot \frac{1}{\beta^3} \int_0^\infty \frac{x^2}{e^x - 1} \dd{x}.
$$

O Mathematica retorna $\int_0^\infty \frac{x^2}{e^x - 1} \dd{x} = 2 \zeta(3) \approx 2.40411$. Portanto
$$
\frac{N}{V} = \frac{2 \zeta(3) k_B^3}{\pi^2 \hbar^3 c^3} \cdot T^3.
$$

\n

(c)


\pagebreak

\section*{5) Condensado de Bose-Einstein: duas dimensões e armadilha parabólica}

<++>

\pagebreak

\section*{6) Paramagnetismo de Pauli}

<++>

\pagebreak




\end{document}
