%%%%%%%%%%%%%%%%%%%%%%%%%%%%%%%%%%%%%%%%%%%%%%%%%%%%%%%%%%%%%%%%%%%%%
% In English:
%    This is a Latex template for São Paulo Research Foudation (FAPESP)
%         reports (annual or final).
%    This is the modified version of the original Latex template from
%         following website.
%    Original Source: http://www.howtotex.com
%    For information about FAPESP, check http://www.fapesp.br/en
%    This template targets mainly on reports in Portuguese language.
%    New additions and changes in the latest version:
%        - Added the possibility of including multiple members in the
%          research team, with the commands \memberA{Name of Member A}
%          \memberB{Name of Member B} \memberC{Name of Member C} etc.
%        - Included commands to define project modality and the research
%          agency (if you want to use the same model for other research
%          agencies such as CAPES, CNPq etc).
%
% In Portuguese:
%    Este é um modelo Latex para relatórios (anual ou final) da Fundação
%         de Amparo à pesquisa do Estado de São Paulo (FAPESP).
%    Esta é uma versão modificada do modelo Latex do site supra mencionado.
%    Para informações sobre a FAPESP, verifique http://www.fapesp.br
%    Esse modelo foca principalmente nos relatórios escritos em Português.
%    Novas adições e alterações na última versão:
%       - Foi adicionada a possibilidade de incluir vários membros no
%         grupo de pesquisas, com os comandos \membroA{Nome do Membro A}
%         \membroB{} \membroC{} etc.
%       - Foram incluídos comandos para definir modalidade de projeto e
%         agência de fomento (caso queira utilizar o mesmo modelo para
%         outras agências, CAPES, CNPq etc).
%
% Author/Autor: André Leon Sampaio Gradvohl, Dr.
% Email:        andre.gradvohl@gmail.com
% Lattes CV:    http://lattes.cnpq.br/9343261628675642
% GitHub: http://gradvohl.github.io/
%
% Last update/Última versão: 19/Feb/2018
%
%%%%%%%%%%%%%%%%%%%%%%%%%%%%%%%%%%%%%%%%%%%%%%%%%%%%%%%%%%%%%%%%%%%%%%
\documentclass[12pt]{report}
\usepackage[a4paper]{geometry}
\usepackage[utf8]{inputenc}
\usepackage[english]{babel}
\usepackage[myheadings]{fullpage}
\usepackage[T1]{fontenc}
\usepackage{fancyhdr}
\usepackage{setspace}
\usepackage{sectsty}
\usepackage{url}

%%% iniciar capitulo na mesma pagina (Mateus) %%%
\usepackage{etoolbox}
\makeatletter %inicio da modificacao (iniciar capitulo na msma pag)
\patchcmd{\chapter}{\if@openright\cleardoublepage\else\clearpage\fi}{}{}{}
\makeatother %fim da modificacao
%%% iniciar capitulo na mesma pagina (Mateus) %%%

%%------
%% Comandos gerais
%% Observação: o arquivo "comandos.tex" tem que estar presente.
%%------
\input{comandos}
%\usepackage{mathtools}
\usepackage{amsthm}     % for definitions and theorems
\usepackage[many]{tcolorbox}    % boxes around definitions and theorems
%\usepackage{amsmath}
%\usepackage{nccmath}
\usepackage{amssymb}    % \ltimes, semi-direct product
%\usepackage{etoolbox}   % for start of Chapter
%\usepackage{amsfonts}
\usepackage{physics}    % for all Physics related
\usepackage{dsfont}     % for the identity matrix symbol \1
%\usepackage{mathrsfs}
\usepackage[notextcomp]{stix}   % font package and some symbols like filled square
%\usepackage{MnSymbol}   % symbols font package

\usepackage{titling}
\usepackage{indentfirst}

\usepackage{bm}
\usepackage[dvipsnames]{xcolor}
\usepackage{cancel}
\usepackage{enumitem}

\usepackage{xurl}
%\usepackage[colorlinks=true]{hyperref} % links have colors
\usepackage{hyperref}  % no colors

\usepackage{float}
\usepackage{graphicx}
\usepackage{subcaption}
%\usepackage{tikz}

\usepackage{ctable}     % tabelas
\renewcommand{\P}{\phantom{+}}  % empty space to indent things
\usepackage{multirow}
\usepackage{tabulary}

%%%%%%%%%%%%%%%%%%%%%%%%%%%%%%%%%%%%%%%%%%%%%%%%%%%

\newcommand{\eps}{\epsilon}
\newcommand{\vphi}{\varphi}
\newcommand{\cte}{\text{cte}}

\newcommand{\N}{{\mathbb{N}}}
\newcommand{\Z}{{\mathbb{Z}}}
%\newcommand{\Q}{{\mathbb{Q}}}
\newcommand{\C}{{\mathbb{C}}}
\renewcommand{\S}{{\hat{S}}}
%\renewcommand{\H}{\s{H}}

\renewcommand{\a}{{\vb{a}}}
\renewcommand{\b}{{\vb{b}}}
\renewcommand{\d}{{\dagger}}
\newcommand{\up}{{\uparrow}}
\newcommand{\down}{{\downarrow}}
\newcommand{\hc}{{\text{h.c.}}}

\newcommand{\ihat}{\bm{\hat{\imath}}}
\newcommand{\jhat}{\bm{\hat{\jmath}}}
\newcommand{\khat}{\bm{\hat{k}}}

\newcommand{\0}{{\vb{0}}}
\newcommand{\1}{\mathds{1}}
\newcommand{\E}{{\vb{E}}}
\newcommand{\B}{{\vb{B}}}
\renewcommand{\u}{{\vb{u}}}
\renewcommand{\v}{{\vb{v}}}
\renewcommand{\r}{{\vb{r}}}
\newcommand{\R}{{\vb{R}}}
\newcommand{\Q}{{\vb{Q}}}
\newcommand{\G}{{\vb{G}}}
\newcommand{\g}{{\vb{g}}}
\renewcommand{\k}{{\vb{k}}}
\newcommand{\K}{{\vb{K}}}
\newcommand{\p}{{\vb{p}}}
\newcommand{\q}{{\vb{q}}}
\newcommand{\F}{{\vb{F}}}
\renewcommand{\t}{{\vb{t}}}
\newcommand{\vtau}{{\bm{\tau}}}
\newcommand{\vdelta}{{\bm{\delta}}}

% COLORED SYMMETRY ELEMENTS
\newcommand{\Ct}{{\textcolor{Cyan}{C_3}}}
\newcommand{\Ctn}[1]{{\textcolor{Cyan}{C_3^{\textcolor{black}{#1}}}}}
\newcommand{\Cs}{{\textcolor{ForestGreen}{C_6}}}
\newcommand{\Csn}[1]{{\textcolor{ForestGreen}{C_6^{\textcolor{black}{#1}}}}}
\newcommand{\sd}{{\textcolor{RoyalBlue}{\sigma_d}}}
\newcommand{\sdn}[1]{{\textcolor{RoyalBlue}{\sigma_d^{\textcolor{black}{#1}}}}}
\newcommand{\sdp}{{\textcolor{RoyalBlue}{\sigma_d'}}}
\newcommand{\sdpp}{{\textcolor{RoyalBlue}{\sigma_d''}}}
\newcommand{\sv}{{\textcolor{Orange}{\sigma_v}}}
\newcommand{\svn}[1]{{\textcolor{Orange}{\sigma_v^{\textcolor{black}{#1}}}}}
\newcommand{\svp}{{\textcolor{Orange}{\sigma_v'}}}
\newcommand{\svpp}{{\textcolor{Orange}{\sigma_v''}}}

\newcommand{\GL}{{\text{GL}}}
\newcommand{\U}{{\text{U}}}

\newcommand{\s}{\sigma}
%\newcommand{\prodint}[2]{\left\langle #1 , #2 \right\rangle}
\newcommand{\cc}[1]{\overline{#1}}
\newcommand{\Eval}[3]{\eval{\left( #1 \right)}_{#2}^{#3}}
\newcommand{\sg}[2]{\{ #1 \mid #2 \}}
\renewcommand{\AA}{{\mathring{\text{A}}}}
\newcommand{\I}{{\mathbb{I}}}
\newcommand{\bP}{{\mathbb{P}}}
\newcommand{\bQ}{{\mathbb{Q}}}

\newcommand{\unit}[1]{\; \mathrm{#1}}

\newcommand{\n}{\medskip}
\newcommand{\e}{\quad \mathrm{and} \quad}
\newcommand{\ou}{\quad \mathrm{or} \quad}
\newcommand{\virg}{\, , \;}
\newcommand{\ptodo}{\forall \,}
\renewcommand{\implies}{\; \Rightarrow \;}
%\newcommand{\eqname}[1]{\tag*{#1}} % Tag equation with name

%\setlength{\droptitle}{-7em}   % título um pouco mais em cima na página
%\makeatletter
%\patchcmd{\chapter}{\if@openright\cleardoublepage\else\clearpage\fi}{}{}{}  % start 'Chapter' at the same page. needs package etoolbox
%\makeatother

%% Theorems, definitions, proofs
\theoremstyle{definition}

%%% defining my own colors %%%
\definecolor{my-blue}{HTML}{f2f4ff}
\definecolor{my-green}{HTML}{f5fcf6}    % a little better: green!5!white
\definecolor{my-cyan}{HTML}{f2fffe}
\definecolor{my-yellow}{HTML}{fffbed}
\definecolor{my-green2}{HTML}{efffdb}

%%% alternative colors %%%
\definecolor{my-pink}{HTML}{fff2f7}
\definecolor{my-teal}{HTML}{ebfffc}

\newtheorem{definition}{Definition}[section]
\tcolorboxenvironment{definition}{
  colback=my-blue,
  %colback=blue!5!white,
  boxrule=0.1pt,
  boxsep=1pt,
  left=2pt,right=2pt,top=2pt,bottom=2pt,
  oversize=2pt,
  sharp corners,
  before skip=\topsep,
  after skip=\topsep,
}

\newtheorem{theorem}{Theorem}[section]
\tcolorboxenvironment{theorem}{
  colback=my-yellow,
  %colback=yellow!22!white!95!black,
  boxrule=0.1pt,
  boxsep=1pt,
  left=2pt,right=2pt,top=2pt,bottom=2pt,
  oversize=2pt,
  sharp corners,
  before skip=\topsep,
  after skip=\topsep,
}

\newtheorem{corollary}{Corollary}[section]
\tcolorboxenvironment{corollary}{
  colback=my-green2,
  boxrule=0.1pt,
  boxsep=1pt,
  left=2pt,right=2pt,top=2pt,bottom=2pt,
  oversize=2pt,
  sharp corners,
  before skip=\topsep,
  after skip=\topsep,
}

\newtheorem{lemma}{Lemma}[section]
\tcolorboxenvironment{lemma}{
  colback=my-cyan,
  boxrule=0.1pt,
  boxsep=1pt,
  left=2pt,right=2pt,top=2pt,bottom=2pt,
  oversize=2pt,
  sharp corners,
  before skip=\topsep,
  after skip=\topsep,
}

\newtheorem{example}{Example}[section]
\tcolorboxenvironment{example}{
  %colback=my-green,
  colback=green!5!white,
  boxrule=0.1pt,
  boxsep=1pt,
  left=2pt,right=2pt,top=2pt,bottom=2pt,
  oversize=2pt,
  sharp corners,
  before skip=\topsep,
  after skip=\topsep,
}

%
%%-----
%% Página de título
%% Observação: As definições que aparecem a seguir comporão a
%%             página de título e a folha de rosto.
%%-----
%% Define o nome da universidade onde o projeto foi desenvolvido.
\universidade{Universidade de São Paulo}
%
%% Define o nome da faculdade onde o projeto foi desenvolvido.
\faculdade{Instituto de Física}
%
%% Define o título do projeto.
\titulo{Multi-orbital topological Anderson models for twisted bilayer graphene}
%
%% Define a agencia de Fomento e a abreviatura. O primeiro argumento é o
%% nome por extenso e o segundo a abreviatura.
%% Ambos os argumentos são obrigatórios
\agFomento{São Paulo Research Foundation}{FAPESP}
%
%% Define o tipo de relatório. Pode ser Anual ou Final.
%% Não é obrigatório definir o tipo de relatório.
%\tipoRelatorio{Final}
%
%% Define a modalidade de Projeto. Pode ser temático, regular, etc.
\modalidadeProjeto{Master's degree}
%
%% Define o número do projeto.
%% Não é obrigatório definir o número do projeto.
\numProjeto{2023/02913-5}
%
%% Define o autor do relatório.
\autor{Mateus Marques}
\orientador{Dr. Luis Gregório G. V. Dias da Silva}
%
%% Define a equipe do projeto (incluindo o pesquisador responsável no comando \membroA{}
\membroA{Mateus Marques}
%% Inclua os demais membros do grupo (máximo +5)
\membroB{Luis Gregório G. V. Dias da Silva}
%\membroC{Francisco}
%\membroD{Joao}
%\membroE{Antonio}
%\membroF{José}
%
%% Define o período da vigência do Projeto.
\periodoVigencia{January 1, 2023 to December 31, 2024}
%
%% Define o período coberto pelo relatório.
\periodoRelatorio{January 1, 2023 to \today}
%
%% Define a cidade onde o projeto foi desenvolvido.
\cidade{São Paulo}

%%-----
%% Página de título
%% Observação: Os comandos a seguir não devem ser mudados,
%%             exceto caso necessário.
%%-----
\begin{document}
%
%% Define a numeração em romanos.
\pagenumbering{roman}
%
%% Gera a folha de título.
\geraTitulo
%
%% Gera a folha de rosto.
\folhaDeRosto
%
%% Escreva aqui o resumo em português. Fica no Resumo do Projeto Proposto
%\Resumo{
%Neste projeto estamos interessado apenas no valor da bolsa. Não fizemos praticamente
%nada nesse tempo e não pretendemos fazer. Eu apenas pretendo entregar este relatório
%para ficar tudo certo.
%}
%
%% Escreva aqui o resumo em inglês. NAO PRECISA
%\Abstract{
%Same thing but in english.
%}
%
%% Adicionará o sumário.
%% Mantenha o \thispagestyle{empty} e \clearpage
\tableofcontents
\thispagestyle{empty}
\clearpage
%
%% Define a numeração em arábicos.
\pagenumbering{arabic}

%%-----
%% Formatação do título da seção
%%-----
\sectionfont{\scshape}

%%-----
%% Corpo do texto
%%-----


%%-----
%% Resumo do projeto
%%-----
\chapter{Summary} \label{chp:abstract}

The twisted bilayer graphene (TBG) is a fascinating system that displays a plethora of intriguing physical phenomena such as correlated insulators, superconductivity \cite{cao2018}, and unconventional quantum Hall effects \cite{unconv_QHE_tbg_2006}, among others. Several recent experiments have demonstrated the rich electronic properties of TBG, which exhibits correlated insulating and superconducting phases. In spite of such a vast amount of experimental data, there are still several open questions regarding the theoretical description of this material in different regimes. While there is some consensus that these phenomena are related to the formation of flat bands near the Fermi level captured by non-interacting band structure models, the description of MATBG as a strongly correlated system is nonetheless challenging. In this Master's project, we propose a description of the correlated electronic properties of MATBG in terms of multi-orbital topological Anderson models (both as a single impurity and as a periodic Anderson lattice), building on the framework recently developed describing MATBG as a topological heavy fermion problem. By employing a combination of analytical and numerical methods, we aim to study the low-energy physics of such models and identify the emergent energy scales and the nature of the Mott transition. We expect that our results can provide some additional insight into the nature of the electronic states in MATBG and suggest new avenues for experimental investigation of this fascinating material system.

\chapter{Accomplishments in the period} \label{chp:accomplishments}

In the year of 2023 our main activities consisted in the three main steps: the familiarization with the basic theory in Condensed Matter Physics, the study of the numerical techniques that will be used in the project and the reading of literature with respect to the system studied in this project (MATBG).

\begin{itemize}
\item The student took the three courses Solid States I, II and Statistical Mechanics. They were very important to provide the necessary tools to study many-body systems. In particular, the MATBG system can be modelled in the second quantization formalism in a way that is very similar to the Hubbard Model, which was studied in the Solid State II course.

\item We extensively studied the Hubbard model \cite{hubbard1963} as a test case to build our numerical DMFT code. In Section \ref{sec:dmft} we describe the techniques and ideas involved in the DMFT method, and show our results in respect to the Mott metal-insulator transition. The student also took the brief course ``Strong coupling theories for metal-insulator transitions: DMFT and beyond'', given by Professor Vladimir Dobrosavljevic at USP in the first semester of 2023.

\item We studied the basic theory of the twisted bilayer graphene system through \cite{handbook2019}, which our main objective was to understand the Bistritzer-MacDonald model \cite{macdonald2011}.
\end{itemize}
<++>

\section{Hubbard model} \label{sec:hubbard}


The Hubbard model was conceived by J. Hubbard with the aim to explore the effects of electron correlations in lattice materials. By its definition, the model is actually simple. We suppose that electrons belong to sites that compose a lattice. They have a constant energy $\eps_d$ (the onsite energy) for belonging to any site. They also have the possibility of hopping to differente sites, which is given by the parameter $t$ (hopping strength). Lastly, when two electrons are on the same site, they have an interaction term given by the parameter $U$. This interaction can be attractive ($U < 0$) or repulsive ($U > 0$), but here we mainly focus in the repulsive case because it leads to the existence of a Mott insulating phase \cite{georges1996} for the half-filled Hubbard model. Figure \ref{fig:hubbard_model-scheme} gives an schematization of the model and an intuition of the $t$ and $U$ parameters in the model.

\n

\begin{figure}[H]
\centering
\includegraphics[width=0.4\linewidth]{fig/hubbard_model-scheme.png}
\caption{Schematization of the Hubbard model on a 2D square lattice. Electrons move to neighboring sites with a hopping $t$, and when two electrons are on the same site they feel a Coulomb repulsion $U$. Figure taken from \cite{thesis_dmft_graz}.}
\label{fig:hubbard_model-scheme}
\end{figure}

Putting together the ideas described in the last paragraph, we construct the hamiltonian of the Hubbard model in second quantization, given by
\begin{equation} \label{eq:hubbard-hamiltonian}
H = \eps_d \sum_{i, \s} n_{i\s} +  \sum_{\nn{i}{j}, \s} t_{ij} (c_{i\s}^\d c_{j\s} + \hc)
+ U \sum_{i} n_{i\up} n_{i\down}.
\end{equation}

\n

The Hubbard model, although somewhat simples to define, does not have an analytic solution for general lattices. Therefore, to this grasp the behaviour of this system, we must work with approximate solutions. Actually, Hubbard himself gave a simple approximate solution to the hamiltonian \ref{eq:hubbard-hamiltonian} in his seminal article, which is today known as ``Hubbard I Approximation'' (HIA). This method uses the technique of retarted Green's functions described by Zubarev \cite{zubarev1960}, which we studied in the book \cite{bruus}.

\n

While the HIA solution gives simples expressions for common observables in the Hubbard model, it does not capture the metal-insulator quantum phase transition in the case of half-filling. In order to describe this transition, we went beyond and started learning the general method of Dynamical Mean-Field Theory (DMFT) \cite{georges1996}, which is gives a reasonable treatment for the effects of electron correlations. In the next section we describe it.

\subsection{DMFT} \label{sec:dmft}

Dynamical mean-field theory (DMFT) is a powerful method to treat strong correlations in materials. Its combination along with Density Functional Theory (DFT) provides a computational tools to determine properties of real correlated electron materials \cite{hauleweb, haule_real_materials}.

\n

DMFT is motivated by the establishment that it is exact \cite{thesis_dmft_graz} in the limit of infinite lattice coordination $z \to \infty$, once there is a self-consistent mapping between the original hamiltonian onto a local impurity model. In this limit, the self-energy $\displaystyle{\lim_{z \to \infty}\Sigma(\k,\omega) = \Sigma(\omega)}$ becomes \textit{local}, i.e., independent of the momentum $\k$.

Here we are going to describe the DMFT method in the special case of the Hubbard model. In this case, the Hubbard hamiltonian is self-consistently mapped \cite{georges1996} onto the Single Anderson Impurity Model (SIAM) \cite{impurity-solvers}, where a level, indexed by $d$, is coupled to a bath
\begin{equation} \label{eq:anderson-hamiltonian}
H_{\text{SIAM}} = \underbrace{U n_{d\up} n_{d\down} + \eps_d \sum_{\s} n_{d\s}}_{H_{\text{imp}}}
+ \underbrace{\sum_{\k,\s} (t_\k c_{d\s}^\d c_{k\s} + \hc)}_{H_{\text{coup}}}
+ \underbrace{\sum_{\k,\s} \eps_\k c_{\k\s}^\d c_{\k\s}}_{H_{\text{bath}}}.
\end{equation}

In equation \ref{eq:anderson-hamiltonian}, $H_{\text{bath}}$ is the hamiltonian of only the bath, $H_{\text{coup}}$ is the coupling between the bath and the single impurity $d$, and $H_{\text{imp}}$ is the hamiltonian of the impurity. $H_{\text{imp}}$, anologously with the Hubbard hamiltonian in equation \ref{eq:hubbard-hamiltonian}, has an interaction term when two electrons are on the same site.

The SIAM is an impurity problems, which is relatively easier to solve numerically than the Hubbard model. There are several methods, known as impurity solvers, available in the literature that attempt to solve it. To give an overview, the reference \cite{impurity-solvers} make a benchmark of some of these impurity solvers.

By solving an impurity problem, we actually mean determining the retarded Green's function of the impurity site $G_{\text{imp}}(\omega)$, because its knowledge allows us compute the spectral function $A(\omega) = - \frac{1}{\pi} \Im{G_{\text{imp}}(\omega)}$ \cite{bruus} and a manifold of important observables related to quantum transport in materials \cite{pedagogical-gfs}.

\begin{figure}[H]
\centering
\includegraphics[width=0.6\linewidth]{fig/dmft-mapping}
\caption{Schematization of the mapping between the lattice problem and a single impurity coupled to an effective bath. Figure taken from \cite{thesis_dmft_graz}.}
\label{fig:dmft-mapping}
\end{figure}

Having in mind this map onto a impurity problem, the DMFT scheme is based on the approximation
\begin{equation} \label{eq:dmft-approx}
\Sigma(\k, \omega) \approx \Sigma(\omega),
\end{equation}
where the self-energy is independent of $\k$, motivated by the fact that this approximation is an equality in the limit where $z \to \infty$. Along with this approximation, we the self-consistency guarantees us that the local lattice Green's function $G_{\text{loc}}(\omega)$ coincides with the impurity model's Green function $G_{\text{imp}}(\omega)$, and this translates into a convergence criterion for the DMFT algorithm.

Hence, using equation \ref{eq:dmft-approx}, the local lattice Green's function in the continuum limit can be written as
\begin{equation} \label{eq:Gf-local}
G_{\text{loc}}(\omega) = \sum_{\k} \frac{1}{\omega-\eps_\k-\Sigma(\k, \omega)} \approx
\int \dd{\eps} \frac{\rho_0(\eps)}{\omega - \eps - \Sigma(\omega)},
\end{equation}
where $\rho_0(\eps)$ is the non-interacting density of states (DOS) and depends only on the considered lattice structure, defined by
$$
\rho_0(\eps) = \sum_{\k} \delta(\eps-\eps_\k).
$$

Now, the DMFT algorithm works as follows. Firstly, we start with an initial guess for the self-energy $\Sigma(\omega)$ and compute $G_{\text{loc}}(\omega)$ by equation \ref{eq:Gf-local}. On the other hand, we compute $G_{\text{imp}}(\omega)$ by solving the SIAM with an impurity solver of choice. We compare $G_{\text{loc}}$ and $G_{\text{imp}}$ with some convergence criterion and, if this criterion is still not satisfied, we update the self-energy by
\begin{equation} \label{eq:self-energy-update}
\Sigma(\omega) = \omega - \eps_d - G_{\text{imp}}^{-1}(\omega).
\end{equation}

\subsection{Bethe lattice} \label{sec:bethe}

In our DMFT work, we focused on the special case of Bethe lattice with infinite coordination $z\to\infty$, which was extensively studied in the literature \cite{georges1996, thesis_dmft_graz} and serves as a benchmark for the DMFT algorithm because the expressions simplify a lot in this lattice. But the Hubbard hamiltonian in equation \ref{eq:hubbard-hamiltonian} only makes sense for a Bethe lattice with $z\to\infty$ if the hopping amplitude is renormalized as \cite{thesis_bruno}
\begin{equation} \label{eq:hopping-renormalization}
t = \frac{t_*}{\sqrt{z}},
\end{equation}
where $t_*$ is the renormalized hopping parameter.

The Bethe lattice is an infinite lattice where each site has $z$ neighbors and any two sites are connected by a unique shortest path. Figure \ref{fig:bethe-lattice} represents a Bethe lattice in the case where $z = 4$.
\begin{figure}[H]
\centering
\includegraphics[width=0.4\linewidth]{fig/bethe-lattice.png}
\caption{Schematization of the Bethe lattice with coordination number $z = 4$. The lattice is actually infinite and all sites equivalent. Figure taken from \cite{thesis_dmft_graz}.}
\label{fig:bethe-lattice}
\end{figure}

For the Bethe lattice, the non-interacting DOS $\rho(\eps)$ has a semicircular form \cite{georges1996}, given by
\begin{equation} \label{eq:bethe-dos}
\rho_0(\eps) = \frac{2}{\pi D} \sqrt{1-\qty(\frac{\omega}{D})^2},
\end{equation}
where $D = 2t^*$ is the half-bandwidth. Here we set $D = 1$, and therefore $t_* = 1/2$.


Using equation \ref{eq:bethe-dos} and integrating equation \ref{eq:Gf-local}, we get the hybridization function \cite{thesis_bruno}
\begin{equation} \label{eq:simple-hybridization-bethe}
\Delta(\omega) = t_*^2 \, G_{\text{imp}}(\omega).
\end{equation}

\subsection{Impurity Solver} \label{sec:impurity-solver}

As our main first objective was to establish a working code for DMFT and to probe the metal-insulator transition in the Hubbard model, we chose to use the Non-Crossing Approximation (NCA) impurity solver. As can be seen in Table 1 of \cite{impurity-solvers}, the NCA method has a low computational-cost while still giving reasonable qualitative results \cite{haule_real_materials, vildosola2015}.

The NCA algorithm is very lengthy to describe but all its details can be found in Section 2.2 of \cite{thesis_bruno}. For our work, we used the NCA code developed by Kristjan Haule available on the website \cite{hauleweb}. The code takes as input the bath hybridization function $\Delta(\omega)$ and returns the Green's function $G_{\text{imp}}(\omega)$. Because of the simplified DMFT equation for the Bethe lattice \ref{eq:simple-hybridization-bethe}, the DMFT loop is reduced to basically applying the impurity solver and updating the hybridization function until convergence is reached. In the next section (Sec. \ref{sec:results}) we present the results obtained with our settings.

\subsection{Results} \label{sec:results}

\begin{figure}[H]
\centering
\includegraphics[width=0.5\textwidth]{fig/dmft/triangle-w0-mu050.png}
\caption{Phase diagram  $U \times T$ for the Hubbard model at half-filling ($\epsilon_d=U/2$) obtained in the DMFT-NCA approximation. The color code shows the average $\rho_{\text{mean}}(0)$. The co-existence region is marked in green.}
\label{fig:triangle-mu050}
\end{figure}

\begin{figure}[H]
\centering
\begin{subfigure}{.5\textwidth}
  \centering
  \includegraphics[width=\linewidth]{fig/dmft/mean2_0-mu045.png}
  \caption{}
  \label{fig:mu045-rho0}
\end{subfigure}%
\begin{subfigure}{.5\textwidth}
  \centering
  \includegraphics[width=\linewidth]{fig/dmft/mean2_n-mu045.png}
  \caption{}
  \label{fig:mu045-n}
\end{subfigure}
\caption{Phase diagrams $U \times T$ for on-site energy $\eps_d = 0.45 \, U$, which show the $\rho_{\text{mean}}$ in (a) and $n_{\text{mean}}$ in (b).}
\label{fig:mu045}
\end{figure}




\begin{figure}[H]
\includegraphics[width=0.49\textwidth]{fig/dmft/fig_diffe-w0-T0002.png}
\hfill
\includegraphics[width=0.49\textwidth]{fig/dmft/fig_diffe-nn-T0002.png}
\includegraphics[width=0.49\textwidth]{fig/dmft/fig_mean2-w0-T0002.png}
\hfill
\includegraphics[width=0.49\textwidth]{fig/dmft/fig_mean2-nn-T0002.png}
\caption{Phase diagram  $U \times \epsilon_d$ for $T=0.002$. Panels labeled by ``difference'' (``mean'') show $\rho_{\text{diff}}$ ($\rho_{\text{mean}}$) and $n_{\text{diff}}$ ($n_{\text{mean}}$).}
%\vspace{-10pt}
\label{fig:Diagram_Drho_U_ed_T0002}
\end{figure}




\begin{figure}[H]
\includegraphics[width=0.49\textwidth]{fig/dmft/coex_mean-df_0-mu=0.460.png}
\hfill
\includegraphics[width=0.49\textwidth]{fig/dmft/coex_mean-df_0-mu=0.470.png}
\includegraphics[width=0.49\textwidth]{fig/dmft/coex_mean-df_0-mu=0.480.png}
\hfill
\includegraphics[width=0.49\textwidth]{fig/dmft/coex_mean-df_0-mu=0.495.png}
\caption{Phase diagrams $U \times T$ for on-site energies $\eps_d/U = 0.46, 0.47, 0.48, 0.495$, which show the $\rho_{\text{mean}}$ and display the growth of the coexistence region as $\eps_d$ approaches $U/2$.}
%\vspace{-10pt}
\label{fig:Diagram_Drho_U_T_coex}
\end{figure}

\section{Twisted bilayer graphene}





\section{Conclusions}

Na seção \ref{sec:solar-nu} simulamos numericamente os neutrinos solares para diferentes reações. Os resultados obtidos na Figura \ref{fig:surv_prob} para a média no ponto de produção da probabilidade de sobrevivência ao sair do Sol estão de acordo com \cite{winslow}. Foi importante a implementação do algoritmo M4 \cite{efficient-nu}, que acelerou a execução do código na ordem de $10^3$ vezes.


Na seção \ref{sec:kamland} simulamos os eventos detectados pelo experimento KamLAND e obtivemos a Figura \ref{fig:kamland-events}, que está qualitativamente de acordo com os resultados de \cite{spectral-distortion}. Ainda mais, os resultados obtidos para o número total de eventos $N_{\text{no-osc}} = 371.5 \pm 24.1$ e $N_{\text{osc}} = 253.1 \pm 16.5$ são estatisticamente compatíveis dentro de uma incerteza com os 258 eventos realmente detectados pelo KamLAND e com a estimativa de $365.2 \pm 23.7$ eventos sem oscilação \cite{spectral-distortion}. Isso nos leva a concluir que os parâmetros atuais do NuFIT \cite{nufit} são compatíveis com o ajuste do KamLAND no período coberto por \cite{spectral-distortion}.



\chapter{Evaluation of the Institutional Support received during the period} \label{chp:apoioInst}
%\chapter{Description and evaluation of the Institutional Support received during the period} \label{chp:apoioInst}

Durante o período correspondente a este relatório, o estudante não fez uso da reserva técnica.



\chapter{Participation in scientific events} \label{chp:particEvento}

O estudante apresentou seu projeto de iniciação científica na $30^{\text{a}}$ edição do \href{https://siicusp.prp.usp.br/pt/home/}{SIICUSP} e também participou de três workshops todos promovidos pelo \href{https://www.ictp-saifr.org/}{ICTP-SAIFR}:
\begin{itemize}
\item Workshop on Strong Electron Correlations in Quantum Materials: Inhomogeneities, Frustration, and Topology
\item APS/ICTP-SAIFR Satellite March Meeting
\item Journal Club, grupo do Luis Gregório.
\item CURSO DE COMUNICAÇÃO E ESCRITA CIENTÍFICA 2023.
\item ``Strong coupling theories for metal-insulator transitions: DMFT and beyond''
%\item \href{http://journeys.ictp-saifr.org/}{V Journeys into Theoretical Physics, 2022}.
%\item \href{https://www.ictp-saifr.org/qc2022/}{School on Quantum Computation, 2022}.
%\item \href{https://www.ictp-saifr.org/cmtm2022/}{II ICTP-SAIFR Condensed Matter Theory in the Metropolis, 2022}.
\end{itemize}



\chapter{Subjects taken during the period}

Durante o período coberto por este relatório, o estudante concluiu sua graduação em Física Bacharelado, havendo cursado apenas duas disciplinas em seu último semestre: ``Filosofia da Física'' e ``Aprendizado de Máquina e Reconhecimento de Padrões''.

%%-----
%% Referências bibliográficas
%%-----
\addcontentsline{toc}{chapter}{\bibname}
%\bibliographystyle{abntex2-num}
\bibliography{bibliografia}
\bibliographystyle{ieeetr}


%%-----
%% Fim do documento
%%-----

\end{document}
