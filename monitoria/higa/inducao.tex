\documentclass[a4paper,fleqn,12pt]{article}
\usepackage[brazilian]{babel}
\usepackage[left=2.5cm,right=2.5cm,top=3cm,bottom=2.5cm]{geometry}
\usepackage{mathtools}
\usepackage{amsthm}
\usepackage{amsmath}
%\usepackage{nccmath}
\usepackage{amssymb}
\usepackage{amsfonts}
\usepackage{physics}
%\usepackage{dsfont}
%\usepackage{mathrsfs}

\usepackage{titling}
\usepackage{indentfirst}

\usepackage{bm}
\usepackage[dvipsnames]{xcolor}
\usepackage{cancel}

\usepackage{xurl}
\usepackage[colorlinks=true]{hyperref}

\usepackage{float}
\usepackage{graphicx}
%\usepackage{tikz}
\usepackage{caption}
\usepackage{subcaption}

%%%%%%%%%%%%%%%%%%%%%%%%%%%%%%%%%%%%%%%%%%%%%%%%%%%

\newcommand{\eps}{\epsilon}
\newcommand{\vphi}{\varphi}
\newcommand{\cte}{\text{cte}}

\newcommand{\N}{\mathbb{N}}
\newcommand{\Z}{\mathbb{Z}}
\newcommand{\Q}{\mathbb{Q}}
\newcommand{\R}{\vb{R}}
\newcommand{\C}{\mathbb{C}}
\renewcommand{\S}{\hat{S}}
%\renewcommand{\H}{\s{H}}

\renewcommand{\a}{\vb{a}}
\newcommand{\nn}{\hat{n}}
\renewcommand{\d}{\dagger}
\newcommand{\up}{\uparrow}
\newcommand{\down}{\downarrow}

\newcommand{\0}{\vb{0}}
%\newcommand{\1}{\mathds{1}}
\newcommand{\E}{\vb{E}}
\newcommand{\B}{\vb{B}}
\renewcommand{\v}{\vb{v}}
\renewcommand{\r}{\vb{r}}
\renewcommand{\k}{\vb{k}}
\newcommand{\p}{\vb{p}}
\newcommand{\q}{\vb{q}}
\newcommand{\F}{\vb{F}}

\newcommand{\s}{\sigma}
%\newcommand{\prodint}[2]{\left\langle #1 , #2 \right\rangle}
\newcommand{\cc}[1]{\overline{#1}}
\newcommand{\Eval}[3]{\eval{\left( #1 \right)}_{#2}^{#3}}

\newcommand{\unit}[1]{\; \mathrm{#1}}

\newcommand{\n}{\medskip}
\newcommand{\e}{\quad \mathrm{e} \quad}
\newcommand{\ou}{\quad \mathrm{ou} \quad}
\newcommand{\virg}{\, , \;}
\newcommand{\ptodo}{\forall \,}
\renewcommand{\implies}{\; \Rightarrow \;}
%\newcommand{\eqname}[1]{\tag*{#1}} % Tag equation with name

\setlength{\droptitle}{-7em}

\theoremstyle{plain}
\newtheorem{theorem}{Teorema}[section]
%\newtheorem{defi}[theorem]{Definição}
\newtheorem{lemma}[theorem]{Lema}
%\newtheorem{corol}[theorem]{Corolário}
%\newtheorem{prop}[theorem]{Proposição}
%\newtheorem{example}{Exemplo}
%
%\newtheorem{inneraxiom}{Axioma}
%\newenvironment{axioma}[1]
%  {\renewcommand\theinneraxiom{#1}\inneraxiom}
%  {\endinneraxiom}
%
%\newtheorem{innerpostulado}{Postulado}
%\newenvironment{postulado}[1]
%  {\renewcommand\theinnerpostulado{#1}\innerpostulado}
%  {\endinnerpostulado}
%
%\newtheorem{innerexercise}{Exercício}
%\newenvironment{exercise}[1]
%  {\renewcommand\theinnerexercise{#1}\innerexercise}
%  {\endinnerexercise}
%
%\newtheorem{innerthm}{Teorema}
%\newenvironment{teorema}[1]
%  {\renewcommand\theinnerthm{#1}\innerthm}
%  {\endinnerthm}
%
\newtheorem{innerlema}{Lema}
\newenvironment{lema}[1]
  {\renewcommand\theinnerlema{#1}\innerlema}
  {\endinnerlema}
%
%\theoremstyle{remark}
%\newtheorem*{hint}{Dica}
%\newtheorem*{notation}{Notação}
%\newtheorem*{obs}{Observação}


\title{\Huge{\textbf{Indução}}}
\author{Mateus Marques}

\begin{document}

\maketitle

\section{Questão 1}


Neste exercício exploraremos o \textit{Princípio da Indução Finita}.

O raciocínio é assim: quando você vê um padrão que funciona para alguns números, por exemplo 1, 2, 3 e 4, você começa a suspeitar que esse padrão deve valer para todos os números, certo?

Por exemplo, defina os números $F_n = 2^{2^n} + 1$, para $n \geq 0$ inteiro.

Temos que $F_0 = 2^{2^0} + 1 = 2^1 + 1 = 3$ é primo.

\n

Quanto é $F_1$?

Ele é primo?
\begin{itemize}
\item Sim.
\item Não.
\end{itemize}

\n

Quanto é $F_2$?

Ele é primo?
\begin{itemize}
\item Sim.
\item Não.
\end{itemize}

\n

Quanto é $F_3$?

Ele é primo?
\begin{itemize}
\item Sim.
\item Não.
\end{itemize}

\n

Quanto é $F_4$?

Ele é primo?
\begin{itemize}
\item Sim.
\item Não.
\end{itemize}

\n

Então deve ser verdade que $F_n$ é primo para todo $n$ natural, correto?
\begin{itemize}
\item Sim. Esta indução é verdadeira.
\item Não. Esta indução é falsa.
\end{itemize}


\section{Questão 2}

Fermat acreditava que sim, que todos os números $F_n$ eram primos (hoje eles são chamados de \textit{números de Fermat}, \url{https://en.wikipedia.org/wiki/Fermat_number}), porém o próximo $F_5 = 4294967297 = 641 \times 6700417$ \textbf{não é primo}. Na realidade, até hoje (1\textsuperscript{\b{o}} de março de 2023) não se sabe se existe outro número de Fermat que seja primo além de $F_0, F_1, F_2, F_3$ e $F_4$. Note que a sequência $F_n = 2^{2^n} + 1$ cresce MUITO RÁPIDO, por exemplo:

$$ F_6 = 18446744073709551617, $$

$$ F_7 = 340282366920938463463374607431768211457. $$

\n\n


O \textit{Princípio da Indução Finita} serve para não cometermos o mesmo erro que o pobre Fermat. A coisa funciona assim:

\n

Seja $P(n)$ uma propriedade sobre um número inteiro $n$, que pode ser verdadeira ou falsa. Por exemplo:
\begin{itemize}
\item $P(n) =$ ``O número $F_n$ é primo''. Nesse caso $P(n)$ é verdadeira para $n = 0, 1, 2, 3$ e $4$, mas é falsa para $n = 5$.
\item $P(n) =$ ``A realidade física tem $n$ dimensões''. Nesse caso talvez $P(3)$ ou $P(4)$ (pelo espaço-tempo da teoria da relatividade) sejam verdadeiras. Mas talvez $P(26)$ seja verdadeira por causa da teoria das cordas.
\item $P(n) =$ ``A soma $1 + 2 + 3 + \cdots + n$ dos $n$ primeiros números naturais vale $\frac{n(n+1)}{2}$''. Nesse caso $P(n)$ é verdadeira para todo $n$, e você vai provar isso no final com o \textit{Princípio da Indução Finita}.
\end{itemize}

\n

Se os dois seguintes itens forem verdadeiros sobre $P(n)$,
\begin{enumerate}
\item (``Base da Indução'') $P(1)$ é verdade.
\item (``Passo de Indução'') Para todo $k$ natural vale que: se $P(k)$ for verdade, então $P(k+1)$ também é verdade.
\end{enumerate}

Então $P(n)$ é verdadeira para todos os números naturais.

\n\n

Por quê? Bem, o raciocínio é recursivo:
\begin{itemize}
\item $P(1)$ é verdade (pela ``Base da Indução''). Apliquemos então o ``Passo da Indução'': Como $P(1)$ é verdade, então $P(2)$ também é verdade.
\item $P(2)$ é verdade pelo item anterior. Então, pelo ``Passo da Indução'', como $P(2)$ é verdade, então $P(3)$ também é verdade.
\item $P(3)$ é verdade pelo item anterior. Então, pelo ``Passo da Indução'', como $P(3)$ é verdade, então $P(4)$ também é verdade.
\item ...
\end{itemize}

E assim por diante, podemos concluir que a ``Base da Indução'' juntamente com o ``Passo da Indução'' estabelecem que $P(n)$ é verdadeira para todo $n$ natural.

\n

Então, se desejamos \textit{provar} que $P(n)$ é verdadeira para todo $n$, basta que provemos a ``Base da Indução'' e o ``Passo da Indução''.

\n\n

Por exemplo, vamos provar que $1^2 + 2^2 + 3^2 + \cdots + n^2 = n(n+1)(2n+1)/6$ por \textit{indução}.
$$
P(n) = \text{``a soma }Q_n = 1^2 + 2^2 + 3^2 + \cdots + n^2 \text{ é igual a }\frac{n(n+1)(2n+1)}{6}\text{''}
$$

\begin{itemize}
\item Base da Indução: é verdade que $1^2 = \frac{1 \cdot (1 + 1) (2 \cdot 1 + 1)}{6}$, logo $P(1)$ é verdadeira.
\item Passo da Indução: Quponha que $P(k)$ seja verdadeira. Isto é $Q_k = 1^2 + 2^2 + 3^2 + \cdots + k^2 = k(k+1)(2k+1)/6$. Calculemos então a soma $Q_{k+1}$
$$
Q_{k+1} = \Big( 1^2 + 2^2 + 3^2 + \cdots + k^2 \Big) + (k+1)^2 = Q_k + (k+1)^2.
$$
Nós podemos $Q_k$, pois admitimos $P(k)$ verdadeira como hipótese. Portanto
$$
Q_{k+1} = \frac{k(k+1)(2k+1)}{6} + (k+1)^2 = \frac{(k+1)}{6} \Big( k(2k+1) + 6(k+1) \Big)
$$
$$
= \frac{(k+1)}{6} \Big( 2k^2 + 7k + 6 \Big) = \frac{(k+1)(k+2)(2k+3)}{6} \implies
$$
$$
\boxed{ Q_{k+1} = \frac{(k+1)\big[(k+1)+2\big]\big[2(k+1)+1\big]}{6}. }
$$
Mas a igualdade destacada acima é justamente a propriedade $P(k+1)$. Portanto, completamos a prova do passo de indução. Mostramos que $P(k)$ verdadeira $\implies P(k+1)$ verdadeira.

Isso \textit{prova} que $1^2 + 2^2 + 3^2 + \cdots + n^2 = n(n+1)(2n+1)/6$ pelo \textit{Princípio da Indução}.
\end{itemize}

\n\n

Agora é sua vez. Prove \textit{por indução} que $S_n = 1 + 2 + 3 + \cdots + n = \frac{n(n+1)}{2}$, repetindo o raciocínio acima.

\n

Você consegue utilizar o desenho abaixo para explicar o porquê de $1 + 2 + 3 + \cdots + n = \frac{n(n+1)}{2}$?

\section{Questão 3}

(Apostol) Descreva a falácia na seguinte ``prova'' por indução:

\n\n

\textit{Afirmação.} $P(n) =$ ``Para qualquer conjunto de $n$ garotas loiras, se pelo menos uma das garotas tem olhos azuis, então todas as $n$ garotas têm olhos azuis.''

\n

\textit{Prova.}

Base da Indução: Vemos que $P(1)$ é verdade, pois para o conjunto de apenas uma garota, é óbvio que uma delas tendo olhos azuis implica que todas (somente esta uma garota) têm olhos azuis.

Passo da Indução: O passo de $P(k) \implies P(k+1)$ pode ser ilustrado indo de $k = 3$ para $k = 4$. Assuma que $P(3)$ seja verdadeira e sejam $G_1, G_2, G_3$ e $G_4$ as quatro garotas loiras tais que pelo menos uma delas, por exemplo $G_1$, tenha olhos azuis. Considerando $G_1$, $G_2$ e $G_3$ em um conjunto e usando $P(3)$, temos que $G_2$ e $G_3$ também têm olhos azuis. Repetindo agora esse processo para $G_1$, $G_2$ e $G_4$, por $P(3)$ novamente, vemos que $G_4$ também tem olhos azuis. Assim, todas as quatro garotas têm olhos azuis. Isso demonstra $P(3) \implies P(4)$. Um argumento similar nos permite fazer o passo da indução de $k$ para $k+1$ em geral.

\n

Se a prova acima estiver correta, isso significa que todas as garotas loiras têm olhos azuis. Então, o que está errado na prova?






\end{document}
