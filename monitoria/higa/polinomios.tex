
\documentclass[a4paper,fleqn,12pt]{article}
\usepackage[brazilian]{babel}
\usepackage[left=2.5cm,right=2.5cm,top=3cm,bottom=2.5cm]{geometry}
\usepackage{mathtools}
\usepackage{amsthm}
\usepackage{amsmath}
%\usepackage{nccmath}
\usepackage{amssymb}
\usepackage{amsfonts}
\usepackage{physics}
%\usepackage{dsfont}
%\usepackage{mathrsfs}

\usepackage{titling}
\usepackage{indentfirst}

\usepackage{bm}
\usepackage[dvipsnames]{xcolor}
\usepackage{cancel}

\usepackage{xurl}
\usepackage[colorlinks=true]{hyperref}

\usepackage{float}
\usepackage{graphicx}
%\usepackage{tikz}
\usepackage{caption}
\usepackage{subcaption}

%%%%%%%%%%%%%%%%%%%%%%%%%%%%%%%%%%%%%%%%%%%%%%%%%%%

\newcommand{\eps}{\epsilon}
\newcommand{\vphi}{\varphi}
\newcommand{\cte}{\text{cte}}

\newcommand{\N}{\mathbb{N}}
\newcommand{\Z}{\mathbb{Z}}
\newcommand{\Q}{\mathbb{Q}}
\newcommand{\R}{\vb{R}}
\newcommand{\C}{\mathbb{C}}
\renewcommand{\S}{\hat{S}}
%\renewcommand{\H}{\s{H}}

\renewcommand{\a}{\vb{a}}
\newcommand{\nn}{\hat{n}}
\renewcommand{\d}{\dagger}
\newcommand{\up}{\uparrow}
\newcommand{\down}{\downarrow}

\newcommand{\0}{\vb{0}}
%\newcommand{\1}{\mathds{1}}
\newcommand{\E}{\vb{E}}
\newcommand{\B}{\vb{B}}
\renewcommand{\v}{\vb{v}}
\renewcommand{\r}{\vb{r}}
\renewcommand{\k}{\vb{k}}
\newcommand{\p}{\vb{p}}
\newcommand{\q}{\vb{q}}
\newcommand{\F}{\vb{F}}

\newcommand{\s}{\sigma}
%\newcommand{\prodint}[2]{\left\langle #1 , #2 \right\rangle}
\newcommand{\cc}[1]{\overline{#1}}
\newcommand{\Eval}[3]{\eval{\left( #1 \right)}_{#2}^{#3}}

\newcommand{\unit}[1]{\; \mathrm{#1}}

\newcommand{\n}{\medskip}
\newcommand{\e}{\quad \mathrm{e} \quad}
\newcommand{\ou}{\quad \mathrm{ou} \quad}
\newcommand{\virg}{\, , \;}
\newcommand{\ptodo}{\forall \,}
\renewcommand{\implies}{\; \Rightarrow \;}
%\newcommand{\eqname}[1]{\tag*{#1}} % Tag equation with name

\setlength{\droptitle}{-7em}

\theoremstyle{plain}
\newtheorem{theorem}{Teorema}[section]
%\newtheorem{defi}[theorem]{Definição}
\newtheorem{lemma}[theorem]{Lema}
%\newtheorem{corol}[theorem]{Corolário}
%\newtheorem{prop}[theorem]{Proposição}
%\newtheorem{example}{Exemplo}
%
%\newtheorem{inneraxiom}{Axioma}
%\newenvironment{axioma}[1]
%  {\renewcommand\theinneraxiom{#1}\inneraxiom}
%  {\endinneraxiom}
%
%\newtheorem{innerpostulado}{Postulado}
%\newenvironment{postulado}[1]
%  {\renewcommand\theinnerpostulado{#1}\innerpostulado}
%  {\endinnerpostulado}
%
%\newtheorem{innerexercise}{Exercício}
%\newenvironment{exercise}[1]
%  {\renewcommand\theinnerexercise{#1}\innerexercise}
%  {\endinnerexercise}
%
%\newtheorem{innerthm}{Teorema}
%\newenvironment{teorema}[1]
%  {\renewcommand\theinnerthm{#1}\innerthm}
%  {\endinnerthm}
%
\newtheorem{innerlema}{Lema}
\newenvironment{lema}[1]
  {\renewcommand\theinnerlema{#1}\innerlema}
  {\endinnerlema}
%
%\theoremstyle{remark}
%\newtheorem*{hint}{Dica}
%\newtheorem*{notation}{Notação}
%\newtheorem*{obs}{Observação}


\title{\Huge{\textbf{Polinômios}}}
\author{Mateus Marques}

%  longdiv.tex  v.1  (1994)  Donald Arseneau  
%
%  Work out and print integer long division problems.  Use:
%       \longdiv{numerator}{denominator}
%  The numerator and denominator (divisor and dividend) must be integers, and
%  the quotient is an integer too.  \longdiv leaves a remainder.
%  Use this in any type of TeX.

\newcount\gpten % (global) power-of-ten -- tells which digit we are doing
\countdef\rtot2 % running total -- remainder so far
\countdef\LDscratch4 % scratch

\def\longdiv#1#2{%
 \vtop{\normalbaselines \offinterlineskip
   \setbox\strutbox\hbox{\vrule height 2.1ex depth .5ex width0ex}%
   \def\showdig{$\underline{\the\LDscratch\strut}$\cr\the\rtot\strut\cr
       \noalign{\kern-.2ex}}%
   \global\rtot=#1\relax
   \count0=\rtot\divide\count0by#2\edef\quotient{\the\count0}%\show\quotient
   % make list macro out of digits in quotient:
   \def\temp##1{\ifx##1\temp\else \noexpand\dodig ##1\expandafter\temp\fi}%
   \edef\routine{\expandafter\temp\quotient\temp}%
   % process list to give power-of-ten:
   \def\dodig##1{\global\multiply\gpten by10 }\global\gpten=1 \routine
   % to display effect of one digit in quotient (zero ignored):
   \def\dodig##1{\global\divide\gpten by10
      \LDscratch =\gpten
      \multiply\LDscratch  by##1%
      \multiply\LDscratch  by#2%
      \global\advance\rtot-\LDscratch \relax
      \ifnum\LDscratch>0 \showdig \fi % must hide \cr in a macro to skip it
   }%
   \tabskip=0pt
   \halign{\hfil##\cr % \halign for entire division problem
     $\quotient$\strut\cr
     #2$\,\overline{\vphantom{\big)}%
     \hbox{\smash{\raise3.5\fontdimen8\textfont3\hbox{$\big)$}}}%
     \mkern2mu \the\rtot}$\cr\noalign{\kern-.2ex}
     \routine \cr % do each digit in quotient
}}}

\endinput % Demonstration below:

\noindent Here are some long division problems

\indent
\longdiv{12345}{13} \quad
\longdiv{123}{1234} \quad
\longdiv{31415926}{2} \quad
\longdiv{81}{3} \quad
\longdiv{1132}{99} \quad
\longdiv{86491}{94}
\bye

\usepackage{polynom}

\begin{document}

\maketitle

\section{Derivada de Polinômios}

A derivada $f'(x)$ de uma função $f(x)$ é definida como
$$
f'(x) = \lim_{h \to 0} \frac{f(x + h) - f(x)}{h}.
$$

No caso particular $f(x) = x^n$, temos que $f(x + h) = (x + h)^n = x^n + n x^{n-1} h + O(h^2)$, onde a notação $O(h^2)$ significa um termo que tem ordem $h^2$ ou maior. Assim, temos

$$
f'(x) = \lim_{h \to 0} \frac{(x + h)^n - x^n}{h} =
\lim_{h \to 0} \frac{\cancel{x^n} + nx^{n-1} h + O(h^2) - \cancel{x^n}}{h} =
\lim_{h \to 0} \qty( n x^{n-1} + \cancelto{0}{\frac{O(h^2)}{h}} )
$$
$$
\implies f'(x) = n x^{n-1}.
$$

Para $n = 3$, temos $f(x) = x^3$ e
$$
f(x + h) = (x + h)^3 = (x + h)^2 (x + h) = (x^2 + 2xh + h^2) (x + h) = x^3 + 3 x^2 h + \textcolor{blue}{3 x h^2 + h^3},
$$

Reconhecemos então o termo $\textcolor{blue}{O(h^2) = 3x h^2 + h^3 = h^2 ( 3x + h )}$, que realmente tem ordem maior ou igual a $h^2$.

Então, para derivar um polinômio é só ``descer a potência $n$ e subtrair um dela'', ou seja, $\displaystyle{\dv{(x^n)}{x} = n x^{n-1}}$.

\n\n

Calcule a derivada dos seguintes polinômios:
\begin{itemize}
\item $f(x) = x^4 + 58 x^3 - 32 x^2 + 19 x + 9$.
\item $g(x) = - 300 x^5 - 11 x^3 + \frac{1}{2} x^2 + 666$.
\item $h(x) = 2 x^{2023} - 5 x^{999999} + 2$.
\end{itemize}

\pagebreak

\section{Divisão de Polinômios}

Dividir polinômios é análogo a fazer uma divisão euclidiana de números inteiros, por exemplo dividir $801$ por $7$:

$$\longdiv{801}{7}$$

Note que acima achamos o quociente $114$ e o resto $3$ que satisfazem $801 = 114 \cdot 7 + 3$.

\n\n

Vamos dividir o polinômio $f(x) = 6x^3 - 2x^2 + x + 3$ pelo polinômio $g(x) = x^2 + x + 1$. A ideia é parecida com o caso de números inteiros. Queremos achar os polinômios quociente $q(x)$ e resto $r(x)$ que satisfazem $f(x) = q(x) \, g(x) + r(x)$.

$$\polylongdiv[style=D]{6x^3-2x^2+x+3}{x^2-x+1}$$

No primeiro passo, multiplicamos $x^2 + x + 1$ por $\boxed{6x}$ e subtraimos, para conseguir cancelar o $\textcolor{blue}{6x^3}$ do $\textcolor{blue}{6x^3} - 2x^2 + x + 3$:

$$\polylongdiv[style=D, stage=4]{6x^3-2x^2+x+3}{x^2-x+1}$$

Com isso obtivemos $4x^2 - 5x + 3$, devemos então multiplicar $x^2 + x + 1$ por $\boxed{4}$ para cancelar o $\textcolor{blue}{4x^2}$ do $\textcolor{blue}{4x^2} - 5x + 3$ e então subtraimos:

$$\polylongdiv[style=D]{6x^3-2x^2+x+3}{x^2-x+1}$$

Isso nos deixa com o quociente $\boxed{q(x) = 6x + 4}$ e o resto foi o que sobrou $r(x) = - x - 1$.

\begin{itemize}
\item Dividir $x^2 - 2x + 1$ por $x - 1$.
\item Dividir $x^5 + 3x^2 - 5$ por $x^2 + 1$.
\item Dividir $-6x^3 + 5x - 8$ por $3x^2 - 1$.
\end{itemize}


\pagebreak

\section{Funções horárias}

Se a função horária de um objeto é dada por uma função $s(t)$, temos que a velocidade média do objeto entre os instantes $t+\Delta t$ e $t$ é dada por:
$$
v_{\text{média}} (t+\Delta t, t) = \frac{s(t+\Delta t) - s(t)}{\Delta t}.
$$

Se calcularmos a velocidade média num intervalo $\Delta t$ muito curto, teremos uma boa aproximação para a velocidade \textbf{instantânea} $v(t)$ do corpo em movimento. De fato,
$$
v(t) = \lim_{\Delta t \to 0} \frac{s(t+\Delta t) - s(t)}{\Delta t} = \lim_{\Delta t \to 0} v_{\text{média}} (t+\Delta t, t).
$$

Assim, temos que $v(t)$ é a derivada da função $s(t)$ e escrevemos
$$
v(t) = s'(t) \quad \text{ou} \quad v(t) = \frac{ds}{dt}(t).
$$

Suponha que o movimento de um corpo seja descrito por $s(t) = t^3 - 5 t^2 + 4t$, sendo a unidade de tempo em segundos e a unidade de comprimento em metros.

\begin{itemize}
\item Em quais instantes o corpo parou?
\item Em qual instante $t_m$ a velocidade média com relação ao início do movimento em $t = 0$ é igual à velocidade instantânea?
\item Para quais intervalos de tempo a distância do corpo com relação ao ponto $s = 0$ é maior que $20 \unit{m}$?
\end{itemize}


\pagebreak

\section{Soma de inteiros}

Talvez você tenha aprendido progressões aritméticas no colégio, de maneira que seja familiar o fato
$$
0 + 1 + 2 + 3 + \ldots + n = \frac{n(n+1)}{2} = \frac{1}{2} n^2 + \frac{1}{2} n.
$$

Uma maneira de mostrar isso é pensar que, como estamos somando $n$ inteiros, sendo que cada um tem uma ordem polinomial de no máximo $n$, então o resultado deve ser um polinômio de ordem $n^2$. Denotando
$$
f(n) = 0 + 1 + 2 + 3 + \ldots + n,
$$
temos se $f(n)$ é um polinômio de ordem 2, então $f(x) = a x^2 + b x + c$. Mas pela definição de $f(n)$ temos que
$$
f(0) = 0 = c,
$$
$$
f(1) = 0 + 1 = a + b + c,
$$
$$
f(2) = 0 + 1 + 2 = 4 a + 2 b + c.
$$

Resolva as três equações acima para $a, b$ e $c$! Você deve obter $c = 0$, $a = \frac{1}{2}$ e $b = \frac{1}{2}$, que é o mesmo resultado que se obtém com progressões aritméticas.

\n

Utilize o mesmo raciocínio para os casos mais difíceis:

\begin{itemize}
\item Descubra uma fórmula para $0^2 + 1^2 + 2^2 + 3^2 + \ldots + n^2$. Essa fórmula deve ser uma polinômio de qual grau? Com essa fórmula calcule $1^2 + 2^2 + 3^2 + \ldots + 100^2$ com a sua mão.
\item Descubra $0^3 + 1^3 + 2^3 + 3^3 + \ldots + n^3$. Deve ser um polinômio de qual grau? Calcule $1^3 + 2^3 + \ldots + 1000^3$ com a sua mão.
\end{itemize}


\pagebreak

\section{Mapa Logístico}

Uma maneira de modelar o crescimento de uma população $x_n$ ao longo dos meses $n$ é pelo crescimento exponencial. A ideia é que o próximo mês terá uma população $x_{n+1}$ proporcional à população $x_n$ do mês anterior. Matematicamente,
$$
x_{n+1} = r x_n,
$$
onde $r$ é a taxa de natalidade da população. Mas veja que esse modelo de crescimento populacional faz com que a população cresça indefinidamente, pois só existe natalidade e nunca mortalidade. Assim, um modelo um pouco melhor consiste em adicionar um fator $(1 - x_n)$ na relação acima, que simula a mortalidade da população quando $x_n$ fica próximo de $1$ (aqui o $1$ significa a população máxima que pode ser atingida).

Nosso modelo de evolução populacional é então dado por
$$
x_{n+1} = r x_n (1 - x_n),
$$
que é conhecido como \textbf{mapa logístico}. Denote $f(x) = r x ( 1 - x )$ e suponha que $0 < r \leq 4$ para toda esta questão.

\n

\begin{enumerate}
\item Qual o máximo de $x_{n+1}$?
\item Suponha que a população convirja para uma população limite $x$. Para achar $x$, deve-se resolver a equação de ponto fixo $f(x) = x$. Pense porque basta resolver essa equação. Qual o valor de $x$ como função de $r$?
\item Agora suponha que a população no limite $n \to \infty$ oscile entre dois valores $x_1$ e $x_2$. Para achar $x_1$ e $x_2$ basta resolver a equação $f(f(x)) = x$. Por quê? Qual os valores de $x_1$ e $x_2$ em função de $r$?
\end{enumerate}

Dica: para resolver $f(f(x)) = x$, note que as raízes de $f(x) = x$ também são raízes de $f(f(x)) = x$. Assim, ao achar as raízes de $f(x) = x$ no item 2, você pode dividir o polinômio $p(x) = f(f(x)) - x$ por essas raízes para diminuir a ordem da equação polinomial $f(f(x)) = x$.


\end{document}
