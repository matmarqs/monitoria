
\documentclass[a4paper,fleqn,12pt]{article}
%\usepackage{mathtools}
\usepackage{amsthm}     % for definitions and theorems
\usepackage[many]{tcolorbox}    % boxes around definitions and theorems
%\usepackage{amsmath}
%\usepackage{nccmath}
\usepackage{amssymb}    % \ltimes, semi-direct product
%\usepackage{etoolbox}   % for start of Chapter
%\usepackage{amsfonts}
\usepackage{physics}    % for all Physics related
\usepackage{dsfont}     % for the identity matrix symbol \1
%\usepackage{mathrsfs}
\usepackage[notextcomp]{stix}   % font package and some symbols like filled square
%\usepackage{MnSymbol}   % symbols font package

\usepackage{titling}
\usepackage{indentfirst}

\usepackage{bm}
\usepackage[dvipsnames]{xcolor}
\usepackage{cancel}
\usepackage{enumitem}

\usepackage{xurl}
%\usepackage[colorlinks=true]{hyperref} % links have colors
\usepackage{hyperref}  % no colors

\usepackage{float}
\usepackage{graphicx}
\usepackage{subcaption}
%\usepackage{tikz}

\usepackage{ctable}     % tabelas
\renewcommand{\P}{\phantom{+}}  % empty space to indent things
\usepackage{multirow}
\usepackage{tabulary}

%%%%%%%%%%%%%%%%%%%%%%%%%%%%%%%%%%%%%%%%%%%%%%%%%%%

\newcommand{\eps}{\epsilon}
\newcommand{\vphi}{\varphi}
\newcommand{\cte}{\text{cte}}

\newcommand{\N}{{\mathbb{N}}}
\newcommand{\Z}{{\mathbb{Z}}}
%\newcommand{\Q}{{\mathbb{Q}}}
\newcommand{\C}{{\mathbb{C}}}
\renewcommand{\S}{{\hat{S}}}
%\renewcommand{\H}{\s{H}}

\renewcommand{\a}{{\vb{a}}}
\renewcommand{\b}{{\vb{b}}}
\renewcommand{\d}{{\dagger}}
\newcommand{\up}{{\uparrow}}
\newcommand{\down}{{\downarrow}}
\newcommand{\hc}{{\text{h.c.}}}

\newcommand{\ihat}{\bm{\hat{\imath}}}
\newcommand{\jhat}{\bm{\hat{\jmath}}}
\newcommand{\khat}{\bm{\hat{k}}}

\newcommand{\0}{{\vb{0}}}
\newcommand{\1}{\mathds{1}}
\newcommand{\E}{{\vb{E}}}
\newcommand{\B}{{\vb{B}}}
\renewcommand{\u}{{\vb{u}}}
\renewcommand{\v}{{\vb{v}}}
\renewcommand{\r}{{\vb{r}}}
\newcommand{\R}{{\vb{R}}}
\newcommand{\Q}{{\vb{Q}}}
\newcommand{\G}{{\vb{G}}}
\newcommand{\g}{{\vb{g}}}
\renewcommand{\k}{{\vb{k}}}
\newcommand{\K}{{\vb{K}}}
\newcommand{\p}{{\vb{p}}}
\newcommand{\q}{{\vb{q}}}
\newcommand{\F}{{\vb{F}}}
\renewcommand{\t}{{\vb{t}}}
\newcommand{\vtau}{{\bm{\tau}}}
\newcommand{\vdelta}{{\bm{\delta}}}

% COLORED SYMMETRY ELEMENTS
\newcommand{\Ct}{{\textcolor{Cyan}{C_3}}}
\newcommand{\Ctn}[1]{{\textcolor{Cyan}{C_3^{\textcolor{black}{#1}}}}}
\newcommand{\Cs}{{\textcolor{ForestGreen}{C_6}}}
\newcommand{\Csn}[1]{{\textcolor{ForestGreen}{C_6^{\textcolor{black}{#1}}}}}
\newcommand{\sd}{{\textcolor{RoyalBlue}{\sigma_d}}}
\newcommand{\sdn}[1]{{\textcolor{RoyalBlue}{\sigma_d^{\textcolor{black}{#1}}}}}
\newcommand{\sdp}{{\textcolor{RoyalBlue}{\sigma_d'}}}
\newcommand{\sdpp}{{\textcolor{RoyalBlue}{\sigma_d''}}}
\newcommand{\sv}{{\textcolor{Orange}{\sigma_v}}}
\newcommand{\svn}[1]{{\textcolor{Orange}{\sigma_v^{\textcolor{black}{#1}}}}}
\newcommand{\svp}{{\textcolor{Orange}{\sigma_v'}}}
\newcommand{\svpp}{{\textcolor{Orange}{\sigma_v''}}}

\newcommand{\GL}{{\text{GL}}}
\newcommand{\U}{{\text{U}}}

\newcommand{\s}{\sigma}
%\newcommand{\prodint}[2]{\left\langle #1 , #2 \right\rangle}
\newcommand{\cc}[1]{\overline{#1}}
\newcommand{\Eval}[3]{\eval{\left( #1 \right)}_{#2}^{#3}}
\newcommand{\sg}[2]{\{ #1 \mid #2 \}}
\renewcommand{\AA}{{\mathring{\text{A}}}}
\newcommand{\I}{{\mathbb{I}}}
\newcommand{\bP}{{\mathbb{P}}}
\newcommand{\bQ}{{\mathbb{Q}}}

\newcommand{\unit}[1]{\; \mathrm{#1}}

\newcommand{\n}{\medskip}
\newcommand{\e}{\quad \mathrm{and} \quad}
\newcommand{\ou}{\quad \mathrm{or} \quad}
\newcommand{\virg}{\, , \;}
\newcommand{\ptodo}{\forall \,}
\renewcommand{\implies}{\; \Rightarrow \;}
%\newcommand{\eqname}[1]{\tag*{#1}} % Tag equation with name

%\setlength{\droptitle}{-7em}   % título um pouco mais em cima na página
%\makeatletter
%\patchcmd{\chapter}{\if@openright\cleardoublepage\else\clearpage\fi}{}{}{}  % start 'Chapter' at the same page. needs package etoolbox
%\makeatother

%% Theorems, definitions, proofs
\theoremstyle{definition}

%%% defining my own colors %%%
\definecolor{my-blue}{HTML}{f2f4ff}
\definecolor{my-green}{HTML}{f5fcf6}    % a little better: green!5!white
\definecolor{my-cyan}{HTML}{f2fffe}
\definecolor{my-yellow}{HTML}{fffbed}
\definecolor{my-green2}{HTML}{efffdb}

%%% alternative colors %%%
\definecolor{my-pink}{HTML}{fff2f7}
\definecolor{my-teal}{HTML}{ebfffc}

\newtheorem{definition}{Definition}[section]
\tcolorboxenvironment{definition}{
  colback=my-blue,
  %colback=blue!5!white,
  boxrule=0.1pt,
  boxsep=1pt,
  left=2pt,right=2pt,top=2pt,bottom=2pt,
  oversize=2pt,
  sharp corners,
  before skip=\topsep,
  after skip=\topsep,
}

\newtheorem{theorem}{Theorem}[section]
\tcolorboxenvironment{theorem}{
  colback=my-yellow,
  %colback=yellow!22!white!95!black,
  boxrule=0.1pt,
  boxsep=1pt,
  left=2pt,right=2pt,top=2pt,bottom=2pt,
  oversize=2pt,
  sharp corners,
  before skip=\topsep,
  after skip=\topsep,
}

\newtheorem{corollary}{Corollary}[section]
\tcolorboxenvironment{corollary}{
  colback=my-green2,
  boxrule=0.1pt,
  boxsep=1pt,
  left=2pt,right=2pt,top=2pt,bottom=2pt,
  oversize=2pt,
  sharp corners,
  before skip=\topsep,
  after skip=\topsep,
}

\newtheorem{lemma}{Lemma}[section]
\tcolorboxenvironment{lemma}{
  colback=my-cyan,
  boxrule=0.1pt,
  boxsep=1pt,
  left=2pt,right=2pt,top=2pt,bottom=2pt,
  oversize=2pt,
  sharp corners,
  before skip=\topsep,
  after skip=\topsep,
}

\newtheorem{example}{Example}[section]
\tcolorboxenvironment{example}{
  %colback=my-green,
  colback=green!5!white,
  boxrule=0.1pt,
  boxsep=1pt,
  left=2pt,right=2pt,top=2pt,bottom=2pt,
  oversize=2pt,
  sharp corners,
  before skip=\topsep,
  after skip=\topsep,
}


\title{\Huge{\textbf{Polinômios}}}
\author{Mateus Marques}

\input{longdiv.tex}
\usepackage{polynom}

\begin{document}

\maketitle

\section{Derivada de Polinômios}

A derivada $f'(x)$ de uma função $f(x)$ é definida como
$$
f'(x) = \lim_{h \to 0} \frac{f(x + h) - f(x)}{h}.
$$

No caso particular $f(x) = x^n$, temos que $f(x + h) = (x + h)^n = x^n + n x^{n-1} h + O(h^2)$, onde a notação $O(h^2)$ significa um termo que tem ordem $h^2$ ou maior. Assim, temos

$$
f'(x) = \lim_{h \to 0} \frac{(x + h)^n - x^n}{h} =
\lim_{h \to 0} \frac{\cancel{x^n} + nx^{n-1} h + O(h^2) - \cancel{x^n}}{h} =
\lim_{h \to 0} \qty( n x^{n-1} + \cancelto{0}{\frac{O(h^2)}{h}} )
$$
$$
\implies f'(x) = n x^{n-1}.
$$

Para $n = 3$, temos $f(x) = x^3$ e
$$
f(x + h) = (x + h)^3 = (x + h)^2 (x + h) = (x^2 + 2xh + h^2) (x + h) = x^3 + 3 x^2 h + \textcolor{blue}{3 x h^2 + h^3},
$$

Reconhecemos então o termo $\textcolor{blue}{O(h^2) = 3x h^2 + h^3 = h^2 ( 3x + h )}$, que realmente tem ordem maior ou igual a $h^2$.

Então, para derivar um polinômio é só ``descer a potência $n$ e subtrair um dela'', ou seja, $\displaystyle{\dv{(x^n)}{x} = n x^{n-1}}$.

\n\n

Calcule a derivada dos seguintes polinômios:
\begin{itemize}
\item $f(x) = x^4 + 58 x^3 - 32 x^2 + 19 x + 9$.
\item $g(x) = - 300 x^5 - 11 x^3 + \frac{1}{2} x^2 + 666$.
\item $h(x) = 2 x^{2023} - 5 x^{999999} + 2$.
\end{itemize}

\pagebreak

\section{Divisão de Polinômios}

Dividir polinômios é análogo a fazer uma divisão euclidiana de números inteiros, por exemplo dividir $801$ por $7$:

$$\longdiv{801}{7}$$

Note que acima achamos o quociente $114$ e o resto $3$ que satisfazem $801 = 114 \cdot 7 + 3$.

\n\n

Vamos dividir o polinômio $f(x) = 6x^3 - 2x^2 + x + 3$ pelo polinômio $g(x) = x^2 + x + 1$. A ideia é parecida com o caso de números inteiros. Queremos achar os polinômios quociente $q(x)$ e resto $r(x)$ que satisfazem $f(x) = q(x) \, g(x) + r(x)$.

$$\polylongdiv[style=D]{6x^3-2x^2+x+3}{x^2-x+1}$$

No primeiro passo, multiplicamos $x^2 + x + 1$ por $\boxed{6x}$ e subtraimos, para conseguir cancelar o $\textcolor{blue}{6x^3}$ do $\textcolor{blue}{6x^3} - 2x^2 + x + 3$:

$$\polylongdiv[style=D, stage=4]{6x^3-2x^2+x+3}{x^2-x+1}$$

Com isso obtivemos $4x^2 - 5x + 3$, devemos então multiplicar $x^2 + x + 1$ por $\boxed{4}$ para cancelar o $\textcolor{blue}{4x^2}$ do $\textcolor{blue}{4x^2} - 5x + 3$ e então subtraimos:

$$\polylongdiv[style=D]{6x^3-2x^2+x+3}{x^2-x+1}$$

Isso nos deixa com o quociente $\boxed{q(x) = 6x + 4}$ e o resto foi o que sobrou $r(x) = - x - 1$.

\begin{itemize}
\item Dividir $x^2 - 2x + 1$ por $x - 1$.
\item Dividir $x^5 + 3x^2 - 5$ por $x^2 + 1$.
\item Dividir $-6x^3 + 5x - 8$ por $3x^2 - 1$.
\end{itemize}


\pagebreak

\section{Mapa Logístico}

\subsection{Questão}

Queremos modelar o crescimento de uma população $x_n$ ao longo dos meses $n$.

\subsection{Questão}

Qual o máximo de $x_{n+1}$?

\subsection{Questão}

Resolver o problema de ponto fixo $f(x) = x$, com $f(x) = rx(1-x)$.

\subsection{Questão}

Resolver o problema de ponto fixo $f(f(x)) = x$.


\subsection{Questão}

Caso $r = 2$. Prove por indução que
$$
x_n = \frac{1}{2} \Big[ 1 - (1-2x_0)^{2^n} \Big]
$$

Caso $r = 4$. Prove por indução que
$$
x_n = \sin[2](2^n \pi \theta),
$$
onde $\theta = \frac{1}{\pi} \sin[-1](\sqrt{x_0})$.




\end{document}
