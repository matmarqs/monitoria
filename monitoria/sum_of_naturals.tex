\documentclass[a4paper,fleqn,12pt]{article}
\usepackage[brazilian]{babel}
\usepackage[left=2.5cm,right=2.5cm,top=3cm,bottom=2.5cm]{geometry}
\usepackage{mathtools}
\usepackage{amsthm}
\usepackage{amsmath}
%\usepackage{nccmath}
\usepackage{amssymb}
\usepackage{amsfonts}
\usepackage{physics}
%\usepackage{dsfont}
%\usepackage{mathrsfs}

\usepackage{titling}
\usepackage{indentfirst}

\usepackage{bm}
\usepackage[dvipsnames]{xcolor}
\usepackage{cancel}

\usepackage{xurl}
\usepackage[colorlinks=true]{hyperref}

\usepackage{float}
\usepackage{graphicx}
%\usepackage{tikz}
\usepackage{caption}
\usepackage{subcaption}

%%%%%%%%%%%%%%%%%%%%%%%%%%%%%%%%%%%%%%%%%%%%%%%%%%%

\newcommand{\eps}{\epsilon}
\newcommand{\vphi}{\varphi}
\newcommand{\cte}{\text{cte}}

\newcommand{\N}{\mathbb{N}}
\newcommand{\Z}{\mathbb{Z}}
\newcommand{\Q}{\mathbb{Q}}
\newcommand{\R}{\vb{R}}
\newcommand{\C}{\mathbb{C}}
\renewcommand{\S}{\hat{S}}
%\renewcommand{\H}{\s{H}}

\renewcommand{\a}{\vb{a}}
\newcommand{\nn}{\hat{n}}
\renewcommand{\d}{\dagger}
\newcommand{\up}{\uparrow}
\newcommand{\down}{\downarrow}

\newcommand{\0}{\vb{0}}
%\newcommand{\1}{\mathds{1}}
\newcommand{\E}{\vb{E}}
\newcommand{\B}{\vb{B}}
\renewcommand{\v}{\vb{v}}
\renewcommand{\r}{\vb{r}}
\renewcommand{\k}{\vb{k}}
\newcommand{\p}{\vb{p}}
\newcommand{\q}{\vb{q}}
\newcommand{\F}{\vb{F}}

\newcommand{\s}{\sigma}
%\newcommand{\prodint}[2]{\left\langle #1 , #2 \right\rangle}
\newcommand{\cc}[1]{\overline{#1}}
\newcommand{\Eval}[3]{\eval{\left( #1 \right)}_{#2}^{#3}}

\newcommand{\unit}[1]{\; \mathrm{#1}}

\newcommand{\n}{\medskip}
\newcommand{\e}{\quad \mathrm{e} \quad}
\newcommand{\ou}{\quad \mathrm{ou} \quad}
\newcommand{\virg}{\, , \;}
\newcommand{\ptodo}{\forall \,}
\renewcommand{\implies}{\; \Rightarrow \;}
%\newcommand{\eqname}[1]{\tag*{#1}} % Tag equation with name

\setlength{\droptitle}{-7em}

\theoremstyle{plain}
\newtheorem{theorem}{Teorema}[section]
%\newtheorem{defi}[theorem]{Definição}
\newtheorem{lemma}[theorem]{Lema}
%\newtheorem{corol}[theorem]{Corolário}
%\newtheorem{prop}[theorem]{Proposição}
%\newtheorem{example}{Exemplo}
%
%\newtheorem{inneraxiom}{Axioma}
%\newenvironment{axioma}[1]
%  {\renewcommand\theinneraxiom{#1}\inneraxiom}
%  {\endinneraxiom}
%
%\newtheorem{innerpostulado}{Postulado}
%\newenvironment{postulado}[1]
%  {\renewcommand\theinnerpostulado{#1}\innerpostulado}
%  {\endinnerpostulado}
%
%\newtheorem{innerexercise}{Exercício}
%\newenvironment{exercise}[1]
%  {\renewcommand\theinnerexercise{#1}\innerexercise}
%  {\endinnerexercise}
%
%\newtheorem{innerthm}{Teorema}
%\newenvironment{teorema}[1]
%  {\renewcommand\theinnerthm{#1}\innerthm}
%  {\endinnerthm}
%
\newtheorem{innerlema}{Lema}
\newenvironment{lema}[1]
  {\renewcommand\theinnerlema{#1}\innerlema}
  {\endinnerlema}
%
%\theoremstyle{remark}
%\newtheorem*{hint}{Dica}
%\newtheorem*{notation}{Notação}
%\newtheorem*{obs}{Observação}


\title{\Huge{\textbf{Soma estranha?}}}
\author{Mateus Marques}

\begin{document}

\maketitle

\section{Questão 1}

Seja \(X\) a soma infinita
$$
X = 1 - 1 + 1 - 1 + 1 - 1 + \ldots + 1 - 1 + \ldots
$$

Se olharmos bem o padrão acima, temos
$$
X = 1 - (1 - 1 + 1 - 1 + \ldots + 1 - 1 + \ldots).
$$

Só que o termo em parênteses é o próprio \(X\)! Portanto:
$$
X = 1 - X.
$$
Com isso, resolvemos para \(X\) e obtemos \(X = 1/2\).

Você concorda com a dedução acima?
\begin{itemize}
\item Sim. (Correto)
\item Não. (Correto)
\end{itemize}

\section{Questão 2}

Agora é sua vez de encontrar o valor para uma soma estranha. Seja \(Y\) a soma infinita
$$
Y = 1 - 2 + 3 - 4 + 5 - 6 + 7 - 8 + \ldots
$$

Se somarmos \(Y + X\) \textit{de maneira esperta} \textbf{termo a termo na mesma vertical}, temos
$$
\begin{cases}
\quad \quad \quad Y = 1 - 2 + 3 - 4 + 5 - 6 + 7 - 8 + \ldots \\
\quad + \\
\quad \quad \quad X = 0 + 1 - 1 + 1 - 1 + 1 - 1 + 1 - \ldots \\
\_\_\_\_\_\_\_\_\_\_\_\_\_\_\_\_\_\_\_\_\_\_\_\_\_\_\_\_\_ \\
\quad Y+X = 1 - 1 + 2 - 3 + 4 - 5 + 6 - 7 + \ldots
\end{cases}
$$
Existe um padrão em
$$
Y+X = 1 - 1 + 2 - 3 + 4 - 5 + 6 - 7 + \ldots
$$
em que você pode substituir uma parte da soma da direita por \(Y\). Fazendo isso e, dado que nós já sabemos o valor de \(X = 1/2\), que valor você descobre para \(Y\)?
\begin{itemize}
\item \(Y = -\infty\)
\item \(Y = +\infty\)
\item \(Y = 1/2\)
\item \(Y = 1/4\) (Correto)
\item \(Y = -1\)
\end{itemize}

\section{Questão 3}

Agora, vamos somar todos os números naturais!
$$
Z = 1 + 2 + 3 + 4 + 5 + 6 + 7 + 8 + \ldots.
$$

Se eu te dissesse que o valor de \(Z\) é diferente de infinito \(+\infty\), você acreditaria?
\begin{itemize}
\item Sim. (Correto)
\item Não. (Correto)
\end{itemize}

\section{Questão 4}

Se subtrairmos \(Z - Y\) \textbf{termo a termo na mesma vertical}, temos
$$
\begin{cases}
\quad \quad \quad Z = 1 + 2 + 3 + 4 + 5 + \; 6 \; + 7 + \, 8 \; + \ldots \\
\quad - \\
\quad \quad \quad Y = 1 - 2 + 3 - 4 + 5 - \, 6 \; + 7 - \, 8 \; - \ldots \\
\_\_\_\_\_\_\_\_\_\_\_\_\_\_\_\_\_\_\_\_\_\_\_\_\_\_\_\_\_ \\
\quad Z-Y = 0 + 4 + 0 + 8 + 0 + 12 + 0 + 16 + \ldots
\end{cases}
$$

Assim, obtemos
$$
Z - Y = 4 + 8 + 12 + 16 + \ldots
$$
Você consegue enxergar o padrão acima na soma da direita? É possível substituir essa soma infinita por alguma das variáveis (\(X\), \(Y\) ou \(Z\)) que já encontramos. Substitua corretamente o padrão. Que valor você encontra para \(Z\)?
\begin{itemize}
\item \(Z = +\infty\)
\item \(Z = \pi\)
\item \(Z = 1\)
\item \(Z = -1/4\)
\item \(Z = -1/12\) (Correto)
\end{itemize}

\section{Questão 5}

A vida faz sentido?
\begin{itemize}
\item Sim.
\item Não. (Correto)
\end{itemize}



\end{document}
