\documentclass[a4paper,fleqn,12pt]{article}
\usepackage[brazilian]{babel}
\usepackage[left=2.5cm,right=2.5cm,top=3cm,bottom=2.5cm]{geometry}
\usepackage{mathtools}
\usepackage{amsthm}
\usepackage{amsmath}
%\usepackage{nccmath}
\usepackage{amssymb}
\usepackage{amsfonts}
\usepackage{physics}
%\usepackage{dsfont}
%\usepackage{mathrsfs}

\usepackage{titling}
\usepackage{indentfirst}

\usepackage{bm}
\usepackage[dvipsnames]{xcolor}
\usepackage{cancel}

\usepackage{xurl}
\usepackage[colorlinks=true]{hyperref}

\usepackage{float}
\usepackage{graphicx}
%\usepackage{tikz}
\usepackage{caption}
\usepackage{subcaption}

%%%%%%%%%%%%%%%%%%%%%%%%%%%%%%%%%%%%%%%%%%%%%%%%%%%

\newcommand{\eps}{\epsilon}
\newcommand{\vphi}{\varphi}
\newcommand{\cte}{\text{cte}}

\newcommand{\N}{\mathbb{N}}
\newcommand{\Z}{\mathbb{Z}}
\newcommand{\Q}{\mathbb{Q}}
\newcommand{\R}{\vb{R}}
\newcommand{\C}{\mathbb{C}}
\renewcommand{\S}{\hat{S}}
%\renewcommand{\H}{\s{H}}

\renewcommand{\a}{\vb{a}}
\newcommand{\nn}{\hat{n}}
\renewcommand{\d}{\dagger}
\newcommand{\up}{\uparrow}
\newcommand{\down}{\downarrow}

\newcommand{\0}{\vb{0}}
%\newcommand{\1}{\mathds{1}}
\newcommand{\E}{\vb{E}}
\newcommand{\B}{\vb{B}}
\renewcommand{\v}{\vb{v}}
\renewcommand{\r}{\vb{r}}
\renewcommand{\k}{\vb{k}}
\newcommand{\p}{\vb{p}}
\newcommand{\q}{\vb{q}}
\newcommand{\F}{\vb{F}}

\newcommand{\s}{\sigma}
%\newcommand{\prodint}[2]{\left\langle #1 , #2 \right\rangle}
\newcommand{\cc}[1]{\overline{#1}}
\newcommand{\Eval}[3]{\eval{\left( #1 \right)}_{#2}^{#3}}

\newcommand{\unit}[1]{\; \mathrm{#1}}

\newcommand{\n}{\medskip}
\newcommand{\e}{\quad \mathrm{e} \quad}
\newcommand{\ou}{\quad \mathrm{ou} \quad}
\newcommand{\virg}{\, , \;}
\newcommand{\ptodo}{\forall \,}
\renewcommand{\implies}{\; \Rightarrow \;}
%\newcommand{\eqname}[1]{\tag*{#1}} % Tag equation with name

\setlength{\droptitle}{-7em}

\theoremstyle{plain}
\newtheorem{theorem}{Teorema}[section]
%\newtheorem{defi}[theorem]{Definição}
\newtheorem{lemma}[theorem]{Lema}
%\newtheorem{corol}[theorem]{Corolário}
%\newtheorem{prop}[theorem]{Proposição}
%\newtheorem{example}{Exemplo}
%
%\newtheorem{inneraxiom}{Axioma}
%\newenvironment{axioma}[1]
%  {\renewcommand\theinneraxiom{#1}\inneraxiom}
%  {\endinneraxiom}
%
%\newtheorem{innerpostulado}{Postulado}
%\newenvironment{postulado}[1]
%  {\renewcommand\theinnerpostulado{#1}\innerpostulado}
%  {\endinnerpostulado}
%
%\newtheorem{innerexercise}{Exercício}
%\newenvironment{exercise}[1]
%  {\renewcommand\theinnerexercise{#1}\innerexercise}
%  {\endinnerexercise}
%
%\newtheorem{innerthm}{Teorema}
%\newenvironment{teorema}[1]
%  {\renewcommand\theinnerthm{#1}\innerthm}
%  {\endinnerthm}
%
\newtheorem{innerlema}{Lema}
\newenvironment{lema}[1]
  {\renewcommand\theinnerlema{#1}\innerlema}
  {\endinnerlema}
%
%\theoremstyle{remark}
%\newtheorem*{hint}{Dica}
%\newtheorem*{notation}{Notação}
%\newtheorem*{obs}{Observação}


\title{\Huge{\textbf{Frações contínuas}}}
\author{Mateus Marques}

\begin{document}

\maketitle

Este é um exercício guiado, o que significa que te guiarei em cada ponto... O objetivo é lhe introduzir a um tipo de fração bem peculiar, chamada fração contínua.

Para começar, vamos nos exercitar com frações comuns. Escreva todas suas respostas de maneira numérica, utilizando o ponto "." como marcador decimal. Por favor não utilize calculadora... Uma dica é: primeiro transforme a expressão em uma fração "bonitinha", como por exemplo $\displaystyle{1 + \frac{1}{3 + \frac{2}{7}} = 1 + \frac{1}{\frac{21 + 2}{7}} = 1 + \frac{7}{23} = \frac{30}{23}}$, e depois faça a divisão na mão.

Quanto é $\displaystyle{1 + \frac{1}{2 + \frac{1}{2}}}$ ?

Agora, quanto é $\displaystyle{1 + \frac{1}{2 + \frac{1}{2 + \frac{1}{2 + \frac{1}{2}}}}}$ aproximadamente? (escreva a resposta com pelo menos 4 casas decimais)?


Vamos continuar só mais um pouco...

Quanto é $\displaystyle{1 + \frac{1}{2 + \frac{1}{2 + \frac{1}{2 + \frac{1}{2 + \frac{1}{2 + \frac{1}{2 + \frac{1}{2}}}}}}}}$ ? (resposta com pelo menos 5 casas decimais)


Por acaso você conhece esse número $1.4142...$ que as respostas anteriores são próximas? Lembra da $\sqrt{2}$?


Por que será? Vou te explicar. Vamos supor que fizéssemos esse processo até o infinito. Ou seja, queremos calcular

$$ \displaystyle{ x = 1 + \frac{1}{2 + \frac{1}{2 + \frac{1}{2 + \, \cdots}}} } $$

Se passarmos o $1$ para o outro lado, e chamando $y = x-1$, temos o seguinte

$$ \displaystyle{ y = x-1 = \frac{1}{2 + \frac{1}{2 + \frac{1}{2 + \, \cdots}}} } $$

Mas perceba que o padrão $\displaystyle{\frac{1}{2 + \frac{1}{2 + \, \cdots}}}$ se repete (para sempre). Então, ao pensarmos um pouco, temos a seguinte recursão:

$$ \displaystyle{ y = \frac{1}{2 + y} } $$

Resolvendo para $y$ (utilizando Bháskara, lembre que $\displaystyle{ax^2 + bx + c = 0 \Rightarrow x = \frac{-b \pm \sqrt{b^2 - 4ac}}{2a}}$)

$$ \displaystyle{ y^2 + 2y - 1 = 0 \Rightarrow y = \sqrt{2} - 1 } $$

Lembrando que $ \displaystyle{ y = x-1 \Rightarrow x = y + 1 } $, obtemos que $x = \sqrt{2}$.

Agora eu te desafio a resolver uma fração contínua! Leve sua solução para monitoria e podemos discuti-la!

Qual é o valor de

$ \displaystyle{ x = 1 + \frac{1}{1 + \frac{1}{1 + \frac{1}{1 + \, \cdots}}} ? } $

\end{document}
