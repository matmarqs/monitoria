\documentclass[a4paper,fleqn,12pt]{article}
%\usepackage{mathtools}
\usepackage{amsthm}     % for definitions and theorems
\usepackage[many]{tcolorbox}    % boxes around definitions and theorems
%\usepackage{amsmath}
%\usepackage{nccmath}
\usepackage{amssymb}    % \ltimes, semi-direct product
%\usepackage{etoolbox}   % for start of Chapter
%\usepackage{amsfonts}
\usepackage{physics}    % for all Physics related
\usepackage{dsfont}     % for the identity matrix symbol \1
%\usepackage{mathrsfs}
\usepackage[notextcomp]{stix}   % font package and some symbols like filled square
%\usepackage{MnSymbol}   % symbols font package

\usepackage{titling}
\usepackage{indentfirst}

\usepackage{bm}
\usepackage[dvipsnames]{xcolor}
\usepackage{cancel}
\usepackage{enumitem}

\usepackage{xurl}
%\usepackage[colorlinks=true]{hyperref} % links have colors
\usepackage{hyperref}  % no colors

\usepackage{float}
\usepackage{graphicx}
\usepackage{subcaption}
%\usepackage{tikz}

\usepackage{ctable}     % tabelas
\renewcommand{\P}{\phantom{+}}  % empty space to indent things
\usepackage{multirow}
\usepackage{tabulary}

%%%%%%%%%%%%%%%%%%%%%%%%%%%%%%%%%%%%%%%%%%%%%%%%%%%

\newcommand{\eps}{\epsilon}
\newcommand{\vphi}{\varphi}
\newcommand{\cte}{\text{cte}}

\newcommand{\N}{{\mathbb{N}}}
\newcommand{\Z}{{\mathbb{Z}}}
%\newcommand{\Q}{{\mathbb{Q}}}
\newcommand{\C}{{\mathbb{C}}}
\renewcommand{\S}{{\hat{S}}}
%\renewcommand{\H}{\s{H}}

\renewcommand{\a}{{\vb{a}}}
\renewcommand{\b}{{\vb{b}}}
\renewcommand{\d}{{\dagger}}
\newcommand{\up}{{\uparrow}}
\newcommand{\down}{{\downarrow}}
\newcommand{\hc}{{\text{h.c.}}}

\newcommand{\ihat}{\bm{\hat{\imath}}}
\newcommand{\jhat}{\bm{\hat{\jmath}}}
\newcommand{\khat}{\bm{\hat{k}}}

\newcommand{\0}{{\vb{0}}}
\newcommand{\1}{\mathds{1}}
\newcommand{\E}{{\vb{E}}}
\newcommand{\B}{{\vb{B}}}
\renewcommand{\u}{{\vb{u}}}
\renewcommand{\v}{{\vb{v}}}
\renewcommand{\r}{{\vb{r}}}
\newcommand{\R}{{\vb{R}}}
\newcommand{\Q}{{\vb{Q}}}
\newcommand{\G}{{\vb{G}}}
\newcommand{\g}{{\vb{g}}}
\renewcommand{\k}{{\vb{k}}}
\newcommand{\K}{{\vb{K}}}
\newcommand{\p}{{\vb{p}}}
\newcommand{\q}{{\vb{q}}}
\newcommand{\F}{{\vb{F}}}
\renewcommand{\t}{{\vb{t}}}
\newcommand{\vtau}{{\bm{\tau}}}
\newcommand{\vdelta}{{\bm{\delta}}}

% COLORED SYMMETRY ELEMENTS
\newcommand{\Ct}{{\textcolor{Cyan}{C_3}}}
\newcommand{\Ctn}[1]{{\textcolor{Cyan}{C_3^{\textcolor{black}{#1}}}}}
\newcommand{\Cs}{{\textcolor{ForestGreen}{C_6}}}
\newcommand{\Csn}[1]{{\textcolor{ForestGreen}{C_6^{\textcolor{black}{#1}}}}}
\newcommand{\sd}{{\textcolor{RoyalBlue}{\sigma_d}}}
\newcommand{\sdn}[1]{{\textcolor{RoyalBlue}{\sigma_d^{\textcolor{black}{#1}}}}}
\newcommand{\sdp}{{\textcolor{RoyalBlue}{\sigma_d'}}}
\newcommand{\sdpp}{{\textcolor{RoyalBlue}{\sigma_d''}}}
\newcommand{\sv}{{\textcolor{Orange}{\sigma_v}}}
\newcommand{\svn}[1]{{\textcolor{Orange}{\sigma_v^{\textcolor{black}{#1}}}}}
\newcommand{\svp}{{\textcolor{Orange}{\sigma_v'}}}
\newcommand{\svpp}{{\textcolor{Orange}{\sigma_v''}}}

\newcommand{\GL}{{\text{GL}}}
\newcommand{\U}{{\text{U}}}

\newcommand{\s}{\sigma}
%\newcommand{\prodint}[2]{\left\langle #1 , #2 \right\rangle}
\newcommand{\cc}[1]{\overline{#1}}
\newcommand{\Eval}[3]{\eval{\left( #1 \right)}_{#2}^{#3}}
\newcommand{\sg}[2]{\{ #1 \mid #2 \}}
\renewcommand{\AA}{{\mathring{\text{A}}}}
\newcommand{\I}{{\mathbb{I}}}
\newcommand{\bP}{{\mathbb{P}}}
\newcommand{\bQ}{{\mathbb{Q}}}

\newcommand{\unit}[1]{\; \mathrm{#1}}

\newcommand{\n}{\medskip}
\newcommand{\e}{\quad \mathrm{and} \quad}
\newcommand{\ou}{\quad \mathrm{or} \quad}
\newcommand{\virg}{\, , \;}
\newcommand{\ptodo}{\forall \,}
\renewcommand{\implies}{\; \Rightarrow \;}
%\newcommand{\eqname}[1]{\tag*{#1}} % Tag equation with name

%\setlength{\droptitle}{-7em}   % título um pouco mais em cima na página
%\makeatletter
%\patchcmd{\chapter}{\if@openright\cleardoublepage\else\clearpage\fi}{}{}{}  % start 'Chapter' at the same page. needs package etoolbox
%\makeatother

%% Theorems, definitions, proofs
\theoremstyle{definition}

%%% defining my own colors %%%
\definecolor{my-blue}{HTML}{f2f4ff}
\definecolor{my-green}{HTML}{f5fcf6}    % a little better: green!5!white
\definecolor{my-cyan}{HTML}{f2fffe}
\definecolor{my-yellow}{HTML}{fffbed}
\definecolor{my-green2}{HTML}{efffdb}

%%% alternative colors %%%
\definecolor{my-pink}{HTML}{fff2f7}
\definecolor{my-teal}{HTML}{ebfffc}

\newtheorem{definition}{Definition}[section]
\tcolorboxenvironment{definition}{
  colback=my-blue,
  %colback=blue!5!white,
  boxrule=0.1pt,
  boxsep=1pt,
  left=2pt,right=2pt,top=2pt,bottom=2pt,
  oversize=2pt,
  sharp corners,
  before skip=\topsep,
  after skip=\topsep,
}

\newtheorem{theorem}{Theorem}[section]
\tcolorboxenvironment{theorem}{
  colback=my-yellow,
  %colback=yellow!22!white!95!black,
  boxrule=0.1pt,
  boxsep=1pt,
  left=2pt,right=2pt,top=2pt,bottom=2pt,
  oversize=2pt,
  sharp corners,
  before skip=\topsep,
  after skip=\topsep,
}

\newtheorem{corollary}{Corollary}[section]
\tcolorboxenvironment{corollary}{
  colback=my-green2,
  boxrule=0.1pt,
  boxsep=1pt,
  left=2pt,right=2pt,top=2pt,bottom=2pt,
  oversize=2pt,
  sharp corners,
  before skip=\topsep,
  after skip=\topsep,
}

\newtheorem{lemma}{Lemma}[section]
\tcolorboxenvironment{lemma}{
  colback=my-cyan,
  boxrule=0.1pt,
  boxsep=1pt,
  left=2pt,right=2pt,top=2pt,bottom=2pt,
  oversize=2pt,
  sharp corners,
  before skip=\topsep,
  after skip=\topsep,
}

\newtheorem{example}{Example}[section]
\tcolorboxenvironment{example}{
  %colback=my-green,
  colback=green!5!white,
  boxrule=0.1pt,
  boxsep=1pt,
  left=2pt,right=2pt,top=2pt,bottom=2pt,
  oversize=2pt,
  sharp corners,
  before skip=\topsep,
  after skip=\topsep,
}


\title{\Huge{\textbf{Indução}}}
\author{Mateus Marques}

\begin{document}

\maketitle

\section{Questão 1}


Neste exercício exploraremos o \textit{Princípio da Indução Finita}.

O raciocínio é assim: quando você vê um padrão que funciona para alguns números, por exemplo 1, 2, 3 e 4, você começa a suspeitar que esse padrão deve valer para todos os números, certo?

Por exemplo, defina os números $F_n = 2^{2^n} + 1$, para $n \geq 0$ inteiro.

Temos que $F_0 = 2^{2^0} + 1 = 2^1 + 1 = 3$ é primo.

\n

Quanto é $F_1$?

Ele é primo?
\begin{itemize}
\item Sim.
\item Não.
\end{itemize}

\n

Quanto é $F_2$?

Ele é primo?
\begin{itemize}
\item Sim.
\item Não.
\end{itemize}

\n

Quanto é $F_3$?

Ele é primo?
\begin{itemize}
\item Sim.
\item Não.
\end{itemize}

\n

Quanto é $F_4$?

Ele é primo?
\begin{itemize}
\item Sim.
\item Não.
\end{itemize}

\n

Então deve ser verdade que $F_n$ é primo para todo $n$ natural, correto?
\begin{itemize}
\item Sim. Esta indução é verdadeira.
\item Não. Esta indução é falsa.
\end{itemize}


\section{Questão 2}

Fermat acreditava que sim, que todos os números $F_n$ eram primos (hoje eles são chamados de \textit{números de Fermat}, \url{https://en.wikipedia.org/wiki/Fermat_number}), porém o próximo $F_5 = 4294967297 = 641 \times 6700417$ \textbf{não é primo}. Na realidade, até hoje (1\textsuperscript{\b{o}} de março de 2023) não se sabe se existe outro número de Fermat que seja primo além de $F_0, F_1, F_2, F_3$ e $F_4$. Note que a sequência $F_n = 2^{2^n} + 1$ cresce MUITO RÁPIDO, por exemplo:

$$ F_6 = 18446744073709551617, $$

$$ F_7 = 340282366920938463463374607431768211457. $$

\n\n


O \textit{Princípio da Indução Finita} serve para não cometermos o mesmo erro que o pobre Fermat. A coisa funciona assim:

\n

Seja $P(n)$ uma propriedade sobre um número inteiro $n$, que pode ser verdadeira ou falsa. Por exemplo:
\begin{itemize}
\item $P(n) =$ ``O número $F_n$ é primo''. Nesse caso $P(n)$ é verdadeira para $n = 0, 1, 2, 3$ e $4$, mas é falsa para $n = 5$.
\item $P(n) =$ ``A realidade física tem $n$ dimensões''. Nesse caso talvez $P(3)$ ou $P(4)$ (pelo espaço-tempo da teoria da relatividade) sejam verdadeiras. Mas talvez $P(26)$ seja verdadeira por causa da teoria das cordas.
\item $P(n) =$ ``A soma $1 + 2 + 3 + \cdots + n$ dos $n$ primeiros números naturais vale $\frac{n(n+1)}{2}$''. Nesse caso $P(n)$ é verdadeira para todo $n$, e você vai provar isso no final com o \textit{Princípio da Indução Finita}.
\end{itemize}

\n

Se os dois seguintes itens forem verdadeiros sobre $P(n)$,
\begin{enumerate}
\item (``Base da Indução'') $P(1)$ é verdade.
\item (``Passo de Indução'') Para todo $k$ natural vale que: se $P(k)$ for verdade, então $P(k+1)$ também é verdade.
\end{enumerate}

Então $P(n)$ é verdadeira para todos os números naturais.

\n\n

Por quê? Bem, o raciocínio é recursivo:
\begin{itemize}
\item $P(1)$ é verdade (pela ``Base da Indução''). Apliquemos então o ``Passo da Indução'': Como $P(1)$ é verdade, então $P(2)$ também é verdade.
\item $P(2)$ é verdade pelo item anterior. Então, pelo ``Passo da Indução'', como $P(2)$ é verdade, então $P(3)$ também é verdade.
\item $P(3)$ é verdade pelo item anterior. Então, pelo ``Passo da Indução'', como $P(3)$ é verdade, então $P(4)$ também é verdade.
\item ...
\end{itemize}

E assim por diante, podemos concluir que a ``Base da Indução'' juntamente com o ``Passo da Indução'' estabelecem que $P(n)$ é verdadeira para todo $n$ natural.

\n

Então, se desejamos \textit{provar} que $P(n)$ é verdadeira para todo $n$, basta que provemos a ``Base da Indução'' e o ``Passo da Indução''.

\n\n

Por exemplo, vamos provar que $1^2 + 2^2 + 3^2 + \cdots + n^2 = n(n+1)(2n+1)/6$ por \textit{indução}.
$$
P(n) = \text{``a soma }Q_n = 1^2 + 2^2 + 3^2 + \cdots + n^2 \text{ é igual a }\frac{n(n+1)(2n+1)}{6}\text{''}
$$

\begin{itemize}
\item Base da Indução: é verdade que $1^2 = \frac{1 \cdot (1 + 1) (2 \cdot 1 + 1)}{6}$, logo $P(1)$ é verdadeira.
\item Passo da Indução: Quponha que $P(k)$ seja verdadeira. Isto é $Q_k = 1^2 + 2^2 + 3^2 + \cdots + k^2 = k(k+1)(2k+1)/6$. Calculemos então a soma $Q_{k+1}$
$$
Q_{k+1} = \Big( 1^2 + 2^2 + 3^2 + \cdots + k^2 \Big) + (k+1)^2 = Q_k + (k+1)^2.
$$
Nós podemos $Q_k$, pois admitimos $P(k)$ verdadeira como hipótese. Portanto
$$
Q_{k+1} = \frac{k(k+1)(2k+1)}{6} + (k+1)^2 = \frac{(k+1)}{6} \Big( k(2k+1) + 6(k+1) \Big)
$$
$$
= \frac{(k+1)}{6} \Big( 2k^2 + 7k + 6 \Big) = \frac{(k+1)(k+2)(2k+3)}{6} \implies
$$
$$
\boxed{ Q_{k+1} = \frac{(k+1)\big[(k+1)+2\big]\big[2(k+1)+1\big]}{6}. }
$$
Mas a igualdade destacada acima é justamente a propriedade $P(k+1)$. Portanto, completamos a prova do passo de indução. Mostramos que $P(k)$ verdadeira $\implies P(k+1)$ verdadeira.

Isso \textit{prova} que $1^2 + 2^2 + 3^2 + \cdots + n^2 = n(n+1)(2n+1)/6$ pelo \textit{Princípio da Indução}.
\end{itemize}

\n\n

Agora é sua vez. Prove \textit{por indução} que $S_n = 1 + 2 + 3 + \cdots + n = \frac{n(n+1)}{2}$, repetindo o raciocínio acima.

\n

Você consegue utilizar o desenho abaixo para explicar o porquê de $1 + 2 + 3 + \cdots + n = \frac{n(n+1)}{2}$?

\section{Questão 3}

(Apostol) Descreva a falácia na seguinte ``prova'' por indução:

\n\n

\textit{Afirmação.} $P(n) =$ ``Para qualquer conjunto de $n$ garotas loiras, se pelo menos uma das garotas tem olhos azuis, então todas as $n$ garotas têm olhos azuis.''

\n

\textit{Prova.}

Base da Indução: Vemos que $P(1)$ é verdade, pois para o conjunto de apenas uma garota, é óbvio que uma delas tendo olhos azuis implica que todas (somente esta uma garota) têm olhos azuis.

Passo da Indução: O passo de $P(k) \implies P(k+1)$ pode ser ilustrado indo de $k = 3$ para $k = 4$. Assuma que $P(3)$ seja verdadeira e sejam $G_1, G_2, G_3$ e $G_4$ as quatro garotas loiras tais que pelo menos uma delas, por exemplo $G_1$, tenha olhos azuis. Considerando $G_1$, $G_2$ e $G_3$ em um conjunto e usando $P(3)$, temos que $G_2$ e $G_3$ também têm olhos azuis. Repetindo agora esse processo para $G_1$, $G_2$ e $G_4$, por $P(3)$ novamente, vemos que $G_4$ também tem olhos azuis. Assim, todas as quatro garotas têm olhos azuis. Isso demonstra $P(3) \implies P(4)$. Um argumento similar nos permite fazer o passo da indução de $k$ para $k+1$ em geral.

\n

Se a prova acima estiver correta, isso significa que todas as garotas loiras têm olhos azuis. Então, o que está errado na prova?






\end{document}
