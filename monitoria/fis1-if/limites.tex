
\documentclass[a4paper,fleqn,12pt]{article}
%\usepackage{mathtools}
\usepackage{amsthm}     % for definitions and theorems
\usepackage[many]{tcolorbox}    % boxes around definitions and theorems
%\usepackage{amsmath}
%\usepackage{nccmath}
\usepackage{amssymb}    % \ltimes, semi-direct product
%\usepackage{etoolbox}   % for start of Chapter
%\usepackage{amsfonts}
\usepackage{physics}    % for all Physics related
\usepackage{dsfont}     % for the identity matrix symbol \1
%\usepackage{mathrsfs}
\usepackage[notextcomp]{stix}   % font package and some symbols like filled square
%\usepackage{MnSymbol}   % symbols font package

\usepackage{titling}
\usepackage{indentfirst}

\usepackage{bm}
\usepackage[dvipsnames]{xcolor}
\usepackage{cancel}
\usepackage{enumitem}

\usepackage{xurl}
%\usepackage[colorlinks=true]{hyperref} % links have colors
\usepackage{hyperref}  % no colors

\usepackage{float}
\usepackage{graphicx}
\usepackage{subcaption}
%\usepackage{tikz}

\usepackage{ctable}     % tabelas
\renewcommand{\P}{\phantom{+}}  % empty space to indent things
\usepackage{multirow}
\usepackage{tabulary}

%%%%%%%%%%%%%%%%%%%%%%%%%%%%%%%%%%%%%%%%%%%%%%%%%%%

\newcommand{\eps}{\epsilon}
\newcommand{\vphi}{\varphi}
\newcommand{\cte}{\text{cte}}

\newcommand{\N}{{\mathbb{N}}}
\newcommand{\Z}{{\mathbb{Z}}}
%\newcommand{\Q}{{\mathbb{Q}}}
\newcommand{\C}{{\mathbb{C}}}
\renewcommand{\S}{{\hat{S}}}
%\renewcommand{\H}{\s{H}}

\renewcommand{\a}{{\vb{a}}}
\renewcommand{\b}{{\vb{b}}}
\renewcommand{\d}{{\dagger}}
\newcommand{\up}{{\uparrow}}
\newcommand{\down}{{\downarrow}}
\newcommand{\hc}{{\text{h.c.}}}

\newcommand{\ihat}{\bm{\hat{\imath}}}
\newcommand{\jhat}{\bm{\hat{\jmath}}}
\newcommand{\khat}{\bm{\hat{k}}}

\newcommand{\0}{{\vb{0}}}
\newcommand{\1}{\mathds{1}}
\newcommand{\E}{{\vb{E}}}
\newcommand{\B}{{\vb{B}}}
\renewcommand{\u}{{\vb{u}}}
\renewcommand{\v}{{\vb{v}}}
\renewcommand{\r}{{\vb{r}}}
\newcommand{\R}{{\vb{R}}}
\newcommand{\Q}{{\vb{Q}}}
\newcommand{\G}{{\vb{G}}}
\newcommand{\g}{{\vb{g}}}
\renewcommand{\k}{{\vb{k}}}
\newcommand{\K}{{\vb{K}}}
\newcommand{\p}{{\vb{p}}}
\newcommand{\q}{{\vb{q}}}
\newcommand{\F}{{\vb{F}}}
\renewcommand{\t}{{\vb{t}}}
\newcommand{\vtau}{{\bm{\tau}}}
\newcommand{\vdelta}{{\bm{\delta}}}

% COLORED SYMMETRY ELEMENTS
\newcommand{\Ct}{{\textcolor{Cyan}{C_3}}}
\newcommand{\Ctn}[1]{{\textcolor{Cyan}{C_3^{\textcolor{black}{#1}}}}}
\newcommand{\Cs}{{\textcolor{ForestGreen}{C_6}}}
\newcommand{\Csn}[1]{{\textcolor{ForestGreen}{C_6^{\textcolor{black}{#1}}}}}
\newcommand{\sd}{{\textcolor{RoyalBlue}{\sigma_d}}}
\newcommand{\sdn}[1]{{\textcolor{RoyalBlue}{\sigma_d^{\textcolor{black}{#1}}}}}
\newcommand{\sdp}{{\textcolor{RoyalBlue}{\sigma_d'}}}
\newcommand{\sdpp}{{\textcolor{RoyalBlue}{\sigma_d''}}}
\newcommand{\sv}{{\textcolor{Orange}{\sigma_v}}}
\newcommand{\svn}[1]{{\textcolor{Orange}{\sigma_v^{\textcolor{black}{#1}}}}}
\newcommand{\svp}{{\textcolor{Orange}{\sigma_v'}}}
\newcommand{\svpp}{{\textcolor{Orange}{\sigma_v''}}}

\newcommand{\GL}{{\text{GL}}}
\newcommand{\U}{{\text{U}}}

\newcommand{\s}{\sigma}
%\newcommand{\prodint}[2]{\left\langle #1 , #2 \right\rangle}
\newcommand{\cc}[1]{\overline{#1}}
\newcommand{\Eval}[3]{\eval{\left( #1 \right)}_{#2}^{#3}}
\newcommand{\sg}[2]{\{ #1 \mid #2 \}}
\renewcommand{\AA}{{\mathring{\text{A}}}}
\newcommand{\I}{{\mathbb{I}}}
\newcommand{\bP}{{\mathbb{P}}}
\newcommand{\bQ}{{\mathbb{Q}}}

\newcommand{\unit}[1]{\; \mathrm{#1}}

\newcommand{\n}{\medskip}
\newcommand{\e}{\quad \mathrm{and} \quad}
\newcommand{\ou}{\quad \mathrm{or} \quad}
\newcommand{\virg}{\, , \;}
\newcommand{\ptodo}{\forall \,}
\renewcommand{\implies}{\; \Rightarrow \;}
%\newcommand{\eqname}[1]{\tag*{#1}} % Tag equation with name

%\setlength{\droptitle}{-7em}   % título um pouco mais em cima na página
%\makeatletter
%\patchcmd{\chapter}{\if@openright\cleardoublepage\else\clearpage\fi}{}{}{}  % start 'Chapter' at the same page. needs package etoolbox
%\makeatother

%% Theorems, definitions, proofs
\theoremstyle{definition}

%%% defining my own colors %%%
\definecolor{my-blue}{HTML}{f2f4ff}
\definecolor{my-green}{HTML}{f5fcf6}    % a little better: green!5!white
\definecolor{my-cyan}{HTML}{f2fffe}
\definecolor{my-yellow}{HTML}{fffbed}
\definecolor{my-green2}{HTML}{efffdb}

%%% alternative colors %%%
\definecolor{my-pink}{HTML}{fff2f7}
\definecolor{my-teal}{HTML}{ebfffc}

\newtheorem{definition}{Definition}[section]
\tcolorboxenvironment{definition}{
  colback=my-blue,
  %colback=blue!5!white,
  boxrule=0.1pt,
  boxsep=1pt,
  left=2pt,right=2pt,top=2pt,bottom=2pt,
  oversize=2pt,
  sharp corners,
  before skip=\topsep,
  after skip=\topsep,
}

\newtheorem{theorem}{Theorem}[section]
\tcolorboxenvironment{theorem}{
  colback=my-yellow,
  %colback=yellow!22!white!95!black,
  boxrule=0.1pt,
  boxsep=1pt,
  left=2pt,right=2pt,top=2pt,bottom=2pt,
  oversize=2pt,
  sharp corners,
  before skip=\topsep,
  after skip=\topsep,
}

\newtheorem{corollary}{Corollary}[section]
\tcolorboxenvironment{corollary}{
  colback=my-green2,
  boxrule=0.1pt,
  boxsep=1pt,
  left=2pt,right=2pt,top=2pt,bottom=2pt,
  oversize=2pt,
  sharp corners,
  before skip=\topsep,
  after skip=\topsep,
}

\newtheorem{lemma}{Lemma}[section]
\tcolorboxenvironment{lemma}{
  colback=my-cyan,
  boxrule=0.1pt,
  boxsep=1pt,
  left=2pt,right=2pt,top=2pt,bottom=2pt,
  oversize=2pt,
  sharp corners,
  before skip=\topsep,
  after skip=\topsep,
}

\newtheorem{example}{Example}[section]
\tcolorboxenvironment{example}{
  %colback=my-green,
  colback=green!5!white,
  boxrule=0.1pt,
  boxsep=1pt,
  left=2pt,right=2pt,top=2pt,bottom=2pt,
  oversize=2pt,
  sharp corners,
  before skip=\topsep,
  after skip=\topsep,
}


\title{\Huge{\textbf{Zoológico}}}
\author{Mateus Marques}

\begin{document}

\maketitle


\section{Conjuntos Numéricos}

\begin{itemize}
\item Perguntar quais conjuntos numéricos eles conhecem: $\N$, $\Z$, $\Q$, $\R$ e $\C$ provavelmente.
\item O que é um número? Apresentar outros conjuntos numéricos: $\Z_p$, $\Q_p$, $\mathbb{F}^q$, $\mathbb{H}$, $\Lambda^n$.
\item Diferenças entre os conjuntos numéricos.
\item $\Q$ é enumerável, prova rápida com a diagonal
\item $\R$ é completo, exemplo de $1 + \frac{1}{2 + \frac{1}{2 + \cdots}} = \sqrt{2} \notin \Q$
\item $\C$ é algebricamente fechado, $x^2 + 1 = 0$.
\end{itemize}


\section{Funções}

\begin{itemize}
\item Definição $f: A \to B$.
\item Exemplos não-matemáticos:
\begin{itemize}
\item Função como uma máquina: constante (joga fora), identidade (faz porra nenhuma), multiplica por 2 (Jesus).
\item Diagrama básicos de Venn. Exemplo mãe e filho.
\item Exemplo do NUSP, altura dos alunos.
\end{itemize}

\item Diferença entre os Cálculos 1, 2, 3 e 4 e exemplos na Física.
\begin{enumerate}
\item $f: \R \to \R$. Trajetória, velocidade, aceleração em 1D.
\item $f: \R^n \to \R$. Potencial, temperatura, densidade.
\item $f: \R^n \to \R^m$. Campo elétrico, magnético, velocidade de fluido.
\item $f: \C \to \C$. Mecânica dos Fluidos. Na real é mais comum $\psi: \R^n \to \C$.
\item Domínios mais complicados, $S[x(t)] = \int_{t_i}^{t_f} L(x(t), \dot{x}(t), t) \dd{t}$.
\item Exemplo de não-função: $\delta(x)$ de Dirac.
\end{enumerate}

\item Funções patológicas que usam $\Q$: Dirichlet, Thomae, Weierstrass.

\begin{minipage}{0.4\textwidth}
$$
f(x) =
\begin{cases}
\frac{1}{q} \text{ se } x = p/q \text{ é racional,} \\
0 \text{ se } x \text{ é irracional.}
\end{cases}
$$
\end{minipage}
\begin{minipage}{0.4\textwidth}
\begin{figure}[H]
  \centering
  \includegraphics[width=1.\linewidth]{fig/thomae.png}
  \label{fig:thomae}
\end{figure}
\end{minipage}

\item Sutileza do domínio e contradomínio. Exemplo disso é $f:\Q \to \Q$, equação funcional $f(x+y) = f(x) + f(y)$. \textbf{Mencionar EDOs}.
\item Gráfico polar: $r(\theta) = \theta$, $r(\theta) = \cos(\theta)$ e $r(\theta) = \cos(5\theta)$.
\item Funções básicas: $x$, $x^n$, $1/x$, $e^x$, $\log(x)$, $\sin(x)$, $\abs{x}$.
\item Jogo de gráficos de funções:
\begin{itemize}
\item $x^2 + \sin(x)$, $e^{-x} \cos(x)$, $x + \frac{1}{x}, \frac{\sin(x)}{x}$, $\sqrt{1+x^2}$.
\item Jogo inverso: $(x^2 - 1)^2$, $\frac{x + \abs{x}}{2}$, $\frac{x}{\abs{x}}$.
\item Composição de funções, inversa.
\item Translações verticais e horizontais, reflexões com módulo.
\item Propriedades: paridade, periodicidade, limites para infinito, raízes, ponto fixo, derivada.
\end{itemize}
\end{itemize}


\section{Cultura}

\begin{itemize}
\item $1 + \frac{1}{2 + \frac{1}{2 + \cdots}} = \sqrt{2}$ como limite.
\item Derivada $\lim_{h \to 0} \frac{f(x+h) - f(x)}{h}$. Polinômios, trigonométricas, exponencial, log.
\item $\lim_{x \to 0} \frac{\sin(x)}{x}$, $\cos(x) = 1 - 2 \sin^2(x/2)$.
\item Exponencial
$$
e^x = \lim_{u \to \infty} \qty(1 + \frac{x}{u})^u,
$$
associada a equação $f_n(x) = \qty(1 + \frac{x}{n}) f_n'(x)$. Com isso dá para resolver $m \dot{v} = -k v$.
\item Mapa logístico, $f(x) = x$ e $f(f(x)) = x$.
\item Integral $\int_0^b x^2 \dd{x}$ usando $1^2 + 2^2 + \ldots + n^2 = f(n)$, com $f(x) = ax^3 + bx^2 + cx + d$.
\end{itemize}


\end{document}
