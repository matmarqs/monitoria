\documentclass[a4paper,10pt]{article}
\usepackage[brazilian]{babel}
\usepackage[left=2.5cm,right=2.5cm,top=3cm,bottom=2.5cm]{geometry}
\usepackage{mathtools}
\usepackage{amsthm}
\usepackage{amsmath}
%\usepackage{nccmath}
\usepackage{amssymb}
\usepackage{amsfonts}
\usepackage{physics}
%\usepackage{dsfont}
%\usepackage{mathrsfs}

\usepackage{titling}
\usepackage{indentfirst}

\usepackage{bm}
\usepackage[dvipsnames]{xcolor}
\usepackage{cancel}

\usepackage{xurl}
\usepackage[colorlinks=true]{hyperref}

\usepackage{float}
\usepackage{graphicx}
%\usepackage{tikz}
\usepackage{caption}
\usepackage{subcaption}

%%%%%%%%%%%%%%%%%%%%%%%%%%%%%%%%%%%%%%%%%%%%%%%%%%%

\newcommand{\eps}{\epsilon}
\newcommand{\vphi}{\varphi}
\newcommand{\cte}{\text{cte}}

\newcommand{\N}{\mathbb{N}}
\newcommand{\Z}{\mathbb{Z}}
\newcommand{\Q}{\mathbb{Q}}
\newcommand{\R}{\vb{R}}
\newcommand{\C}{\mathbb{C}}
\renewcommand{\S}{\hat{S}}
%\renewcommand{\H}{\s{H}}

\renewcommand{\a}{\vb{a}}
\newcommand{\nn}{\hat{n}}
\renewcommand{\d}{\dagger}
\newcommand{\up}{\uparrow}
\newcommand{\down}{\downarrow}

\newcommand{\0}{\vb{0}}
%\newcommand{\1}{\mathds{1}}
\newcommand{\E}{\vb{E}}
\newcommand{\B}{\vb{B}}
\renewcommand{\v}{\vb{v}}
\renewcommand{\r}{\vb{r}}
\renewcommand{\k}{\vb{k}}
\newcommand{\p}{\vb{p}}
\newcommand{\q}{\vb{q}}
\newcommand{\F}{\vb{F}}

\newcommand{\s}{\sigma}
%\newcommand{\prodint}[2]{\left\langle #1 , #2 \right\rangle}
\newcommand{\cc}[1]{\overline{#1}}
\newcommand{\Eval}[3]{\eval{\left( #1 \right)}_{#2}^{#3}}

\newcommand{\unit}[1]{\; \mathrm{#1}}

\newcommand{\n}{\medskip}
\newcommand{\e}{\quad \mathrm{e} \quad}
\newcommand{\ou}{\quad \mathrm{ou} \quad}
\newcommand{\virg}{\, , \;}
\newcommand{\ptodo}{\forall \,}
\renewcommand{\implies}{\; \Rightarrow \;}
%\newcommand{\eqname}[1]{\tag*{#1}} % Tag equation with name

\setlength{\droptitle}{-7em}

\theoremstyle{plain}
\newtheorem{theorem}{Teorema}[section]
%\newtheorem{defi}[theorem]{Definição}
\newtheorem{lemma}[theorem]{Lema}
%\newtheorem{corol}[theorem]{Corolário}
%\newtheorem{prop}[theorem]{Proposição}
%\newtheorem{example}{Exemplo}
%
%\newtheorem{inneraxiom}{Axioma}
%\newenvironment{axioma}[1]
%  {\renewcommand\theinneraxiom{#1}\inneraxiom}
%  {\endinneraxiom}
%
%\newtheorem{innerpostulado}{Postulado}
%\newenvironment{postulado}[1]
%  {\renewcommand\theinnerpostulado{#1}\innerpostulado}
%  {\endinnerpostulado}
%
%\newtheorem{innerexercise}{Exercício}
%\newenvironment{exercise}[1]
%  {\renewcommand\theinnerexercise{#1}\innerexercise}
%  {\endinnerexercise}
%
%\newtheorem{innerthm}{Teorema}
%\newenvironment{teorema}[1]
%  {\renewcommand\theinnerthm{#1}\innerthm}
%  {\endinnerthm}
%
\newtheorem{innerlema}{Lema}
\newenvironment{lema}[1]
  {\renewcommand\theinnerlema{#1}\innerlema}
  {\endinnerlema}
%
%\theoremstyle{remark}
%\newtheorem*{hint}{Dica}
%\newtheorem*{notation}{Notação}
%\newtheorem*{obs}{Observação}


\title{\Huge{\textbf{Exercícios 1}}}
\author{Mateus Marques}

\begin{document}

\maketitle

\section{}

Qual sistema não é adequadamente descrito como um oscilador amortecido?

\begin{itemize}
\item Um pêndulo na atmosfera.
\item Um circuito RLC.
\item Um bola de bilhar rolando sobre a mesa de sinuca.
\item Uma corda de violão vibrando até parar.
\end{itemize}


\section{}

A diferença de potencial entre as extremidades de um capacitor, resistor e indutor são, respectivamente, $V_C = \frac{Q}{C}$, $V_R = R I$ e $V_L = L \dv{I}{t}$.

É montado um circuito RLC, ou seja, os três componentes capacitor (capacitância $C$), resistor (resistência $R$) e indutor (indutância $L$) conectados em série.

As valores nominais de resistência e indutância dos componentes são $R = 1 \unit{\Omega}$, $L = 40 \unit{mH}$, $C = 1 \unit{\mu F}$.

No instante inicial $t = 0$, não há corrente pelo circuito e a voltagem entre as extremidades do capacitor é $V_0 = 0.1 \unit{V}$.

$$
\ddot{Q} + \frac{R}{L} \dot{Q} + \frac{Q}{LC} = 0
$$
$$
\ddot{V} + \frac{R}{L} \dot{V} + \frac{V}{LC} = 0
$$
$$
\ddot{V} + \gamma \dot{V} + \omega_0^2 V = 0,
$$
onde $\gamma = R/L$ e $\omega_0 = 1/\sqrt{LC}$.

Subamortecimento é o caso $\gamma/2 < \omega_0$, ou seja, $R < 2 \sqrt{L/C}$.

Qual o módulo da corrente máxima que o circuito atinge depois do instante inicial?
$$
Q(t) = A \, e^{-\frac{\gamma}{2} t} \cos(\omega t + \phi), \quad \omega = \sqrt{\omega_0^2 - (\gamma/2)^2}.
$$
$$
I(t) = - \frac{\gamma}{2} A \, e^{-\frac{\gamma}{2} t} \cos(\omega t + \phi) - \omega A \, e^{-\frac{\gamma}{2} t} \sin(\omega t + \phi).
$$
$$
I(0) = 0 = - \frac{\gamma}{2} A \, \cos(\phi) - \omega A \, \sin(\phi) \implies
$$
$$
\tan(\phi) = -\frac{\gamma}{2 \omega}.
$$
$$
\cos(\phi) = \frac{1}{\sqrt{1+\tan^2\phi}} = \frac{\omega}{\sqrt{\omega^2 + (\gamma/2)^2}} = \frac{\omega}{\omega_0}.
$$
$$
Q_0 = CV_0 = A \cos(\phi) \implies A = C V_0 \frac{\omega_0}{\omega}.
$$

Qual a carga no capacitor depois de um período de oscilação?
$$
Q\qty(\frac{2\pi}{\omega}) = C V_0 \exp(-\frac{\pi \gamma}{\omega}).
$$

\section{}

Uma corpo de massa $m = 50 \unit{g}$ que oscila preso a uma mola de constante elástica $k = 1.25 \unit{N \cdot m^{-1}}$ é sujeito a uma força de resistência do ar proporcional à sua velocidade, sendo $b = 3 \unit{g \cdot s^{-1}}$ a constante de proporcionalidade. No instante inicial $t = 0$ a mola está em sua posição de equilíbrio e o corpo tem velocidade inicial não-nula $v_0 = 1 \unit{m \cdot s^{-1}}$.

Qual o período $T$ de oscilação do movimento? $T = 2 \pi / \omega$, $\omega = \sqrt{\qty(\sqrt{\frac{k}{m}})^2 - \qty(\frac{b}{2m})^2}$.

Sendo $x(t)$ a posição do corpo com respeito à posição de equilíbrio, definimos o desvio médio $\sigma_x(t)$ num período do movimento $\sigma_x(t)$ como
$$
\sigma_x(t) = \int_{t + T}^T x^2(t) \dd{t}
$$

Quanto vale $\sigma_x(4T)$? É uma conta braba pra fazer exata...

Qual o valor da razão $\displaystyle{\frac{\sigma_x(60T)}{\sigma_x(0)}}$? $\sigma_x$ é proporcional a $e^{-\frac{b}{2m}t}$, logo a resposta é $e^{-\frac{b}{m} \cdot 30 T}$.

\end{document}
