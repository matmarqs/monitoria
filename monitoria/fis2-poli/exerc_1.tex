\documentclass[a4paper,10pt]{article}
%\usepackage{mathtools}
\usepackage{amsthm}     % for definitions and theorems
\usepackage[many]{tcolorbox}    % boxes around definitions and theorems
%\usepackage{amsmath}
%\usepackage{nccmath}
\usepackage{amssymb}    % \ltimes, semi-direct product
%\usepackage{etoolbox}   % for start of Chapter
%\usepackage{amsfonts}
\usepackage{physics}    % for all Physics related
\usepackage{dsfont}     % for the identity matrix symbol \1
%\usepackage{mathrsfs}
\usepackage[notextcomp]{stix}   % font package and some symbols like filled square
%\usepackage{MnSymbol}   % symbols font package

\usepackage{titling}
\usepackage{indentfirst}

\usepackage{bm}
\usepackage[dvipsnames]{xcolor}
\usepackage{cancel}
\usepackage{enumitem}

\usepackage{xurl}
%\usepackage[colorlinks=true]{hyperref} % links have colors
\usepackage{hyperref}  % no colors

\usepackage{float}
\usepackage{graphicx}
\usepackage{subcaption}
%\usepackage{tikz}

\usepackage{ctable}     % tabelas
\renewcommand{\P}{\phantom{+}}  % empty space to indent things
\usepackage{multirow}
\usepackage{tabulary}

%%%%%%%%%%%%%%%%%%%%%%%%%%%%%%%%%%%%%%%%%%%%%%%%%%%

\newcommand{\eps}{\epsilon}
\newcommand{\vphi}{\varphi}
\newcommand{\cte}{\text{cte}}

\newcommand{\N}{{\mathbb{N}}}
\newcommand{\Z}{{\mathbb{Z}}}
%\newcommand{\Q}{{\mathbb{Q}}}
\newcommand{\C}{{\mathbb{C}}}
\renewcommand{\S}{{\hat{S}}}
%\renewcommand{\H}{\s{H}}

\renewcommand{\a}{{\vb{a}}}
\renewcommand{\b}{{\vb{b}}}
\renewcommand{\d}{{\dagger}}
\newcommand{\up}{{\uparrow}}
\newcommand{\down}{{\downarrow}}
\newcommand{\hc}{{\text{h.c.}}}

\newcommand{\ihat}{\bm{\hat{\imath}}}
\newcommand{\jhat}{\bm{\hat{\jmath}}}
\newcommand{\khat}{\bm{\hat{k}}}

\newcommand{\0}{{\vb{0}}}
\newcommand{\1}{\mathds{1}}
\newcommand{\E}{{\vb{E}}}
\newcommand{\B}{{\vb{B}}}
\renewcommand{\u}{{\vb{u}}}
\renewcommand{\v}{{\vb{v}}}
\renewcommand{\r}{{\vb{r}}}
\newcommand{\R}{{\vb{R}}}
\newcommand{\Q}{{\vb{Q}}}
\newcommand{\G}{{\vb{G}}}
\newcommand{\g}{{\vb{g}}}
\renewcommand{\k}{{\vb{k}}}
\newcommand{\K}{{\vb{K}}}
\newcommand{\p}{{\vb{p}}}
\newcommand{\q}{{\vb{q}}}
\newcommand{\F}{{\vb{F}}}
\renewcommand{\t}{{\vb{t}}}
\newcommand{\vtau}{{\bm{\tau}}}
\newcommand{\vdelta}{{\bm{\delta}}}

% COLORED SYMMETRY ELEMENTS
\newcommand{\Ct}{{\textcolor{Cyan}{C_3}}}
\newcommand{\Ctn}[1]{{\textcolor{Cyan}{C_3^{\textcolor{black}{#1}}}}}
\newcommand{\Cs}{{\textcolor{ForestGreen}{C_6}}}
\newcommand{\Csn}[1]{{\textcolor{ForestGreen}{C_6^{\textcolor{black}{#1}}}}}
\newcommand{\sd}{{\textcolor{RoyalBlue}{\sigma_d}}}
\newcommand{\sdn}[1]{{\textcolor{RoyalBlue}{\sigma_d^{\textcolor{black}{#1}}}}}
\newcommand{\sdp}{{\textcolor{RoyalBlue}{\sigma_d'}}}
\newcommand{\sdpp}{{\textcolor{RoyalBlue}{\sigma_d''}}}
\newcommand{\sv}{{\textcolor{Orange}{\sigma_v}}}
\newcommand{\svn}[1]{{\textcolor{Orange}{\sigma_v^{\textcolor{black}{#1}}}}}
\newcommand{\svp}{{\textcolor{Orange}{\sigma_v'}}}
\newcommand{\svpp}{{\textcolor{Orange}{\sigma_v''}}}

\newcommand{\GL}{{\text{GL}}}
\newcommand{\U}{{\text{U}}}

\newcommand{\s}{\sigma}
%\newcommand{\prodint}[2]{\left\langle #1 , #2 \right\rangle}
\newcommand{\cc}[1]{\overline{#1}}
\newcommand{\Eval}[3]{\eval{\left( #1 \right)}_{#2}^{#3}}
\newcommand{\sg}[2]{\{ #1 \mid #2 \}}
\renewcommand{\AA}{{\mathring{\text{A}}}}
\newcommand{\I}{{\mathbb{I}}}
\newcommand{\bP}{{\mathbb{P}}}
\newcommand{\bQ}{{\mathbb{Q}}}

\newcommand{\unit}[1]{\; \mathrm{#1}}

\newcommand{\n}{\medskip}
\newcommand{\e}{\quad \mathrm{and} \quad}
\newcommand{\ou}{\quad \mathrm{or} \quad}
\newcommand{\virg}{\, , \;}
\newcommand{\ptodo}{\forall \,}
\renewcommand{\implies}{\; \Rightarrow \;}
%\newcommand{\eqname}[1]{\tag*{#1}} % Tag equation with name

%\setlength{\droptitle}{-7em}   % título um pouco mais em cima na página
%\makeatletter
%\patchcmd{\chapter}{\if@openright\cleardoublepage\else\clearpage\fi}{}{}{}  % start 'Chapter' at the same page. needs package etoolbox
%\makeatother

%% Theorems, definitions, proofs
\theoremstyle{definition}

%%% defining my own colors %%%
\definecolor{my-blue}{HTML}{f2f4ff}
\definecolor{my-green}{HTML}{f5fcf6}    % a little better: green!5!white
\definecolor{my-cyan}{HTML}{f2fffe}
\definecolor{my-yellow}{HTML}{fffbed}
\definecolor{my-green2}{HTML}{efffdb}

%%% alternative colors %%%
\definecolor{my-pink}{HTML}{fff2f7}
\definecolor{my-teal}{HTML}{ebfffc}

\newtheorem{definition}{Definition}[section]
\tcolorboxenvironment{definition}{
  colback=my-blue,
  %colback=blue!5!white,
  boxrule=0.1pt,
  boxsep=1pt,
  left=2pt,right=2pt,top=2pt,bottom=2pt,
  oversize=2pt,
  sharp corners,
  before skip=\topsep,
  after skip=\topsep,
}

\newtheorem{theorem}{Theorem}[section]
\tcolorboxenvironment{theorem}{
  colback=my-yellow,
  %colback=yellow!22!white!95!black,
  boxrule=0.1pt,
  boxsep=1pt,
  left=2pt,right=2pt,top=2pt,bottom=2pt,
  oversize=2pt,
  sharp corners,
  before skip=\topsep,
  after skip=\topsep,
}

\newtheorem{corollary}{Corollary}[section]
\tcolorboxenvironment{corollary}{
  colback=my-green2,
  boxrule=0.1pt,
  boxsep=1pt,
  left=2pt,right=2pt,top=2pt,bottom=2pt,
  oversize=2pt,
  sharp corners,
  before skip=\topsep,
  after skip=\topsep,
}

\newtheorem{lemma}{Lemma}[section]
\tcolorboxenvironment{lemma}{
  colback=my-cyan,
  boxrule=0.1pt,
  boxsep=1pt,
  left=2pt,right=2pt,top=2pt,bottom=2pt,
  oversize=2pt,
  sharp corners,
  before skip=\topsep,
  after skip=\topsep,
}

\newtheorem{example}{Example}[section]
\tcolorboxenvironment{example}{
  %colback=my-green,
  colback=green!5!white,
  boxrule=0.1pt,
  boxsep=1pt,
  left=2pt,right=2pt,top=2pt,bottom=2pt,
  oversize=2pt,
  sharp corners,
  before skip=\topsep,
  after skip=\topsep,
}


\title{\Huge{\textbf{Exercícios 1}}}
\author{Mateus Marques}

\begin{document}

\maketitle

\section{}

Qual sistema não é adequadamente descrito como um oscilador amortecido?

\begin{itemize}
\item Um pêndulo na atmosfera.
\item Um circuito RLC.
\item Um bola de bilhar rolando sobre a mesa de sinuca.
\item Uma corda de violão vibrando até parar.
\end{itemize}


\section{}

A diferença de potencial entre as extremidades de um capacitor, resistor e indutor são, respectivamente, $V_C = \frac{Q}{C}$, $V_R = R I$ e $V_L = L \dv{I}{t}$.

É montado um circuito RLC, ou seja, os três componentes capacitor (capacitância $C$), resistor (resistência $R$) e indutor (indutância $L$) conectados em série.

As valores nominais de resistência e indutância dos componentes são $R = 1 \unit{\Omega}$, $L = 40 \unit{mH}$, $C = 1 \unit{\mu F}$.

No instante inicial $t = 0$, não há corrente pelo circuito e a voltagem entre as extremidades do capacitor é $V_0 = 0.1 \unit{V}$.

$$
\ddot{Q} + \frac{R}{L} \dot{Q} + \frac{Q}{LC} = 0
$$
$$
\ddot{V} + \frac{R}{L} \dot{V} + \frac{V}{LC} = 0
$$
$$
\ddot{V} + \gamma \dot{V} + \omega_0^2 V = 0,
$$
onde $\gamma = R/L$ e $\omega_0 = 1/\sqrt{LC}$.

Subamortecimento é o caso $\gamma/2 < \omega_0$, ou seja, $R < 2 \sqrt{L/C}$.

Qual o módulo da corrente máxima que o circuito atinge depois do instante inicial?
$$
Q(t) = A \, e^{-\frac{\gamma}{2} t} \cos(\omega t + \phi), \quad \omega = \sqrt{\omega_0^2 - (\gamma/2)^2}.
$$
$$
I(t) = - \frac{\gamma}{2} A \, e^{-\frac{\gamma}{2} t} \cos(\omega t + \phi) - \omega A \, e^{-\frac{\gamma}{2} t} \sin(\omega t + \phi).
$$
$$
I(0) = 0 = - \frac{\gamma}{2} A \, \cos(\phi) - \omega A \, \sin(\phi) \implies
$$
$$
\tan(\phi) = -\frac{\gamma}{2 \omega}.
$$
$$
\cos(\phi) = \frac{1}{\sqrt{1+\tan^2\phi}} = \frac{\omega}{\sqrt{\omega^2 + (\gamma/2)^2}} = \frac{\omega}{\omega_0}.
$$
$$
Q_0 = CV_0 = A \cos(\phi) \implies A = C V_0 \frac{\omega_0}{\omega}.
$$

Qual a carga no capacitor depois de um período de oscilação?
$$
Q\qty(\frac{2\pi}{\omega}) = C V_0 \exp(-\frac{\pi \gamma}{\omega}).
$$

\section{}

Uma corpo de massa $m = 50 \unit{g}$ que oscila preso a uma mola de constante elástica $k = 1.25 \unit{N \cdot m^{-1}}$ é sujeito a uma força de resistência do ar proporcional à sua velocidade, sendo $b = 3 \unit{g \cdot s^{-1}}$ a constante de proporcionalidade. No instante inicial $t = 0$ a mola está em sua posição de equilíbrio e o corpo tem velocidade inicial não-nula $v_0 = 1 \unit{m \cdot s^{-1}}$.

Qual o período $T$ de oscilação do movimento? $T = 2 \pi / \omega$, $\omega = \sqrt{\qty(\sqrt{\frac{k}{m}})^2 - \qty(\frac{b}{2m})^2}$.

Sendo $x(t)$ a posição do corpo com respeito à posição de equilíbrio, definimos o desvio médio $\sigma_x(t)$ num período do movimento $\sigma_x(t)$ como
$$
\sigma_x(t) = \int_{t + T}^T x^2(t) \dd{t}
$$

Quanto vale $\sigma_x(4T)$? É uma conta braba pra fazer exata...

Qual o valor da razão $\displaystyle{\frac{\sigma_x(60T)}{\sigma_x(0)}}$? $\sigma_x$ é proporcional a $e^{-\frac{b}{2m}t}$, logo a resposta é $e^{-\frac{b}{m} \cdot 30 T}$.

\end{document}
