\documentclass[a4paper,10pt]{article}
\usepackage[brazilian]{babel}
\usepackage[left=2.5cm,right=2.5cm,top=3cm,bottom=2.5cm]{geometry}
\usepackage{mathtools}
\usepackage{amsthm}
\usepackage{amsmath}
%\usepackage{nccmath}
\usepackage{amssymb}
\usepackage{amsfonts}
\usepackage{physics}
%\usepackage{dsfont}
%\usepackage{mathrsfs}

\usepackage{titling}
\usepackage{indentfirst}

\usepackage{bm}
\usepackage[dvipsnames]{xcolor}
\usepackage{cancel}

\usepackage{xurl}
\usepackage[colorlinks=true]{hyperref}

\usepackage{float}
\usepackage{graphicx}
%\usepackage{tikz}
\usepackage{caption}
\usepackage{subcaption}

%%%%%%%%%%%%%%%%%%%%%%%%%%%%%%%%%%%%%%%%%%%%%%%%%%%

\newcommand{\eps}{\epsilon}
\newcommand{\vphi}{\varphi}
\newcommand{\cte}{\text{cte}}

\newcommand{\N}{\mathbb{N}}
\newcommand{\Z}{\mathbb{Z}}
\newcommand{\Q}{\mathbb{Q}}
\newcommand{\R}{\vb{R}}
\newcommand{\C}{\mathbb{C}}
\renewcommand{\S}{\hat{S}}
%\renewcommand{\H}{\s{H}}

\renewcommand{\a}{\vb{a}}
\newcommand{\nn}{\hat{n}}
\renewcommand{\d}{\dagger}
\newcommand{\up}{\uparrow}
\newcommand{\down}{\downarrow}

\newcommand{\0}{\vb{0}}
%\newcommand{\1}{\mathds{1}}
\newcommand{\E}{\vb{E}}
\newcommand{\B}{\vb{B}}
\renewcommand{\v}{\vb{v}}
\renewcommand{\r}{\vb{r}}
\renewcommand{\k}{\vb{k}}
\newcommand{\p}{\vb{p}}
\newcommand{\q}{\vb{q}}
\newcommand{\F}{\vb{F}}

\newcommand{\s}{\sigma}
%\newcommand{\prodint}[2]{\left\langle #1 , #2 \right\rangle}
\newcommand{\cc}[1]{\overline{#1}}
\newcommand{\Eval}[3]{\eval{\left( #1 \right)}_{#2}^{#3}}

\newcommand{\unit}[1]{\; \mathrm{#1}}

\newcommand{\n}{\medskip}
\newcommand{\e}{\quad \mathrm{e} \quad}
\newcommand{\ou}{\quad \mathrm{ou} \quad}
\newcommand{\virg}{\, , \;}
\newcommand{\ptodo}{\forall \,}
\renewcommand{\implies}{\; \Rightarrow \;}
%\newcommand{\eqname}[1]{\tag*{#1}} % Tag equation with name

\setlength{\droptitle}{-7em}

\theoremstyle{plain}
\newtheorem{theorem}{Teorema}[section]
%\newtheorem{defi}[theorem]{Definição}
\newtheorem{lemma}[theorem]{Lema}
%\newtheorem{corol}[theorem]{Corolário}
%\newtheorem{prop}[theorem]{Proposição}
%\newtheorem{example}{Exemplo}
%
%\newtheorem{inneraxiom}{Axioma}
%\newenvironment{axioma}[1]
%  {\renewcommand\theinneraxiom{#1}\inneraxiom}
%  {\endinneraxiom}
%
%\newtheorem{innerpostulado}{Postulado}
%\newenvironment{postulado}[1]
%  {\renewcommand\theinnerpostulado{#1}\innerpostulado}
%  {\endinnerpostulado}
%
%\newtheorem{innerexercise}{Exercício}
%\newenvironment{exercise}[1]
%  {\renewcommand\theinnerexercise{#1}\innerexercise}
%  {\endinnerexercise}
%
%\newtheorem{innerthm}{Teorema}
%\newenvironment{teorema}[1]
%  {\renewcommand\theinnerthm{#1}\innerthm}
%  {\endinnerthm}
%
\newtheorem{innerlema}{Lema}
\newenvironment{lema}[1]
  {\renewcommand\theinnerlema{#1}\innerlema}
  {\endinnerlema}
%
%\theoremstyle{remark}
%\newtheorem*{hint}{Dica}
%\newtheorem*{notation}{Notação}
%\newtheorem*{obs}{Observação}


\title{\Huge{\textbf{Exercícios 3}}}
\author{Mateus Marques}

\begin{document}

\maketitle

\section{}

Um pêndulo simples de comprimento $L$ é colocado para oscilar no instante inicial $t = 0$ em um meio viscoso sobre uma força periódica $F(t) = 3 F_0 \cos(\Omega t) + 4 F_0 \sin(\Omega t)$, sendo que a massa $m$ em sua extremidade sofre uma força de resistência do ar da forma $F_{\text{ar}} = - \rho \dv{\theta}{t}$. Considerando que se passou muito tempo depois do instante inicial, de maneira a entrarmos no regime estacionário, qual é o maior deslocamento com relação à posição de equilíbrio?

\n

\textbf{SOLUÇÃO}: Podemos reescrever a força $F(t)$ como
$$
F(t) = 5 F_0 \qty[\frac{3}{5} \cos(\Omega t) - \qty(\frac{4}{5}) \sin(\Omega t)] = 5 F_0 \cos(\Omega t + \Delta),
$$
onde a fase $\Delta = \tan(-4/3)$. Como a fase $\Delta$ é irrelevante para a amplitude máxima, esta é
$$
A = \frac{5 F_0}{m \sqrt{(\omega_0^2 - \omega^2)^2 + \gamma^2 \omega^2}}.
$$


\n



\section{}


Uma massa $m$ presa a mola de constante elástica $k$ oscila num meio viscoso com coeficiente $\rho$ sob ação de uma força externa $F(t) = F_0 \cos(\Omega t)$. Considerando que já se passou tempo suficiente para o sistema atingir o regime estacionário, qual
\begin{itemize}
\item A maior taxa de variação de energia no tempo ($\dv{E}{t}$)?

\textbf{SOLUÇÃO:} Equação (4.5.5) do Moysés.
$$
\dv{E}{t} = m \omega (\omega^2 - \omega_0^2) A^2 \cdot \frac{1}{2} \sin[2(\omega t+\vphi)].
$$
Colocando o seno igual a 1, a maior taxa de variação de energia no tempo é $m\omega(\omega^2-\omega_0^2) A^2$.

\item Supondo que podemos ajustar a frequência $\omega$ de oscilação da força $F(t)$, qual a maior potência média realizada pela força, para qual $\omega$ ela ocorre?

\textbf{SOLUÇÃO:} Equação (4.5.9) do Moysés.
$$
\cc{P}(\omega) = \frac{\gamma F_0^2 \omega^2}{2m \qty[(\omega^2-\omega_0^2)^2 + \gamma^2\omega^2]}.
$$
Derivar e igualar a zero obtém-se a resposta.

\end{itemize}


\end{document}
