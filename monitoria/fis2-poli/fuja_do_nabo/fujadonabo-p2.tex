\documentclass[a4paper,12pt]{article}
%\usepackage{mathtools}
\usepackage{amsthm}     % for definitions and theorems
\usepackage[many]{tcolorbox}    % boxes around definitions and theorems
%\usepackage{amsmath}
%\usepackage{nccmath}
\usepackage{amssymb}    % \ltimes, semi-direct product
%\usepackage{etoolbox}   % for start of Chapter
%\usepackage{amsfonts}
\usepackage{physics}    % for all Physics related
\usepackage{dsfont}     % for the identity matrix symbol \1
%\usepackage{mathrsfs}
\usepackage[notextcomp]{stix}   % font package and some symbols like filled square
%\usepackage{MnSymbol}   % symbols font package

\usepackage{titling}
\usepackage{indentfirst}

\usepackage{bm}
\usepackage[dvipsnames]{xcolor}
\usepackage{cancel}
\usepackage{enumitem}

\usepackage{xurl}
%\usepackage[colorlinks=true]{hyperref} % links have colors
\usepackage{hyperref}  % no colors

\usepackage{float}
\usepackage{graphicx}
\usepackage{subcaption}
%\usepackage{tikz}

\usepackage{ctable}     % tabelas
\renewcommand{\P}{\phantom{+}}  % empty space to indent things
\usepackage{multirow}
\usepackage{tabulary}

%%%%%%%%%%%%%%%%%%%%%%%%%%%%%%%%%%%%%%%%%%%%%%%%%%%

\newcommand{\eps}{\epsilon}
\newcommand{\vphi}{\varphi}
\newcommand{\cte}{\text{cte}}

\newcommand{\N}{{\mathbb{N}}}
\newcommand{\Z}{{\mathbb{Z}}}
%\newcommand{\Q}{{\mathbb{Q}}}
\newcommand{\C}{{\mathbb{C}}}
\renewcommand{\S}{{\hat{S}}}
%\renewcommand{\H}{\s{H}}

\renewcommand{\a}{{\vb{a}}}
\renewcommand{\b}{{\vb{b}}}
\renewcommand{\d}{{\dagger}}
\newcommand{\up}{{\uparrow}}
\newcommand{\down}{{\downarrow}}
\newcommand{\hc}{{\text{h.c.}}}

\newcommand{\ihat}{\bm{\hat{\imath}}}
\newcommand{\jhat}{\bm{\hat{\jmath}}}
\newcommand{\khat}{\bm{\hat{k}}}

\newcommand{\0}{{\vb{0}}}
\newcommand{\1}{\mathds{1}}
\newcommand{\E}{{\vb{E}}}
\newcommand{\B}{{\vb{B}}}
\renewcommand{\u}{{\vb{u}}}
\renewcommand{\v}{{\vb{v}}}
\renewcommand{\r}{{\vb{r}}}
\newcommand{\R}{{\vb{R}}}
\newcommand{\Q}{{\vb{Q}}}
\newcommand{\G}{{\vb{G}}}
\newcommand{\g}{{\vb{g}}}
\renewcommand{\k}{{\vb{k}}}
\newcommand{\K}{{\vb{K}}}
\newcommand{\p}{{\vb{p}}}
\newcommand{\q}{{\vb{q}}}
\newcommand{\F}{{\vb{F}}}
\renewcommand{\t}{{\vb{t}}}
\newcommand{\vtau}{{\bm{\tau}}}
\newcommand{\vdelta}{{\bm{\delta}}}

% COLORED SYMMETRY ELEMENTS
\newcommand{\Ct}{{\textcolor{Cyan}{C_3}}}
\newcommand{\Ctn}[1]{{\textcolor{Cyan}{C_3^{\textcolor{black}{#1}}}}}
\newcommand{\Cs}{{\textcolor{ForestGreen}{C_6}}}
\newcommand{\Csn}[1]{{\textcolor{ForestGreen}{C_6^{\textcolor{black}{#1}}}}}
\newcommand{\sd}{{\textcolor{RoyalBlue}{\sigma_d}}}
\newcommand{\sdn}[1]{{\textcolor{RoyalBlue}{\sigma_d^{\textcolor{black}{#1}}}}}
\newcommand{\sdp}{{\textcolor{RoyalBlue}{\sigma_d'}}}
\newcommand{\sdpp}{{\textcolor{RoyalBlue}{\sigma_d''}}}
\newcommand{\sv}{{\textcolor{Orange}{\sigma_v}}}
\newcommand{\svn}[1]{{\textcolor{Orange}{\sigma_v^{\textcolor{black}{#1}}}}}
\newcommand{\svp}{{\textcolor{Orange}{\sigma_v'}}}
\newcommand{\svpp}{{\textcolor{Orange}{\sigma_v''}}}

\newcommand{\GL}{{\text{GL}}}
\newcommand{\U}{{\text{U}}}

\newcommand{\s}{\sigma}
%\newcommand{\prodint}[2]{\left\langle #1 , #2 \right\rangle}
\newcommand{\cc}[1]{\overline{#1}}
\newcommand{\Eval}[3]{\eval{\left( #1 \right)}_{#2}^{#3}}
\newcommand{\sg}[2]{\{ #1 \mid #2 \}}
\renewcommand{\AA}{{\mathring{\text{A}}}}
\newcommand{\I}{{\mathbb{I}}}
\newcommand{\bP}{{\mathbb{P}}}
\newcommand{\bQ}{{\mathbb{Q}}}

\newcommand{\unit}[1]{\; \mathrm{#1}}

\newcommand{\n}{\medskip}
\newcommand{\e}{\quad \mathrm{and} \quad}
\newcommand{\ou}{\quad \mathrm{or} \quad}
\newcommand{\virg}{\, , \;}
\newcommand{\ptodo}{\forall \,}
\renewcommand{\implies}{\; \Rightarrow \;}
%\newcommand{\eqname}[1]{\tag*{#1}} % Tag equation with name

%\setlength{\droptitle}{-7em}   % título um pouco mais em cima na página
%\makeatletter
%\patchcmd{\chapter}{\if@openright\cleardoublepage\else\clearpage\fi}{}{}{}  % start 'Chapter' at the same page. needs package etoolbox
%\makeatother

%% Theorems, definitions, proofs
\theoremstyle{definition}

%%% defining my own colors %%%
\definecolor{my-blue}{HTML}{f2f4ff}
\definecolor{my-green}{HTML}{f5fcf6}    % a little better: green!5!white
\definecolor{my-cyan}{HTML}{f2fffe}
\definecolor{my-yellow}{HTML}{fffbed}
\definecolor{my-green2}{HTML}{efffdb}

%%% alternative colors %%%
\definecolor{my-pink}{HTML}{fff2f7}
\definecolor{my-teal}{HTML}{ebfffc}

\newtheorem{definition}{Definition}[section]
\tcolorboxenvironment{definition}{
  colback=my-blue,
  %colback=blue!5!white,
  boxrule=0.1pt,
  boxsep=1pt,
  left=2pt,right=2pt,top=2pt,bottom=2pt,
  oversize=2pt,
  sharp corners,
  before skip=\topsep,
  after skip=\topsep,
}

\newtheorem{theorem}{Theorem}[section]
\tcolorboxenvironment{theorem}{
  colback=my-yellow,
  %colback=yellow!22!white!95!black,
  boxrule=0.1pt,
  boxsep=1pt,
  left=2pt,right=2pt,top=2pt,bottom=2pt,
  oversize=2pt,
  sharp corners,
  before skip=\topsep,
  after skip=\topsep,
}

\newtheorem{corollary}{Corollary}[section]
\tcolorboxenvironment{corollary}{
  colback=my-green2,
  boxrule=0.1pt,
  boxsep=1pt,
  left=2pt,right=2pt,top=2pt,bottom=2pt,
  oversize=2pt,
  sharp corners,
  before skip=\topsep,
  after skip=\topsep,
}

\newtheorem{lemma}{Lemma}[section]
\tcolorboxenvironment{lemma}{
  colback=my-cyan,
  boxrule=0.1pt,
  boxsep=1pt,
  left=2pt,right=2pt,top=2pt,bottom=2pt,
  oversize=2pt,
  sharp corners,
  before skip=\topsep,
  after skip=\topsep,
}

\newtheorem{example}{Example}[section]
\tcolorboxenvironment{example}{
  %colback=my-green,
  colback=green!5!white,
  boxrule=0.1pt,
  boxsep=1pt,
  left=2pt,right=2pt,top=2pt,bottom=2pt,
  oversize=2pt,
  sharp corners,
  before skip=\topsep,
  after skip=\topsep,
}


\title{\Huge{\textbf{Fuja do Nabo - P2}}}
\date{\vspace{-10ex}}

\begin{document}

\maketitle

\section{Questões 7-10 da P1 de 2022}

Um corpo preso na extremidade de uma mola e imerso em um
fluido viscoso, executa um movimento harmˆonico amortecido no regime subcr´ıtico conforme a equa¸c˜ao
hor´aria:
$$
x(t) = 4 e^{-3t} \cos(4t + \phi) \unit{m}.
$$
O corpo tem massa M = 1 kg e possui velocidade nula no instante de tempo t = 0 s. Sabendo que
a equa¸c˜ao diferencial deste movimento é
$$
\ddot{x} + \gamma \dot{x} + \omega_0^2x = 0.
$$
Determine:

\begin{enumerate}
\item A frequˆencia natural de oscila¸c˜ao ω0 e a constante de viscosidade ρ = γM:

\item A fase inicial ϕ do movimento:

\item O tempo necess´ario para que a amplitude m´axima do movimento se reduza `a metade do valor inicial:

\item Supondo que seja poss´ıvel variar apenas γ, qual deveria ser seu valor para o caso
do amortecimento cr´ıtico? Da mesma forma, supondo que seja poss´ıvel variar apenas ω0, qual
deveria ser seu valor para o caso do amortecimento supercr´ıtico?
\end{enumerate}

\section{Questões 1-3 da P2 de 2022}

Um bloco de massa m = 2 kg, preso a duas molas idˆenticas, com
comprimentos naturais iguais a l0 = 0,5 m e constantes el´asticas iguais a k = 400 N/m, est˜ao dentro
de um recipiente com um l´ıquido, cujo coeficiente de atrito viscoso vale ρ = 80 kg/s. Uma for¸ca
harmˆonica de amplitude F0 = 100 N e frequˆencia angular ω = 10 rad/s atua sobre o bloco ao
longo da dire¸c˜ao horizontal. O atrito entre o bloco e o fundo do recipiente ´e desprez´ıvel. Usando o
referencial da figura, determine:
\begin{figure}[H]
\centering
\includegraphics[width=0.6\linewidth]{fig/duas_molas.png}
\label{fig:duas_molas}
\end{figure}

\begin{enumerate}
\item A express˜ao da for¸ca el´astica que as molas exercem sobre o corpo, pode ser escrita
como:

\item A frequˆencia natural de oscila¸c˜ao, ω0, do sistema vale:

\item A solu¸c˜ao da equa¸c˜ao de movimento no regime estacion´ario, em fun¸c˜ao de t,
considerando xeq=0 m:
\end{enumerate}


\section{Questões 1-3 da REC de 2022}

O gr´afico de x(t), mostrado na figura abaixo, representa a equa¸c˜ao
hor´aria de um oscilador criticamente amortecido, para um sistema composto de um corpo de massa
m = 1, 0 kg preso a uma mola de constante el´astica k e imerso em um l´ıquido viscoso, de coeficiente
de resistˆencia viscosa ρ. A equa¸c˜ao hor´aria pode ser escrita como x(t) = e− γ
2 t(A + Bt).
\begin{figure}[H]
\centering
\includegraphics[width=0.6\linewidth]{fig/grafico_rec.png}
\label{fig:grafico_rec}
\end{figure}

\begin{enumerate}
\item Determine os valores de A e B.

\item Determine o coeficiente de resistˆencia viscosa ρ e a constante el´astica k da mola.

\item Determine o valor da velocidade inicial v0 do oscilador.
\end{enumerate}


\section{Questões 2-5 da SUB de 2022}

O gr´afico abaixo representa a equa¸c˜ao hor´aria x(t) de um oscilador,
para um sistema composto por um bloco de massa m = 1 kg preso a uma mola de constante el´astica
k que est´a na posi¸c˜ao horizontal. Este sistema est´a imerso em um l´ıquido viscoso com coeficiente
de resistˆencia ρ = 0, 4 Ns/m e cuja for¸ca de resistˆencia ´e proporcional `a velocidade do corpo que se
movimenta em seu interior. Dado que a velocidade inicial do bloco ´e v0 = −2 m/s, considerando o
intervalo de tempo mostrado no gr´afico, responda:

\begin{figure}[H]
\centering
\includegraphics[width=0.6\linewidth]{fig/grafico_sub}
\label{fig:grafico_sub}
\end{figure}

\begin{enumerate}
\item Dado o movimento do bloco no sistema descrito acima, qual a equa¸c˜ao hor´aria que
descreve este sistema?

\item Determine a constante el´astica da mola k.

\item Qual a velocidade do bloco no instante t = 3 s

\item Suponha agora que este mesmo sistema seja retirado do l´ıquido viscoso e passe a
oscilar num meio sem resistˆencia. Neste novo regime, o deslocamento m´aximo ´e xmax=10 m e a
constante de fase ´e nula. Qual ser´a a nova equa¸c˜ao hor´aria do sistema?
\end{enumerate}





\end{document}
