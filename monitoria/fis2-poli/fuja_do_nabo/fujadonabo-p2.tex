\documentclass[a4paper,10pt]{article}
%\usepackage{mathtools}
\usepackage{amsthm}     % for definitions and theorems
\usepackage[many]{tcolorbox}    % boxes around definitions and theorems
%\usepackage{amsmath}
%\usepackage{nccmath}
\usepackage{amssymb}    % \ltimes, semi-direct product
%\usepackage{etoolbox}   % for start of Chapter
%\usepackage{amsfonts}
\usepackage{physics}    % for all Physics related
\usepackage{dsfont}     % for the identity matrix symbol \1
%\usepackage{mathrsfs}
\usepackage[notextcomp]{stix}   % font package and some symbols like filled square
%\usepackage{MnSymbol}   % symbols font package

\usepackage{titling}
\usepackage{indentfirst}

\usepackage{bm}
\usepackage[dvipsnames]{xcolor}
\usepackage{cancel}
\usepackage{enumitem}

\usepackage{xurl}
%\usepackage[colorlinks=true]{hyperref} % links have colors
\usepackage{hyperref}  % no colors

\usepackage{float}
\usepackage{graphicx}
\usepackage{subcaption}
%\usepackage{tikz}

\usepackage{ctable}     % tabelas
\renewcommand{\P}{\phantom{+}}  % empty space to indent things
\usepackage{multirow}
\usepackage{tabulary}

%%%%%%%%%%%%%%%%%%%%%%%%%%%%%%%%%%%%%%%%%%%%%%%%%%%

\newcommand{\eps}{\epsilon}
\newcommand{\vphi}{\varphi}
\newcommand{\cte}{\text{cte}}

\newcommand{\N}{{\mathbb{N}}}
\newcommand{\Z}{{\mathbb{Z}}}
%\newcommand{\Q}{{\mathbb{Q}}}
\newcommand{\C}{{\mathbb{C}}}
\renewcommand{\S}{{\hat{S}}}
%\renewcommand{\H}{\s{H}}

\renewcommand{\a}{{\vb{a}}}
\renewcommand{\b}{{\vb{b}}}
\renewcommand{\d}{{\dagger}}
\newcommand{\up}{{\uparrow}}
\newcommand{\down}{{\downarrow}}
\newcommand{\hc}{{\text{h.c.}}}

\newcommand{\ihat}{\bm{\hat{\imath}}}
\newcommand{\jhat}{\bm{\hat{\jmath}}}
\newcommand{\khat}{\bm{\hat{k}}}

\newcommand{\0}{{\vb{0}}}
\newcommand{\1}{\mathds{1}}
\newcommand{\E}{{\vb{E}}}
\newcommand{\B}{{\vb{B}}}
\renewcommand{\u}{{\vb{u}}}
\renewcommand{\v}{{\vb{v}}}
\renewcommand{\r}{{\vb{r}}}
\newcommand{\R}{{\vb{R}}}
\newcommand{\Q}{{\vb{Q}}}
\newcommand{\G}{{\vb{G}}}
\newcommand{\g}{{\vb{g}}}
\renewcommand{\k}{{\vb{k}}}
\newcommand{\K}{{\vb{K}}}
\newcommand{\p}{{\vb{p}}}
\newcommand{\q}{{\vb{q}}}
\newcommand{\F}{{\vb{F}}}
\renewcommand{\t}{{\vb{t}}}
\newcommand{\vtau}{{\bm{\tau}}}
\newcommand{\vdelta}{{\bm{\delta}}}

% COLORED SYMMETRY ELEMENTS
\newcommand{\Ct}{{\textcolor{Cyan}{C_3}}}
\newcommand{\Ctn}[1]{{\textcolor{Cyan}{C_3^{\textcolor{black}{#1}}}}}
\newcommand{\Cs}{{\textcolor{ForestGreen}{C_6}}}
\newcommand{\Csn}[1]{{\textcolor{ForestGreen}{C_6^{\textcolor{black}{#1}}}}}
\newcommand{\sd}{{\textcolor{RoyalBlue}{\sigma_d}}}
\newcommand{\sdn}[1]{{\textcolor{RoyalBlue}{\sigma_d^{\textcolor{black}{#1}}}}}
\newcommand{\sdp}{{\textcolor{RoyalBlue}{\sigma_d'}}}
\newcommand{\sdpp}{{\textcolor{RoyalBlue}{\sigma_d''}}}
\newcommand{\sv}{{\textcolor{Orange}{\sigma_v}}}
\newcommand{\svn}[1]{{\textcolor{Orange}{\sigma_v^{\textcolor{black}{#1}}}}}
\newcommand{\svp}{{\textcolor{Orange}{\sigma_v'}}}
\newcommand{\svpp}{{\textcolor{Orange}{\sigma_v''}}}

\newcommand{\GL}{{\text{GL}}}
\newcommand{\U}{{\text{U}}}

\newcommand{\s}{\sigma}
%\newcommand{\prodint}[2]{\left\langle #1 , #2 \right\rangle}
\newcommand{\cc}[1]{\overline{#1}}
\newcommand{\Eval}[3]{\eval{\left( #1 \right)}_{#2}^{#3}}
\newcommand{\sg}[2]{\{ #1 \mid #2 \}}
\renewcommand{\AA}{{\mathring{\text{A}}}}
\newcommand{\I}{{\mathbb{I}}}
\newcommand{\bP}{{\mathbb{P}}}
\newcommand{\bQ}{{\mathbb{Q}}}

\newcommand{\unit}[1]{\; \mathrm{#1}}

\newcommand{\n}{\medskip}
\newcommand{\e}{\quad \mathrm{and} \quad}
\newcommand{\ou}{\quad \mathrm{or} \quad}
\newcommand{\virg}{\, , \;}
\newcommand{\ptodo}{\forall \,}
\renewcommand{\implies}{\; \Rightarrow \;}
%\newcommand{\eqname}[1]{\tag*{#1}} % Tag equation with name

%\setlength{\droptitle}{-7em}   % título um pouco mais em cima na página
%\makeatletter
%\patchcmd{\chapter}{\if@openright\cleardoublepage\else\clearpage\fi}{}{}{}  % start 'Chapter' at the same page. needs package etoolbox
%\makeatother

%% Theorems, definitions, proofs
\theoremstyle{definition}

%%% defining my own colors %%%
\definecolor{my-blue}{HTML}{f2f4ff}
\definecolor{my-green}{HTML}{f5fcf6}    % a little better: green!5!white
\definecolor{my-cyan}{HTML}{f2fffe}
\definecolor{my-yellow}{HTML}{fffbed}
\definecolor{my-green2}{HTML}{efffdb}

%%% alternative colors %%%
\definecolor{my-pink}{HTML}{fff2f7}
\definecolor{my-teal}{HTML}{ebfffc}

\newtheorem{definition}{Definition}[section]
\tcolorboxenvironment{definition}{
  colback=my-blue,
  %colback=blue!5!white,
  boxrule=0.1pt,
  boxsep=1pt,
  left=2pt,right=2pt,top=2pt,bottom=2pt,
  oversize=2pt,
  sharp corners,
  before skip=\topsep,
  after skip=\topsep,
}

\newtheorem{theorem}{Theorem}[section]
\tcolorboxenvironment{theorem}{
  colback=my-yellow,
  %colback=yellow!22!white!95!black,
  boxrule=0.1pt,
  boxsep=1pt,
  left=2pt,right=2pt,top=2pt,bottom=2pt,
  oversize=2pt,
  sharp corners,
  before skip=\topsep,
  after skip=\topsep,
}

\newtheorem{corollary}{Corollary}[section]
\tcolorboxenvironment{corollary}{
  colback=my-green2,
  boxrule=0.1pt,
  boxsep=1pt,
  left=2pt,right=2pt,top=2pt,bottom=2pt,
  oversize=2pt,
  sharp corners,
  before skip=\topsep,
  after skip=\topsep,
}

\newtheorem{lemma}{Lemma}[section]
\tcolorboxenvironment{lemma}{
  colback=my-cyan,
  boxrule=0.1pt,
  boxsep=1pt,
  left=2pt,right=2pt,top=2pt,bottom=2pt,
  oversize=2pt,
  sharp corners,
  before skip=\topsep,
  after skip=\topsep,
}

\newtheorem{example}{Example}[section]
\tcolorboxenvironment{example}{
  %colback=my-green,
  colback=green!5!white,
  boxrule=0.1pt,
  boxsep=1pt,
  left=2pt,right=2pt,top=2pt,bottom=2pt,
  oversize=2pt,
  sharp corners,
  before skip=\topsep,
  after skip=\topsep,
}


\usepackage[shortlabels]{enumitem}

\title{\Huge{\textbf{Fuja do Nabo - P2}}}
\date{\vspace{-10ex}}

\begin{document}

\maketitle

\subsection*{1. Questões 7-10 da P1 de 2022}

Um corpo preso na extremidade de uma mola e imerso em um
fluido viscoso, executa um movimento harmônico amortecido no regime subcrítico conforme a equação
horária:
$$
x(t) = 4 e^{-3t} \cos(4t + \phi) \unit{m}.
$$
O corpo tem massa $M=1 \unit{kg}$ e possui velocidade nula no instante de tempo $t = 0 \unit{s}$. Sabendo que
a equação diferencial deste movimento é
$$
\ddot{x} + \gamma \dot{x} + \omega_0^2x = 0.
$$
Determine:

\begin{enumerate}[(a)]
\item A frequência natural de oscilação $\omega_0$ e a constante de viscosidade $b = \gamma M$.

\item A fase inicial $\phi$ do movimento.

\item O tempo necessário para que a amplitude máxima do movimento se reduza à metade do valor inicial.

\item Supondo que seja possível variar apenas $\gamma$, qual deveria ser seu valor para o caso
do amortecimento crítico? Da mesma forma, supondo que seja possível variar apenas $\omega_0$, qual
deveria ser seu valor para o caso do amortecimento supercrítico?
\end{enumerate}

\n\n

\subsection*{2. Questões 1-3 da P2 de 2022}

Um bloco de massa $m = 2 \unit{kg}$, preso a duas molas idênticas, com
comprimentos naturais iguais a $l_0 = 0.5 \unit{m}$ e constantes elásticas iguais a $k = 400 \unit{N/m}$, estão dentro
de um recipiente com um líquido, cujo coeficiente de atrito viscoso vale $b = 80 \unit{kg/s}$. Uma força
harmônica de amplitude $F_0 = 100 \unit{N}$ e frequência angular $\omega = 10 \unit{rad/s}$ atua sobre o bloco ao
longo da direção horizontal. O atrito entre o bloco e o fundo do recipiente é desprezível. Usando o
referencial da figura, determine:
\begin{figure}[H]
\centering
\includegraphics[width=0.7\linewidth]{fig/duas_molas.png}
\label{fig:duas_molas}
\end{figure}

Determine:
\begin{enumerate}[(a)]
\item A expressão da força elástica que as molas exercem sobre o corpo.

\item A frequência natural de oscilação $\omega_0$.

\item A solução da equação de movimento no regime estacionário, em função de $t$,
considerando $x_{\text{eq}}=0 \unit{m}$.
\end{enumerate}


\subsection*{3. Questões 1-3 da REC de 2022}

O gráfico de $x(t)$, mostrado na figura abaixo, representa a equação
horária de um oscilador criticamente amortecido, para um sistema composto de um corpo de massa
$m = 1 \unit{kg}$ preso a uma mola de constante elástica $k$ e imerso em um líquido viscoso, de coeficiente
de resistência viscosa $b$. A equação horária pode ser escrita como $x(t) = e^{-\frac{\gamma}{2} t} (A + Bt)$.
\begin{figure}[H]
\centering
\includegraphics[width=0.45\linewidth]{fig/grafico_rec.png}
\label{fig:grafico_rec}
\end{figure}

\begin{enumerate}[(a)]
\item Determine os valores de $A$ e $B$.

\item Determine o coeficiente de resistência viscosa $b$ e a constante elástica $k$ da mola.

\item Determine o valor da velocidade inicial $v_0$ do oscilador.
\end{enumerate}


\subsection*{4. Questões 2-5 da SUB de 2022}

O gráfico abaixo representa a equação horária $x(t)$ de um oscilador,
para um sistema composto por um bloco de massa $m = 1 \unit{kg}$ preso a uma mola de constante elástica
$k$ que está na posição horizontal. Este sistema está imerso em um líquido viscoso com coeficiente
de resistência $b = 0.4 \unit{Ns/m}$ e cuja força de resistência é proporcional à velocidade do corpo que se
movimenta em seu interior. Dado que a velocidade inicial do bloco é $v_0 = −2 \unit{m/s}$, considerando o
intervalo de tempo mostrado no gráfico, responda:

\begin{figure}[H]
\centering
\includegraphics[width=0.39\linewidth]{fig/grafico_sub}
\label{fig:grafico_sub}
\end{figure}

\begin{enumerate}[(a)]
\item Dado o movimento do bloco no sistema descrito acima, qual a equação horária que
descreve este sistema?

\item Determine a constante elástica da mola $k$.

\item Qual a velocidade do bloco no instante $t = 3 \unit{s}$?

\item Suponha agora que este mesmo sistema seja retirado do líquido viscoso e passe a
oscilar num meio sem resistência. Neste novo regime, o deslocamento máximo é $x_{\text{max}}=10 \unit{m}$ e a
constante de fase é nula. Qual será a nova equação horária do sistema?
\end{enumerate}

\n

\subsection*{5. Questões 4-6 da P2 de 2022}

Considere um sistema mecânico de massa $m = 10 \unit{kg}$ no regime
de amortecimento fraco $(\gamma \ll \omega_0)$ na presença de uma força externa $F_0$, cuja curva de ressonância,
com $A_{\text{max}} = 2\unit{m}$, é apresentada na figura abaixo:
\begin{figure}[H]
\centering
\includegraphics[width=0.6\linewidth]{fig/grafico-A2-p2.png}
\label{fig:grafico-A2-p2}
\end{figure}

\begin{enumerate}[(a)]
\item Quanto vale a constante de viscosidade $b$?

\item Determine o fator de qualidade $Q$ do sistema.

\item Quanto vale a força externa $F_0$?
\end{enumerate}

\n

\subsection*{6. Questão 2 da 2\textsuperscript{\underline{a}} da Lista de Exercícios}

Um corpo de massa $m$ desliza sobre um plano horizontal sem atrito sujeito a três
forças: uma força elástica resultante da ação de uma mola de constante elástica
$k$, uma força devido à resistência viscosa do meio, caracterizada pela constante de
resistência viscosa $b$ e uma força externa periódica $F(t) = F_0 \cos(\Omega t)$, sendo $\Omega$ a
frequência externa.

\begin{enumerate}[(a)]
\item Escreva a equação diferencial que descreve o movimento do corpo e encontre a
sua solução estacionária.

\item Considerando que $m = 50 \unit{kg}$, $k = 5000 \unit{N/m}$, $F_0 = 50 \unit{N}$ e $b = 500 \unit{kg/s}$, calcule
a frequência natural do sistema e o seu fator de qualidade $Q$.

\item No regime estacionário, usando os valores do item anterior, determine o valor
de $\Omega$ para o qual a amplitude $A$ do movimento é máxima.

\item No regime estacionário, usando os valores do item (b), determine o valor da
amplitude máxima.
\end{enumerate}

\n

\subsection*{7. Questão 5 da REC de 2022}

Um oscilador unidimensional não amortecido, de massa $m = 0.50 \unit{kg}$ e frequência
própria $\omega_0 = 2.0 \unit{s^{-1}}$, move-se sobre um plano horizontal sob a ação de uma força externa não
periódica $F(t) = F_0 e^{-\beta t}$, com $F_0 = 40 \unit{N}$ e $\beta = 6.0 \unit{s^{-1}}$. Inicialmente o oscilador
encontra-se em repouso na posição de equilíbrio. Lembrando que a solução de uma equação diferencial não homogênea
é igual à solução da homogênea somada à solução particular (equação completa), determine a função
$x(t)$ que descreve o movimento do oscilador, com as condições iniciais acima.

\pagebreak

\section*{Gabarito}

\subsection*{1}

\begin{enumerate}[(a)]
\item $\omega_0 = 5 \unit{rad/s}$; $b = 6 \unit{kg/s}$.

\item $\phi = \arctan(-\frac{3}{4})$.

\item $t = \frac{\ln(2)}{3}$.

\item $\gamma = 10 \unit{s^{-1}}$; $\omega_0 < 3 \unit{rad/s}$.
\end{enumerate}

\subsection*{2}

\begin{enumerate}[(a)]
\item $F = -2k(x-l_0)$.

\item $\omega_0 = 20 \unit{rad/s}$.

\item $x(t) = 0.1 \cos[10t - \arctan(4/3)] \unit{m}$.
\end{enumerate}

\subsection*{3}

\begin{enumerate}[(a)]
\item $A = 0.5 \unit{m}$ e $B = -0.5 \unit{m/s}$.

\item $b = 1 \unit{kg/s}$ e $k = 0.25 \unit{N/m}$.

\item $v_0 = -0.75 \unit{m/s}$.
\end{enumerate}

\subsection*{4}

\begin{enumerate}[(a)]
\item $x(t) = 10 e^{-0.2 t} \cos(\pi t) \unit{m}$.

\item $k = (\pi^2 + 0.04) \unit{kg/s^2}$.

\item $v(t = 3 \unit{s}) = 2 \, e^{-0.6} \unit{m/s}$.

\item $x(t) = 10 \cos(\sqrt{\pi^2 + 0.04} \, t) \unit{m}$.
\end{enumerate}

\subsection*{5}

\begin{enumerate}[(a)]
\item $b = 20 \unit{kg/s}$.

\item $Q = 5$.

\item $F_0 = 400 \unit{N}$.
\end{enumerate}


\subsection*{6}

\begin{enumerate}[(a)]
\item $\ddot{x} + \gamma \ddot{x} + \omega_0^2 \dot{x} = \frac{F_0}{m} \cos(\Omega t)$;
$x(t) = A(\Omega) \cos(\Omega t + \phi)$, com
$$
A(\Omega) = \frac{F_0}{m \sqrt{(\omega_0^2-\Omega^2)^2 + \gamma^2\Omega^2}}
\e
\phi = - \arctan(\frac{\gamma \Omega}{\omega_0^2 - \Omega^2}).
$$

\item $\omega_0 = 10 \unit{s^{-1}}$, $Q = 1$.

\item $\Omega_{\text{max}} = 5\sqrt{2} \unit{s^{-1}}$.

\item $A_{\text{max}} = \frac{1}{50\sqrt{3}} \unit{m}$.
\end{enumerate}

\subsection*{7}

\begin{enumerate}[(a)]
\item $x(t) = 2 \qty[e^{-6t} - \cos(2t) + 3 \sin(2t)] \unit{m}$.
\end{enumerate}

\end{document}
