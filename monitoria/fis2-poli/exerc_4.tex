\documentclass[a4paper,10pt]{article}
%\usepackage{mathtools}
\usepackage{amsthm}     % for definitions and theorems
\usepackage[many]{tcolorbox}    % boxes around definitions and theorems
%\usepackage{amsmath}
%\usepackage{nccmath}
\usepackage{amssymb}    % \ltimes, semi-direct product
%\usepackage{etoolbox}   % for start of Chapter
%\usepackage{amsfonts}
\usepackage{physics}    % for all Physics related
\usepackage{dsfont}     % for the identity matrix symbol \1
%\usepackage{mathrsfs}
\usepackage[notextcomp]{stix}   % font package and some symbols like filled square
%\usepackage{MnSymbol}   % symbols font package

\usepackage{titling}
\usepackage{indentfirst}

\usepackage{bm}
\usepackage[dvipsnames]{xcolor}
\usepackage{cancel}
\usepackage{enumitem}

\usepackage{xurl}
%\usepackage[colorlinks=true]{hyperref} % links have colors
\usepackage{hyperref}  % no colors

\usepackage{float}
\usepackage{graphicx}
\usepackage{subcaption}
%\usepackage{tikz}

\usepackage{ctable}     % tabelas
\renewcommand{\P}{\phantom{+}}  % empty space to indent things
\usepackage{multirow}
\usepackage{tabulary}

%%%%%%%%%%%%%%%%%%%%%%%%%%%%%%%%%%%%%%%%%%%%%%%%%%%

\newcommand{\eps}{\epsilon}
\newcommand{\vphi}{\varphi}
\newcommand{\cte}{\text{cte}}

\newcommand{\N}{{\mathbb{N}}}
\newcommand{\Z}{{\mathbb{Z}}}
%\newcommand{\Q}{{\mathbb{Q}}}
\newcommand{\C}{{\mathbb{C}}}
\renewcommand{\S}{{\hat{S}}}
%\renewcommand{\H}{\s{H}}

\renewcommand{\a}{{\vb{a}}}
\renewcommand{\b}{{\vb{b}}}
\renewcommand{\d}{{\dagger}}
\newcommand{\up}{{\uparrow}}
\newcommand{\down}{{\downarrow}}
\newcommand{\hc}{{\text{h.c.}}}

\newcommand{\ihat}{\bm{\hat{\imath}}}
\newcommand{\jhat}{\bm{\hat{\jmath}}}
\newcommand{\khat}{\bm{\hat{k}}}

\newcommand{\0}{{\vb{0}}}
\newcommand{\1}{\mathds{1}}
\newcommand{\E}{{\vb{E}}}
\newcommand{\B}{{\vb{B}}}
\renewcommand{\u}{{\vb{u}}}
\renewcommand{\v}{{\vb{v}}}
\renewcommand{\r}{{\vb{r}}}
\newcommand{\R}{{\vb{R}}}
\newcommand{\Q}{{\vb{Q}}}
\newcommand{\G}{{\vb{G}}}
\newcommand{\g}{{\vb{g}}}
\renewcommand{\k}{{\vb{k}}}
\newcommand{\K}{{\vb{K}}}
\newcommand{\p}{{\vb{p}}}
\newcommand{\q}{{\vb{q}}}
\newcommand{\F}{{\vb{F}}}
\renewcommand{\t}{{\vb{t}}}
\newcommand{\vtau}{{\bm{\tau}}}
\newcommand{\vdelta}{{\bm{\delta}}}

% COLORED SYMMETRY ELEMENTS
\newcommand{\Ct}{{\textcolor{Cyan}{C_3}}}
\newcommand{\Ctn}[1]{{\textcolor{Cyan}{C_3^{\textcolor{black}{#1}}}}}
\newcommand{\Cs}{{\textcolor{ForestGreen}{C_6}}}
\newcommand{\Csn}[1]{{\textcolor{ForestGreen}{C_6^{\textcolor{black}{#1}}}}}
\newcommand{\sd}{{\textcolor{RoyalBlue}{\sigma_d}}}
\newcommand{\sdn}[1]{{\textcolor{RoyalBlue}{\sigma_d^{\textcolor{black}{#1}}}}}
\newcommand{\sdp}{{\textcolor{RoyalBlue}{\sigma_d'}}}
\newcommand{\sdpp}{{\textcolor{RoyalBlue}{\sigma_d''}}}
\newcommand{\sv}{{\textcolor{Orange}{\sigma_v}}}
\newcommand{\svn}[1]{{\textcolor{Orange}{\sigma_v^{\textcolor{black}{#1}}}}}
\newcommand{\svp}{{\textcolor{Orange}{\sigma_v'}}}
\newcommand{\svpp}{{\textcolor{Orange}{\sigma_v''}}}

\newcommand{\GL}{{\text{GL}}}
\newcommand{\U}{{\text{U}}}

\newcommand{\s}{\sigma}
%\newcommand{\prodint}[2]{\left\langle #1 , #2 \right\rangle}
\newcommand{\cc}[1]{\overline{#1}}
\newcommand{\Eval}[3]{\eval{\left( #1 \right)}_{#2}^{#3}}
\newcommand{\sg}[2]{\{ #1 \mid #2 \}}
\renewcommand{\AA}{{\mathring{\text{A}}}}
\newcommand{\I}{{\mathbb{I}}}
\newcommand{\bP}{{\mathbb{P}}}
\newcommand{\bQ}{{\mathbb{Q}}}

\newcommand{\unit}[1]{\; \mathrm{#1}}

\newcommand{\n}{\medskip}
\newcommand{\e}{\quad \mathrm{and} \quad}
\newcommand{\ou}{\quad \mathrm{or} \quad}
\newcommand{\virg}{\, , \;}
\newcommand{\ptodo}{\forall \,}
\renewcommand{\implies}{\; \Rightarrow \;}
%\newcommand{\eqname}[1]{\tag*{#1}} % Tag equation with name

%\setlength{\droptitle}{-7em}   % título um pouco mais em cima na página
%\makeatletter
%\patchcmd{\chapter}{\if@openright\cleardoublepage\else\clearpage\fi}{}{}{}  % start 'Chapter' at the same page. needs package etoolbox
%\makeatother

%% Theorems, definitions, proofs
\theoremstyle{definition}

%%% defining my own colors %%%
\definecolor{my-blue}{HTML}{f2f4ff}
\definecolor{my-green}{HTML}{f5fcf6}    % a little better: green!5!white
\definecolor{my-cyan}{HTML}{f2fffe}
\definecolor{my-yellow}{HTML}{fffbed}
\definecolor{my-green2}{HTML}{efffdb}

%%% alternative colors %%%
\definecolor{my-pink}{HTML}{fff2f7}
\definecolor{my-teal}{HTML}{ebfffc}

\newtheorem{definition}{Definition}[section]
\tcolorboxenvironment{definition}{
  colback=my-blue,
  %colback=blue!5!white,
  boxrule=0.1pt,
  boxsep=1pt,
  left=2pt,right=2pt,top=2pt,bottom=2pt,
  oversize=2pt,
  sharp corners,
  before skip=\topsep,
  after skip=\topsep,
}

\newtheorem{theorem}{Theorem}[section]
\tcolorboxenvironment{theorem}{
  colback=my-yellow,
  %colback=yellow!22!white!95!black,
  boxrule=0.1pt,
  boxsep=1pt,
  left=2pt,right=2pt,top=2pt,bottom=2pt,
  oversize=2pt,
  sharp corners,
  before skip=\topsep,
  after skip=\topsep,
}

\newtheorem{corollary}{Corollary}[section]
\tcolorboxenvironment{corollary}{
  colback=my-green2,
  boxrule=0.1pt,
  boxsep=1pt,
  left=2pt,right=2pt,top=2pt,bottom=2pt,
  oversize=2pt,
  sharp corners,
  before skip=\topsep,
  after skip=\topsep,
}

\newtheorem{lemma}{Lemma}[section]
\tcolorboxenvironment{lemma}{
  colback=my-cyan,
  boxrule=0.1pt,
  boxsep=1pt,
  left=2pt,right=2pt,top=2pt,bottom=2pt,
  oversize=2pt,
  sharp corners,
  before skip=\topsep,
  after skip=\topsep,
}

\newtheorem{example}{Example}[section]
\tcolorboxenvironment{example}{
  %colback=my-green,
  colback=green!5!white,
  boxrule=0.1pt,
  boxsep=1pt,
  left=2pt,right=2pt,top=2pt,bottom=2pt,
  oversize=2pt,
  sharp corners,
  before skip=\topsep,
  after skip=\topsep,
}


\title{\Huge{\textbf{Exercícios 4}}}
\author{Mateus Marques}

\begin{document}

\maketitle

\section{Compreensão Semana 9}

Com respeito à direção de vibração e à direção de propagação de uma onda, classifique os seguintes fenômenos como ondas longitudinais, transversais ou compostas (longitudinais + transversais).

\begin{itemize}
\item Ondas formadas na corda de um violão. (transversais)
\item Ondas no oceano. (compostas)
\item Ondas eletromagnéticas. (transversais)
\item Ondas sonoras. (longitudinais)
\item Ondas gravitacionais. (transversais)
\end{itemize}


\section{Aprofundamento Semana 9}

Para este exercício, defina as funções de uma variável
$$
\theta_H(x) = \frac{x +  \abs{x}}{2x} \e \phi(x) = \exp(-\frac{1}{1-x^2}).
$$

Agora considere que uma onda \textbf{exótica} da forma
$$
y(x,t) = A \, \theta_H(1 - (kx - \omega t)) \, \theta_H(1 + (kx - \omega t)) \, \phi(kx - \omega t).
$$ se propaga num barbante preso nas extremidades, onde $A = 2 \unit{cm}$, $\omega = 10 \unit{s^{-1}}$ e $k = 1 \unit{cm^{-1}}$.

\n

Para conseguir visualizar, plote $\theta_H(x)$, $f(x) = \theta_H(1-x) \theta_H(1+x)$, $\phi(x)$ e $g(x) = y(x, 0)$ no Desmos. Plote também as derivadas $g'(x)$ e $g''(x)$.

\n\n

\textit{Comentário 1: A ideia do produto $\theta_H(1-x) \theta_H(1+x)$ (é uma caixa de largura 2 e altura 1) é delimitar o domínio para $x \in [-1, 1]$ e tornar a onda \textbf{finita}. A função $\phi(x)$ parece uma gaussiana, mas ela decai para $0$ para $x \to \pm 1$ (a gaussiana normal se estende até o infinito). Quando $x$ está fora do intervalo $(-1, 1)$ a função $\phi(x)$ é muito louca, por isso adicionei $\theta_H(1-x) \theta_H(1+x)$ para tornar ela bem comportada e $y(x,t)$ ser uma onda finita de fato.}

\n

\textit{Comentário 2: A ideia deste exercício é que ele seja resolvido com auxílio do Desmos mesmo. Mas todos os itens podem ser resolvidos de forma analítica (A função $\phi(x)$ é bem comportada).}

\n\n

\begin{itemize}

\item Pergunta de sanidade: a função $y(x,t)$ é uma onda de verdade? Ela obedece à equação de onda
$$
\pdv[2]{y}{x} = \frac{1}{v^2} \pdv[2]{y}{t} \; ?
$$
Qual a velocidade $v$?

R: Sim, pois $y(x,t)$ é da forma $F(kx - \omega t)$. Logo, é onda que se propaga para direita com velocidade $v = \omega/k = 10 \unit{cm/s}$.

\item Num instante $t = 3 \unit{s}$, onde se localiza o ponto mais alto dessa onda?

A posição desse ponto é $x = \frac{\omega}{k} \, t$.

\item Qual é a largura $L$ dessa onda?

A função $\phi(x)$ se anula em $x = \pm 1$. Portanto a largura de $\phi(x)$ é 2. Como $y(x, 0)$ possui o termo $\phi(kx)$, a largura é portanto $L = 2/k$.

\item Qual a velocidade máxima (vertical) de um ponto da corda?

Pelo \textit{Comentário 1}, para todos efeitos práticos podemos ignorar $\theta_H(1-x) \theta_H(1+x)$ e dar atenção apenas ao termo $A \phi(kx - \omega t)$.

Temos que
$$
\phi'(x) = - \frac{2x}{(1-x^2)^2} \exp(-\frac{1}{1-x^2}) \e
\phi''(x) = \frac{(6x^4-2)}{(1-x^2)^4} \exp(-\frac{1}{1-x^2})
$$

A velocidade máxima de um ponto da corda ocorre quando sua aceleração $a_y = 0$.
$$
a_y = \pdv[2]{y}{t} = \omega^2 A \phi''(kx - \omega t) = 0 \implies
6 (kx - \omega t)^4 - 2 = 0 \implies \boxed{kx - \omega t =  \pm \sqrt[4]{\frac{1}{3}}.}
$$

Inserindo esse argumento em $v_y = \partial y / \partial t$, temos (para $t = 0$ por exemplo)
$$
v_{\text{max}} = \abs{\pdv{y\qty(x = \frac{1}{k} \sqrt[4]{\frac{1}{3}}, t=0)}{t}} = \omega A \abs{ \phi\qty(\sqrt[4]{\frac{1}{3}}) } = 0.79843 \, \omega A = 15.9686 \unit{cm/s}.
$$

\end{itemize}


\end{document}
