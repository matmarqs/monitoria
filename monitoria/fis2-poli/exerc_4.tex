\documentclass[a4paper,10pt]{article}
\usepackage[brazilian]{babel}
\usepackage[left=2.5cm,right=2.5cm,top=3cm,bottom=2.5cm]{geometry}
\usepackage{mathtools}
\usepackage{amsthm}
\usepackage{amsmath}
%\usepackage{nccmath}
\usepackage{amssymb}
\usepackage{amsfonts}
\usepackage{physics}
%\usepackage{dsfont}
%\usepackage{mathrsfs}

\usepackage{titling}
\usepackage{indentfirst}

\usepackage{bm}
\usepackage[dvipsnames]{xcolor}
\usepackage{cancel}

\usepackage{xurl}
\usepackage[colorlinks=true]{hyperref}

\usepackage{float}
\usepackage{graphicx}
%\usepackage{tikz}
\usepackage{caption}
\usepackage{subcaption}

%%%%%%%%%%%%%%%%%%%%%%%%%%%%%%%%%%%%%%%%%%%%%%%%%%%

\newcommand{\eps}{\epsilon}
\newcommand{\vphi}{\varphi}
\newcommand{\cte}{\text{cte}}

\newcommand{\N}{\mathbb{N}}
\newcommand{\Z}{\mathbb{Z}}
\newcommand{\Q}{\mathbb{Q}}
\newcommand{\R}{\vb{R}}
\newcommand{\C}{\mathbb{C}}
\renewcommand{\S}{\hat{S}}
%\renewcommand{\H}{\s{H}}

\renewcommand{\a}{\vb{a}}
\newcommand{\nn}{\hat{n}}
\renewcommand{\d}{\dagger}
\newcommand{\up}{\uparrow}
\newcommand{\down}{\downarrow}

\newcommand{\0}{\vb{0}}
%\newcommand{\1}{\mathds{1}}
\newcommand{\E}{\vb{E}}
\newcommand{\B}{\vb{B}}
\renewcommand{\v}{\vb{v}}
\renewcommand{\r}{\vb{r}}
\renewcommand{\k}{\vb{k}}
\newcommand{\p}{\vb{p}}
\newcommand{\q}{\vb{q}}
\newcommand{\F}{\vb{F}}

\newcommand{\s}{\sigma}
%\newcommand{\prodint}[2]{\left\langle #1 , #2 \right\rangle}
\newcommand{\cc}[1]{\overline{#1}}
\newcommand{\Eval}[3]{\eval{\left( #1 \right)}_{#2}^{#3}}

\newcommand{\unit}[1]{\; \mathrm{#1}}

\newcommand{\n}{\medskip}
\newcommand{\e}{\quad \mathrm{e} \quad}
\newcommand{\ou}{\quad \mathrm{ou} \quad}
\newcommand{\virg}{\, , \;}
\newcommand{\ptodo}{\forall \,}
\renewcommand{\implies}{\; \Rightarrow \;}
%\newcommand{\eqname}[1]{\tag*{#1}} % Tag equation with name

\setlength{\droptitle}{-7em}

\theoremstyle{plain}
\newtheorem{theorem}{Teorema}[section]
%\newtheorem{defi}[theorem]{Definição}
\newtheorem{lemma}[theorem]{Lema}
%\newtheorem{corol}[theorem]{Corolário}
%\newtheorem{prop}[theorem]{Proposição}
%\newtheorem{example}{Exemplo}
%
%\newtheorem{inneraxiom}{Axioma}
%\newenvironment{axioma}[1]
%  {\renewcommand\theinneraxiom{#1}\inneraxiom}
%  {\endinneraxiom}
%
%\newtheorem{innerpostulado}{Postulado}
%\newenvironment{postulado}[1]
%  {\renewcommand\theinnerpostulado{#1}\innerpostulado}
%  {\endinnerpostulado}
%
%\newtheorem{innerexercise}{Exercício}
%\newenvironment{exercise}[1]
%  {\renewcommand\theinnerexercise{#1}\innerexercise}
%  {\endinnerexercise}
%
%\newtheorem{innerthm}{Teorema}
%\newenvironment{teorema}[1]
%  {\renewcommand\theinnerthm{#1}\innerthm}
%  {\endinnerthm}
%
\newtheorem{innerlema}{Lema}
\newenvironment{lema}[1]
  {\renewcommand\theinnerlema{#1}\innerlema}
  {\endinnerlema}
%
%\theoremstyle{remark}
%\newtheorem*{hint}{Dica}
%\newtheorem*{notation}{Notação}
%\newtheorem*{obs}{Observação}


\title{\Huge{\textbf{Exercícios 4}}}
\author{Mateus Marques}

\begin{document}

\maketitle

\section{Compreensão Semana 9}

Com respeito à direção de vibração e à direção de propagação de uma onda, classifique os seguintes fenômenos como ondas longitudinais, transversais ou compostas (longitudinais + transversais).

\begin{itemize}
\item Ondas formadas na corda de um violão. (transversais)
\item Ondas no oceano. (compostas)
\item Ondas eletromagnéticas. (transversais)
\item Ondas sonoras. (longitudinais)
\item Ondas gravitacionais. (transversais)
\end{itemize}


\section{Aprofundamento Semana 9}

Para este exercício, defina as funções de uma variável
$$
\theta_H(x) = \frac{x +  \abs{x}}{2x} \e \phi(x) = \exp(-\frac{1}{1-x^2}).
$$

Agora considere que uma onda \textbf{exótica} da forma
$$
y(x,t) = A \, \theta_H(1 - (kx - \omega t)) \, \theta_H(1 + (kx - \omega t)) \, \phi(kx - \omega t).
$$ se propaga num barbante preso nas extremidades, onde $A = 2 \unit{cm}$, $\omega = 10 \unit{s^{-1}}$ e $k = 1 \unit{cm^{-1}}$.

\n

Para conseguir visualizar, plote $\theta_H(x)$, $f(x) = \theta_H(1-x) \theta_H(1+x)$, $\phi(x)$ e $g(x) = y(x, 0)$ no Desmos. Plote também as derivadas $g'(x)$ e $g''(x)$.

\n\n

\textit{Comentário 1: A ideia do produto $\theta_H(1-x) \theta_H(1+x)$ (é uma caixa de largura 2 e altura 1) é delimitar o domínio para $x \in [-1, 1]$ e tornar a onda \textbf{finita}. A função $\phi(x)$ parece uma gaussiana, mas ela decai para $0$ para $x \to \pm 1$ (a gaussiana normal se estende até o infinito). Quando $x$ está fora do intervalo $(-1, 1)$ a função $\phi(x)$ é muito louca, por isso adicionei $\theta_H(1-x) \theta_H(1+x)$ para tornar ela bem comportada e $y(x,t)$ ser uma onda finita de fato.}

\n

\textit{Comentário 2: A ideia deste exercício é que ele seja resolvido com auxílio do Desmos mesmo. Mas todos os itens podem ser resolvidos de forma analítica (A função $\phi(x)$ é bem comportada).}

\n\n

\begin{itemize}

\item Pergunta de sanidade: a função $y(x,t)$ é uma onda de verdade? Ela obedece à equação de onda
$$
\pdv[2]{y}{x} = \frac{1}{v^2} \pdv[2]{y}{t} \; ?
$$
Qual a velocidade $v$?

R: Sim, pois $y(x,t)$ é da forma $F(kx - \omega t)$. Logo, é onda que se propaga para direita com velocidade $v = \omega/k = 10 \unit{cm/s}$.

\item Num instante $t = 3 \unit{s}$, onde se localiza o ponto mais alto dessa onda?

A posição desse ponto é $x = \frac{\omega}{k} \, t$.

\item Qual é a largura $L$ dessa onda?

A função $\phi(x)$ se anula em $x = \pm 1$. Portanto a largura de $\phi(x)$ é 2. Como $y(x, 0)$ possui o termo $\phi(kx)$, a largura é portanto $L = 2/k$.

\item Qual a velocidade máxima (vertical) de um ponto da corda?

Pelo \textit{Comentário 1}, para todos efeitos práticos podemos ignorar $\theta_H(1-x) \theta_H(1+x)$ e dar atenção apenas ao termo $A \phi(kx - \omega t)$.

Temos que
$$
\phi'(x) = - \frac{2x}{(1-x^2)^2} \exp(-\frac{1}{1-x^2}) \e
\phi''(x) = \frac{(6x^4-2)}{(1-x^2)^4} \exp(-\frac{1}{1-x^2})
$$

A velocidade máxima de um ponto da corda ocorre quando sua aceleração $a_y = 0$.
$$
a_y = \pdv[2]{y}{t} = \omega^2 A \phi''(kx - \omega t) = 0 \implies
6 (kx - \omega t)^4 - 2 = 0 \implies \boxed{kx - \omega t =  \pm \sqrt[4]{\frac{1}{3}}.}
$$

Inserindo esse argumento em $v_y = \partial y / \partial t$, temos (para $t = 0$ por exemplo)
$$
v_{\text{max}} = \abs{\pdv{y\qty(x = \frac{1}{k} \sqrt[4]{\frac{1}{3}}, t=0)}{t}} = \omega A \abs{ \phi\qty(\sqrt[4]{\frac{1}{3}}) } = 0.79843 \, \omega A = 15.9686 \unit{cm/s}.
$$

\end{itemize}


\end{document}
