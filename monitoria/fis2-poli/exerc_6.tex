\documentclass[a4paper,10pt]{article}
\usepackage[brazilian]{babel}
\usepackage[left=2.5cm,right=2.5cm,top=3cm,bottom=2.5cm]{geometry}
\usepackage{mathtools}
\usepackage{amsthm}
\usepackage{amsmath}
%\usepackage{nccmath}
\usepackage{amssymb}
\usepackage{amsfonts}
\usepackage{physics}
%\usepackage{dsfont}
%\usepackage{mathrsfs}

\usepackage{titling}
\usepackage{indentfirst}

\usepackage{bm}
\usepackage[dvipsnames]{xcolor}
\usepackage{cancel}

\usepackage{xurl}
\usepackage[colorlinks=true]{hyperref}

\usepackage{float}
\usepackage{graphicx}
%\usepackage{tikz}
\usepackage{caption}
\usepackage{subcaption}

%%%%%%%%%%%%%%%%%%%%%%%%%%%%%%%%%%%%%%%%%%%%%%%%%%%

\newcommand{\eps}{\epsilon}
\newcommand{\vphi}{\varphi}
\newcommand{\cte}{\text{cte}}

\newcommand{\N}{\mathbb{N}}
\newcommand{\Z}{\mathbb{Z}}
\newcommand{\Q}{\mathbb{Q}}
\newcommand{\R}{\vb{R}}
\newcommand{\C}{\mathbb{C}}
\renewcommand{\S}{\hat{S}}
%\renewcommand{\H}{\s{H}}

\renewcommand{\a}{\vb{a}}
\newcommand{\nn}{\hat{n}}
\renewcommand{\d}{\dagger}
\newcommand{\up}{\uparrow}
\newcommand{\down}{\downarrow}

\newcommand{\0}{\vb{0}}
%\newcommand{\1}{\mathds{1}}
\newcommand{\E}{\vb{E}}
\newcommand{\B}{\vb{B}}
\renewcommand{\v}{\vb{v}}
\renewcommand{\r}{\vb{r}}
\renewcommand{\k}{\vb{k}}
\newcommand{\p}{\vb{p}}
\newcommand{\q}{\vb{q}}
\newcommand{\F}{\vb{F}}

\newcommand{\s}{\sigma}
%\newcommand{\prodint}[2]{\left\langle #1 , #2 \right\rangle}
\newcommand{\cc}[1]{\overline{#1}}
\newcommand{\Eval}[3]{\eval{\left( #1 \right)}_{#2}^{#3}}

\newcommand{\unit}[1]{\; \mathrm{#1}}

\newcommand{\n}{\medskip}
\newcommand{\e}{\quad \mathrm{e} \quad}
\newcommand{\ou}{\quad \mathrm{ou} \quad}
\newcommand{\virg}{\, , \;}
\newcommand{\ptodo}{\forall \,}
\renewcommand{\implies}{\; \Rightarrow \;}
%\newcommand{\eqname}[1]{\tag*{#1}} % Tag equation with name

\setlength{\droptitle}{-7em}

\theoremstyle{plain}
\newtheorem{theorem}{Teorema}[section]
%\newtheorem{defi}[theorem]{Definição}
\newtheorem{lemma}[theorem]{Lema}
%\newtheorem{corol}[theorem]{Corolário}
%\newtheorem{prop}[theorem]{Proposição}
%\newtheorem{example}{Exemplo}
%
%\newtheorem{inneraxiom}{Axioma}
%\newenvironment{axioma}[1]
%  {\renewcommand\theinneraxiom{#1}\inneraxiom}
%  {\endinneraxiom}
%
%\newtheorem{innerpostulado}{Postulado}
%\newenvironment{postulado}[1]
%  {\renewcommand\theinnerpostulado{#1}\innerpostulado}
%  {\endinnerpostulado}
%
%\newtheorem{innerexercise}{Exercício}
%\newenvironment{exercise}[1]
%  {\renewcommand\theinnerexercise{#1}\innerexercise}
%  {\endinnerexercise}
%
%\newtheorem{innerthm}{Teorema}
%\newenvironment{teorema}[1]
%  {\renewcommand\theinnerthm{#1}\innerthm}
%  {\endinnerthm}
%
\newtheorem{innerlema}{Lema}
\newenvironment{lema}[1]
  {\renewcommand\theinnerlema{#1}\innerlema}
  {\endinnerlema}
%
%\theoremstyle{remark}
%\newtheorem*{hint}{Dica}
%\newtheorem*{notation}{Notação}
%\newtheorem*{obs}{Observação}


\newcommand{\fbat}{f_{\text{bat}}}

\title{\Huge{\textbf{Exercícios 6}}}
\author{Mateus Marques}

\begin{document}

\maketitle

\section{Compreensão Semana 11}

\textbf{Exercício 6.2 do Moysés 2}

\n

Tome a densidade do ar como $1.3 \unit{kg/m^3}$ e a velocidade do som como $340 \unit{m/s}$.

\n

O alto-falante de um aparelho de som emite $1 \unit{W}$ de potência sonora na frequência
$f = 100 \unit{Hz}$. Admitindo que o som se distribui uniformemente em todas as direções,
determine, num ponto situado a $2 \unit{m}$ de distância do alto-falante:

\n\n

(a) o nível sonoro em dB;

\n

Solução: Temos que $I = \frac{P}{4\pi r^2}$ ($P = 1\unit{W}$ e $r=2\unit{m}$), portanto
$$
\alpha = 10 \log_{10}\qty(\frac{I}{I_0}) =
10 \log_{10}\qty[\frac{1 \unit{W}}{4\pi \, (2 \unit{m})^2 \cdot (10^{-12} \unit{W/m^2})}] = 103 \unit{dB}.
$$

\n

(b) a amplitude de pressão;

\n

Solução: $\rho_0 = 1.3 \unit{kg/m^3}$, $v = 340 \unit{m/s}$
$$
\wp^2 = 2 \rho v I \implies \boxed{\wp = 4.2 \unit{Pa}.}
$$

\n

(c) a amplitude de deslocamento.

\n

Solução: $k = \frac{2\pi f}{v} = \frac{2\pi \cdot 100\unit{Hz}}{340 \unit{m/s}} = 1.848 \unit{m^{-1}}$. A amplitude de deslocamento é
$$
U = \frac{\wp}{\rho_0 v^2 k} = 1.51 \times 10^{-5} = 0.015 \unit{mm}.
$$

\n

\n

(d) A que distância do alto-falante o nível sonoro estará 10 dB abaixo do calculado em (a)?
$$
10 \log_{10}\qty(\frac{I'}{I_0}) =
\alpha - 10 = 10 \qty[\log_{10}\qty(\frac{I}{I_0}) - \log_{10}(10)] = 10 \log_{10}\qty(\frac{I}{10 I_0}).
$$

Solução: A nova distância $d$ é dada então por
$$
\frac{\cancel{P}}{\cancel{4\pi} d^2} = \frac{1}{10} \cdot \frac{\cancel{P}}{\cancel{4\pi} r^2} \implies
d = r \sqrt{10} = 6.32 \unit{m}
$$


\section{Aprofundamento Semana 11}

\textbf{Exercício 27 da Lista 2 - 2022}


Duas cordas idênticas sob a mesma tensão $F$ possuem uma frequência fundamental igual a $f_0$. A seguir, a tensão em uma delas é aumentada de um valor bastante pequeno $\Delta F$.

\n\n

(a) Se elas são tocadas em suas frequências fundamentais, a frequência do batimento produzido é
\begin{enumerate}[(a)]
\item $ f_{\text{bat}} = \frac{1}{2} f_0 \frac{\Delta F}{F}. $
\item $ f_{\text{bat}} = \frac{1}{2} f_0 \frac{F}{\Delta F}. $
\item $ f_{\text{bat}} = f_0 \frac{\Delta F}{F}. $
\item $ f_{\text{bat}} = f_0 \frac{F}{\Delta F}. $
\item $ f_{\text{bat}} = 2 f_0 \frac{\Delta F}{F}. $
\end{enumerate}


\n

Solução: a frequência de batimento é $\fbat = f_1 - f_2$. Onde $f_1 = \frac{v_1}{2L}$, $f_2 = \frac{v_2}{2L} = F_0$ e $v_j = \sqrt{F_j/\mu}$. No caso temos $F_1 = F + \Delta F$ e $F_2 = F$. Portanto
$$
\fbat = \frac{1}{2L\sqrt{\mu}} \qty(\sqrt{F+\Delta F} - \sqrt{F}) \approx
\frac{1}{2L\sqrt{\mu}} \dv{\sqrt{F}}{F} \Delta F = \frac{1}{4L\sqrt{\mu}} \frac{\Delta F}{\sqrt{F}} =
\frac{1}{2} \qty(\frac{1}{2L} \sqrt{\frac{F}{\mu}}) \frac{\Delta F}{\sqrt{F}} =\frac{1}{2} f_0 \frac{\Delta F}{F}.
$$


\n

(b) Duas cordas de violino idênticas, quando estão em ressonância e esticadas com
a mesma tensão, possuem uma frequência fundamental igual a $440 \unit{Hz}$. Uma
das cordas é afinada novamente, tendo sua tensão aumentada. Quando isso é
feito, ouvimos $3$ batimentos a cada dois segundos quando as duas cordas são puxadas
simultaneamente em seus centros. Em que porcentagem variou a tensão na corda?

\n

Solução: Temos $\fbat = 1.5 \unit{Hz}$ e $f_0 = 440 \unit{Hz}$, portanto
$$
\frac{\Delta F}{F} = \frac{2 \fbat}{f_0} = \frac{3}{440} \approx 0.0068 = 0.68\%.
$$




\end{document}
