%%%%%%%%%%%%%%%%%%%%%%%%%%%%%%%%%%%%%%%%%%%%%%%%%%%%%%%%%%%%%%%%%%%%%
% In English:
%    This is a list of commands specification for FAPESP reports.
%
% In Portuguese:
%    Esta é uma lista de especificação de comandos para relatórios
% da Fundação de Amparo à pesquisa do Estado de São Paulo (FAPESP).
%
% Author/Autor: André Leon Sampaio Gradvohl, Dr.
% Email:        andre.gradvohl@gmail.com
% Lattes CV:    http://lattes.cnpq.br/9343261628675642
%
% Last update/Última versão: 11/Sep/2016
%%%%%%%%%%%%%%%%%%%%%%%%%%%%%%%%%%%%%%%%%%%%%%%%%%%%%%%%%%%%%%%%%%%%%%

\newcommand{\HRule}[1]{\rule{\linewidth}{#1}}
\setcounter{tocdepth}{3}
\setcounter{secnumdepth}{3}

\newcommand{\titulo}[1]{\def\meuTitulo{#1}}
\newcommand{\tituloIngles}[1]{\def\meuTituloIngles{#1}}
\newcommand{\numProjeto}[1]{\def\numFAP{#1}}
\newcommand{\tipoRelatorio}[1]{\def\tipoRelat{#1 }} %o espaço depois do #1 é importante
\newcommand{\modalidadeProjeto}[1]{\def\modProjeto{#1}}
\newcommand{\agFomento}[2]{\def\agFom{#1} \def\siglaAgFom{#2}} %extenso Sigla
\newcommand{\autor}[1]{\def\nomeAutor{#1}}
\newcommand{\cidade}[1]{\def\nomeCidade{#1}}
\newcommand{\universidade}[1]{\def\nomeUniversidade{#1}}
\newcommand{\faculdade}[1]{\def\nomeFaculdade{#1}}
\newcommand{\periodoVigencia}[1]{\def\periodVig{#1}}
\newcommand{\periodoRelatorio}[1]{\def\periodRelat{#1}}
\newcommand{\orientador}[1]{\def\nomeOrientador{#1}}

\newcommand\namegroup[3]{%
   \begin{minipage}[t]{0.4\textwidth}
   \hspace{0.6cm}
   \begin{tikzpicture}
   \node[anchor=south east,inner sep = 0](signF){\includegraphics[width=#3]{#2}};
   \end{tikzpicture}
   \vspace*{0.1cm}  % leave some space above the horizontal line
   \hrule
   \vspace{1mm} % just a bit more whitespace below the line
   \centering
   \begin{tabular}[t]{c}
   #1
   \end{tabular}
   \end{minipage}}

\author{}
\date{}

%Definição de membros da equipe de pesquisas
\newcommand{\membroA}[1]{\def\nomeMembroA{#1}}
\newcommand{\membroB}[1]{\def\nomeMembroB{#1}}
\newcommand{\membroC}[1]{\def\nomeMembroC{#1}}
\newcommand{\membroD}[1]{\def\nomeMembroD{#1}}
\newcommand{\membroE}[1]{\def\nomeMembroE{#1}}
\newcommand{\membroF}[1]{\def\nomeMembroF{#1}}

\newcommand{\Figure}[1]{Figura~\ref{fig:#1}}
\newcommand{\Table}[1] {Tabela~\ref{#1}}
\newcommand{\Equation}[1] {Equa\c{c}\~ao~\ref{#1}}
\newcommand{\addFigure}[3] { %Parametros scale, fig_name, caption
    \begin{figure}[!hbt]
      \centering
      \includegraphics[scale=#1]{figures/}
      \caption{#3}\label{fig:#2}
    \end{figure}
}

\newcommand{\geraTitulo}{
\clearpage
\begin{titlepage}
  \begin{center}
      \vspace*{-3cm}
       { \setstretch{.5}
         \textsc{\nomeUniversidade} \\
         \HRule{.2pt}\\
         \textsc{\nomeFaculdade}
       }

       \vspace{5.5cm}

       \Large \textbf{\textsc{\meuTitulo}}
 	  \HRule{1.5pt} \\ [0.5cm]
       \linespread{1}
       \large Report on the activities
       %\ifdefined\tipoRelat
       %     \tipoRelat
       %\fi
       of the
       \ifdefined\modProjeto
           \modProjeto
       \fi
       ~project supported by the \agFom. \\
   	   \HRule{1.5pt} \\ [0.5cm]

       \ifdefined\numFAP
          Grant \texttt{\#\numFAP}, \siglaAgFom
          \\ [0.5cm]
       \fi
        Author: \nomeAutor \\
        Advisor: \nomeOrientador \\[5.cm]

        % aqui começa a adicao das linhas de assinatura
        %Aluno: \tikz\draw [thick,solid] (0,0) -- (6,0); \\[.5cm]
        %Orientador: \tikz\draw [thick,solid] (0,0) -- (5.5,0);


        %% Assinatura assinatura ASSINATURA
        %\namegroup{Researcher}{example-image-a}
        \namegroup{Researcher}{fig/placeholder.png}{5cm}
        \hspace{1.5cm}
        \namegroup{Advisor}{fig/placeholder.png}{1cm}

        %\begin{minipage}[t]{0.4\textwidth}
        %\vspace*{1.5cm}
        %\hrule
        %\vspace{1mm}
        %\centering
        %\begin{tabular}[t]{c}
        %Aluno
        %\end{tabular}
        %\hspace{1.5cm}
        %\begin{minipage}[t]{0.4\textwidth}
        %\vspace*{1.5cm}
        %\hrule
        %\vspace{1mm}
        %\centering
        %\begin{tabular}[t]{c}
        %Aluno
        %\end{tabular}
        %\end{minipage}
        %%\hfill

        \vfill

        {\normalsize  \nomeCidade, \today}
 \end{center}
 \end{titlepage}
}

\usepackage{titlesec}
\titleformat{\chapter}{\normalfont\LARGE\bfseries}{\thechapter}{1em}{}
\titlespacing*{\chapter}{0pt}{3.5ex plus 1ex minus .2ex}{2.3ex plus .2ex}

%----------------------------------------------------------------------
% Cabeçalho e rodapé
%----------------------------------------------------------------------
\pagestyle{fancy}
\fancyhf{} % Limpa todos os campos de header and footer fields
\renewcommand{\headrulewidth}{0pt}
\fancyfoot[R]{\thepage}

\addto\captions{\renewcommand{\contentsname}{Summary}}
\addto\captions{\renewcommand{\bibname}{Bibliography}}

%------
% Resumo e Abstract
%------
%%\newcommand{\Resumo}[1]{
%%   \begin{otherlanguage}{portuguese}
%%       \addcontentsline{toc}{chapter}{Resumo}
%%       \begin{abstract} \thispagestyle{plain} \setcounter{page}{2}
%%          #1
%%        \end{abstract}
%%   \end{otherlanguage}
%%} %end \Resumo

\newcommand{\Abstract}[1]{
   \begin{otherlanguage}{english}
      \addcontentsline{toc}{chapter}{Abstract}
      \begin{abstract} \thispagestyle{plain} \setcounter{page}{3}
       #1
      \end{abstract}
    \end{otherlanguage}
} %end \abstract

%------
% Folha de rosto
%------


\newcommand{\folhaDeRosto}{
   \chapter*{Project}
   \addcontentsline{toc}{chapter}{General Project Information}
   \begin{itemize}
      \item Title:
            \begin{itemize}\item[] \textbf{\meuTitulo} \end{itemize}
      \item Advisor:
            \begin{itemize}\item[]\textbf{Prof. \nomeOrientador}\end{itemize}
      \item Project host institution:
            \begin{itemize}
               \item[]\textbf{\nomeFaculdade \ da \nomeUniversidade}
            \end{itemize}
      \item Research team:
            \begin{itemize}
               \ifdefined\nomeMembroA
                 \item[]\textbf{\nomeMembroA}
               \else
                 \item[]\textbf{\nomeAutor}
               \fi
               \ifx\nomeMembroB\undefined\else \item[]\textbf{\nomeMembroB}\fi
               \ifx\nomeMembroC\undefined\else \item[]\textbf{\nomeMembroC}\fi
               \ifx\nomeMembroD\undefined\else \item[]\textbf{\nomeMembroD}\fi
               \ifx\nomeMembroE\undefined\else \item[]\textbf{\nomeMembroE}\fi
               \ifx\nomeMembroF\undefined\else \item[]\textbf{\nomeMembroF}\fi
             \end{itemize}

          \ifdefined \numFAP
             \item Research project number:
             \begin{itemize}
                 \item[]\textbf{\numFAP}
             \end{itemize}
          \fi
       \item Validity period:
            \begin{itemize}
               \item[]\textbf{\periodVig}
            \end{itemize}
       \item Period covered by this scientific report:
            \begin{itemize}
               \item[]\textbf{\periodRelat}
            \end{itemize}
   \end{itemize}
   \clearpage
}


%\newcommand{\folhaDeRosto}{
%   \chapter*{Informações Gerais do Projeto}
%   \addcontentsline{toc}{chapter}{Informações Gerais do Projeto}
%   \begin{itemize}
%      \item Título do projeto:
%            \begin{itemize}\item[] \textbf{\meuTitulo} \end{itemize}
%      \item Nome do pesquisador responsável:
%            \begin{itemize}\item[]\textbf{Prof. \nomeOrientador}\end{itemize}
%      \item Instituição sede do projeto:
%            \begin{itemize}
%               \item[]\textbf{\nomeFaculdade \ da \nomeUniversidade}
%            \end{itemize}
%      \item Equipe de pesquisa:
%            \begin{itemize}
%               \ifdefined\nomeMembroA
%                 \item[]\textbf{\nomeMembroA}
%               \else
%                 \item[]\textbf{\nomeAutor}
%               \fi
%               \ifx\nomeMembroB\undefined\else \item[]\textbf{\nomeMembroB}\fi
%               \ifx\nomeMembroC\undefined\else \item[]\textbf{\nomeMembroC}\fi
%               \ifx\nomeMembroD\undefined\else \item[]\textbf{\nomeMembroD}\fi
%               \ifx\nomeMembroE\undefined\else \item[]\textbf{\nomeMembroE}\fi
%               \ifx\nomeMembroF\undefined\else \item[]\textbf{\nomeMembroF}\fi
%             \end{itemize}
%
%          \ifdefined \numFAP
%             \item Número do projeto de pesquisa:
%             \begin{itemize}
%                 \item[]\textbf{\numFAP}
%             \end{itemize}
%          \fi
%       \item Período de vigência:
%            \begin{itemize}
%               \item[]\textbf{\periodVig}
%            \end{itemize}
%       \item Período coberto por este relatório científico:
%            \begin{itemize}
%               \item[]\textbf{\periodRelat}
%            \end{itemize}
%   \end{itemize}
%   \clearpage
%}
