\documentclass[a4paper,10pt]{article}
\usepackage[brazilian]{babel}
\usepackage[left=2.5cm,right=2.5cm,top=3cm,bottom=2.5cm]{geometry}
\usepackage{mathtools}
\usepackage{amsthm}
\usepackage{amsmath}
%\usepackage{nccmath}
\usepackage{amssymb}
\usepackage{amsfonts}
\usepackage{physics}
\usepackage{dsfont}
%\usepackage{mathrsfs}

\usepackage{titling}
\usepackage{indentfirst}

\usepackage{bm}
\usepackage[dvipsnames]{xcolor}
\usepackage{cancel}

\usepackage{xurl}
\usepackage[colorlinks=true]{hyperref}

\usepackage{float}
\usepackage{graphicx}
%\usepackage{tikz}
\usepackage{caption}
\usepackage{subcaption}

%%%%%%%%%%%%%%%%%%%%%%%%%%%%%%%%%%%%%%%%%%%%%%%%%%%

\newcommand{\eps}{\epsilon}
\newcommand{\vphi}{\varphi}
\newcommand{\cte}{\text{cte}}

\newcommand{\N}{\mathbb{N}}
\newcommand{\Z}{\mathbb{Z}}
\newcommand{\Q}{\mathbb{Q}}
\newcommand{\R}{\vb{R}}
\newcommand{\C}{\mathbb{C}}
\renewcommand{\S}{\vb{S}}
%\renewcommand{\H}{\s{H}}

\renewcommand{\a}{\vb{a}}
\renewcommand{\d}{\dagger}
\newcommand{\up}{\uparrow}
\newcommand{\down}{\downarrow}
\newcommand{\hc}{\text{h.c.}}

\newcommand{\0}{\vb{0}}
\newcommand{\1}{\mathds{1}}
\newcommand{\E}{\vb{E}}
\newcommand{\B}{\vb{B}}
\renewcommand{\v}{\vb{v}}
\renewcommand{\r}{\vb{r}}
\renewcommand{\k}{\vb{k}}
\newcommand{\p}{\vb{p}}
\newcommand{\q}{\vb{q}}
\newcommand{\F}{\vb{F}}
\newcommand{\A}{\vb{A}}

\newcommand{\s}{\sigma}
\newcommand{\nn}{\hat{n}}
\newcommand{\cc}[1]{\overline{#1}}
\newcommand{\Eval}[3]{\eval{\left( #1 \right)}_{#2}^{#3}}

\newcommand{\unit}[1]{\; \mathrm{#1}}

\newcommand{\n}{\medskip}
\newcommand{\e}{\quad \mathrm{e} \quad}
\newcommand{\ou}{\quad \mathrm{ou} \quad}
\newcommand{\virg}{\, , \;}
\newcommand{\ptodo}{\forall \,}
\renewcommand{\implies}{\; \Rightarrow \;}
%\newcommand{\eqname}[1]{\tag*{#1}} % Tag equation with name

\setlength{\droptitle}{-7em}

\theoremstyle{plain}
\newtheorem{theorem}{Teorema}[section]
%\newtheorem{defi}[theorem]{Definição}
\newtheorem{lemma}[theorem]{Lema}
%\newtheorem{corol}[theorem]{Corolário}
%\newtheorem{prop}[theorem]{Proposição}
%\newtheorem{example}{Exemplo}
%
%\newtheorem{inneraxiom}{Axioma}
%\newenvironment{axioma}[1]
%  {\renewcommand\theinneraxiom{#1}\inneraxiom}
%  {\endinneraxiom}
%
%\newtheorem{innerpostulado}{Postulado}
%\newenvironment{postulado}[1]
%  {\renewcommand\theinnerpostulado{#1}\innerpostulado}
%  {\endinnerpostulado}
%
%\newtheorem{innerexercise}{Exercício}
%\newenvironment{exercise}[1]
%  {\renewcommand\theinnerexercise{#1}\innerexercise}
%  {\endinnerexercise}
%
%\newtheorem{innerthm}{Teorema}
%\newenvironment{teorema}[1]
%  {\renewcommand\theinnerthm{#1}\innerthm}
%  {\endinnerthm}
%
\newtheorem{innerlema}{Lema}
\newenvironment{lema}[1]
  {\renewcommand\theinnerlema{#1}\innerlema}
  {\endinnerlema}
%
%\theoremstyle{remark}
%\newtheorem*{hint}{Dica}
%\newtheorem*{notation}{Notação}
%\newtheorem*{obs}{Observação}


\title{\Huge{\textbf{Lista 1 - Matéria Condensada 2}}}
\author{Mateus Marques}

\begin{document}

\maketitle


\section{Teoria de perturbação degenerada}

A hamiltoniana perturbada é dada por $H = H_0 + \lambda V$. Seja $\ket{n \alpha_n}$ os autovetores de $H$ com autovalores $E_n$ (onde $\alpha_n$ indexa a degenerescência da energia $E_n$) que formam uma base no espaço de Hilbert $\mathcal{H}$. Seja $\mathcal{H}_n$ o subespaço degenerado de energia $E_n$. Podemos dividir o espaço total numa soma direta $\mathcal{H} = \mathcal{H}_n \oplus \mathcal{H}_n^\perp$, onde $\mathcal{H}_n^\perp$ é o espaço ortogonal a $\mathcal{H}_n$. Em termos de projeções, definimos
$$
P = \sum_{\alpha_n} \ketbra{n \alpha_n} \e
Q = \sum_{m \neq n} \sum_{\alpha_m} \ketbra{m \alpha_m}, \text{ com }
P + Q = 1, PQ = QP = 0,
$$
onde $P$ projeta em $\mathcal{H}_n$ e $Q$ projeta em $\mathcal{H}_n^\perp$.

Queremos encontrar um novo autovalor $E(\lambda)$ da hamiltoniana perturbada $H_0 + \lambda V$ que nasce da degenerescência em $E_n$, ou seja $\displaystyle{\lim_{\lambda \to 0} E(\lambda) = E_n}$. Temos que

\begin{equation} \label{eq:pertub}
(E - H_0) \ket{\psi} = \lambda V \ket{\psi}.
\end{equation}

Considere o operador $R$, que é a inversa de $(E - H_0)$ dentro do subespaço $\mathcal{H}_n^\perp$
$$
R = \sum_{m \neq n} \sum_{\alpha_m} \frac{\ketbra{m \alpha_m}}{E - E_m}.
$$

Note que $R$ satisfaz $R (E - H_0) = (E - H_0) R = Q$. Aplicando $R$ dos dois lados de \ref{eq:pertub}:
$$
Q \ket{\psi} = \lambda R V \ket{\psi}.
$$

Portanto, temos
\begin{equation} \label{eq:root}
\ket{\psi} = P \ket{\psi} + Q \ket{\psi} =
P \ket{\psi} + \lambda R V \ket{\psi}.
\end{equation}

Através de \ref{eq:root}, podemos obter uma equação perturbativa para $\ket{\psi}$ em função de $P \ket{\psi}$:
\begin{equation} \label{eq:recorr}
\ket{\psi} = P \ket{\psi} + \lambda R V P \ket{\psi} + \lambda^2 R V R V P \ket{\psi} + O(\lambda^3).
\end{equation}

Agora, temos que $P \ket{\psi}$ se escreve da forma
$$
P \ket{\psi} = \sum_{\alpha_n} \ket{n \alpha_n} \braket{n\alpha_n}{\psi} =
\sum_{\alpha_n} c_{\alpha_n} \ket{n \alpha_n}, \text{ com } c_{\alpha_n} = \braket{n\alpha_n}{\psi}.
$$

Aplicando $\bra{n \alpha_n}$ na equação \ref{eq:pertub} e substituindo \ref{eq:recorr} até primeira ordem $\lambda$, temos
$$
(E - E_n) c_{\alpha_n} = \lambda \mel{n \alpha_n}{V}{\psi} =
\lambda \bra{n \alpha_n} V \Bigg\{ 1 + \lambda R V + O(\lambda^2) \Bigg\} P \ket{\psi} =
$$
$$
=
\lambda \bra{n \alpha_n} V \Bigg\{ 1 + \lambda R V + O(\lambda^2) \Bigg\} \sum_{\beta_n} c_{\beta_n} \ket{n \beta_n} \implies
$$
\begin{equation} \label{eq:sol1}
(E - E_n) c_{\alpha_n} =
\sum_{\beta_n} \Bigg\{
\lambda \mel{n \alpha_n}{V}{n \beta_n} + \lambda^2 \sum_{\substack{m \neq n \\ \gamma_m}}
\frac{\mel{n \alpha_n}{V}{m \gamma_m} \mel{m \gamma_m}{V}{n \beta_n}}{E_n - E_m} + O(\lambda^3)
\Bigg\} c_{\beta_n},
\end{equation}
onde expandimos
$$
R = \sum_{m \neq n} \sum_{\alpha_m} \frac{\ketbra{m \alpha_m}}{\textcolor{red}{E} - E_m}
= \sum_{m \neq n} \sum_{\alpha_m} \frac{\ketbra{m \alpha_m}}{\textcolor{red}{E_n} - E_m} + O(\lambda).
$$

Agora basta que analisemos a equação \ref{eq:sol1}. Perceba que temos liberdade para escolher $\ket{n \alpha_n}$ como bem entendermos (dentro do subespaço degenerado $\mathcal{H}_n$). Em particular, escolhamos $\ket{n \alpha_n}$ de maneira que $\displaystyle{\lim_{\lambda \to 0} P \ket{\psi} (\lambda) = \ket{n \alpha_n}}$. Dessa maneira, temos que os coeficientes $c_{\beta_n}(\lambda)$ obedecem
\begin{equation} \label{eq:coeff}
c_{\beta_n}(\lambda) = \delta_{\alpha_n \beta_n} + O(\lambda).
\end{equation}

Substituindo \ref{eq:coeff} na equação \ref{eq:sol1}, obtemos a correção de segunda ordem de $E$ (proporcional a $\lambda^2$) como
$$
E^{(2)} = \sum_{\substack{m \neq n \\ \gamma_m}} \frac{\abs{\mel{n \alpha_n}{V}{m \gamma_m}}^2}{E_n - E_m}.
\quad \qedsymbol
$$

Observação: É claro que em teoria de perturbação degenerada existem \textit{splittings} que acontecem em primeira e segunda ordem. De certa maneira, estamos ignorando esses splittings e focando em obter a energia média em segunda ordem.



\pagebreak



\section{Operadores de criação e aniquilação}

Os operadores bosônicos satisfazem $[a_i, a_j] = [a_i^\dagger, a_j^\dagger] = 0$ e $[a_i, a_j^\dagger] = \delta_{ij}$. Usarei bastante a propriedade básica do comutador $[A, BC] = [A, B] C + B [A, C]$.

\n

(a) Definindo $\nn_j = a_j^\d a_j$, temos então
$$
[a_j^\d, \nn_j] = [a_j^\d, a_j^\d a_j] = a_j^\d \cancelto{\boxed{= -1}}{[a_j^\d, a_j]} + \cancelto{0}{[a_j^\d,a_j^\d]} a_j = - a_j^\d.
$$

$$
[a_j, \nn_j] = [a_j, a_j^\d a_j] = a_j^\d \cancelto{0}{[a_j, a_j]} + \cancelto{\boxed{=1}}{[a_j,a_j^\d]} a_j = a_j.
$$

\n

(b) Temos
$$
[a_i, (a_j^\d a_k^\d) (a_l a_m) ] = [a_i, a_j^\d a_k^\d] a_l a_m + a_j^\d a_k^\d \cancelto{0}{[a_i, a_l a_m]} =
$$
$$
(\cancelto{\boxed{= \delta_{ij}}}{[a_i, a_j^\d]} a_k^\d + a_j^\d \cancelto{\boxed{= \delta_{ik}}}{[a_i, a_k^\d]}) a_l a_m =
\delta_{ij} a_k^\d a_l a_m  +  \delta_{ik} a_j^\d a_m a_l.
$$

\n

Para operadores fermiônicos temos $\{c_i, c_j\} = \{c_i^\dagger, c_j^\dagger\} = 0$ e $\{c_i, c_j^\dagger\} = \delta_{ij}$. Também usarei a propriedade elementar do (anti)comutador $[A, BC] = \{A, B\} C - B \{A, C\}$.

\n

(c) Temos então
$$
[c_j^\d, \nn_j] = [ c_j^\d, c_j^\d c_j ] = \cancelto{0}{\{c_j^\d, c_j^\d\}} c_j - c_j^\d \cancelto{\boxed{=1}}{\{c_j^\d, c_j\}} = - c_j^\d.
$$
$$
[c_j, \nn_j] = [ c_j, c_j^\d c_j ] = \cancelto{\boxed{=1}}{\{c_j, c_j^\d\}} c_j - c_j^\d \cancelto{0}{\{c_j, c_j\}} = c_j.
$$

\n

(d) Expandindo o comutador em dois comutadores:
$$
[c_i, (c_j^\d c_k^\d) (c_l c_m)] = [ c_i, c_j^\d c_k^\d ] c_l c_m - c_j^\d c_k^\d \cancelto{0}{[ c_i, c_l c_m ]} =
$$
E agora em anticomutadores:
$$
= (\{c_i, c_j^\d\} c_k^\d - c_j^\d \{c_i, c_k^\d\}) c_l c_m = (\delta_{ij} c_k^\d - c_j^\d \delta_{ik}) c_l c_m
= \delta_{ij} c_k^\d c_l c_m + \delta_{ik} c_j^\d c_m c_l.
$$

\n

(e) Para a última relação, veja que $i \neq j$ (sítios distintos) e que basta considerar o caso $n_i = n_j = 1$, pois o caso $n_i = 0$ ou $n_j = 0$ é trivial, já que ambos os lados da relação dão zero ($c_i$ ou $c_j$ aniquilará $\ket{n_i n_j}$). Assim, no caso $n_i = n_j = 1$ nós temos
$$
\ev{c_k^\d c_l^\d c_i c_j}{n_i n_j} = \ev{c_j (c_i c_k^\d c_l^\d c_i c_j) c_i^\d c_j^\d}{00}
$$

Usando $(c_i c_k^\d) c_l^\d = (\delta_{ki} - c_k^\d c_i) c_l^\d = \delta_{ki} c_l^\d - c_k^\d (c_i c_l^\d) = \delta_{ki} c_l^\d - \delta_{li} c_k^\d - c_k^\d c_l c_i$, temos
$$
c_j (c_i c_k^\d c_l^\d) c_i c_j c_i^\d c_j^\d =
c_j (\delta_{ki} c_l^\d - \delta_{li} c_k^\d - c_k^\d c_l \cancelto{\boxed{c_i^2 = 0}}{c_i}) \textcolor{red}{c_i c_j} c_i^\d c_j^\d =
\textcolor{red}{-} (\delta_{ki} c_j c_l^\d - \delta_{li} c_j c_k^\d ) \textcolor{red}{c_j c_i} c_i^\d c_j^\d =
$$
$$
- \Big(\delta_{ki} (\delta_{lj} - c_l^\d \cancelto{\boxed{c_j^2 = 0}}{c_j}) - \delta_{li} (\delta_{kj} - c_k^\d \cancelto{\boxed{c_j^2 = 0}}{c_j}) \Big)
c_j c_i c_i^\d c_j^\d =
(\delta_{kj} \delta_{li} - \delta_{ki} \delta_{lj} ) c_j c_i c_i^\d c_j^\d =
$$
$$
= (\delta_{kj} \delta_{li} - \delta_{ki} \delta_{lj} ) c_j (1 - \nn_i) c_j^\d =
(\delta_{kj} \delta_{li} - \delta_{ki} \delta_{lj} ) (1 - \nn_i) (1 - \nn_j).
$$

Portanto
$$
\ev{c_k^\d c_l^\d c_i c_j}{n_i n_j} = (\delta_{kj} \delta_{li} - \delta_{ki} \delta_{lj} ) \ev{(1 - \nn_i) (1 - \nn_j)}{00}
= \delta_{kj} \delta_{li} - \delta_{ki} \delta_{lj} = (\delta_{kj} \delta_{li} - \delta_{ki} \delta_{lj}) n_i n_j,
$$
pois $n_i = n_j = 1$. Assim, temos que a relação é válida para todos $n_i, n_j \in \{0, 1\}$.




\pagebreak



\section{Interação de spin gerada pela repulsão Coulombiana}

(a) Como $U$ não depende do spin, temos que
$$
\mel{i\s_1, k \s_2}{U}{j \s_3, m \s_4} =
\braket{\s_1, \s_2}{\s_3, \s_4} \mel{ik}{U}{jm} =
\delta_{\s_1 \s_3} \delta_{\s_2, \s_4} \mel{ik}{U}{jm}.
$$

Assim, claramente temos
$$
I = \frac{1}{2} \sum_{ijkm} \sum_{\s_1\s_2\s_3\s_4}
\mel{i\s_1, k \s_2}{U}{j \s_3, m \s_4} c_{i\s_1}^\d c_{k\s_2}^\d c_{m\s_4} c_{j\s_3}=
$$
$$
\frac{1}{2} \sum_{ijkm} \sum_{\s_1\s_2}
\mel{ik}{U}{jm} c_{i\s_1}^\d c_{k\s_2}^\d c_{m\s_2} c_{j\s_1}
$$

Renomeando os índices $k = i'$, $m = j'$, $\s_1 = \s$ e $\s_2 = \s'$, obtemos
$$
I =
\frac{1}{2} \sum_{ii'jj'} \sum_{\s\s'}
\mel{ii'}{U}{jj'} c_{i\s}^\d c_{i'\s'}^\d c_{j'\s'} c_{j\s}.
$$

\n

(b) A contribuição direta $I_0$ é dada por
$$
I_0 =
\frac{1}{2} \sum_{\substack{i = j \\ i' = j' \\ i \neq i'}}
\sum_{\s\s'} \mel{ii'}{U}{jj'} c_{i\s}^\d c_{i'\s'}^\d c_{j'\s'} c_{j\s} =
\frac{1}{2} \sum_{i \neq i'} \sum_{\s\s'}
\mel{ii'}{U}{ii'} c_{i\s}^\d c_{i'\s'}^\d c_{i'\s'} c_{i\s} =
$$
$$
= \frac{1}{2} \sum_{i \neq i'}
\mel{ii'}{U}{ii'} \sum_{\s\s'} \nn_{i\s} \nn_{i'\s'} =
\sum_{i \neq i'}
\qty[ \textcolor{blue}{ \frac{1}{2} \mel{ii'}{U}{ii'} \nn_{i} \nn_{i'} } ].
$$

(c) O termo de troca $I_{X}$ é
$$
I_{X} =
\frac{1}{2} \sum_{\substack{j = i' \\ j' = i \\ i \neq i'}}
\sum_{\s\s'} \mel{ii'}{U}{jj'} c_{i\s}^\d c_{i'\s'}^\d c_{j'\s'} c_{j\s} =
\frac{1}{2} \sum_{i \neq i'} \sum_{\s\s'}
\mel{ii'}{U}{i'i} c_{i\s}^\d c_{i'\s'}^\d c_{i\s'} c_{i'\s} =
$$
$$
= \sum_{i \neq i'}
\mel{ii'}{U}{i'i} \frac{1}{2} \sum_{\s\s'} c_{i\s}^\d c_{i'\s'}^\d c_{i\s'} c_{i'\s}.
$$

Agora calcularemos $\vb{S}_i \vdot \vb{S}_{i'}$. Para isso, utilizarei a seguinte propriedade das matrizes de Pauli:
\begin{lema}{1}
Vale a seguinte relação de completeza das matrizes de Pauli:
$$
\bm{\s}_{\alpha \beta} \vdot \bm{\s}_{\gamma \delta} =
\sum_{k} \s_{\alpha \beta}^k \s_{\gamma \delta}^k =
2 \delta_{\alpha \delta} \delta_{\beta \gamma} - \delta_{\alpha \beta} \delta_{\gamma \delta}.
$$
\end{lema}

\begin{proof}
Dada uma matriz $M$ complexa $2 \times 2$ arbitrária, temos que $M = a_0 I + \sum_{k} a_k \s^k$, com $a_0, a_k \in \C$. Lembrando as propriedades básicas de traço $\tr(\s^k) = 0$ e $\tr(\s^j \s^k) = 2 \delta_{jk}$, temos que
$$
a_0 = \frac{1}{2} \tr(M) =
\frac{1}{2} \sum_{\gamma} M_{\gamma \gamma} =
\frac{1}{2} \sum_{\gamma \delta} \delta_{\gamma \delta} M_{\delta \gamma}
$$
$$
a_k = \frac{1}{2} \tr(\s^k M) =
\frac{1}{2} \sum_{\gamma} (\s^k M)_{\gamma \gamma} =
\frac{1}{2} \sum_{\gamma \delta}
\s^k_{\gamma \delta} M_{\delta \gamma}
$$

Substituindo $a_0, a_k$ e inserindo alguns $\delta$'s de Kronecker, temos
$$
2 M_{\alpha \beta} = 2 a_0 \delta_{\alpha \beta} + \sum_{k} 2 a_k \s^k_{\alpha \beta}
\implies
2 \sum_{\gamma \delta} \delta_{\alpha \delta} \delta_{\beta \gamma} M_{\delta \gamma}=
2 M_{\alpha \beta} =
\sum_{\gamma \delta} \delta_{\alpha \beta} \delta_{\gamma \delta} M_{\delta \gamma} +
\sum_{k} \s^k_{\alpha \beta}
\sum_{\gamma \delta} \s^k_{\gamma \delta} M_{\delta \gamma} \implies
$$
$$
\sum_{\gamma \delta} \qty(
2 \delta_{\alpha \delta} \delta_{\beta \gamma}
- \delta_{\alpha \beta} \delta_{\gamma \delta}
- \sum_{k} \s^k_{\alpha \beta} \s^k_{\gamma \delta}
) M_{\delta \gamma} = 0, \quad \ptodo M.
$$

Como a equação acima vale para toda matriz $M$, temos que a expressão dentro do parênteses deve ser zero.
\end{proof}

Temos então
$$
\vb{S}_i \vdot \vb{S}_{i'} =
\frac{1}{4} \sum_{\alpha \beta \gamma \delta}
c_{i\alpha}^\d c_{i\beta} c_{i'\gamma}^\d c_{i'\delta}
\qty( \sum_{k} \s^k_{\alpha \beta} \s^k_{\gamma \delta} )
= \frac{1}{4} \sum_{\alpha \beta \gamma \delta}
c_{i\alpha}^\d c_{i\beta} c_{i'\gamma}^\d c_{i'\delta}
\Big( 2 \delta_{\alpha \delta} \delta_{\beta \gamma}
- \delta_{\alpha \beta} \delta_{\gamma \delta} \Big) =
$$
$$
= \frac{1}{4} \qty( 2 \sum_{\alpha \beta}
c_{i\alpha}^\d c_{i\beta} c_{i'\beta}^\d c_{i'\alpha}
- \sum_{\alpha \gamma}
c_{i\alpha}^\d c_{i\alpha} c_{i'\gamma}^\d c_{i'\gamma}
)
= - \frac{1}{2} \sum_{\s \s'}
c_{i\s}^\d c_{i'\s'}^\d c_{i\s'} c_{i'\s} -
\frac{1}{4} \nn_i \nn_{i'}.
$$

Portanto
$$
I_X =
\sum_{i \neq i'}
\mel{ii'}{U}{i'i} \frac{1}{2} \sum_{\s\s'} c_{i\s}^\d c_{i'\s'}^\d c_{i\s'} c_{i'\s}=
\sum_{i \neq i'}
\qty{ \textcolor{blue}{ - \mel{ii'}{U}{i'i} \qty(
\vb{S}_i \vdot \vb{S}_{i'} + \frac{1}{4} \nn_i \nn_{i'} ) } }.
$$

A expressão acima para $I_X$ explica a regra de Hund que vemos na Química do Ensino Médio. Já que $U = e^2 / 4 \pi r > 0$, temos que $\mel{ii'}{U}{i'i} \geq 0$. Assim, para minimizar o termo de trocar $I_X$, é necessário que os spins $\vb{S}_i$ se alinhem. Por isso, ao preencher os elétrons nos orbitais, é energeticamente favorável primeiro preencher cada orbital com o mesmo alinhamento de spin, que é a regra de Hund essencialmente.


\pagebreak



\section{Representação de spins em termos de férmions e bósons}

Não escreverei o índice $i$ do sítio, pois todos os operadores aqui farão referência ao mesmo sítio $i$ fixo.

\n

(a) Neste item provarei que $[ \S^a, \S^b ] = i \eps_{abc} \S^c$ usando a representação fermiônica de Abrikosov.
$$
[ \S^a, \S^b ] =
\frac{1}{4} \sum_{\alpha \beta \gamma \delta}
\s^a_{\alpha \beta} \s^b_{\gamma \delta}
[ c_\alpha^\d c_\beta, c_\gamma^\d c_\delta ].
$$

Por propriedades do (anti)comutador temos
$$
[ c_\alpha^\d c_\beta, c_\gamma^\d c_\delta ] =
c_\alpha^\d [ c_\beta, c_\gamma^\d c_\delta ] +
[ c_\alpha^\d , c_\gamma^\d c_\delta ] c_\beta =
$$
$$
= c_\alpha^\d \Big( \{ c_\beta, c_\gamma^\d \} c_\delta -
c_\gamma \cancelto{0}{\{c_\beta, c_\delta\}} \Big) +
\Big( \cancelto{0}{\{ c_\alpha^\d , c_\gamma^\d \}} c_\delta -
c_\gamma^\d \{c_\alpha^\d, c_\delta\} \Big) c_\beta =
\delta_{\beta \gamma} c_\alpha^\d c_\delta -
\delta_{\alpha\delta} c_\gamma^\d c_\beta.
$$

Continuando:
$$
[ \S^a, \S^b ] =
\frac{1}{4} \sum_{\alpha \beta \gamma \delta}
\s^a_{\alpha \beta} \s^b_{\gamma \delta}
\Big( \delta_{\beta \gamma} c_\alpha^\d c_\delta -
\delta_{\alpha\delta} c_\gamma^\d c_\beta \Big) =
$$
$$
\frac{1}{4} \qty{ \sum_{\alpha \delta} c_\alpha^\d c_\delta
\sum_{\beta} \s^a_{\alpha \beta} \s^b_{\beta \delta}
-
\sum_{\beta \gamma} c_\gamma^\d c_\beta
\sum_{\alpha} \s^b_{\gamma \alpha} \s^a_{\alpha \beta}
} =
$$
$$
= \frac{1}{4} \qty{ \sum_{\alpha \delta} c_\alpha^\d c_\delta
(\s^a \s^b)_{\alpha \delta}
-
\sum_{\beta \gamma} c_\gamma^\d c_\beta
(\s^b \s^a)_{\gamma \beta}
} =
$$
trocando $\delta$ por $\beta$ na primeira somatória e $\gamma$ por $\alpha$ na segunda, e depois usando que $[\s^a, \s^b] = 2i \eps_{abc} \s^c$:
$$
= \frac{1}{4} \qty{ \sum_{\alpha \beta} c_\alpha^\d c_\beta
(\s^a \s^b)_{\alpha \beta}
-
\sum_{\alpha \beta} c_\alpha^\d c_\beta
(\s^b \s^a)_{\alpha \beta}
} =
\frac{1}{4} \sum_{\alpha \beta} c_\alpha^\d c_\beta
\Big(
(\s^a \s^b)_{\alpha \beta}
-
(\s^b \s^a)_{\alpha \beta} \Big) =
$$
$$
= \frac{1}{4} \sum_{\alpha \beta} c_\alpha^\d c_\beta
[\s^a , \s^b]_{\alpha \beta} =
i \eps_{abc} \, \frac{1}{2} \sum_{\alpha \beta}
c_\alpha^\d \s^c_{\alpha \beta} c_\beta = i \eps_{abc} \S^c.
$$

\n

(b) Neste item provarei que $[\S^z, \S^\pm] = \pm \S^\pm$ e $[\S^+, \S^-] = 2 \S^z$, que são as relações de comutação usuais para os operadores $\S^\pm, \S^z$. Denote $\nn = a^\d a$ e $F(\nn) = \qty(2S - \nn)^{1/2}$ para facilitar a notação.
$$
[\S^z, \S^+] = [ \cancelto{0}{S} - \nn , F(\nn) a] =
- \Big(
\cancelto{0}{[ \nn, F(\nn) ]} a + F(\nn) [ \nn, a ]
\Big) =
F(\nn) a = \S^+.
$$
$$
[\S^z, \S^-] = [ \cancelto{0}{S} - \nn , a^\d F(\nn)] =
- \Big(
[ \nn, a^\d ] F(\nn) + a^\d \cancelto{0}{[ \nn, F(\nn) ]}
\Big) =
- a^\d F(\nn) = - \S^-.
$$
$$
[\S^+, \S^-] = [ F(\nn) a, a^\d F(\nn) ] =
F(\nn) a a^\d F(\nn) - a^\d F(\nn)^2 a =
$$
$$
= F(\nn) (1 + \nn) F(\nn) - a^\d (2S - \nn) a =
$$
$$
= 2S - \nn + \cancel{2S \nn} - \nn^2 - \cancel{2S \nn} + (a^\d \nn) a =
$$
$$
= 2S - \nn - \nn^2 + (-a^\d + \nn a^\d) a =
$$
$$
= 2S - \nn - \cancel{\nn^2} - \nn + \cancel{\nn^2} =
$$
$$
= 2 (S - \nn) = 2 \S^z.
$$



\pagebreak



\section{Termodinâmica para bósons livres}


(a) A matriz densidade do sistema é $\hat{\rho} = \frac{e^{-\beta(\hat{H} - \mu \hat{N})}}{Z}$, com $Z = \tr(e^{-\beta(\hat{H} - \mu \hat{N})})$. Assim
$$
Z = \sum_{n_1 = 0}^{\infty} \cdots \sum_{n_N = 0}^{\infty}
e^{-\beta \sum_{j} (\eps_j - \mu) n_j} =
\prod_j \qty[ \sum_{n_j = 0}^{\infty}
\qty( e^{-\beta (\eps_j - \mu)} )^{n_j} ] =
\prod_j \frac{1}{1 - e^{-\beta(\eps_j - \mu)}},
$$
mas note que a série da P.G. acima só converge se $\mu < \eps_j, \ptodo j$. Temos
$$
\ln Z = - \sum_{j} \ln( 1 - e^{-\beta (\eps_j - \mu)} ),
$$
$$
\pdv{Z}{\eps_j} = \sum_{n_1 = 0}^{\infty} \cdots \sum_{n_N = 0}^{\infty}
(- \beta n_j) e^{-\beta \sum_{j} (\eps_j - \mu) n_j} =
- \beta \tr( \nn_j e^{-\beta(\hat{H} - \mu \hat{N})} ) \implies
$$
$$
\frac{1}{Z} \pdv{Z}{\eps_j} = - \beta \tr(\nn_j \hat{\rho}) = - \beta \ev{\nn_j}
\implies
\boxed{ \ev{\nn_j} = - \frac{1}{\beta} \pdv{\ln Z}{\eps_j}. }
$$

Portanto
$$
\ev{\nn_j} = - \frac{1}{\beta} \cdot
\frac{ - \beta e^{-\beta(\eps_j - \mu)}}{1 - e^{-\beta(\eps_j - \mu)}} \implies
\boxed{ \ev{\nn_j} = \frac{1}{e^{\beta (\eps_j - \mu)} - 1}. }
$$

Quando $\eps_j \to \mu^+$, note que $\ev{\nn_j} \to +\infty$. Diferentemente dos férmions, a distribuição $\ev{\nn_j}$ dos bósons tem $-1$ no denominador ao invés de um $+1$.

\n

(b) Primeiramente, como a relação de dispersão é $\eps(k) = ck$ onde $k$ é uma variável contínua com $0 \leq k \leq K$ ($K$ é um cutoff máximo), temos que $\mu < \eps(k) = ck, \ptodo k \implies \mu = 0$.

Fazendo a substituição $\sum_{j} = \frac{L}{2\pi} \int_0^K \dd{k}$, a energia interna $U$ é dada por
$$
U = \ev{\hat{H}} = \sum_{j} \eps_j \ev{\nn_j} =
\frac{L}{2\pi} \int_0^K \dd{k} \frac{c k}{e^{\beta ck} - 1} =
$$
$$
= \frac{L}{2\pi \beta^2 c} \int_0^K \frac{\beta c k}{e^{\beta ck} - 1} \beta c \dd{k} \implies
U = \frac{L}{2\pi \beta^2 c} \int_0^{\beta c K} \frac{x}{e^{x} - 1} \dd{x}.
$$

No limite em que $T \to 0$, temos que $\beta \to \infty$ e
$$
U \approx \frac{L}{2\pi \beta^2 c} \int_0^\infty \frac{x}{e^x - 1} \dd{x} =
\frac{L}{2\pi \beta^2 c} \cdot \frac{\pi^2}{6} =
\frac{k_{\text{B}}^2 L \pi}{12 c} \, T^2,
$$
onde calculei a integral $\int_0^\infty \frac{x}{e^x - 1} \dd{x} = \frac{\pi^2}{6}$ no Mathematica.

\n

Finalmente, no limite $T \to 0$:
$$
c_v = \pdv{U}{T} \approx \frac{\pi L \, k_{\text{B}}^2}{6 c} \, T.
$$


\end{document}
