\documentclass[a4paper,12pt]{article}
%\usepackage{mathtools}
\usepackage{amsthm}     % for definitions and theorems
\usepackage[many]{tcolorbox}    % boxes around definitions and theorems
%\usepackage{amsmath}
%\usepackage{nccmath}
\usepackage{amssymb}    % \ltimes, semi-direct product
%\usepackage{etoolbox}   % for start of Chapter
%\usepackage{amsfonts}
\usepackage{physics}    % for all Physics related
\usepackage{dsfont}     % for the identity matrix symbol \1
%\usepackage{mathrsfs}
\usepackage[notextcomp]{stix}   % font package and some symbols like filled square
%\usepackage{MnSymbol}   % symbols font package

\usepackage{titling}
\usepackage{indentfirst}

\usepackage{bm}
\usepackage[dvipsnames]{xcolor}
\usepackage{cancel}
\usepackage{enumitem}

\usepackage{xurl}
%\usepackage[colorlinks=true]{hyperref} % links have colors
\usepackage{hyperref}  % no colors

\usepackage{float}
\usepackage{graphicx}
\usepackage{subcaption}
%\usepackage{tikz}

\usepackage{ctable}     % tabelas
\renewcommand{\P}{\phantom{+}}  % empty space to indent things
\usepackage{multirow}
\usepackage{tabulary}

%%%%%%%%%%%%%%%%%%%%%%%%%%%%%%%%%%%%%%%%%%%%%%%%%%%

\newcommand{\eps}{\epsilon}
\newcommand{\vphi}{\varphi}
\newcommand{\cte}{\text{cte}}

\newcommand{\N}{{\mathbb{N}}}
\newcommand{\Z}{{\mathbb{Z}}}
%\newcommand{\Q}{{\mathbb{Q}}}
\newcommand{\C}{{\mathbb{C}}}
\renewcommand{\S}{{\hat{S}}}
%\renewcommand{\H}{\s{H}}

\renewcommand{\a}{{\vb{a}}}
\renewcommand{\b}{{\vb{b}}}
\renewcommand{\d}{{\dagger}}
\newcommand{\up}{{\uparrow}}
\newcommand{\down}{{\downarrow}}
\newcommand{\hc}{{\text{h.c.}}}

\newcommand{\ihat}{\bm{\hat{\imath}}}
\newcommand{\jhat}{\bm{\hat{\jmath}}}
\newcommand{\khat}{\bm{\hat{k}}}

\newcommand{\0}{{\vb{0}}}
\newcommand{\1}{\mathds{1}}
\newcommand{\E}{{\vb{E}}}
\newcommand{\B}{{\vb{B}}}
\renewcommand{\u}{{\vb{u}}}
\renewcommand{\v}{{\vb{v}}}
\renewcommand{\r}{{\vb{r}}}
\newcommand{\R}{{\vb{R}}}
\newcommand{\Q}{{\vb{Q}}}
\newcommand{\G}{{\vb{G}}}
\newcommand{\g}{{\vb{g}}}
\renewcommand{\k}{{\vb{k}}}
\newcommand{\K}{{\vb{K}}}
\newcommand{\p}{{\vb{p}}}
\newcommand{\q}{{\vb{q}}}
\newcommand{\F}{{\vb{F}}}
\renewcommand{\t}{{\vb{t}}}
\newcommand{\vtau}{{\bm{\tau}}}
\newcommand{\vdelta}{{\bm{\delta}}}

% COLORED SYMMETRY ELEMENTS
\newcommand{\Ct}{{\textcolor{Cyan}{C_3}}}
\newcommand{\Ctn}[1]{{\textcolor{Cyan}{C_3^{\textcolor{black}{#1}}}}}
\newcommand{\Cs}{{\textcolor{ForestGreen}{C_6}}}
\newcommand{\Csn}[1]{{\textcolor{ForestGreen}{C_6^{\textcolor{black}{#1}}}}}
\newcommand{\sd}{{\textcolor{RoyalBlue}{\sigma_d}}}
\newcommand{\sdn}[1]{{\textcolor{RoyalBlue}{\sigma_d^{\textcolor{black}{#1}}}}}
\newcommand{\sdp}{{\textcolor{RoyalBlue}{\sigma_d'}}}
\newcommand{\sdpp}{{\textcolor{RoyalBlue}{\sigma_d''}}}
\newcommand{\sv}{{\textcolor{Orange}{\sigma_v}}}
\newcommand{\svn}[1]{{\textcolor{Orange}{\sigma_v^{\textcolor{black}{#1}}}}}
\newcommand{\svp}{{\textcolor{Orange}{\sigma_v'}}}
\newcommand{\svpp}{{\textcolor{Orange}{\sigma_v''}}}

\newcommand{\GL}{{\text{GL}}}
\newcommand{\U}{{\text{U}}}

\newcommand{\s}{\sigma}
%\newcommand{\prodint}[2]{\left\langle #1 , #2 \right\rangle}
\newcommand{\cc}[1]{\overline{#1}}
\newcommand{\Eval}[3]{\eval{\left( #1 \right)}_{#2}^{#3}}
\newcommand{\sg}[2]{\{ #1 \mid #2 \}}
\renewcommand{\AA}{{\mathring{\text{A}}}}
\newcommand{\I}{{\mathbb{I}}}
\newcommand{\bP}{{\mathbb{P}}}
\newcommand{\bQ}{{\mathbb{Q}}}

\newcommand{\unit}[1]{\; \mathrm{#1}}

\newcommand{\n}{\medskip}
\newcommand{\e}{\quad \mathrm{and} \quad}
\newcommand{\ou}{\quad \mathrm{or} \quad}
\newcommand{\virg}{\, , \;}
\newcommand{\ptodo}{\forall \,}
\renewcommand{\implies}{\; \Rightarrow \;}
%\newcommand{\eqname}[1]{\tag*{#1}} % Tag equation with name

%\setlength{\droptitle}{-7em}   % título um pouco mais em cima na página
%\makeatletter
%\patchcmd{\chapter}{\if@openright\cleardoublepage\else\clearpage\fi}{}{}{}  % start 'Chapter' at the same page. needs package etoolbox
%\makeatother

%% Theorems, definitions, proofs
\theoremstyle{definition}

%%% defining my own colors %%%
\definecolor{my-blue}{HTML}{f2f4ff}
\definecolor{my-green}{HTML}{f5fcf6}    % a little better: green!5!white
\definecolor{my-cyan}{HTML}{f2fffe}
\definecolor{my-yellow}{HTML}{fffbed}
\definecolor{my-green2}{HTML}{efffdb}

%%% alternative colors %%%
\definecolor{my-pink}{HTML}{fff2f7}
\definecolor{my-teal}{HTML}{ebfffc}

\newtheorem{definition}{Definition}[section]
\tcolorboxenvironment{definition}{
  colback=my-blue,
  %colback=blue!5!white,
  boxrule=0.1pt,
  boxsep=1pt,
  left=2pt,right=2pt,top=2pt,bottom=2pt,
  oversize=2pt,
  sharp corners,
  before skip=\topsep,
  after skip=\topsep,
}

\newtheorem{theorem}{Theorem}[section]
\tcolorboxenvironment{theorem}{
  colback=my-yellow,
  %colback=yellow!22!white!95!black,
  boxrule=0.1pt,
  boxsep=1pt,
  left=2pt,right=2pt,top=2pt,bottom=2pt,
  oversize=2pt,
  sharp corners,
  before skip=\topsep,
  after skip=\topsep,
}

\newtheorem{corollary}{Corollary}[section]
\tcolorboxenvironment{corollary}{
  colback=my-green2,
  boxrule=0.1pt,
  boxsep=1pt,
  left=2pt,right=2pt,top=2pt,bottom=2pt,
  oversize=2pt,
  sharp corners,
  before skip=\topsep,
  after skip=\topsep,
}

\newtheorem{lemma}{Lemma}[section]
\tcolorboxenvironment{lemma}{
  colback=my-cyan,
  boxrule=0.1pt,
  boxsep=1pt,
  left=2pt,right=2pt,top=2pt,bottom=2pt,
  oversize=2pt,
  sharp corners,
  before skip=\topsep,
  after skip=\topsep,
}

\newtheorem{example}{Example}[section]
\tcolorboxenvironment{example}{
  %colback=my-green,
  colback=green!5!white,
  boxrule=0.1pt,
  boxsep=1pt,
  left=2pt,right=2pt,top=2pt,bottom=2pt,
  oversize=2pt,
  sharp corners,
  before skip=\topsep,
  after skip=\topsep,
}


\title{\Huge{\textbf{Lista 1}}}
\author{Mateus Marques}

\begin{document}

\maketitle


\section{Teoria de perturbação degenerada}

A hamiltoniana perturbada é dada por $H = H_0 + \lambda V$. Seja $\ket{n \alpha_n}$ os autovetores de $H$ com autovalores $E_n$ (onde $\alpha_n$ indexa a degenerescência da energia $E_n$) que formam uma base no espaço de Hilbert $\s{H}$. Seja $\s{H}_n$ o subespaço degenerado de energia $E_n$. Podemos dividir o espaço total numa soma direta $\s{H} = \s{H}_n \oplus \s{H}_n^\perp$, onde $\s{H}_n^\perp$ é o espaço ortogonal a $\s{H}_n$. Em termos de projeções, definimos
$$
P = \sum_{\alpha_n} \ketbra{n \alpha_n} \e
Q = \sum_{m \neq n} \sum_{\alpha_m} \ketbra{m \alpha_m}, \text{ com }
P + Q = 1, PQ = QP = 0,
$$
onde $P$ projeta em $\s{H}_n$ e $Q$ projeta em $\s{H}_n^\perp$.

Queremos encontrar um novo autovalor $E(\lambda)$ da hamiltoniana perturbada $H_0 + \lambda V$ que nasce da degenerescência em $E_n$, ou seja $\displaystyle{\lim_{\lambda \to 0} E(\lambda) = E_n}$. Temos que

\begin{equation} \label{eq:pertub}
(E - H_0) \ket{\psi} = \lambda V \ket{\psi}.
\end{equation}

Considere o operador $R$, que é a inversa de $(E - H_0)$ dentro do subespaço $\s{H}_n^\perp$
$$
R = \sum_{m \neq n} \sum_{\alpha_m} \frac{\ketbra{m \alpha_m}}{E - E_m}.
$$

Note que $R$ satisfaz $R (E - H_0) = (E - H_0) R = Q$. Aplicando $R$ dos dois lados de \ref{eq:pertub}:
$$
Q \ket{\psi} = \lambda R V \ket{\psi}.
$$

Portanto, temos
\begin{equation} \label{eq:root}
\ket{\psi} = P \ket{\psi} + Q \ket{\psi} =
P \ket{\psi} + \lambda R V \ket{\psi}.
\end{equation}

Através de \ref{eq:root}, podemos obter uma equação perturbativa para $\ket{\psi}$ em função de $P \ket{\psi}$:
\begin{equation} \label{eq:recorr}
\ket{\psi} = P \ket{\psi} + \lambda R V P \ket{\psi} + \lambda^2 R V R V P \ket{\psi} + O(\lambda^3).
\end{equation}

Agora, temos que $P \ket{\psi}$ se escreve da forma
$$
P \ket{\psi} = \sum_{\alpha_n} \ket{n \alpha_n} \braket{n\alpha_n}{\psi} =
\sum_{\alpha_n} c_{\alpha_n} \ket{n \alpha_n}, \text{ com } c_{\alpha_n} = \braket{n\alpha_n}{\psi}.
$$

Aplicando $\bra{n \alpha_n}$ na equação \ref{eq:pertub} e substituindo \ref{eq:recorr} até primeira ordem $\lambda$, temos
$$
(E - E_n) c_{\alpha_n} = \lambda \mel{n \alpha_n}{V}{\psi} =
\lambda \bra{n \alpha_n} V \Bigg\{ 1 + \lambda R V + O(\lambda^2) \Bigg\} P \ket{\psi} =
$$
$$
=
\lambda \bra{n \alpha_n} V \Bigg\{ 1 + \lambda R V + O(\lambda^2) \Bigg\} \sum_{\beta_n} c_{\beta_n} \ket{n \beta_n} \implies
$$
\begin{equation} \label{eq:sol1}
(E - E_n) c_{\alpha_n} =
\sum_{\beta_n} \Bigg\{
\lambda \mel{n \alpha_n}{V}{n \beta_n} + \lambda^2 \sum_{\substack{m \neq n \\ \gamma_m}}
\frac{\mel{n \alpha_n}{V}{m \gamma_m} \mel{m \gamma_m}{V}{n \beta_n}}{E_n - E_m} + O(\lambda^3)
\Bigg\} c_{\beta_n},
\end{equation}
onde expandimos
$$
R = \sum_{m \neq n} \sum_{\alpha_m} \frac{\ketbra{m \alpha_m}}{\textcolor{red}{E} - E_m}
= \sum_{m \neq n} \sum_{\alpha_m} \frac{\ketbra{m \alpha_m}}{\textcolor{red}{E_n} - E_m} + O(\lambda).
$$

Agora basta que analisemos a equação \ref{eq:sol1}. Perceba que temos liberdade para escolher $\ket{n \alpha_n}$ como bem entendermos (dentro do subespaço degenerado $\s{H}_n$). Em particular, escolhamos $\ket{n \alpha_n}$ de maneira que $\displaystyle{\lim_{\lambda \to 0} P \ket{\psi} (\lambda) = \ket{n \alpha_n}}$. Dessa maneira, temos que os coeficientes $c_{\beta_n}(\lambda)$ obedecem
\begin{equation} \label{eq:coeff}
c_{\beta_n}(\lambda) = \delta_{\alpha_n \beta_n} + O(\lambda).
\end{equation}

Substituindo \ref{eq:coeff} na equação \ref{eq:sol1}, obtemos a correção de segunda ordem de $E$ (proporcional a $\lambda^2$) como
$$
E^{(2)} = \sum_{\substack{m \neq n \\ \gamma_m}} \frac{\abs{\mel{n \alpha_n}{V}{m \gamma_m}}^2}{E_n - E_m}.
\quad \qedsymbol
$$

Observação: É claro que em teoria de perturbação degenerada existem \textit{splittings} que acontecem em primeira e segunda ordem. De certa maneira, estamos ignorando esses splittings e focando em obter a energia média em segunda ordem.



\pagebreak



\section{Operadores de criação e aniquilação}

oi


\pagebreak



\section{Interação de spin gerada pela repulsão Coulombiana}



\pagebreak



\section{Representação de spins em termos de férmions e bósons}



\pagebreak



\section{Termodinâmica para bósons livres}



\end{document}
