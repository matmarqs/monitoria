\documentclass[a4paper,10pt]{article}
\usepackage[brazilian]{babel}
\usepackage[left=2.5cm,right=2.5cm,top=3cm,bottom=2.5cm]{geometry}
\usepackage{mathtools}
\usepackage{amsthm}
\usepackage{amsmath}
%\usepackage{nccmath}
\usepackage{amssymb}
\usepackage{amsfonts}
\usepackage{physics}
%\usepackage{dsfont}
%\usepackage{mathrsfs}

\usepackage{titling}
\usepackage{indentfirst}

\usepackage{bm}
\usepackage[dvipsnames]{xcolor}
\usepackage{cancel}

\usepackage{xurl}
\usepackage[colorlinks=true]{hyperref}

\usepackage{float}
\usepackage{graphicx}
%\usepackage{tikz}
\usepackage{caption}
\usepackage{subcaption}

%%%%%%%%%%%%%%%%%%%%%%%%%%%%%%%%%%%%%%%%%%%%%%%%%%%

\newcommand{\eps}{\epsilon}
\newcommand{\vphi}{\varphi}
\newcommand{\cte}{\text{cte}}

\newcommand{\N}{\mathbb{N}}
\newcommand{\Z}{\mathbb{Z}}
\newcommand{\Q}{\mathbb{Q}}
\newcommand{\R}{\vb{R}}
\newcommand{\C}{\mathbb{C}}
\renewcommand{\S}{\hat{S}}
%\renewcommand{\H}{\s{H}}

\renewcommand{\a}{\vb{a}}
\newcommand{\nn}{\hat{n}}
\renewcommand{\d}{\dagger}
\newcommand{\up}{\uparrow}
\newcommand{\down}{\downarrow}

\newcommand{\0}{\vb{0}}
%\newcommand{\1}{\mathds{1}}
\newcommand{\E}{\vb{E}}
\newcommand{\B}{\vb{B}}
\renewcommand{\v}{\vb{v}}
\renewcommand{\r}{\vb{r}}
\renewcommand{\k}{\vb{k}}
\newcommand{\p}{\vb{p}}
\newcommand{\q}{\vb{q}}
\newcommand{\F}{\vb{F}}

\newcommand{\s}{\sigma}
%\newcommand{\prodint}[2]{\left\langle #1 , #2 \right\rangle}
\newcommand{\cc}[1]{\overline{#1}}
\newcommand{\Eval}[3]{\eval{\left( #1 \right)}_{#2}^{#3}}

\newcommand{\unit}[1]{\; \mathrm{#1}}

\newcommand{\n}{\medskip}
\newcommand{\e}{\quad \mathrm{e} \quad}
\newcommand{\ou}{\quad \mathrm{ou} \quad}
\newcommand{\virg}{\, , \;}
\newcommand{\ptodo}{\forall \,}
\renewcommand{\implies}{\; \Rightarrow \;}
%\newcommand{\eqname}[1]{\tag*{#1}} % Tag equation with name

\setlength{\droptitle}{-7em}

\theoremstyle{plain}
\newtheorem{theorem}{Teorema}[section]
%\newtheorem{defi}[theorem]{Definição}
\newtheorem{lemma}[theorem]{Lema}
%\newtheorem{corol}[theorem]{Corolário}
%\newtheorem{prop}[theorem]{Proposição}
%\newtheorem{example}{Exemplo}
%
%\newtheorem{inneraxiom}{Axioma}
%\newenvironment{axioma}[1]
%  {\renewcommand\theinneraxiom{#1}\inneraxiom}
%  {\endinneraxiom}
%
%\newtheorem{innerpostulado}{Postulado}
%\newenvironment{postulado}[1]
%  {\renewcommand\theinnerpostulado{#1}\innerpostulado}
%  {\endinnerpostulado}
%
%\newtheorem{innerexercise}{Exercício}
%\newenvironment{exercise}[1]
%  {\renewcommand\theinnerexercise{#1}\innerexercise}
%  {\endinnerexercise}
%
%\newtheorem{innerthm}{Teorema}
%\newenvironment{teorema}[1]
%  {\renewcommand\theinnerthm{#1}\innerthm}
%  {\endinnerthm}
%
\newtheorem{innerlema}{Lema}
\newenvironment{lema}[1]
  {\renewcommand\theinnerlema{#1}\innerlema}
  {\endinnerlema}
%
%\theoremstyle{remark}
%\newtheorem*{hint}{Dica}
%\newtheorem*{notation}{Notação}
%\newtheorem*{obs}{Observação}


\title{\Huge{\textbf{Lista 1}}}
\author{Mateus Marques}

\begin{document}

\maketitle


\section{Teoria de perturbação degenerada}

A hamiltoniana perturbada é dada por $H = H_0 + \lambda V$. Seja $\ket{n \alpha_n}$ os autovetores de $H$ com autovalores $E_n$ (onde $\alpha_n$ indexa a degenerescência da energia $E_n$) que formam uma base no espaço de Hilbert $\s{H}$. Seja $\s{H}_n$ o subespaço degenerado de energia $E_n$. Podemos dividir o espaço total numa soma direta $\s{H} = \s{H}_n \oplus \s{H}_n^\perp$, onde $\s{H}_n^\perp$ é o espaço ortogonal a $\s{H}_n$. Em termos de projeções, definimos
$$
P = \sum_{\alpha_n} \ketbra{n \alpha_n} \e
Q = \sum_{m \neq n} \sum_{\alpha_m} \ketbra{m \alpha_m}, \text{ com }
P + Q = 1, PQ = QP = 0,
$$
onde $P$ projeta em $\s{H}_n$ e $Q$ projeta em $\s{H}_n^\perp$.

Queremos encontrar um novo autovalor $E(\lambda)$ da hamiltoniana perturbada $H_0 + \lambda V$ que nasce da degenerescência em $E_n$, ou seja $\displaystyle{\lim_{\lambda \to 0} E(\lambda) = E_n}$. Temos que

\begin{equation} \label{eq:pertub}
(E - H_0) \ket{\psi} = \lambda V \ket{\psi}.
\end{equation}

Considere o operador $R$, que é a inversa de $(E - H_0)$ dentro do subespaço $\s{H}_n^\perp$
$$
R = \sum_{m \neq n} \sum_{\alpha_m} \frac{\ketbra{m \alpha_m}}{E - E_m}.
$$

Note que $R$ satisfaz $R (E - H_0) = (E - H_0) R = Q$. Aplicando $R$ dos dois lados de \ref{eq:pertub}:
$$
Q \ket{\psi} = \lambda R V \ket{\psi}.
$$

Portanto, temos
\begin{equation} \label{eq:root}
\ket{\psi} = P \ket{\psi} + Q \ket{\psi} =
P \ket{\psi} + \lambda R V \ket{\psi}.
\end{equation}

Através de \ref{eq:root}, podemos obter uma equação perturbativa para $\ket{\psi}$ em função de $P \ket{\psi}$:
\begin{equation} \label{eq:recorr}
\ket{\psi} = P \ket{\psi} + \lambda R V P \ket{\psi} + \lambda^2 R V R V P \ket{\psi} + O(\lambda^3).
\end{equation}

Agora, temos que $P \ket{\psi}$ se escreve da forma
$$
P \ket{\psi} = \sum_{\alpha_n} \ket{n \alpha_n} \braket{n\alpha_n}{\psi} =
\sum_{\alpha_n} c_{\alpha_n} \ket{n \alpha_n}, \text{ com } c_{\alpha_n} = \braket{n\alpha_n}{\psi}.
$$

Aplicando $\bra{n \alpha_n}$ na equação \ref{eq:pertub} e substituindo \ref{eq:recorr} até primeira ordem $\lambda$, temos
$$
(E - E_n) c_{\alpha_n} = \lambda \mel{n \alpha_n}{V}{\psi} =
\lambda \bra{n \alpha_n} V \Bigg\{ 1 + \lambda R V + O(\lambda^2) \Bigg\} P \ket{\psi} =
$$
$$
=
\lambda \bra{n \alpha_n} V \Bigg\{ 1 + \lambda R V + O(\lambda^2) \Bigg\} \sum_{\beta_n} c_{\beta_n} \ket{n \beta_n} \implies
$$
\begin{equation} \label{eq:sol1}
(E - E_n) c_{\alpha_n} =
\sum_{\beta_n} \Bigg\{
\lambda \mel{n \alpha_n}{V}{n \beta_n} + \lambda^2 \sum_{\substack{m \neq n \\ \gamma_m}}
\frac{\mel{n \alpha_n}{V}{m \gamma_m} \mel{m \gamma_m}{V}{n \beta_n}}{E_n - E_m} + O(\lambda^3)
\Bigg\} c_{\beta_n},
\end{equation}
onde expandimos
$$
R = \sum_{m \neq n} \sum_{\alpha_m} \frac{\ketbra{m \alpha_m}}{\textcolor{red}{E} - E_m}
= \sum_{m \neq n} \sum_{\alpha_m} \frac{\ketbra{m \alpha_m}}{\textcolor{red}{E_n} - E_m} + O(\lambda).
$$

Agora basta que analisemos a equação \ref{eq:sol1}. Perceba que temos liberdade para escolher $\ket{n \alpha_n}$ como bem entendermos (dentro do subespaço degenerado $\s{H}_n$). Em particular, escolhamos $\ket{n \alpha_n}$ de maneira que $\displaystyle{\lim_{\lambda \to 0} P \ket{\psi} (\lambda) = \ket{n \alpha_n}}$. Dessa maneira, temos que os coeficientes $c_{\beta_n}(\lambda)$ obedecem
\begin{equation} \label{eq:coeff}
c_{\beta_n}(\lambda) = \delta_{\alpha_n \beta_n} + O(\lambda).
\end{equation}

Substituindo \ref{eq:coeff} na equação \ref{eq:sol1}, obtemos a correção de segunda ordem de $E$ (proporcional a $\lambda^2$) como
$$
E^{(2)} = \sum_{\substack{m \neq n \\ \gamma_m}} \frac{\abs{\mel{n \alpha_n}{V}{m \gamma_m}}^2}{E_n - E_m}.
\quad \qedsymbol
$$

Observação: É claro que em teoria de perturbação degenerada existem \textit{splittings} que acontecem em primeira e segunda ordem. De certa maneira, estamos ignorando esses splittings e focando em obter a energia média em segunda ordem.



\pagebreak



\section{Operadores de criação e aniquilação}

Os operadores bosônicos satisfazem $[a_i, a_j] = [a_i^\dagger, a_j^\dagger] = 0$ e $[a_i, a_j^\dagger] = \delta_{ij}$. Usarei bastante a propriedade básica do comutador $[A, BC] = [A, B] C + B [A, C]$.

\n

(a) Definindo $\nn_j = a_j^\d a_j$, temos então
$$
[a_j^\d, \nn_j] = [a_j^\d, a_j^\d a_j] = a_j^\d \cancelto{\boxed{= -1}}{[a_j^\d, a_j]} + \cancelto{0}{[a_j^\d,a_j^\d]} a_j = - a_j^\d.
$$

$$
[a_j, \nn_j] = [a_j, a_j^\d a_j] = a_j^\d \cancelto{0}{[a_j, a_j]} + \cancelto{\boxed{=1}}{[a_j,a_j^\d]} a_j = a_j.
$$

\n

(b) Temos
$$
[a_i, (a_j^\d a_k^\d) (a_l a_m) ] = [a_i, a_j^\d a_k^\d] a_l a_m + a_j^\d a_k^\d \cancelto{0}{[a_i, a_l a_m]} =
$$
$$
(\cancelto{\boxed{= \delta_{ij}}}{[a_i, a_j^\d]} a_k^\d + a_j^\d \cancelto{\boxed{= \delta_{ik}}}{[a_i, a_k^\d]}) a_l a_m =
\delta_{ij} a_k^\d a_l a_m  +  \delta_{ik} a_j^\d a_m a_l.
$$

\n

Para operadores fermiônicos temos $\{c_i, c_j\} = \{c_i^\dagger, c_j^\dagger\} = 0$ e $\{c_i, c_j^\dagger\} = \delta_{ij}$. Também usarei a propriedade elementar do (anti)comutador $[A, BC] = \{A, B\} C - B \{A, C\}$.

\n

(c) Temos então
$$
[c_j^\d, \nn_j] = [ c_j^\d, c_j^\d c_j ] = \cancelto{0}{\{c_j^\d, c_j^\d\}} c_j - c_j^\d \cancelto{\boxed{=1}}{\{c_j^\d, c_j\}} = - c_j^\d.
$$
$$
[c_j, \nn_j] = [ c_j, c_j^\d c_j ] = \cancelto{\boxed{=1}}{\{c_j, c_j^\d\}} c_j - c_j^\d \cancelto{0}{\{c_j, c_j\}} = c_j.
$$

\n

(d) Expandindo o comutador em dois comutadores:
$$
[c_i, (c_j^\d c_k^\d) (c_l c_m)] = [ c_i, c_j^\d c_k^\d ] c_l c_m - c_j^\d c_k^\d \cancelto{0}{[ c_i, c_l c_m ]} =
$$
E agora em anticomutadores:
$$
= (\{c_i, c_j^\d\} c_k^\d - c_j^\d \{c_i, c_k^\d\}) c_l c_m = (\delta_{ij} c_k^\d - c_j^\d \delta_{ik}) c_l c_m
= \delta_{ij} c_k^\d c_l c_m + \delta_{ik} c_j^\d c_m c_l.
$$

\n

(e) Para a última relação, veja que $i \neq j$ (sítios distintos) e que basta considerar o caso $n_i = n_j = 1$, pois o caso $n_i = 0$ ou $n_j = 0$ é trivial, já que ambos os lados da relação dão zero ($c_i$ ou $c_j$ aniquilará $\ket{n_i n_j}$). Assim, no caso $n_i = n_j = 1$ nós temos
$$
\ev{c_k^\d c_l^\d c_i c_j}{n_i n_j} = \ev{c_j (c_i c_k^\d c_l^\d c_i c_j) c_i^\d c_j^\d}{00}
$$

Usando $(c_i c_k^\d) c_l^\d = (\delta_{ki} - c_k^\d c_i) c_l^\d = \delta_{ki} c_l^\d - c_k^\d (c_i c_l^\d) = \delta_{ki} c_l^\d - \delta_{li} c_k^\d - c_k^\d c_l c_i$, temos
$$
c_j (c_i c_k^\d c_l^\d) c_i c_j c_i^\d c_j^\d =
c_j (\delta_{ki} c_l^\d - \delta_{li} c_k^\d - c_k^\d c_l \cancelto{\boxed{c_i^2 = 0}}{c_i}) \textcolor{red}{c_i c_j} c_i^\d c_j^\d =
\textcolor{red}{-} (\delta_{ki} c_j c_l^\d - \delta_{li} c_j c_k^\d ) \textcolor{red}{c_j c_i} c_i^\d c_j^\d =
$$
$$
- \Big(\delta_{ki} (\delta_{lj} - c_l^\d \cancelto{\boxed{c_j^2 = 0}}{c_j}) - \delta_{li} (\delta_{kj} - c_k^\d \cancelto{\boxed{c_j^2 = 0}}{c_j}) \Big)
c_j c_i c_i^\d c_j^\d =
(\delta_{kj} \delta_{li} - \delta_{ki} \delta_{lj} ) c_j c_i c_i^\d c_j^\d =
$$
$$
= (\delta_{kj} \delta_{li} - \delta_{ki} \delta_{lj} ) c_j (1 - \nn_i) c_j^\d =
(\delta_{kj} \delta_{li} - \delta_{ki} \delta_{lj} ) (1 - \nn_i) (1 - \nn_j).
$$

Portanto
$$
\ev{c_k^\d c_l^\d c_i c_j}{n_i n_j} = (\delta_{kj} \delta_{li} - \delta_{ki} \delta_{lj} ) \ev{(1 - \nn_i) (1 - \nn_j)}{00}
= \delta_{kj} \delta_{li} - \delta_{ki} \delta_{lj} = (\delta_{kj} \delta_{li} - \delta_{ki} \delta_{lj}) n_i n_j,
$$
pois $n_i = n_j = 1$. Assim, temos que a relação é válida para todos $n_i, n_j \in \{0, 1\}$.




\pagebreak



\section{Interação de spin gerada pela repulsão Coulombiana}



\pagebreak



\section{Representação de spins em termos de férmions e bósons}



\pagebreak



\section{Termodinâmica para bósons livres}



\end{document}
