\documentclass[a4paper,10pt]{article}
\usepackage[brazilian]{babel}
\usepackage[left=2.5cm,right=2.5cm,top=3cm,bottom=2.5cm]{geometry}
\usepackage{mathtools}
\usepackage{amsthm}
\usepackage{amsmath}
%\usepackage{nccmath}
\usepackage{amssymb}
\usepackage{amsfonts}
\usepackage{physics}
%\usepackage{dsfont}
%\usepackage{mathrsfs}

\usepackage{titling}
\usepackage{indentfirst}

\usepackage{bm}
\usepackage[dvipsnames]{xcolor}
\usepackage{cancel}

\usepackage{xurl}
\usepackage[colorlinks=true]{hyperref}

\usepackage{float}
\usepackage{graphicx}
%\usepackage{tikz}
\usepackage{caption}
\usepackage{subcaption}

%%%%%%%%%%%%%%%%%%%%%%%%%%%%%%%%%%%%%%%%%%%%%%%%%%%

\newcommand{\eps}{\epsilon}
\newcommand{\vphi}{\varphi}
\newcommand{\cte}{\text{cte}}

\newcommand{\N}{\mathbb{N}}
\newcommand{\Z}{\mathbb{Z}}
\newcommand{\Q}{\mathbb{Q}}
\newcommand{\R}{\vb{R}}
\newcommand{\C}{\mathbb{C}}
\renewcommand{\S}{\hat{S}}
%\renewcommand{\H}{\s{H}}

\renewcommand{\a}{\vb{a}}
\newcommand{\nn}{\hat{n}}
\renewcommand{\d}{\dagger}
\newcommand{\up}{\uparrow}
\newcommand{\down}{\downarrow}

\newcommand{\0}{\vb{0}}
%\newcommand{\1}{\mathds{1}}
\newcommand{\E}{\vb{E}}
\newcommand{\B}{\vb{B}}
\renewcommand{\v}{\vb{v}}
\renewcommand{\r}{\vb{r}}
\renewcommand{\k}{\vb{k}}
\newcommand{\p}{\vb{p}}
\newcommand{\q}{\vb{q}}
\newcommand{\F}{\vb{F}}

\newcommand{\s}{\sigma}
%\newcommand{\prodint}[2]{\left\langle #1 , #2 \right\rangle}
\newcommand{\cc}[1]{\overline{#1}}
\newcommand{\Eval}[3]{\eval{\left( #1 \right)}_{#2}^{#3}}

\newcommand{\unit}[1]{\; \mathrm{#1}}

\newcommand{\n}{\medskip}
\newcommand{\e}{\quad \mathrm{e} \quad}
\newcommand{\ou}{\quad \mathrm{ou} \quad}
\newcommand{\virg}{\, , \;}
\newcommand{\ptodo}{\forall \,}
\renewcommand{\implies}{\; \Rightarrow \;}
%\newcommand{\eqname}[1]{\tag*{#1}} % Tag equation with name

\setlength{\droptitle}{-7em}

\theoremstyle{plain}
\newtheorem{theorem}{Teorema}[section]
%\newtheorem{defi}[theorem]{Definição}
\newtheorem{lemma}[theorem]{Lema}
%\newtheorem{corol}[theorem]{Corolário}
%\newtheorem{prop}[theorem]{Proposição}
%\newtheorem{example}{Exemplo}
%
%\newtheorem{inneraxiom}{Axioma}
%\newenvironment{axioma}[1]
%  {\renewcommand\theinneraxiom{#1}\inneraxiom}
%  {\endinneraxiom}
%
%\newtheorem{innerpostulado}{Postulado}
%\newenvironment{postulado}[1]
%  {\renewcommand\theinnerpostulado{#1}\innerpostulado}
%  {\endinnerpostulado}
%
%\newtheorem{innerexercise}{Exercício}
%\newenvironment{exercise}[1]
%  {\renewcommand\theinnerexercise{#1}\innerexercise}
%  {\endinnerexercise}
%
%\newtheorem{innerthm}{Teorema}
%\newenvironment{teorema}[1]
%  {\renewcommand\theinnerthm{#1}\innerthm}
%  {\endinnerthm}
%
\newtheorem{innerlema}{Lema}
\newenvironment{lema}[1]
  {\renewcommand\theinnerlema{#1}\innerlema}
  {\endinnerlema}
%
%\theoremstyle{remark}
%\newtheorem*{hint}{Dica}
%\newtheorem*{notation}{Notação}
%\newtheorem*{obs}{Observação}


\title{\Huge{\textbf{Lista 4 - Matéria Condensada 2}}}
\author{Mateus Marques}

\begin{document}

\maketitle

\section{Modelo de Hubbard}

(a) Na questão 3 da Lista 1 eu cheguei a calcular $\S_i \vdot \S_j$. Lá eu obtive
$$
\S_i \vdot \S_j = -\frac{1}{2} \sum_{\s, \s'} c_{i\s}^\d c_{j\s'}^\d c_{i\s'} c_{j\s} - \frac{1}{4} n_i n_{j}.
$$

Temos então que para $i=j$:
$$
\S_i^2 = -\frac{1}{2} \sum_{\s,\s'} c_{i\s}^\d n_{i\s'} c_{i\s} - \frac{1}{4} n_i^2.
$$

Usando que $[c_{i\s}^\d, n_{i\s'}] = - \delta_{\s\s'} c_{i\s}^\d$ e $n_i^2 = (n_{i\up} + n_{i\down})^2 = n_i + 2 n_{i\up} n_{\down}$, temos
$$
\S_i^2 = -\frac{1}{2} \sum_{\s,\s'} \qty(n_{i\s'} c_{i\s}^\d - \delta_{\s\s'} c_{i\s}^\d ) c_{i\s} - \frac{1}{4} (n_i + 2 n_{i\up} n_{i\down}) \implies
$$
$$
\S_i^2 = -\frac{1}{2} \sum_{\s,\s'} n_{i\s'} n_{i\s} + \sum_{\s} n_{i\s} - \frac{1}{4} (n_i + 2 n_{i\up} n_{i\down}) \implies
$$
$$
\S_i^2 = -\frac{1}{2} (n_i + 2 n_{i\up} n_{\down}) + n_i - \frac{1}{4} n_i - \frac{1}{2} n_{i\up} n_{i\down} =
\frac{1}{4} n_i - \frac{3}{2} n_{i\up} n_{i\down} \implies
$$
$$
n_{i\up} n_{i\down} = -\frac{2}{3} \S_i^2 + \frac{1}{2} n_i \implies
\boxed{
U \sum_{i} n_{i\up} n_{i\down} =
-\frac{2}{3} \sum_{i} (\S_i^2) + \frac{U}{2} \sum_{i} n_i.
}
$$

Para mostrar que $[H, N_e] = 0$, é óbvio que cada $n_{i\up}$ e $n_{i\down}$ comutam com os termos \textit{interaction} e \textit{on-site} da hamiltoniana. Assim, falta mostrar que $N_e$ comuta com o termo de \textit{hopping}. Note que
$$
[c_{i\s}^\d c_{j\s} + c_{j\s}^\d c_{i\s}, n_{k\s'}] =
\Big([c_{i\s}^\d, n_{k\s'}] c_{j\s} + c_{i\s}^\d [c_{j\s}, n_{k\s'}]\Big) +
\Big([c_{j\s}^\d, n_{k\s'}] c_{i\s} + c_{j\s}^\d [c_{i\s}, n_{k\s'}]\Big)
$$
$$
=
\delta_{\s\s'} \Big(-c_{i\s}^\d c_{j\s} \delta_{ik} + c_{i\s}^\d c_{j\s} \delta_{jk} -
c_{j\s}^\d c_{i\s} \delta_{jk} + c_{j\s}^\d c_{i\s} \delta_{ik} \Big)
$$

Assim, para qualquer spin $\s' = \; \up, \down$, temos
$$
\qty[\sum_{\nn{i}{j}, \s} (c_{i\s}^\d c_{j\s} + c_{j\s}^\d c_{i\s}),
\sum_{k} n_{k\s'} ] =
\delta_{\s\s'} \sum_{\nn{i}{j}, \s}
\qty( \cancelto{0}{\boxed{-c_{i\s}^\d c_{j\s} + c_{i\s}^\d c_{j\s}}} \; \cancelto{0}{\boxed{- c_{j\s}^\d c_{i\s} + c_{j\s}^\d c_{i\s}}} ) = 0.
$$

Como $N_e = \sum_{\s} \sum_{k} n_{k\s}$, vemos que $[H, N_e] = 0$.

\n

Para a magnetização $M^z$, também é óbvio que ela comuta com os termos de \textit{interaction} e \textit{on-site}. Escrevendo $2 M^z = \sum_{k} n_{k\up} - \sum_{k} n_{k\down}$, do resultado anterior é direto que $M^z$ também comuta com o termo de \textit{hopping}. Isso nos dá que $[H, M^z] = 0$.

\pagebreak

\section{Antiferromagnetismo no modelo de Hubbard em campo médio}


\pagebreak

\section{Modelo de Heisenberg na rede triangular}


\pagebreak

\section{Transformação de Bogoliubov bosônica}



\pagebreak

\section{Ondas de spin}



\end{document}
