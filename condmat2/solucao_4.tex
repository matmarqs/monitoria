\documentclass[a4paper,10pt]{article}
%\usepackage{mathtools}
\usepackage{amsthm}     % for definitions and theorems
\usepackage[many]{tcolorbox}    % boxes around definitions and theorems
%\usepackage{amsmath}
%\usepackage{nccmath}
\usepackage{amssymb}    % \ltimes, semi-direct product
%\usepackage{etoolbox}   % for start of Chapter
%\usepackage{amsfonts}
\usepackage{physics}    % for all Physics related
\usepackage{dsfont}     % for the identity matrix symbol \1
%\usepackage{mathrsfs}
\usepackage[notextcomp]{stix}   % font package and some symbols like filled square
%\usepackage{MnSymbol}   % symbols font package

\usepackage{titling}
\usepackage{indentfirst}

\usepackage{bm}
\usepackage[dvipsnames]{xcolor}
\usepackage{cancel}
\usepackage{enumitem}

\usepackage{xurl}
%\usepackage[colorlinks=true]{hyperref} % links have colors
\usepackage{hyperref}  % no colors

\usepackage{float}
\usepackage{graphicx}
\usepackage{subcaption}
%\usepackage{tikz}

\usepackage{ctable}     % tabelas
\renewcommand{\P}{\phantom{+}}  % empty space to indent things
\usepackage{multirow}
\usepackage{tabulary}

%%%%%%%%%%%%%%%%%%%%%%%%%%%%%%%%%%%%%%%%%%%%%%%%%%%

\newcommand{\eps}{\epsilon}
\newcommand{\vphi}{\varphi}
\newcommand{\cte}{\text{cte}}

\newcommand{\N}{{\mathbb{N}}}
\newcommand{\Z}{{\mathbb{Z}}}
%\newcommand{\Q}{{\mathbb{Q}}}
\newcommand{\C}{{\mathbb{C}}}
\renewcommand{\S}{{\hat{S}}}
%\renewcommand{\H}{\s{H}}

\renewcommand{\a}{{\vb{a}}}
\renewcommand{\b}{{\vb{b}}}
\renewcommand{\d}{{\dagger}}
\newcommand{\up}{{\uparrow}}
\newcommand{\down}{{\downarrow}}
\newcommand{\hc}{{\text{h.c.}}}

\newcommand{\ihat}{\bm{\hat{\imath}}}
\newcommand{\jhat}{\bm{\hat{\jmath}}}
\newcommand{\khat}{\bm{\hat{k}}}

\newcommand{\0}{{\vb{0}}}
\newcommand{\1}{\mathds{1}}
\newcommand{\E}{{\vb{E}}}
\newcommand{\B}{{\vb{B}}}
\renewcommand{\u}{{\vb{u}}}
\renewcommand{\v}{{\vb{v}}}
\renewcommand{\r}{{\vb{r}}}
\newcommand{\R}{{\vb{R}}}
\newcommand{\Q}{{\vb{Q}}}
\newcommand{\G}{{\vb{G}}}
\newcommand{\g}{{\vb{g}}}
\renewcommand{\k}{{\vb{k}}}
\newcommand{\K}{{\vb{K}}}
\newcommand{\p}{{\vb{p}}}
\newcommand{\q}{{\vb{q}}}
\newcommand{\F}{{\vb{F}}}
\renewcommand{\t}{{\vb{t}}}
\newcommand{\vtau}{{\bm{\tau}}}
\newcommand{\vdelta}{{\bm{\delta}}}

% COLORED SYMMETRY ELEMENTS
\newcommand{\Ct}{{\textcolor{Cyan}{C_3}}}
\newcommand{\Ctn}[1]{{\textcolor{Cyan}{C_3^{\textcolor{black}{#1}}}}}
\newcommand{\Cs}{{\textcolor{ForestGreen}{C_6}}}
\newcommand{\Csn}[1]{{\textcolor{ForestGreen}{C_6^{\textcolor{black}{#1}}}}}
\newcommand{\sd}{{\textcolor{RoyalBlue}{\sigma_d}}}
\newcommand{\sdn}[1]{{\textcolor{RoyalBlue}{\sigma_d^{\textcolor{black}{#1}}}}}
\newcommand{\sdp}{{\textcolor{RoyalBlue}{\sigma_d'}}}
\newcommand{\sdpp}{{\textcolor{RoyalBlue}{\sigma_d''}}}
\newcommand{\sv}{{\textcolor{Orange}{\sigma_v}}}
\newcommand{\svn}[1]{{\textcolor{Orange}{\sigma_v^{\textcolor{black}{#1}}}}}
\newcommand{\svp}{{\textcolor{Orange}{\sigma_v'}}}
\newcommand{\svpp}{{\textcolor{Orange}{\sigma_v''}}}

\newcommand{\GL}{{\text{GL}}}
\newcommand{\U}{{\text{U}}}

\newcommand{\s}{\sigma}
%\newcommand{\prodint}[2]{\left\langle #1 , #2 \right\rangle}
\newcommand{\cc}[1]{\overline{#1}}
\newcommand{\Eval}[3]{\eval{\left( #1 \right)}_{#2}^{#3}}
\newcommand{\sg}[2]{\{ #1 \mid #2 \}}
\renewcommand{\AA}{{\mathring{\text{A}}}}
\newcommand{\I}{{\mathbb{I}}}
\newcommand{\bP}{{\mathbb{P}}}
\newcommand{\bQ}{{\mathbb{Q}}}

\newcommand{\unit}[1]{\; \mathrm{#1}}

\newcommand{\n}{\medskip}
\newcommand{\e}{\quad \mathrm{and} \quad}
\newcommand{\ou}{\quad \mathrm{or} \quad}
\newcommand{\virg}{\, , \;}
\newcommand{\ptodo}{\forall \,}
\renewcommand{\implies}{\; \Rightarrow \;}
%\newcommand{\eqname}[1]{\tag*{#1}} % Tag equation with name

%\setlength{\droptitle}{-7em}   % título um pouco mais em cima na página
%\makeatletter
%\patchcmd{\chapter}{\if@openright\cleardoublepage\else\clearpage\fi}{}{}{}  % start 'Chapter' at the same page. needs package etoolbox
%\makeatother

%% Theorems, definitions, proofs
\theoremstyle{definition}

%%% defining my own colors %%%
\definecolor{my-blue}{HTML}{f2f4ff}
\definecolor{my-green}{HTML}{f5fcf6}    % a little better: green!5!white
\definecolor{my-cyan}{HTML}{f2fffe}
\definecolor{my-yellow}{HTML}{fffbed}
\definecolor{my-green2}{HTML}{efffdb}

%%% alternative colors %%%
\definecolor{my-pink}{HTML}{fff2f7}
\definecolor{my-teal}{HTML}{ebfffc}

\newtheorem{definition}{Definition}[section]
\tcolorboxenvironment{definition}{
  colback=my-blue,
  %colback=blue!5!white,
  boxrule=0.1pt,
  boxsep=1pt,
  left=2pt,right=2pt,top=2pt,bottom=2pt,
  oversize=2pt,
  sharp corners,
  before skip=\topsep,
  after skip=\topsep,
}

\newtheorem{theorem}{Theorem}[section]
\tcolorboxenvironment{theorem}{
  colback=my-yellow,
  %colback=yellow!22!white!95!black,
  boxrule=0.1pt,
  boxsep=1pt,
  left=2pt,right=2pt,top=2pt,bottom=2pt,
  oversize=2pt,
  sharp corners,
  before skip=\topsep,
  after skip=\topsep,
}

\newtheorem{corollary}{Corollary}[section]
\tcolorboxenvironment{corollary}{
  colback=my-green2,
  boxrule=0.1pt,
  boxsep=1pt,
  left=2pt,right=2pt,top=2pt,bottom=2pt,
  oversize=2pt,
  sharp corners,
  before skip=\topsep,
  after skip=\topsep,
}

\newtheorem{lemma}{Lemma}[section]
\tcolorboxenvironment{lemma}{
  colback=my-cyan,
  boxrule=0.1pt,
  boxsep=1pt,
  left=2pt,right=2pt,top=2pt,bottom=2pt,
  oversize=2pt,
  sharp corners,
  before skip=\topsep,
  after skip=\topsep,
}

\newtheorem{example}{Example}[section]
\tcolorboxenvironment{example}{
  %colback=my-green,
  colback=green!5!white,
  boxrule=0.1pt,
  boxsep=1pt,
  left=2pt,right=2pt,top=2pt,bottom=2pt,
  oversize=2pt,
  sharp corners,
  before skip=\topsep,
  after skip=\topsep,
}


\title{\Huge{\textbf{Lista 4 - Matéria Condensada 2}}}
\author{Mateus Marques}

\begin{document}

\maketitle

\section{Fórmula de Drude revisitada}

(a) Neste caso, o campo elétrico $\E(t) = \E(\omega) e^{-i\omega t}$ é estático (e o magnético zero $\vb{H}=\0$), porém dependente do tempo. Dessa forma, a função distribuição $f(\k, t)$ depende apenas do momento $\k$ e do tempo $t$. A equação de Boltzmann dentro da aproximação do tempo de relaxação então fica
$$
\qty[\pdv{t} + \dot{\k} \vdot \grad_{\k}] f(\k, t) =
-\frac{1}{\tau} \qty[f(\k,t) - f_0(\k)],
$$
onde $f_0(\k) = \frac{1}{\exp[\beta(\eps_n(\k)-\mu)] + 1}$ é a distribuição de Fermi-Dirac. Da equação semiclássica
$$
\hbar \dot{\k} = -e \qty[\E + \frac{1}{c} \v_n(\k) \cp \vb{H}],
$$
temos que $\dot{\k} = - \frac{e \E(t)}{\hbar} = -\frac{e \E(\omega)}{\hbar} e^{-i\omega t}$ (colocarei $\hbar=1$). Definindo $\delta f(\k,t) = f(\k,t) - f_0(\k)$ e utilizando a aproximação linear que $\delta f(\k,t) \propto \E(t) = \E(\omega) e^{-i\omega t}$, temos $\pdv{t} \delta f(\k,t) = - i\omega \delta f(\k,t)$. Portanto:
$$
\pdv{t}\delta f(\k,t) + \frac{1}{\tau} \delta f(\k, t) =  - \dot{\k} \vdot \grad_{\k}  [f_0(\k) + \delta f(\k, t) ] \implies
$$
$$
\delta f(\k,t) = \frac{e \tau}{1-i\omega \tau} \, \E(t) \vdot \grad_{\k}  [ f_0(\k) + \delta f(\k, t) ].
$$
Note que a equação acima é um equação de ponto fixo para $\delta f(\k,t)$. Mas como nos basta a aproximação linear onde $\delta f(\k,t)$ é proporcional a $\E(t)$, ficamos com
$$
\delta f(\k,t) \simeq \frac{e \tau}{1-i\omega \tau} \, \E(t) \vdot \grad_{\k}  f_0(\k).
$$
Utilizando a expressão para a densidade de corrente em duas dimensões
$$
\vb{j}_n = -e \int \frac{\dd[2]{\k}}{(2\pi)^2} \v_n(\k) f(\k),
$$
temos
$$
\vb{j}_n = -e \int \frac{\dd[2]{\k}}{(2\pi)^2} \v_n(\k)
\qty{ \cancelto{0}{f_0(\k)} +
\frac{\tau[\eps_n(\k)]}{1-i\omega \tau[\eps_n(\k)]} e\E \vdot \grad_{\k} f_0(\k)} =
$$
$$
= e^2 \int \frac{\dd[2]{\k}}{(2\pi)^2} \v_n(\k)
\frac{\tau[\eps_n(\k)]}{1-i\omega \tau[\eps_n(\k)]} \qty(-\pdv{f_0(\k)}{\eps(\k)}) \, \E \vdot
\cancelto{\v_n(\k)}{\grad_k \eps_n(\k)} =
$$
$$
= e^2 \int \frac{\dd[2]{\k}}{(2\pi)^2} \v_n(\k) ( \v_n(\k) \vdot \E )
\frac{\tau[\eps_n(\k)]\qty(-\partial f_0 / \partial \eps)_{\eps = \eps_n(\k)}}{1-i\omega \tau[\eps_n(\k)]}.
$$
Como definimos $j^{\mu} = \sum_{\nu} \s_{\mu\nu} E^{\nu}$, vemos facilmente que
\begin{equation} \label{eq:cond}
\sigma_{\mu\nu}^{(n)}(\omega) = e^2 \int \frac{\dd[2]{\k}}{(2\pi)^2}
\frac{v_n^{\mu}(\k) v_n^{\nu}(\k) \, \tau[\eps_n(\k)] \, \qty(-\partial f_0 / \partial \eps)_{\eps = \eps_n(\k)}}
{1-i\omega \tau[\eps_n(\k)]}.
\end{equation}

Supondo que $\tau[\eps_n(\k)]$ tenha um valor bem definido $\tau_F$ na superfície de Fermi e que, por isotropia em duas dimensões $2v^2 = v_F^2$, no limite $T \to 0$ temos $\qty(-\partial f_0 / \partial \eps)_{\eps = \eps_n(\k)} \to \delta(\eps_n(\k) - E_F)$ e portanto
$$
\sigma(\omega) = e^2 \int \frac{\dd[2]{\k}}{(2\pi)^2}
\frac{\v(\k) \v(\k) \, \tau[\eps_n(\k)] \, \delta(\eps_n(\k) - E_F)}
{1-i\omega \tau[\eps_n(\k)]} =
e^2 \frac{v_F^2 \, \tau_F }{2(1-i\omega \tau_F)}
\int \frac{\dd[2]{\k}}{(2\pi)^2} \, \delta(\eps_n(\k) - E_F) \implies
$$
$$
\s(\omega) = e^2 \frac{v_F^2 \, \tau_F }{2(1-i\omega \tau_F)} \, \rho(E_F).
$$

Agora, supondo uma dispersão de elétrons livres com massa efetiva $m^*$, temos $E(\k) = \frac{\hbar^2 \k^2}{2m^*}$. Em duas dimensões sabemos que $E_F \propto n$, de maneira que $\rho(E_F) = \dv{n}{E_F} = n/E_F = \frac{2 n}{m^* v_F^2}$. Substituindo obtemos
$$
\s(\omega) = \frac{\s_0}{1-i\omega \tau_F}, \quad \s_0 = \frac{n e^2 \tau_F}{m^*}.
$$

(b) Da expressão de Drude vemos que
$$
\Re[\s(\omega)] = \frac{ne^2}{m^*} \, \frac{\tau}{1+\omega^2 \tau^2}.
$$

Colocando $ne^2/m^* = 1$ e plotando $\Re[\s(\omega)]$ para diferentes valores de $\tau$, obtemos:
\begin{figure}[H]
\centering
\includegraphics[width=0.75\textwidth]{fig/resigma.png}
\caption{$\omega \times \Re[\s(\omega)]$ para diferentes valores de $\tau$. Vemos que para $\tau \to \infty$ o peso espectral localiza-se em $\omega = 0$, pois $\Re[\s(\omega)]$ é uma lorentziana.}
\label{fig:resigma}
\end{figure}

Devido à representação da função $\delta$ de Dirac por lorentzianas
$$
\delta(\omega) = \lim_{\eta \to \infty} \frac{1}{\pi} \cdot \frac{\eta}{1 + \omega^2 \eta^2},
$$
temos que no limite $\tau \to \infty$:
$$
\Re[\s(\omega)] = \frac{ne^2 \pi}{m^*} \delta(\omega).
$$

Definindo $\omega_p^2 = \frac{4\pi n e^2}{m^*}$, obtemos que
$$
\Re[\s(\omega)] = \frac{\omega_p^2}{4} \delta(\omega) \e
\int_0^{\infty} \Re[\s(\omega)] \dd{\omega} = \frac{\omega_p^2}{8}.
$$


\pagebreak

\section{Condutividade eletrônica}

Por equipartição da energia (isotropia) temos na superfície de Fermi $3v^2 = v_F^2$. Para baixas temperaturas $T \to 0$ temos então
$$
\s = \frac{2e^2 v_F^2 \tau_F}{3} \int \frac{\dd[3]{\k}}{(2\pi)^3}
\qty(-\pdv{f}{\eps})_{\eps=\eps(\k)} =
\frac{2e^2 v_F^2 \tau_F}{3} \int \frac{\dd[3]{\k}}{(2\pi)^3}
\int \dd{\eps} \delta(\eps - \eps(\k)) \qty(-\pdv{f}{\eps}) =
$$
$$
\s = \frac{2e^2 v_F^2 \tau_F}{3} \int \dd{\eps} \rho(\eps) \qty(-\pdv{f}{\eps}).
$$

Integrando por partes:
$$
\int \dd{\eps} \rho(\eps) \qty(-\pdv{f}{\eps}) =
\int \dd{\eps} f(\eps) \dv{\rho}{\eps}.
$$

A partir de agora assumiremos $E_F = 0$ e $T_F = D = k_B = 1$.

\begin{itemize}
\item (a) Metal: $\rho(\eps) = \frac{4}{\pi} \sqrt{1-\eps^2}$. Para $T \to 0$ podemos usar a expansão de Sommerfeld
$$
\int_{-\infty}^{\infty} \dd{\eps} f(\eps) G(\eps) \simeq
\int_{-\infty}^{\mu} G(\eps) + \frac{\pi^2}{6} T^2 \eval{\dv{G}{\eps}}_{\eps=\mu}.
$$
Colocando $G(\eps) = \dd{\rho}/\dd{\eps}$ temos
$$
\s = \frac{2e^2 v_F^2 \tau_F}{3}
\int_{-\infty}^{\infty} \dd{\eps} f(\eps) \dv{\rho}{\eps} =
\frac{2e^2 v_F^2 \tau_F}{3}
\qty{\rho(E_F) + \frac{\pi^2}{6} T^2 \eval{\dv[2]{\rho}{\eps}}_{\eps=E_F}}.
$$
Assim, da expressão $\s = \s_0 - A T^2$, reconhecemos
$$
\s_0 = \frac{2e^2 v_F^2 \tau_F}{3} \rho(E_F) = \frac{8e^2 v_F^2 \tau_F^2}{3\pi} \e
A = -\frac{2e^2v_F^2\tau_F}{3} \frac{\pi^2}{6} \eval{\dv[2]{\rho}{\eps}}_{\eps=E_F}
= \frac{4 \pi e^2v_F^2\tau_F}{9}.
$$

Essa DOS semicircular reflete o comportamento de um metal, já que a $\rho(\eps)$ é não-nulo para todo $-D \leq \eps \leq D$. Em particular $\rho(E_F) > 0$ nos dá uma condutividade finita em $T = 0$.

\item (b) Semimetal: $\rho(\eps)=2\abs{\eps}$. Essa DOS representa um semimetal pois $\rho(E_F) = 0$, porém $\rho(\eps) > 0$ para $\eps \neq E_F$.

\item (c) Semicondutor: $\rho(\eps) = \frac{\eps}{\sqrt{\eps^2-\Delta^2}}$, $\Delta < \eps < 1$. Em geral essa DOS representa um isolante de bandas, já que ela é gerada por uma dispersão da forma $\eps(\k) = \sqrt{\k^2 + \Delta^2}$ com a existência de um gap $\Delta$. Um semicondutor nada mais é do que um isolante com um gap $\Delta$ pequeno.
\end{itemize}

Observação: Basicamente temos $T \to 0$. Então em baixas energias (perto de $E_F$) temos que cada sistema se comporta como o $\rho(\eps)$ da categoria correspondente (metal, semimetal ou semicondutor). Ainda mais, podemos sempre setar $E_F = 0$. Então as escolhas de $\rho(\eps)$ e $E_F = 0$ feitas aqui são gerais.

Os gráficos das densidades de estados $\rho(\eps)$ e das condutividades $\s(T)$ são exibidos nas Figuras \ref{fig:dos-systems} e \ref{fig:sigma}.

\begin{figure}[H]
\centering
\includegraphics[width=0.4\textwidth]{fig/dos-systems.png}
\caption{Densidade de estados para os sistemas metal, semimetal e semicondutor.}
\label{fig:dos-systems}
\end{figure}
Note que obtivemos $1.908$ ao invés de $2$ para o caso semicondutor (com gap $\Delta = 0.3$). Percebi que na verdade temos $\int \frac{x}{\sqrt{x^2-\Delta^2}} \dd{x} = \sqrt{x^2 - \Delta^2}$, de maneira que
$$
\int_{-1}^{-\Delta} \frac{\abs{\eps}}{\eps^2-\Delta^2} \dd{\eps} + \int_{\Delta}^{1} \frac{\eps}{\eps^2-\Delta^2} \dd{\eps} = 2 \sqrt{1-\Delta^2}.
$$
Assim, a integral não é exatamente $2$, a não ser que $\Delta = 0$.

\begin{figure}[H]
\centering
\includegraphics[width=0.6\textwidth]{fig/sigma.png}
\caption{Condutividade para os diferentes sistemas físicos. No semicondutor utilizei $\Delta = 0.3$.}
\label{fig:sigma}
\end{figure}

Note que $\sigma_{\text{metal}}$ é finita para $T = 0$, e como podemos ver na Figura \ref{fig:sigma}, ela se comporta como $\s_0 - AT^2$ para temperaturas baixas (comportamento quadrático). É a única condutividade que decresce com o aumento da temperatura, as condutividades do semimetal e semicondutor são nulas para $T = 0$, porém aumentam com a temperatura.

No caso do semimetal, a condutividade é zero para $T = 0$, porém é positiva para todo $T > 0$. Ela tem um comportamento linear para $T \approx 0$.

No semicondutor, a condutividade é zero para $T = 0$ e permanece essencialmente zero para até um certo valor de $T \approx 0.05$ na Figura \ref{fig:sigma}.

Vemos que o fato da condutividade permanecer zero para temperaturas pequenas no caso semicondutor é explicada pela existência do gap $\Delta$. Assim, podemos estudar o valor de $\Delta$ somente a partir de $\s(T)$ para baixas temperaturas. Quanto maior for a faixa de temperaturas em que $\s(T)$ permanece zero, maior é o gap $\Delta$.

\pagebreak

\section{Desordem e localização de Anderson}

(a) É intuitivo pensar que a banda associada aos operadores $d_i$ é localizada pois, na hamiltoniana considerada, existem termos de hopping entre $c_i$ e $c_j$ (sítios diferentes) e entre $c_i$ e $d_i$ (mesmo sítio), mas não existe hopping entre $d_i$ e $d_j$.

Assumindo que o espaçamento de rede é $a$, tomamos as transformadas de Fourier
$$
c_j^\d = \frac{1}{\sqrt{N}} \sum_{k} c_k^\d e^{-ikaj} \e
d_j^\d = \frac{1}{\sqrt{N}} \sum_{k} d_k^\d e^{-ikaj}.
$$

Temos então
$$
H = -t \sum_{j} \qty(c_j^\d c_{j+1} + c_{j+1}^\d c_j) +
\lambda \sum_{j} d_j^\d d_j +
V \sum_{j} \qty(c_j^\d d_j + d_j^\d c_j) \implies
$$
$$
H = \frac{1}{N} \sum_{k,k'}
\cancelto{\boxed{= N \delta_{k,k'}}}{\boxed{\sum_{j} e^{-i(k-k')ja}}}
\qty{
-t \qty(e^{ik'a} c_k^\d c_{k'} + \hc) +
\lambda \qty(d_k^\d d_{k'}) +
V \qty(c_k^\d d_{k'} + \hc)
} \implies
$$
$$
H = \sum_{k}
\qty{
-2t \cos(ka) (c_k^\d c_k) +
\lambda (d_k^\d d_k) +
V (c_k^\d d_k + d_k^\d c_k)
} \implies
$$
$$
\boxed{ H = \sum_{k}
\begin{pmatrix}
c_k^\d & d_k^\d
\end{pmatrix}
\begin{pmatrix}
-2t \cos(ka) & V \\
V & \lambda
\end{pmatrix}
\begin{pmatrix}
c_k \\ d_k
\end{pmatrix}. }
$$

Os autovalores da matriz acima são
$$
E_\pm(k) =
\frac{\qty[\lambda - 2t \cos(ka)] \pm
\sqrt{\qty[\lambda - 2t \cos(ka)]^2 + 4 V^2}}
{2}.
$$

Os mínimos e máximos de $E_{\pm}(k)$ e máximos sempre ocorrem em $k = 0$ ou $k = \pm \frac{\pi}{a}$. Assim, as larguras $d_\pm$ das respectivas bandas $\pm$ são
$$
d_\pm = \abs{E_\pm\qty(\frac{\pi}{a}) - E_\pm(0)} =
2\abs{t} \cdot \abs{1 \pm
\frac{2 \lambda}{\sqrt{(\lambda+2t)^2+4V^2}+\sqrt{(\lambda-2t)^2+4V^2}}}.
$$

Das expressões de $d_\pm$ acima, concluimos que ambas as bandas são dispersivas (possuem larguras não-nulas), a não ser que $t = 0$ ou que $\lambda \gg 1$ (nesse caso uma das bandas tem largura muito pequena e a outra uma largura da ordem de $t$).

\n

O parâmetro $V$ é o gap entre as duas bandas. Temos um isolante (em $T = 0$) quando $V \neq 0$.

O parâmetro $t$ controla a largura e o tipo das duas bandas. Quando $t > 0$ temos bandas do tipo elétron e quando $t < 0$ temos bandas do tipo buraco.

O parâmetro $\lambda$ controla o peso (assimetria) entre as duas bandas. Se $\lambda > 0$ a banda de cima tem largura maior $d_+ > d_-$. Para $\lambda < 0$ a banda de baixo tem largura maior $d_- > d_+$.

\begin{figure}[H]
\centering
\begin{subfigure}{.46\textwidth}
  \centering
  \includegraphics[width=0.95\linewidth]{fig/bandas-anderson1.png}
  \label{fig:bandas-anderson1}
\end{subfigure}
\begin{subfigure}{.46\textwidth}
  \centering
  \includegraphics[width=0.95\linewidth]{fig/bandas-anderson2.png}
  \label{fig:bandas-anderson2}
\end{subfigure}
\label{fig:bandas-anderson11}
\end{figure}

\begin{figure}[H]
\centering
\begin{subfigure}{.46\textwidth}
  \centering
  \includegraphics[width=0.95\linewidth]{fig/bandas-anderson3.png}
  \label{fig:bandas-anderson3}
\end{subfigure}
\begin{subfigure}{.46\textwidth}
  \centering
  \includegraphics[width=0.95\linewidth]{fig/bandas-anderson4.png}
  \label{fig:bandas-anderson4}
\end{subfigure}
\label{fig:bandas-anderson12}
\end{figure}

(b) A função de escala
$$
\beta(g) = \dv{\ln g}{\ln L} = \frac{L}{g} \dv{g}{L}
$$
quantifica a resposta (independente da escala, devido ao fator $L/g$) da condutância adimensional $g$ com respeito ao comprimento $L$ do sistema. Se a derivada em questão for positiva, a condutância aumenta ao aumentarmos o sistema (comportamento metálico). Do contrário, temos um comportamento isolante.

Em aula fizemos uma estimativa para o comportamento $\beta(g)$ para a teoria de localização de Anderson. Obtivemos o seguinte gráfico:
\begin{figure}[H]
\centering
\includegraphics[width=0.75\textwidth]{fig/beta.png}
\caption{Comportamento qualitativo da função de escala $\beta(g)$ para a teoria de localização de Anderson.}
\label{fig:beta}
\end{figure}

Veja que para $d \leq 2$, temos $\beta(g) < 0$ negativo sempre. Isso mostra que não podemos ter estados estendidos nesse caso, todos são localizados (o sistema se comporta como isolante). Na presença de um campo magnético os estados do bulk continuam localizados, porém os da borda conseguem transmitir corrente elétrica. É o que acontece no efeito Hall quântico inteiro.

\pagebreak

\section{Efeito Hall}

(a) A condição de quantização de Bohr-Sommerfeld nos dá que $2\pi R = n \lambda \iff k R = n$ (onde $\lambda = \frac{2\pi}{k}$ é o comprimento de onda de de Broglie). Para um elétron em 2D sob a ação de um campo magnético uniforme perpendicular ao plano, ele executará um movimento circular uniforme. Temos então que $\frac{mv^2}{R} = e v B \implies p = mv = e B R = \hbar k \implies e B R^2 = n \hbar$. O fluxo magnético é
$$
\Phi_B = B \cdot \pi R^2 = n \times \frac{h}{2e},
$$
onde $\Phi_0 = h/2e$ é o quanta de fluxo. Cada órbita com $n$ comprimento de ondas engloba $n \times \Phi_0$ de fluxo magnético.

\n

(b) Seja $c^\d$ o operador de criação em um sítio que denominamos como a origem em nosso lattice. Os operadores de criação $c_j^\d$ em outros sítios podem ser obtidos então através da aplicação do operador de translação $T(\r_j) = \exp[-\frac{i}{\hbar} \p \vdot \r_j] = \exp[-\frac{i}{\hbar} \int_{\0}^{\r_j} \p \vdot \dd{\r}]$, de maneira que
$$
c_j^\d = \exp[-\frac{i}{\hbar} \int_{\0}^{\r_j} \p \vdot \dd{\r}] c^\d.
$$
Como sabemos, podemos incluir o campo magnético através da substituição $\p \to \p - q \A$, de maneira que
$$
c_j^\d \to \exp[i\frac{q}{\hbar} \int_{\0}^{\r_j} \A \vdot \dd{\r}] c_j^\d
$$
e portanto
$$
t_{ij} c_j^\d c_i \to t_{ij} e^{i \Phi_{ij}} c_j^\d c_i, \quad
\Phi_{ij} = \frac{q}{\hbar}\int_{\r_i}^{\r_j} \A \vdot \dd{\r}.
$$

Do item (a) temos que
$$
n \Phi_0 = \Phi_B = \iint \B \vdot \dd{\vb{S}} = \iint (\curl{\A}) \vdot \dd{\vb{S}} = \oint \A \vdot \dd{\r} \implies n \propto \Phi_{jj}.
$$

O gauge de Landau é $\A = Bx \vu{y}$. O modelo tight-binding da rede quadrada em 2D para primeiros vizinhos então nos dá
$$
\Phi_{(x,y),(x+a,y)} = \frac{q}{\hbar} \int_{(x,y)}^{(x+a,y)} (B x \vu{y}) \vdot (dx \vu{x}) = 0
$$
$$
\Phi_{(x,y),(x,y+a)} = \frac{q}{\hbar} \int_{(x,y)}^{(x,y+a)} (B x \vu{y}) \vdot (dy \vu{y}) = \frac{qBxa}{\hbar}.
$$

A hamiltoniana desse problema então ficaria
$$
H = -t \sum_{\R}
\qty{
\qty[c^\d_{(\R+a\vu{x})} c_{\R} + c_{\R}^\d c_{(\R+a\vu{x})}] +
\qty[e^{i\frac{qBxa}{\hbar}} c^\d_{(\R+a\vu{y})} c_{\R} + e^{-i\frac{qBxa}{\hbar}}c_{\R}^\d c_{(\R+a\vu{y})}]
}.
$$
$$
H = -\frac{t}{N} \sum_{\k,\k'}
\qty{
\qty[
\cancelto{\boxed{= N \delta_{\k,\k'}}}{\boxed{\sum_{\R} e^{-i(\k-\k')\vdot\r}}}
e^{-ik_x a} c_{\k}^\d c_{\k'} + \hc] +
\qty[
\cancelto{\boxed{= N \delta_{\k-\frac{qBa}{\hbar}\vu{x},\k'}}}{\boxed{\sum_{\R} e^{-i\qty(\k-\k'-\frac{qBa}{\hbar}\vu{x})\vdot\r}}}
e^{-ik_y a} c_{\k}^\d c_{\k'} + \hc]
}
$$
$$
H = -t \sum_{\k} \qty{
\qty[e^{-ik_x a} c_{\k}^\d c_{\k} +
e^{-ik_y a} c_{\k}^\d c_{\qty(\k - \frac{qBa}{\hbar} \vu{x})}
] + \hc
}.
$$

Como a \href{https://en.wikipedia.org/wiki/Hofstadter's_butterfly}{borboleta de Hofstadter} exibe um padrão bem complexo, não tenho muita certeza de como intuir sobre a densidade de estados desse problema. Porém, olhando a Figura \ref{fig:butterfly}, sabemos quais são as energias possíveis para o problema. Note que apesar delas serem discretas, o espectro é simétrico em torno do zero. Como calculamos na Questão 4 da Lista 3, a hamiltoniana de tight-binding na rede quadrada 2D tem uma divergência logarítmica em $\eps = 0$. Eu chutaria que a DOS na presença de campo magnético também terá divergências logarítmicas, porém em pontos diferentes de zero e de maneira simétrica. Um esboço do que eu estou pensando está na Figura \ref{fig:dos-esboco}.
\begin{figure}[H]
\centering
\includegraphics[width=0.5\textwidth]{fig/butterfly.png}
\caption{Espectro da borboleta de Hofstadter, energias possíveis $\eps_\alpha$ em função da razão de fluxo $\alpha$.}
\label{fig:butterfly}
\end{figure}

\begin{figure}[H]
\centering
\includegraphics[width=0.75\textwidth]{fig/dos-esboco.png}
\caption{Esboço do que eu intuo que seria a ``cara'' da densidade de estados do tight-binding rede quadrada 2D na presença de campo magnético.}
\label{fig:dos-esboco}
\end{figure}


\pagebreak

\section{Número de Chern para um modelo de duas bandas}

É direto obter que
$$
\abs{\vb{h}(\vb{k})}^2 =
\Delta^2 + 2 (1 + \cos k_x \cos k_y - \Delta (\cos k_x + \cos k_y)) + \Delta^2.
$$
$$
\begin{vmatrix}
h_x & h_y & h_z \\
\pdv{h_x}{k_x} & \pdv{h_y}{k_x} & \pdv{h_z}{k_x} \\
\pdv{h_x}{k_y} & \pdv{h_y}{k_z} & \pdv{h_z}{k_y} \\
\end{vmatrix}
=
\begin{pmatrix}
\sin k_x & \sin k_y & \Delta - \cos k_x - \cos k_y \\
\cos k_x &  0 & \sin k_x \\
0 & \cos k_y & \sin k_y \\
\end{pmatrix}
=
\Delta \cos k_x \cos k_y - \cos k_x - \cos k_y.
$$

Portanto,
$$
C = \frac{1}{4\pi} \int_{-\pi}^{\pi} \int_{-\pi}^{\pi}
\frac{\Delta \cos k_x \cos k_y - \cos k_x - \cos k_y}{\qty[\Delta^2 + 2 (1 + \cos k_x \cos k_y - \Delta (\cos k_x + \cos k_y)) + \Delta^2]}
\dd{k_x} \dd{k_y}
$$

Integrando a expressão acima numericamente com a função \texttt{scipy.integrate.dblquad} do Python, obtemos a Figura \ref{fig:chern}:
\begin{figure}[H]
\centering
\includegraphics[width=0.75\textwidth]{fig/chern.png}
\caption{Número de Chern em função do parâmetro $\Delta$.}
\label{fig:chern}
\end{figure}

\pagebreak

\section{Condutividade Hall}

Da equação \ref{eq:cond}, substituindo $\grad_{\k}f = \pdv{f(\k)}{\eps_n(\k)} \grad_{\k}\eps_n(\k)$ ($f$ é a distribuição de Fermi-Dirac), $\v_n(\k) = \frac{1}{\hbar} \grad_{\k}\eps_n(\k)$ e aproximando $1-i\omega\tau \approx -i\omega\tau$ no limite $\omega\tau \gg 1$, temos
$$
\bm{\s}^{(n)}(\omega) = \frac{e^2}{i\omega \hbar^2}
\int \frac{\dd[2]{\k}}{(2\pi)^2} \grad_{\k}\eps_n(\k) \grad_{\k} f.
$$

 Integrando por partes as componentes $\mu, \nu$:
$$
\int \dd[2]{\k} \pdv{\eps_n(\k)}{k_\mu} \pdv{f(\k)}{k_\nu} =
\cancelto{0}{\boxed{\int \dd[2]{\k} \pdv{k_\nu} \qty[\pdv{\eps_n(\k)}{k_\mu} f(\k)]}} -
\int \dd[2]{\k} \pdv{\eps_n(\k)}{k_\mu}{k_\nu} f(\k).
$$

O termo superficial acima é zero pois o integrando é periódico dentro na borda da zona de Brillouin.

Em $T=0$, a distribuição de Fermi-Dirac faz a integral ser somente sobre os estados ocupados, portanto
$$
\s^{(n)}_{\mu\nu}(\omega) = \frac{e^2}{i\omega \hbar^2}
\int \frac{\dd[2]{\k}}{(2\pi)^2} \pdv{\eps_n(\k)}{k_\mu} \pdv{f(\k)}{k_\nu} =
- \frac{e^2}{i\omega}
\int_{\text{ocupados}} \frac{\dd[2]{\k}}{(2\pi)^2} \frac{1}{\hbar^2} \pdv{\eps_n(\k)}{k_\mu k_\nu}.
$$

O resultado do problema 3(b) da Lista 3 para o tensor massa efetiva é
$$
\frac{1}{\hbar^2} \pdv{\eps_n(\k)}{k_\mu}{k_\nu} =
\frac{\delta_{\mu\nu}}{m_e} +
\qty(\frac{\hbar}{m_e})^2
\sum_{m\neq n}
\frac{
\mel**{n}{\frac{1}{i}\pdv{r_\mu}}{m} \mel**{m}{\frac{1}{i}\pdv{r_\nu}}{n} +
\mel**{n}{\frac{1}{i}\pdv{r_\nu}}{m} \mel**{m}{\frac{1}{i}\pdv{r_\mu}}{n}
}{\eps_n(\k) - \eps_{m}(\k)}.
$$

O problema que estamos tratando envolve um campo elétrico alternado $\E(t) = \E(\omega) e^{-i\omega t}$, ou seja, uma frequência $\omega$ de onda eletromagnética bem definida. A essa frequência correspondem excitações de energia $\delta(\pm \hbar \omega + \eps_n(\k) - \eps_m(\k))$ do espectro. No limite $\hbar \omega \to 0$, então é razoável aproximarmos
$$
\frac{1}{\hbar^2} \pdv{\eps_n(\k)}{k_\mu}{k_\nu} =
\frac{\delta_{\mu\nu}}{m_e} +
\qty(\frac{\hbar}{m_e})^2
\sum_{m\neq n} \qty{
\frac{\mel**{n}{\frac{1}{i}\pdv{r_\mu}}{m} \mel**{m}{\frac{1}{i}\pdv{r_\nu}}{n}}
{\hbar \omega + \eps_n(\k) - \eps_{m}(\k)} +
\frac{\mel**{n}{\frac{1}{i}\pdv{r_\nu}}{m} \mel**{m}{\frac{1}{i}\pdv{r_\mu}}{n}}
{-\hbar \omega + \eps_n(\k) - \eps_{m}(\k)}
} .
$$

Expandimos então para $\hbar \omega \to 0$:
$$
\frac{1}{\pm \hbar \omega + \eps_n(\k) - \eps_m(\k)} =
\boxed{\frac{1}{\eps_n(\k) - \eps_m(\k)}}
\mp \frac{\hbar\omega}{(\eps_n(\k) - \eps_m(\k))^2}.
$$

Como estamos interessados na condutividade $\s_{xy}$, note que ela deve se manter a invariante sobre a rotação $x \to y$, $y \to -x$. Mas, com relação ao primeiro termo destacado acima, as derivadas $\pdv{x}$ e $\pdv{y}$ tomam um sinal de menos na expressão de $\s_{y,-x}$, de maneira que o termo destacado é anulado.

Além disso, da equação de Heisenberg temos $i\hbar \dot{r}_\mu = [r_\mu, H] = i\hbar \pdv{H}{p_\mu} \implies \dot{r}_\mu = \pdv{H}{p_\mu}$. Podemos então fazer a substiuição $p_\mu = \frac{1}{i} \hbar \pdv{r_\mu} = m_e \dot{r}_\mu = m_e \pdv{H}{p_\mu} = m_e \hbar^{-1} \pdv{H}{k_\mu}$.

Dessa maneira, ficamos com
$$
\s^{(n)}_{xy}(T=0) =i\frac{e^2}{\hbar}\int_{\text{BZ}}\frac{\dd[2]{\k}}{(2\pi)^2}
\sum_{m\neq n}
\qty[
\frac{
\mel**{n}{\partial_x H}{m} \mel**{m}{\partial_y H}{n} -
\mel**{n}{\partial_y H}{m} \mel**{m}{\partial_x H}{n}
}{(\eps_n(\k) - \eps_{m}(\k))^2} ]
$$

Usando que $(E_n - E_m)^2 = 4h^2$ e a equação (26) das notas de aula sobre Isolantes Topológicos para a curvatura de Berry, chegamos que
$$
\s^{(n)}_{xy}(T=0) =
\frac{e^2}{\hbar} \int_{\text{BZ}} \frac{\dd[2]{\k}}{(2\pi)^2} B_n(\k) =
\frac{e^2}{h} C_n.
$$

\end{document}
