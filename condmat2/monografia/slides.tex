%\documentclass[aspectratio=169]{beamer}
\documentclass[8pt,aspectratio=169,xcolor={table,dvipsnames,usenames}]{beamer}
\usefonttheme[onlymath]{serif}
\usepackage[brazilian]{babel}


\mode<presentation>
{
  \usetheme{Madrid}      % or try Darmstadt, Madrid, Warsaw, ...
  \usecolortheme{dolphin} % or try albatross, beaver, crane, ...
  \usefonttheme{default}  % or try serif, structurebold, ...
  \setbeamertemplate{navigation symbols}{}
  \setbeamertemplate{caption}[numbered]
  \setbeamertemplate{headline}{}
}

%\usepackage[T1]{fontenc}
%\usepackage[brazilian]{babel}
%\usepackage[utf8]{inputenc}
%\usepackage[left=2.5cm,right=2.5cm,top=3cm,bottom=2.5cm]{geometry}
%\usepackage{mathtools}
%\usepackage{amsthm}
%\usepackage{amsmath}
%\usepackage{nccmath}
%\usepackage{amssymb}
%\usepackage{amsfonts}
\usepackage{physics}
%\usepackage{dsfont}
%\usepackage{mathrsfs}

%\usepackage{titling}
\usepackage{indentfirst}

\usepackage{bm}
%\usepackage[dvipsnames]{xcolor}    % it crashes with beamer
\usepackage{xcolor}
\usepackage{cancel}

\usepackage{xurl}
%\usepackage[colorlinks=true]{hyperref}  % crashes with beamer

%% code
%\definecolor{bg}{rgb}{0.90,0.90,0.90}
%\usepackage{minted}


\usepackage{float}
\usepackage{graphicx}
\usepackage{tikz}
\usepackage{caption}
\usepackage{subcaption}

%%%%%%%%%%%%%%%%%%%%%%%%%%%%%%%%%%%%%%%%%%%%%%%%%%%

\newcommand{\eps}{\epsilon}
\newcommand{\vphi}{\varphi}
\newcommand{\cte}{\text{cte}}

\newcommand{\N}{\mathbb{N}}
\newcommand{\Z}{\mathbb{Z}}
\newcommand{\Q}{\mathbb{Q}}
\newcommand{\R}{\vb{R}}
\newcommand{\C}{\mathbb{C}}
\renewcommand{\S}{\vb{S}}
%\renewcommand{\H}{\s{H}}

\renewcommand{\a}{\vb{a}}
\renewcommand{\d}{\dagger}
\newcommand{\up}{\uparrow}
\newcommand{\down}{\downarrow}
\newcommand{\hc}{\text{h.c.}}

\newcommand{\0}{\vb{0}}
\newcommand{\1}{\mathds{1}}
\newcommand{\E}{\vb{E}}
\newcommand{\B}{\vb{B}}
\renewcommand{\v}{\vb{v}}
\renewcommand{\r}{\vb{r}}
\renewcommand{\k}{\vb{k}}
\newcommand{\K}{{\vb{K}}}
\newcommand{\p}{{\vb{p}}}
\newcommand{\q}{{\vb{q}}}
\newcommand{\F}{{\vb{F}}}
\newcommand{\A}{{\vb{A}}}

\newcommand{\s}{\sigma}
\newcommand{\nn}[2]{\left\langle #1 , #2 \right\rangle}
\newcommand{\cc}[1]{\overline{#1}}
\newcommand{\Eval}[3]{\eval{\left( #1 \right)}_{#2}^{#3}}

\newcommand{\unit}[1]{\; \mathrm{#1}}

\newcommand{\n}{\medskip}
\newcommand{\e}{\quad \mathrm{e} \quad}
\newcommand{\ou}{\quad \mathrm{ou} \quad}
\newcommand{\virg}{\, , \;}
\newcommand{\ptodo}{\forall \,}
\renewcommand{\implies}{\; \Rightarrow \;}
%\newcommand{\eqname}[1]{\tag*{#1}} % Tag equation with name

%\setlength{\droptitle}{-6em}   % crashes with beamer


%%%%%%%%%%%%%%%%%%%%%%%%%%%%%%%%%%%%%%%%%%%%%%%%%%%%%%%%%

\title[Supercondutividade não-convencional]{\LARGE{Supercondutividade não-convencional}}
\author[Mateus Marques]{
\large{Mateus Marques
}}
\date{\today}
%\titlegraphic{\includegraphics[height=1.5cm]{logos/ifusp.png} \,\, \includegraphics[height=1.5cm]{logos/scientiavinces.png}}

\begin{document}

\begin{frame}
  \titlepage
\end{frame}

\begin{frame}{Supercondutores que vemos por aí}

\begin{figure}[H]
\centering
\includegraphics[width=0.7\textwidth]{fig/levitating.jpg}
\caption{Supercondutor de cuprato levitando sobre um ímã pelo efeito Meissner.}
\label{fig:levitating}
\end{figure}

\end{frame}

%%%%%%%%%%%%%%%%%%%%%%%%%%%%%%%%%%%%%%%%%%%%%%%%%%%%%%%%%%%%%%%%%%%%%%%%%%%%%%%%%%%%%%%%%%%%%%%%%

\begin{frame}{Diferentes supercondutores}

\begin{center}
\begin{tabular}{ |p{3cm}||p{3cm}|p{3cm}|p{3cm}|  }
\hline
\multicolumn{4}{|c|}{Exemplos de supercondutores/superfluidos} \\
\hline
Supercondutor & Simetria & $T_c \approx$ & Categoria \\
\hline
Hg                   & s-wave   & $4.2   \, \text{K}$      & BCS \\
MgB$_2$              & s-wave   & $39    \, \text{K}$      & BCS \\
$^3$He               & p-wave   & $2.5   \, \text{mK}$     & Superfluido \\
Sr$_2$RuO$_4$        & $-$      & $0.93  \, \text{K}$      & $-$ \\
YBCO                 & d-wave   & $93  \, \text{K}$        & Cuprato \\
BSCCO-2223           & d-wave   & $108   \, \text{K}$      & Cuprato \\
UPt$_3$              & f-wave   & $0.51  \, \text{K}$      & Heavy-Fermion \\
K$_3$C$_{60}$        & $-$      & $18    \, \text{K}$      & Orgânico  \\
LaO$_{1-x}$F$_x$FeAs & $-$      & $26    \, \text{K}$      & Ferro  \\
MATBG                & $-$      & $1.7    \, \text{K}$     & Grafeno  \\
\hline
\end{tabular}
\end{center}

\end{frame}


%%%%%%%%%%%%%%%%%%%%%%%%%%%%%%%%%%%%%%%%%%%%%%%%%%%%%%%%%%%%%%%%%%%%%%%%%%%%%%%%%%%%%%%%%%%%%%%%%

\begin{frame}{Objetivos}

\begin{itemize}
\item Passar uma ideia introdutória de como generalizar a teoria BCS para estudar a \textbf{fenomenologia} da supercondutividade não-convencional.

\n\n

\item A inclusão de interações repulsivas e magnéticas causam anisotropia na função de gap.

\n\n

\item Entender a distinção básica das diferentes simetrias (jargão) s, p, d, f-wave.

\n\n

\item Existem mecanismos além dos fônons que geram interações atrativas?

\n\n

\item Foco específico nos cupratos e na simetria d-wave.
\end{itemize}

%%%%%%%%%%%%%%%%%%%%%%%%%%%%%%%%%%%%%%%%%%%%%%%%%%%%%%%%%%%%%%%%%%%%%%%%%%%%%%%%%%%%%%%%%%%%%%%%%


\end{frame}

\begin{frame}{Hipóteses da teoria BCS}

\begin{itemize}
\item O estado normal é um líquido de Fermi (metal).

\item Elétrons formam pares de Cooper pela interação atrativa mediada por fônons.

\n

\begin{figure}[H]
\centering
\includegraphics[width=0.5\linewidth]{fig/phonon.png}
\label{fig:phonon}
\end{figure}

\n

\item Os pares de Cooper são bósons e condensam num estado coerente, formando um superfluido carregado.

\item Dispersão $E_{\k} = \sqrt{\eps_{\k}^2 + \abs{\Delta}^2}$, onde o gap $\Delta$ não depende de $\k$ (esfericamente simétrico, ou s-wave).
\end{itemize}

\end{frame}

%%%%%%%%%%%%%%%%%%%%%%%%%%%%%%%%%%%%%%%%%%%%%%%%%%%%%%%%%%%%%%%%%%%%%%%%%%%%%%%%%%%%%%%%%%%%%%%%%


\begin{frame}{Equação do Gap}

Aproveitaremos parte do formalismo da teoria BCS. O raciocínio ainda se baseia na formação de pares de Cooper. Consideremos uma interação do tipo BCS
$$
H_I =
\frac{1}{V} \sum_{\k,\k'} V_{\k,\k'} (c_{\k\up}^\d c_{-\k\down}^\d) (c_{-\k'\down} c_{\k'\up}).
$$

Se definirmos o gap $\Delta_{\k} = \sum_{\k'} V_{\k,\k'} \ev{c_{-\k'\down} c_{\k'\up}}$ e aplicarmos o procedimento de campo médio $AB \simeq A\ev{B} + B\ev{A} - \ev{A}\ev{B}$, generalizamos a equação do gap
\begin{equation} \label{eq:gapeq}
\Delta_{\k} = - \frac{1}{V} \sum_{\k'} V_{\k,\k'} \frac{\Delta_{\k'}}{2 E_{\k'}} \tanh(\frac{\beta E_{\k'}}{2}).
\end{equation}

Devido ao sinal negativo acima, se $V_{\k,\k'}$ não for sempre negativo, a função de gap $\Delta_{\k}$ poderá ser negativa em alguns pontos, de maneira a surgirem nós. Esse fenômeno acontece em várias classes de supercondutores: orgânicos, heavy-fermion, cupratos e baseados em ferro.

\end{frame}


%%%%%%%%%%%%%%%%%%%%%%%%%%%%%%%%%%%%%%%%%%%%%%%%%%%%%%%%%%%%%%%%%%%%%%%%%%%%%%%%%%%%%%%%%%%%%%%%%

\begin{frame}{Anisotropia}

Tendo em mente que queremos capturar os nós da função de gap, faremos considerações gerais sobre potenciais para estudar como a anisotropia pode surgir. Consideremos dois potenciais, um potencial repulsivo genérico
$$
V = \frac{1}{2} \sum_{\substack{\k_1,\k_2,\q \\ \s, \s'}} V_{\q} c_{\k_1+\q,\s}^\d c_{\k_2+\q,\s'}^\d c_{\k_2,\s'} c_{\k_1,\s},
$$
e um magnético
$$
V_{\text{mag}} = \sum_{i,j} J_{ij} \, \vb{S}_i \vdot \vb{S}_j = \frac{1}{2} \sum_{\q} J_{\q} \, \vb{S}_{-\q} \vdot \vb{S}_{\q}.
$$

\n

Analisaremos primeiro o potencial repulsivo $V$.

\end{frame}

%%%%%%%%%%%%%%%%%%%%%%%%%%%%%%%%%%%%%%%%%%%%%%%%%%%%%%%%%%%%%%%%%%%%%%%%%%%%%%%%%%%%%%%%%%%%%%%%%


\begin{frame}{Potencial repulsivo (1)}

Como estamos interessados na projeção nos pares de Cooper (momento total zero), consideramos que $\k_1 = -\k_2 = \k'$, $\k_1 + \q = -(\k_2 - \q) = \k$ e $\q = \k - \k'$. A interação resultante geral pode ser separada em termos de acordo com os spins
$$
V_{BCS} = \frac{1}{2} \sum_{\substack{\k,\k' \\ \s, \s'}} V_{\k-\k'} c_{\k\s}^\d c_{-\k\s'}^\d c_{-\k'\s'}c_{\k'\s} =
V_{BCS}^{\up\down} + V_{BCS}^{\up\up} + V_{BCS}^{\down\down}.
$$

Foquemos primeiramente no termo mais familiar
$$
V_{BCS}^{\up\down} = \sum_{\k,\k'} V_{\k-\k'} (c_{\k\up}^\d c_{-\k\down}^\d) (c_{-\k'\down} c_{\k'\up}) =
\sum_{\k,\k'} V_{\k-\k'} \Psi_{\k}^\d \Psi_{\k}.
$$

\end{frame}


%%%%%%%%%%%%%%%%%%%%%%%%%%%%%%%%%%%%%%%%%%%%%%%%%%%%%%%%%%%%%%%%%%%%%%%%%%%%%%%%%%%%%%%%%%%%%%%%%

\begin{frame}{Paridade e troca de spin}

Consideremos as propriedades de paridade $P$ e spin exchange $X$ da função de onda $F(\k)_{\alpha\beta} = \braket{\k\alpha,-\k\beta}{\k_P}$ do par de Cooper, definidas por
$$
P F(\k)_{\alpha\beta} = F(-\k)_{\alpha\beta},
$$
$$
X F(\k)_{\alpha\beta} = F(\k)_{\beta\alpha}.
$$

O operador de troca de spin distingue singletos $X = -1$ de tripletos $X = +1$.

\n

Obs: Lembremos que um par de spins $\ket{\up\down}$ não é singleto nem tripleto. De fato, o singleto é $(\ket{\up\down} - \ket{\down\up})/\sqrt{2}$ e o tripleto é composto por $(\ket{\up\down} + \ket{\down\up})/\sqrt{2}$, $\ket{\up\up}$ e $\ket{\down\down}$.

\n

Se aplicarmos $X$ e $P$ em $\braket{\k\alpha,-\k\beta}{\k_P}$ estaremos trocando os dois férmions que compõem o par, adquirindo uma fase $-1$. Portanto, $XP = -1$ no subespaço dos pares de Cooper.

\n

Temos então que pares com paridade par são singletos $(P,X) = (+, -)$ e com paridade ímpar são tripletos $(P,X) = (-,+)$.

\end{frame}


%%%%%%%%%%%%%%%%%%%%%%%%%%%%%%%%%%%%%%%%%%%%%%%%%%%%%%%%%%%%%%%%%%%%%%%%%%%%%%%%%%%%%%%%%%%%%%%%%

\begin{frame}{Singleto e tripleto}

Com autovalores dos operadores $P$ e $X$ em mente, podemos decompor a interação em partes simétrica (par, singleto) e antissimétrica (ímpar, tripleto):
$$
V_{BCS}^{\up\down} = \sum_{\k,\k'}
\Bigg[
\overbrace{\qty(\frac{V_{\k-\k'} + V_{\k+\k'}}{2})}^{V_{\k,\k'}^S} +
\overbrace{\qty(\frac{V_{\k-\k'} - V_{\k+\k'}}{2})}^{V_{\k,\k'}^T}
\Bigg] \Psi_{\k}^\d \Psi_{\k}.
$$

O primeiro termo acima espalha singletos e o segundo tripletos, representados por
$$
\Psi_{\k}^{S\d} = (c_{\k\up}^\d c_{-\k\down}^\d + c_{-\k\up}^\d c_{\k\down}^\d),
\quad \Psi_{\k}^{S\d} = +\Psi_{-\k}^{S\d},
$$
$$
\Psi_{\k}^{T\d} = (c_{\k\up}^\d c_{-\k\down}^\d - c_{-\k\up}^\d c_{\k\down}^\d),
\quad \Psi_{\k}^{T\d} = -\Psi_{-\k}^{T\d}.
$$

E escrevemos
$$
V_{BCS}^{\up\down} =
\frac{1}{4} \sum_{\k,\k'}
\qty[
V_{\k,\k'}^S \Psi_{\k}^{S\d} \Psi_{\k'}^{S} +
V_{\k,\k'}^T \Psi_{\k}^{T\d} \Psi_{\k'}^{T}
].
$$

\end{frame}


%%%%%%%%%%%%%%%%%%%%%%%%%%%%%%%%%%%%%%%%%%%%%%%%%%%%%%%%%%%%%%%%%%%%%%%%%%%%%%%%%%%%%%%%%%%%%%%%%

\begin{frame}{Potencial repulsivo (2)}

Os outros termos $V_{BCS}^{\up\up}$ e $V_{BCS}^{\down\down}$ somente envolvem tripletos, que interagem via $V_{\k,\k'}^T$. Se definirmos o vetor tripleto
$$
\va*{\Psi}_{\k}^T = \sum_{\alpha\beta} c_{\k\alpha}^\d \qty(\va*{\s} i \s_2)_{\alpha\beta} c_{-\k\beta}^\d =
\begin{cases}
\; \; \; \; c_{\k\down}^\d c_{-\k\down} - c_{\k\up}^\d c_{-\k\up}^\d \; , \quad (x) \\
\; i (c_{\k\down}^\d c_{-\k\down} + c_{\k\up}^\d c_{-\k\up}^\d), \quad (y) \\
\; \; \; \; c_{\k\up}^\d c_{-\k\down}^\d + c_{-\k\up}^\d c_{\k\down}^\d \; , \quad (z),
\end{cases}
$$
podemos representar as interações de spins paralelos de maneira compacta. No final obtemos
$$
V_{BCS} =
V_{BCS}^{\up\down} + V_{BCS}^{\up\up} + V_{BCS}^{\down\down} =
\frac{1}{4} \sum_{\k,\k'}
\qty(
V_{\k,\k'}^S \Psi_{\k}^{S\d} \Psi_{\k'}^S +
V_{\k,\k'}^T \va*{\Psi}_{\k}^{T\d} \va*{\Psi}_{\k'}^T
).
$$

\end{frame}


%%%%%%%%%%%%%%%%%%%%%%%%%%%%%%%%%%%%%%%%%%%%%%%%%%%%%%%%%%%%%%%%%%%%%%%%%%%%%%%%%%%%%%%%%%%%%%%%%


\begin{frame}{Observações sobre o potencial repulsivo}

\begin{itemize}
\item Ao estudarmos os harmônicos esféricos $Y_{\ell m}$ no contexto do átomo de Hidrogênio, vemos que a paridade está associada ao número quântico de momento angular orbital $\ell$, de maneira que $Y_{\ell m}(-\r) = (-1)^\ell Y_{\ell m}(\r)$.

\n

\item Pares de Cooper com função de onda de singleto envolvem $\ell = 0, 2, \ldots$ (s, d, $\ldots$ wave) e de tripleto envolvem valores ímpares $\ell = 1, 3, \ldots$ (p, f, $\ldots$ wave).

\begin{figure}[H]
\centering
\includegraphics[width=0.27\linewidth]{fig/d-orbitals.png}
\caption{Harmônicos esféricos $Y_{2m}$ dos orbitais $d$.}
\label{fig:d-orbitals}
\end{figure}

\item Já que o potencial repulsivo $V_{BCS}$ se dissocia em interações de singleto e tripleto, podemos estudá-las separadamente. Em especial, só precisamos considerar o termo de tripleto $V^T_{\k,\k'}$ se estivermos interessados em descrever supercondutividade p-wave ou f-wave, por exemplo.

\n

\item Para um acoplamento somente de singleto, ficamos com a familiar
$$
V_{BCS} = \sum_{\k,\k'} V_{\k,\k'}^S (c_{\k\up}^\d c_{-\k\down}^\d) (c_{-\k'\down} c_{\k'\up}).
$$
\end{itemize}

\end{frame}


%%%%%%%%%%%%%%%%%%%%%%%%%%%%%%%%%%%%%%%%%%%%%%%%%%%%%%%%%%%%%%%%%%%%%%%%%%%%%%%%%%%%%%%%%%%%%%%%%

\begin{frame}{Interação magnética (1)}

Analisemos agora a interação magnética (do tipo Heisenberg)
$$
V_{\text{mag}} = \frac{1}{2} \sum_{\q} J_{\q} \, \va*{S}_{-\q} \vdot \va*{S}_{\q} =
\frac{1}{2} \sum_{\substack{\k_1,\k_2,\q \\ \alpha\beta\gamma\delta}} J_{\q} c_{\k_1+\q,\alpha}^\d c_{\k_2-\q,\gamma}^\d
\qty(\frac{\va*{\s}}{2})_{\alpha\beta} \qty(\frac{\va*{\s}}{2})_{\gamma\delta} c_{\k_2\delta} c_{\k_1\beta},
$$
onde $J_{\q}$ é a interação efetiva dos spins. Por exemplo, para spins em uma rede quadrada interagindo só com primeiros vizinhos temos $J_{\q} = 2 J [\cos(q_x a) + \cos(q_y a)]$ (tight-binding).

\n

Considerando o produto escalar $\qty(\frac{\va*{\s}}{2})_{\alpha\beta} \cdot \qty(\frac{\va*{\s}}{2})_{\gamma\delta} \equiv \va*{S}_1 \vdot \va*{S}_2$, lembremos que seus autovalores são $+\frac{1}{4}$ para o tripleto e $-\frac{3}{4}$ para o singleto (lembre da estrutura hiperfina do hidrogênio):
$$
\va*{S}_1 \vdot \va*{S}_2 =
\begin{cases}
\; +\frac{1}{4} \quad \text{(tripleto)}, \\
\; -\frac{3}{4} \quad \text{(singleto)}.
\end{cases}
$$

\end{frame}


%%%%%%%%%%%%%%%%%%%%%%%%%%%%%%%%%%%%%%%%%%%%%%%%%%%%%%%%%%%%%%%%%%%%%%%%%%%%%%%%%%%%%%%%%%%%%%%%%


\begin{frame}{Interação magnética (2)}

Já que as partes simétricas e antissimétricas da interação filtram os pares singleto e tripleto, temos que esses autovalores entram como prefatores no potencial, de maneira que
$$
V_{\k,\k'}^S = -\frac{3}{4} \qty(\frac{J_{\k-\k'} + J_{\k+\k'}}{2}), \quad
V_{\k,\k'}^T = +\frac{1}{4} \qty(\frac{J_{\k-\k'} - J_{\k+\k'}}{2}).
$$

Interações antiferromagnéticas ($J > 0 \implies V_{\k,\k'}^S < 0$) causam uma interação atrativa nos pares de singleto e interações ferromagnéticas ($J < 0 \implies V_{\k,\k'}^T < 0$) causam interação atrativa nos pares de tripleto.
$$
\begin{cases}
\; \text{interação antiferromagnética} \leftrightarrow \text{pares singleto anisotrópicos (d-wave),} \\
\; \text{interação ferromagnética} \leftrightarrow \text{pares tripleto anisotrópicos (p-wave).}
\end{cases}
$$

A saber, o estado normal dos cupratos é em geral um isolante de Mott com ordem antiferromagnética. No resto de nossa discussão daremos foco especial a eles.

\end{frame}

%%%%%%%%%%%%%%%%%%%%%%%%%%%%%%%%%%%%%%%%%%%%%%%%%%%%%%%%%%%%%%%%%%%%%%%%%%%%%%%%%%%%%%%%%%%%%%%%%


\begin{frame}{Fatos sobre os cupratos}

\begin{figure}
\centering
   \includegraphics[width=0.475\textwidth]{fig/cuo2.png}
   \quad \quad
   \includegraphics[width=0.32\textwidth]{fig/cuprate-phasediag.png}
   \caption{Propriedades gerais dos cupratos}
\end{figure}


\centering
\begin{minipage}[b]{0.75\textwidth}
  \begin{itemize}
  \item Os cupratos consistem de várias camadas de CuO$_2$, que possui uma rede quadrada.
  \item Seu estado normal é um isolante de Mott com ordem antiferromagnética.
  \item A interação atrativa elétron-elétron não é causada por fônons, mas sim por interação antiferromagnética, capaz de gerar supercondutividade d-wave.
  \item Do lado direito do diagrama (superdopado) eles se comportam como líquido de Fermi. Essa é a nossa justificativa para aplicarmos fenomenologia BCS.
  \end{itemize}
\end{minipage}
\raisebox{-1\baselineskip}{\includegraphics[width=0.072\textwidth]{fig/bscco-unitcell.png}}

\end{frame}

%%%%%%%%%%%%%%%%%%%%%%%%%%%%%%%%%%%%%%%%%%%%%%%%%%%%%%%%%%%%%%%%%%%%%%%%%%%%%%%%%%%%%%%%%%%%%%%%%%%

\begin{frame}{Modelando os cupratos}

Consideramos um modelo simplificado de um supercondutor d-wave para os cupratos, onde os elétrons se movem em uma rede quadrada com dispersão $\eps_{\k} = -2t (\cos k_x a + \cos k_y a)$. Incluimos a repulsão de Coulomb por meio de um termo de Hubbard e também consideramos interações antiferromagnéticas para primeiros vizinhos:
$$
H = \sum_{\k} \eps_{\k} c_{\k\s}^\d c_{\k\s} + \sum_{j} U n_{j\up} n_{j\down} + J \sum_{\nn{i}{j}} \va*{S}_i \vdot \va*{S}_j.
$$

Supondo que $U, J \ll t$, tratamos o material como um líquido de Fermi com uma interação BCS de singleto
$$
V^{\text{singleto}}(\q) = U - \frac{3J}{2} (\cos q_x a + \cos q_y a),
$$
onde o fator $-\frac{3}{2}$ surgiu daquele $-\frac{3}{4}$ da discussão anterior.

A interação $V_{\k,\k'}$ é obtida a partir da simetrização
$$
V_{\k,\k'} = \frac{1}{2}
\Big[
V^{\text{singleto}}(\k-\k') + V^{\text{singleto}}(\k+\k')
\Big] =
U - \frac{3}{2} (c_x c_{x'} + c_y c_{y'}),
$$
onde denotamos $c_x = \cos(k_x a)$ e $c_y = \cos(k_y a)$. A hamiltoniana BCS é então
$$
H_{BCS} = \sum_{\k\s} \eps_{\k} c_{\k\s}^\d c_{\k\s} +
\sum_{\k,\k'} \qty[U - \frac{3}{2} (c_x c_{x'} + c_y c_{y'})]
c_{\k\up}^\d c_{-\k\down}^\d c_{-\k'\down} c_{\k'\up}.
$$

\end{frame}

%%%%%%%%%%%%%%%%%%%%%%%%%%%%%%%%%%%%%%%%%%%%%%%%%%%%%%%%%%%%%%%%%%%%%%%%%%%%%%%%

\begin{frame}{Termos s-wave e d-wave}

Podemos separar a interação $V_{\k,\k'} = V_{\k,\k'}^s + V_{\k,\k'}^d$ em um termo s-wave e d-wave:
$$
V_{\k,\k'}^s =
\overbrace{U}^{\text{s-wave}} -
\overbrace{\frac{3}{4} J (c_x + c_y) (c_{x'} + c_{y'})}^{\text{s-wave estendida}}
\quad \text{(s-wave)},
$$
$$
V_{\k,\k'}^d = - \frac{3}{4} J (c_x - c_y) (c_{x'} - c_{y'})
\quad \text{(d-wave)},
$$
onde o termo s-wave é invariante por rotações de $90^\circ$ e o termo d-wave troca de sinal, $V_{\k,\k'}^s = + V_{\k, R\k'}^s$ e $V_{\k,\k'}^d = - V_{\k, R\k'}^d$ com $R\k = (-k_y, k_x)$.

\n

Se analisarmos a equação \ref{eq:gapeq} do gap
$$
\Delta_{\k} = - \int \frac{\dd[2]{\k'}}{(2\pi)^2} (V_{\k,\k'}^s + V_{\k,\k'}^d) \frac{\Delta_{\k'}}{2 E_{\k'}} \tanh(\frac{\beta E_{\k'}}{2}),
$$
é possível escrever também $\Delta_{\k} = \Delta_{\k}^s + \Delta_{\k}^d$, com
$$
\Delta_{\k}^s = \Delta_1 + \Delta_ 2 (c_x + c_y) = + \Delta_{R\k}^s,
$$
$$
\Delta_{\k}^d = \Delta_d (c_x - c_y) = - \Delta_{R\k}^d.
$$

Os dois termos $\Delta_{\k,\k'}^s$ e $\Delta_{\k,\k'}^d$ se acoplam somente com as respectivas interações s-wave e d-wave na equação do gap, pois $\int \frac{\dd[2]{\k'}}{(2\pi)^2} V_{\k,\k'}^s \Delta^d_{\k'} (\ldots) = 0$ e $\int \frac{\dd[2]{\k'}}{(2\pi)^2} V_{\k,\k'}^d \Delta^s_{\k'} (\ldots) = 0$, devido às integrais mudarem de sinal sob uma rotação de $90^\circ$ e, portanto, serem nulas. Concluimos que as duas simetrias s-wave e d-wave são desacopladas e, em particular, a simetria d-wave é ortogonal ao potencial de Coulomb local $U$.


\end{frame}

%%%%%%%%%%%%%%%%%%%%%%%%%%%%%%%%%%%%%%%%%%%%%%%%%%%%%%%%%%%%%%%%%%%%%%%%%%%%%%%%

\begin{frame}{Simetria d-wave}

\n

Analisando o gap d-wave $\Delta_{\k}^d = \Delta_d (c_x - c_y)$, vemos que ele possui nós nas diagonais $k_x = \pm k_y$, de maneira que essa simetria d-wave tem a forma do orbital $d_{x^2-y^2}$ da Figura \ref{fig:d-orbitals}. A energia das quasepartículas $E_{\k} = \sqrt{\eps_{\k}^2 + \Delta_d^2 (c_x - c_y)^2}$ se anula na interseção dessas diagonais (onde $\Delta_{\k} = 0$) com a superfície de Fermi (onde $\eps_{\k} = 0$), de maneira a formar cones de Dirac nesses pontos (nodais), ilustrados na Figura \ref{fig:fermisurf}:

\begin{figure}[H]
\centering
\includegraphics[width=0.45\linewidth]{fig/fermisurf.png}
\caption{Cones de Dirac nos pontos nodais, que se localizam na interseção da superfície de Fermi (FS) com as retas $k_x = \pm k_y$.}
\label{fig:fermisurf}
\end{figure}


\end{frame}

%%%%%%%%%%%%%%%%%%%%%%%%%%%%%%%%%%%%%%%%%%%%%%%%%%%%%%%%%%%%%%%%%%%%%%%%%%%%%%%%

\begin{frame}{Gap d-wave}

Utilizando a equação do gap, podemos fazer algumas aproximações para ter uma ideia da física envolvida. Supondo que o preenchimento ao redor de $\Gamma (\k = \0)$ seja pequeno, temos $\eps_{\k} = -2t (c_x + c_y) \simeq -4t  + t k^2$ e o gap
$$
\Delta_{\k}^d = \Delta_d (c_x - c_y) \simeq -\frac{\Delta_d}{2 k_F^2}
\qty[\qty(\frac{k_x}{k_F})^2 - \qty(\frac{k_y}{k_F})^2] = - \Delta_0 \cos(2\theta),
\quad \Delta_0 = \frac{\Delta_d}{2 k_F^2}, \k = (k_x, k_y) = \abs{\k} e^{i\theta}.
$$
Perceba que $\Delta^d(\theta) \propto \cos(2\theta)$ remete ao harmônico esférico do orbital $d_{x^2 - y^2}$.

\n

Colocando um cutoff na energia $\abs{\eps} \leq \omega_0$ e assumindo que a densidade de estados por spin seja constante $\rho(E_F) = \frac{1}{4\pi t}$, obtemos a equação do gap:
$$
1 = \frac{3 J}{4} \rho(E_F) \int_{-\omega_0}^{\omega_0} \dd{\eps}
\int_0^{2\pi} \frac{\dd{\theta}}{2\pi} \cos[2](2\theta) \frac{\tanh(\frac{\beta E}{2})}{2E}, \quad E = \sqrt{\eps^2 + [\Delta_0 \cos(2\theta)]^2}.
$$

Aproximando grosseiramente $\cos[2](2\theta)$ pela sua média $1/2$, chegamos numa equação aproximada (similar à BCS) para $T_c$:
$$
1 = \frac{3J}{8} \rho(E_F) \int_0^{\omega_0} \dd{\eps} \,
\frac{\tanh(\frac{\eps}{2T_c})}{\eps},
$$
onde é possível obter $T_c \sim 1.13 \, \omega_0 e^{-\frac{8}{3 J \rho(E_F)}}$.


\end{frame}

%%%%%%%%%%%%%%%%%%%%%%%%%%%%%%%%%%%%%%%%%%%%%%%%%%%%%%%%%%%%%%%%%%%%%%%%%%%%%%%%

\begin{frame}{Densidade de estados d-wave}

Também podemos calcular a densidade de estados aproximada tomando uma média sobre o ângulo $\theta$:
$$
\frac{N_d(E)}{N(0)} = \dv{\eps}{E} =
\Re{\int_0^{2\pi} \frac{\dd{\theta}}{2\pi}
\frac{\abs{E}}{\sqrt{(E - i\delta)^2 + [\Delta_0 \cos(2\theta)]^2}}
},
$$
em que obtemos a Figura \ref{fig:dwavedos}:

\begin{figure}[H]
\centering
\includegraphics[width=0.25\textwidth]{fig/dwavedos.png}
\caption{Densidade de estados $N_d(E)/N(0)$ aproximada para um supercondutor d-wave.}
\label{fig:dwavedos}
\end{figure}

Note que não temos mais uma densidade de estados gapeada, pois de fato há cones de Dirac pelos nós nas diagonais $k_x = \pm k_y$. A Figura \ref{fig:dwavedos} na verdade parece muito com a DOS do grafeno.

\end{frame}

%%%%%%%%%%%%%%%%%%%%%%%%%%%%%%%%%%%%%%%%%%%%%%%%%%%%%%%%%%%%%%%%%%%%%%%%%%%%%%%%

\begin{frame}{Simetria s-wave estendida}

Exploremos por final a componente s-wave do gap $\Delta_{\k}^s = \Delta_1 + \Delta_2 (c_x + c_y)$. A equação do gap correspondente é
$$
\Delta_{\k}^s = - \int \frac{\dd[2]{\k'}}{(2\pi)^2}
\overbrace{\qty[U - \frac{3}{4} J (c_x + c_y)(c_{x'}+ c_{y'})]}^{V_{\k,\k'}^s}
\frac{\Delta_{\k'}^s}{2 E_{\k'}} \, \tanh(\frac{\beta E_{\k'}}{2}).
$$

Ela é mais complicada pois existe acoplamento entre o termo local $\Delta_1$ e o termo s-wave estendido $\Delta_2 (c_x + c_y)$.

\n

De forma simplificada, assumindo uma única superfície de Fermi, um cutoff $\abs{\eps} \leq \omega_0$ e expandindo $\k, \k' \ll t$ próximo do ponto $\Gamma$ de maneira que $c_x \simeq c_y \simeq 1$, temos que a interação efetiva é da ordem de $V_{\k,\k'}^s \simeq U - 3J$.
\n

Isso indica que, para uma única superfície de Fermi, a atração s-wave estendida é suprimida pela interação de Coulomb $U$. Isso faz com que a interação efetiva seja reduzida.

\n

Pensando na fórmula da temperatura crítica BCS para a componente s-wave $T_c^s \sim 1.13 \, \omega_0 e^{-\frac{1}{g \rho(E_F)}}$, essa redução na constante efetiva de acoplamento $g \sim V_{\k,\k'}^s \simeq U-3J$ faz com que a temperatura crítica s-wave diminua.

\n

Essas estimativas indicam que nos cupratos o $T_c^d$ da componente d-wave é maior que $T_c^s$ (que foi suprimido por $U$), fazendo com que o pareamento d-wave seja predominante.

\end{frame}

%%%%%%%%%%%%%%%%%%%%%%%%%%%%%%%%%%%%%%%%%%%%%%%%%%%%%%%%%%%%%%%%%%%%%%%%%%%%%%%%

\begin{frame}{Referências}

\footnotesize

\begin{thebibliography}{10}
\bibitem{coleman}
\alert{P. Coleman.}
\newblock {Introduction to Many-Body Physics.}
\newblock {Cambridge University Press, 2015.}

\bibitem{timm}
\alert{Carsten Timm.}
\newblock {Theory of Superconductivity.}
\newblock {Version: March 24, 2023.}

\bibitem{tsuei}
\alert{C. C. Tsuei and J. R. Kirtley}
\newblock {Pairing symmetry in cuprate superconductors.}
\newblock {\textit{Rev. Mod. Phys.}, 72, 969, October 2000.}

\bibitem{wiki-cuprate}
\alert{Wikipédia contributors.}
\newblock {Cuprate superconductor - Wikipédia}
\newblock {\url{https://en.wikipedia.org/wiki/Cuprate_superconductor}.}

\end{thebibliography}


\end{frame}

\end{document}
