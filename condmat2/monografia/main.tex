\documentclass[a4paper,11pt]{article}
\usepackage[brazilian]{babel}
\usepackage[left=2.5cm,right=2.5cm,top=3cm,bottom=2.5cm]{geometry}
\usepackage{mathtools}
\usepackage{amsthm}
\usepackage{amsmath}
%\usepackage{nccmath}
\usepackage{amssymb}
\usepackage{amsfonts}
\usepackage{physics}
%\usepackage{dsfont}
%\usepackage{mathrsfs}

\usepackage{titling}
\usepackage{indentfirst}

\usepackage{bm}
\usepackage[dvipsnames]{xcolor}
\usepackage{cancel}

\usepackage{xurl}
\usepackage[colorlinks=true]{hyperref}

\usepackage{float}
\usepackage{graphicx}
%\usepackage{tikz}
\usepackage{caption}
\usepackage{subcaption}

%%%%%%%%%%%%%%%%%%%%%%%%%%%%%%%%%%%%%%%%%%%%%%%%%%%

\newcommand{\eps}{\epsilon}
\newcommand{\vphi}{\varphi}
\newcommand{\cte}{\text{cte}}

\newcommand{\N}{\mathbb{N}}
\newcommand{\Z}{\mathbb{Z}}
\newcommand{\Q}{\mathbb{Q}}
\newcommand{\R}{\vb{R}}
\newcommand{\C}{\mathbb{C}}
\renewcommand{\S}{\hat{S}}
%\renewcommand{\H}{\s{H}}

\renewcommand{\a}{\vb{a}}
\newcommand{\nn}{\hat{n}}
\renewcommand{\d}{\dagger}
\newcommand{\up}{\uparrow}
\newcommand{\down}{\downarrow}

\newcommand{\0}{\vb{0}}
%\newcommand{\1}{\mathds{1}}
\newcommand{\E}{\vb{E}}
\newcommand{\B}{\vb{B}}
\renewcommand{\v}{\vb{v}}
\renewcommand{\r}{\vb{r}}
\renewcommand{\k}{\vb{k}}
\newcommand{\p}{\vb{p}}
\newcommand{\q}{\vb{q}}
\newcommand{\F}{\vb{F}}

\newcommand{\s}{\sigma}
%\newcommand{\prodint}[2]{\left\langle #1 , #2 \right\rangle}
\newcommand{\cc}[1]{\overline{#1}}
\newcommand{\Eval}[3]{\eval{\left( #1 \right)}_{#2}^{#3}}

\newcommand{\unit}[1]{\; \mathrm{#1}}

\newcommand{\n}{\medskip}
\newcommand{\e}{\quad \mathrm{e} \quad}
\newcommand{\ou}{\quad \mathrm{ou} \quad}
\newcommand{\virg}{\, , \;}
\newcommand{\ptodo}{\forall \,}
\renewcommand{\implies}{\; \Rightarrow \;}
%\newcommand{\eqname}[1]{\tag*{#1}} % Tag equation with name

\setlength{\droptitle}{-7em}

\theoremstyle{plain}
\newtheorem{theorem}{Teorema}[section]
%\newtheorem{defi}[theorem]{Definição}
\newtheorem{lemma}[theorem]{Lema}
%\newtheorem{corol}[theorem]{Corolário}
%\newtheorem{prop}[theorem]{Proposição}
%\newtheorem{example}{Exemplo}
%
%\newtheorem{inneraxiom}{Axioma}
%\newenvironment{axioma}[1]
%  {\renewcommand\theinneraxiom{#1}\inneraxiom}
%  {\endinneraxiom}
%
%\newtheorem{innerpostulado}{Postulado}
%\newenvironment{postulado}[1]
%  {\renewcommand\theinnerpostulado{#1}\innerpostulado}
%  {\endinnerpostulado}
%
%\newtheorem{innerexercise}{Exercício}
%\newenvironment{exercise}[1]
%  {\renewcommand\theinnerexercise{#1}\innerexercise}
%  {\endinnerexercise}
%
%\newtheorem{innerthm}{Teorema}
%\newenvironment{teorema}[1]
%  {\renewcommand\theinnerthm{#1}\innerthm}
%  {\endinnerthm}
%
\newtheorem{innerlema}{Lema}
\newenvironment{lema}[1]
  {\renewcommand\theinnerlema{#1}\innerlema}
  {\endinnerlema}
%
%\theoremstyle{remark}
%\newtheorem*{hint}{Dica}
%\newtheorem*{notation}{Notação}
%\newtheorem*{obs}{Observação}


\title{\Huge{\textbf{Monografia}}}
\author{Mateus Marques}

\begin{document}

\maketitle

\section{Introdução}

\begin{itemize}
\item Temos como objetivo discutir supercondutividade não-convencional fenomenologicamente.
\item Ainda não levamos em consideração a interação repulsiva (Coulomb) entre elétrons na teoria BCS, somente atrativa quando escrevemos
$$
U_{\k\k'} =
\begin{cases}
\; -g, \text{ se } E_F \leq \eps(\k), \eps(\k') \leq E_F + \hbar \omega_D \\
\; 0, \text{ caso contrário.}
\end{cases}
$$
\item Para que a função de onda do par de Cooper (e o gap) sejam anisotrópicos, devemos considerar também a repulsão entre os elétrons.
\item A tal da supercondutividade não-convencional tem relação com os supercondutores em questão serem anisotrópicos, sendo a interação repulsiva de Coulomb responsável por tal anisotropia. No contexto dessa anisotropia, a função de onda dos pares de Cooper (formada por pares de elétron com momento angular orbital $\ell \geq 1$) e a função de gap possuem nós.
\item Dois exemplos de pares com funções de onda anisotrópicas são os pares \textit{p-wave} do superfluido $^3$He e os pares \textit{d-wave} dos supercondutores de alta temperatura baseados em cupratos.
\end{itemize}

\section{Aprender é recordar}

Em nossas aulas sobre fônons deduzimos a interação efetiva elétron-elétron mediada por fônons, sendo dada por
$$
H_{\text{int}} = \sum_{\k,\p,\q} \abs{g_{\q}}^2 c_{\k+\q} c_{\p-\q}^\d c_{\p} c_{\k} \,
\frac{\hbar \omega_{\q}}{[\eps(\p-\q) - \eps(p)]^2 - \hbar^2 \omega_{\q}^2},
$$
e foi argumentado que essa interação é atrativa para $\abs{\eps(\p-\q) - \eps(\q)} < \hbar \omega_{\q}$. \textbf{Incluir o conceito atração retardada}. Mas considerando apenas $H_{\text{int}}$, estamos ignorando o termo repulsivo da interação de Coulomb, que justamente é responsável pela anisotropia (além de \textit{s-wave}).

\section{Teoria BCS com gap dependente do momento}

Em BCS tínhamos
$$
H = \sum_{\k\s} \eps_{\k\s} c_{\k\s}^\d c_{\k\s} +
\frac{1}{V} \sum_{\k,\k'} U_{\k,\k'} (c_{\k\up}^\d c_{-\k\down}^\d) (c_{-\k'\down} c_{\k'\up}).
$$

A simetria \textit{s-wave} veio de termos assumido a interação $U_{\k,\k'}$ isotrópica.

Assumindo um gap dependente do momento $\Delta_{\k} = \sum_{\k'} U_{k, k'} \ev{c_{-\k'\down}c_{\k'\up}}$, generalizamos a equação do gap vista em aula para
\begin{equation} \label{eq:gapeq}
\Delta_{\k} = - \sum_{\k'} U_{\k,\k'} \frac{\Delta_{\k'}}{2 E_{\k'}} \tanh(\frac{\beta E_{\k'}}{2}).
\end{equation}

Devido ao sinal de menos do lado direito da equação \ref{eq:gapeq}, se a interação $U_{\k,\k'}$ for sempre negativa (atrativa), então a equação \ref{eq:gapeq} é satisfeita por uma função gap sempre positiva. Mas, em geral $U_{\k,\k'}$ será positiva para alguns momentos, de maneira que a função de gap também será negativa em alguns pontos, surgindo pontos onde ela troca de sinal (nós).


\section{Acoplamento}

Se definirmos o operador paridade $P$ e o operador troca de spin $X$ para as funções de onda dos pares de Cooper, temos que $XP = -1$. Por isso, estados com paridade par são necessariamente singletos e de paridade ímpar tripletos.

\pagebreak


\section{Fatos experimentais}

From \url{https://en.wikipedia.org/wiki/Cuprate_superconductor}

\textbf{Introcution Section}

Cuprate superconductors are a family of high-temperature superconducting materials made of layers of copper oxides (CuO2) alternating with layers of other metal oxides, which act as charge reservoirs.

\textbf{Structure Section}

Cuprates are layered materials, consisting of superconducting planes of copper oxide, separated by layers containing ions such as lanthanum, barium, strontium, which act as a charge reservoir, doping electrons or holes into the copper-oxide planes. Thus the structure is described as a superlattice of superconducting CuO2 layers separated by spacer layers, resulting in a structure often closely related to the perovskite structure. Superconductivity takes place within the copper-oxide (CuO2) sheets, with only weak coupling between adjacent CuO2 planes, making the properties close to that of a two-dimensional material. Electrical currents flow within the CuO2 sheets, resulting in a large anisotropy in normal conducting and superconducting properties, with a much higher conductivity parallel to the CuO2 plane than in the perpendicular direction.

Critical superconducting temperatures depend on the chemical compositions, cations substitutions and oxygen content. Chemical formulae of superconducting materials generally contain fractional numbers to describe the doping required for superconductivity. There are several families of cuprate superconductors which can be categorized by the elements they contain and the number of adjacent copper-oxide layers in each superconducting block. For example, YBCO and BSCCO can alternatively be referred to as Y123 and Bi2201/Bi2212/Bi2223 depending on the number of layers in each superconducting block (n). The superconducting transition temperature has been found to peak at an optimal doping value (p=0.16) and an optimal number of layers in each superconducting block, typically n=3.

The undoped "parent" or "mother" compounds are Mott insulators with long-range antiferromagnetic order at sufficiently low temperatures. Single band models are generally considered to be enough to describe the electronic properties.

Cuprate superconductors usually feature copper oxides in both the oxidation states 3+ and 2+. For example, YBa2Cu3O7 is described as Y3+(Ba2+)2(Cu3+)(Cu2+)2(O2−)7. The copper 2+ and 3+ ions tend to arrange themselves in a checkerboard pattern, a phenomenon known as charge ordering.[8] All superconducting cuprates are layered materials having a complex structure described as a superlattice of superconducting CuO2 layers separated by spacer layers, where the misfit strain between different layers and dopants in the spacers induce a complex heterogeneity that in the superstripes scenario is intrinsic for high-temperature superconductivity.

\textbf{Superconducting Mechanism Section}

Similarities between the low-temperature antiferromagnetic state in undoped materials and the low-temperature superconducting state that emerges upon doping, primarily the dx2-y2 orbital state of the Cu2+ ions, suggest that electron-phonon coupling is less relevant in cuprates. Recent work on the Fermi surface has shown that nesting occurs at four points in the antiferromagnetic Brillouin zone where spin waves exist and that the superconducting energy gap is larger at these points. The weak isotope effects observed for most cuprates contrast with conventional superconductors that are well described by BCS theory.

\textbf{QUORA}

\url{https://www.quora.com/What-is-convincing-experimental-evidence-for-d-wave-pairing-symmetry-in-high-Tc-cuprate-superconductors-1}

(D. A. Wollman, et al. Phys. Rev. Lett. 71 2134 (1993)) and vortices in tri-crystal boundaries (C.C. Tsuei andJ.R. Kirtley, et al. Phys. Rev. Lett. 73 593 (1994)).

(K. A. Moler et al Phys. Rev. B 55, 3954 (1997))

(W. N. Hardy et al, Phys. Rev. Lett. 70, 3999 (1993))

(Z-X Shen et al. Phys. Rev. Lett. 70 1553 (1993)).

\n

Qual a diferença entre \textit{s-wave} e \textit{d-wave}?

\url{https://www.quora.com/What-is-a-d-wave-superconductor-vs-an-s-wave-superconductor}

tl;dr: s-wave superconductors have a superconducting gap which is isotropic in all directions, whereas dx2−y2

superconductors have a superconducting gap which is anisotropic and is identically zero at four line nodes located at the diagonals of the Brillouin zone.

A superconducting wavefunction has both a spin and an orbital (spatial) component. The spin component can be in a singlet state
(Cooper pairs of opposite spin, S=0) or in a triplet state
(Cooper pairs of the same spin, S=1; yes, this exists). The orbital component can have angular momentum l=0 (s), l=1 (p), l=2 (d), l=3(f), and so on. As a first order approximation, you can think of the shapes of spherical harmonics

, when considering the orbital component, with the caveat that the crystal lattice and Fermiology can make the situation more complex in real materials. Because Fermions obey antisymmetric exchange (switching two electrons corresponds to a sign change), if the spin part of the wavefunction is antisymmetric (Singlet Cooper pairs), then the orbital part has to be even (l=0,2...). An s-wave superconductor corresponds to S=0 and l=0, while a d-wave superconductor corresponds to S=0 and l=2.

Now lets make the answer more concrete in terms of experimentally measured parameters. The spatial part of the superconducting order parameter can be expressed at ≈Δ(k)eıϕ(k)
, where Δ(k) is the magnitude of the superconducting gap and ϕ(k)

is the phase of the order parameter, both of which may have a momentum-dependence in k-space. The magnitude of a superconducting gap roughly represents the energy required to break a Cooper pair. The phase is a factor that the superconducting wavefunction acquires spontaneously below the transition temperature (Tc), akin to how a ferromagnet spontaneously picks a magnetization direction. The phase of a superconducting order parameter is relevant for analyzing Josephson junctions and also for visualizing non-s-wave superconductors.

In momentum space, an s-wave superconducting gap has isotropic magnitude in all directions*, and it has a fixed phase in all directions. A d-wave gap breaks rotational symmetry such that different regions of the Brilliouin zone will have the phase of the order parameter differ by 180-degrees. At the boundary between these regions, the magnitude of the superconducting gap will be identically zero, because the phase switches signs. If a Fermi surface intersects this boundary, it will feature special spots called 'nodes' where the superconducting gap is exactly zero. If you have small Fermi surfaces which do not cross this boundary, the phase will differ by 180 degrees between Fermi surfaces, but there will be no nodes in the magnitude of the gap. Two examples (in 2D) are shown: a dx2−y2
gap and a dxy gap. These differ in the position of the nodes (along the Brillouin zone diagonal vs the boundary).

*s-wave gaps can have variations in magnitude around the Fermi surface or even 'accidental' nodes where the gap goes to zero, but not in a symmetry-protected way, such that small perturbations can turn the node into a finite gap.

As a closing comment, I want to give examples of superconductors with different kinds of pairing symmetries.

    s-wave (S=0, l=0): all elemental superconductors such as Al, Nb, and Pb
    p-wave (S=1,l=1): Sr2RuO4
    d-wave (S=0, l=2): Cuprate high temperature superconductors
    f-wave (S=1, l=3): UPt3



\end{document}
