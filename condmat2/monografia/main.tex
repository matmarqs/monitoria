\documentclass[a4paper,10pt]{article}
\usepackage[brazilian]{babel}
\usepackage[left=2.5cm,right=2.5cm,top=3cm,bottom=2.5cm]{geometry}
\usepackage{mathtools}
\usepackage{amsthm}
\usepackage{amsmath}
%\usepackage{nccmath}
\usepackage{amssymb}
\usepackage{amsfonts}
\usepackage{physics}
%\usepackage{dsfont}
%\usepackage{mathrsfs}

\usepackage{titling}
\usepackage{indentfirst}

\usepackage{bm}
\usepackage[dvipsnames]{xcolor}
\usepackage{cancel}

\usepackage{xurl}
\usepackage[colorlinks=true]{hyperref}

\usepackage{float}
\usepackage{graphicx}
%\usepackage{tikz}
\usepackage{caption}
\usepackage{subcaption}

%%%%%%%%%%%%%%%%%%%%%%%%%%%%%%%%%%%%%%%%%%%%%%%%%%%

\newcommand{\eps}{\epsilon}
\newcommand{\vphi}{\varphi}
\newcommand{\cte}{\text{cte}}

\newcommand{\N}{\mathbb{N}}
\newcommand{\Z}{\mathbb{Z}}
\newcommand{\Q}{\mathbb{Q}}
\newcommand{\R}{\vb{R}}
\newcommand{\C}{\mathbb{C}}
\renewcommand{\S}{\hat{S}}
%\renewcommand{\H}{\s{H}}

\renewcommand{\a}{\vb{a}}
\newcommand{\nn}{\hat{n}}
\renewcommand{\d}{\dagger}
\newcommand{\up}{\uparrow}
\newcommand{\down}{\downarrow}

\newcommand{\0}{\vb{0}}
%\newcommand{\1}{\mathds{1}}
\newcommand{\E}{\vb{E}}
\newcommand{\B}{\vb{B}}
\renewcommand{\v}{\vb{v}}
\renewcommand{\r}{\vb{r}}
\renewcommand{\k}{\vb{k}}
\newcommand{\p}{\vb{p}}
\newcommand{\q}{\vb{q}}
\newcommand{\F}{\vb{F}}

\newcommand{\s}{\sigma}
%\newcommand{\prodint}[2]{\left\langle #1 , #2 \right\rangle}
\newcommand{\cc}[1]{\overline{#1}}
\newcommand{\Eval}[3]{\eval{\left( #1 \right)}_{#2}^{#3}}

\newcommand{\unit}[1]{\; \mathrm{#1}}

\newcommand{\n}{\medskip}
\newcommand{\e}{\quad \mathrm{e} \quad}
\newcommand{\ou}{\quad \mathrm{ou} \quad}
\newcommand{\virg}{\, , \;}
\newcommand{\ptodo}{\forall \,}
\renewcommand{\implies}{\; \Rightarrow \;}
%\newcommand{\eqname}[1]{\tag*{#1}} % Tag equation with name

\setlength{\droptitle}{-7em}

\theoremstyle{plain}
\newtheorem{theorem}{Teorema}[section]
%\newtheorem{defi}[theorem]{Definição}
\newtheorem{lemma}[theorem]{Lema}
%\newtheorem{corol}[theorem]{Corolário}
%\newtheorem{prop}[theorem]{Proposição}
%\newtheorem{example}{Exemplo}
%
%\newtheorem{inneraxiom}{Axioma}
%\newenvironment{axioma}[1]
%  {\renewcommand\theinneraxiom{#1}\inneraxiom}
%  {\endinneraxiom}
%
%\newtheorem{innerpostulado}{Postulado}
%\newenvironment{postulado}[1]
%  {\renewcommand\theinnerpostulado{#1}\innerpostulado}
%  {\endinnerpostulado}
%
%\newtheorem{innerexercise}{Exercício}
%\newenvironment{exercise}[1]
%  {\renewcommand\theinnerexercise{#1}\innerexercise}
%  {\endinnerexercise}
%
%\newtheorem{innerthm}{Teorema}
%\newenvironment{teorema}[1]
%  {\renewcommand\theinnerthm{#1}\innerthm}
%  {\endinnerthm}
%
\newtheorem{innerlema}{Lema}
\newenvironment{lema}[1]
  {\renewcommand\theinnerlema{#1}\innerlema}
  {\endinnerlema}
%
%\theoremstyle{remark}
%\newtheorem*{hint}{Dica}
%\newtheorem*{notation}{Notação}
%\newtheorem*{obs}{Observação}


\title{\Huge{\textbf{Monografia}}}
\author{Mateus Marques}

\begin{document}

\maketitle

\section{Introdução}

\begin{itemize}
\item Temos como objetivo discutir supercondutividade não-convencional fenomenologicamente.
\item Ainda não levamos em consideração a interação repulsiva (Coulomb) entre elétrons na teoria BCS, somente atrativa quando escrevemos
$$
U_{\k\k'} =
\begin{cases}
\; -g, \text{ se } E_F \leq \eps(\k), \eps(\k') \leq E_F + \hbar \omega_D \\
\; 0, \text{ caso contrário.}
\end{cases}
$$
\item Para que a função de onda do par de Cooper (e o gap) sejam anisotrópicos, devemos considerar também a repulsão entre os elétrons.
\item A tal da supercondutividade não-convencional tem relação com os supercondutores em questão serem anisotrópicos, sendo a interação repulsiva de Coulomb responsável por tal anisotropia. No contexto dessa anisotropia, a função de onda dos pares de Cooper (formada por pares de elétron com momento angular orbital $\ell \geq 1$) e a função de gap possuem nós.
\item Dois exemplos de pares com funções de onda anisotrópicas são os pares \textit{p-wave} do superfluido $^3$He e os pares \textit{d-wave} dos supercondutores de alta temperatura baseados em cupratos.
\end{itemize}

\section{Aprender é recordar}

Em nossas aulas sobre fônons deduzimos a interação efetiva elétron-elétron mediada por fônons, sendo dada por
$$
H_{\text{int}} = \sum_{\k,\p,\q} \abs{g_{\q}}^2 c_{\k+\q} c_{\p-\q}^\d c_{\p} c_{\k} \,
\frac{\hbar \omega_{\q}}{[\eps(\p-\q) - \eps(p)]^2 - \hbar^2 \omega_{\q}^2},
$$
e foi argumentado que essa interação é atrativa para $\abs{\eps(\p-\q) - \eps(\q)} < \hbar \omega_{\q}$. \textbf{Incluir o conceito atração retardada}. Mas considerando apenas $H_{\text{int}}$, estamos ignorando o termo repulsivo da interação de Coulomb, que justamente é responsável pela anisotropia (além de \textit{s-wave}).

\section{Teoria BCS com gap dependente do momento}

Em BCS tínhamos
$$
H = \sum_{\k\s} \eps_{\k\s} c_{\k\s}^\d c_{\k\s} +
\sum_{\k,\k'} V_{\k,\k'} (c_{\k\up}^\d c_{-\k\down}^\d) (c_{-\k'\down} c_{\k'\up})
$$

Assumindo um gap dependente do momento $\Delta_{\k} = \ev{c_{-\k\down}c_{\k\up}}$, generalizamos a equação do gap vista em aula para
\begin{equation} \label{eq:gapeq}
\Delta_{\k} = - \sum_{\k'} V_{\k,\k'} \frac{\Delta_{\k'}}{2 E_{\k'}} \tanh(\frac{\beta E_{\k'}}{2}).
\end{equation}

Devido ao sinal de menos do lado direito da equação \ref{eq:gapeq}



\end{document}
