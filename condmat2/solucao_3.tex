\documentclass[a4paper,10pt]{article}
\usepackage[brazilian]{babel}
\usepackage[left=2.5cm,right=2.5cm,top=3cm,bottom=2.5cm]{geometry}
\usepackage{mathtools}
\usepackage{amsthm}
\usepackage{amsmath}
%\usepackage{nccmath}
\usepackage{amssymb}
\usepackage{amsfonts}
\usepackage{physics}
%\usepackage{dsfont}
%\usepackage{mathrsfs}

\usepackage{titling}
\usepackage{indentfirst}

\usepackage{bm}
\usepackage[dvipsnames]{xcolor}
\usepackage{cancel}

\usepackage{xurl}
\usepackage[colorlinks=true]{hyperref}

\usepackage{float}
\usepackage{graphicx}
%\usepackage{tikz}
\usepackage{caption}
\usepackage{subcaption}

%%%%%%%%%%%%%%%%%%%%%%%%%%%%%%%%%%%%%%%%%%%%%%%%%%%

\newcommand{\eps}{\epsilon}
\newcommand{\vphi}{\varphi}
\newcommand{\cte}{\text{cte}}

\newcommand{\N}{\mathbb{N}}
\newcommand{\Z}{\mathbb{Z}}
\newcommand{\Q}{\mathbb{Q}}
\newcommand{\R}{\vb{R}}
\newcommand{\C}{\mathbb{C}}
\renewcommand{\S}{\hat{S}}
%\renewcommand{\H}{\s{H}}

\renewcommand{\a}{\vb{a}}
\newcommand{\nn}{\hat{n}}
\renewcommand{\d}{\dagger}
\newcommand{\up}{\uparrow}
\newcommand{\down}{\downarrow}

\newcommand{\0}{\vb{0}}
%\newcommand{\1}{\mathds{1}}
\newcommand{\E}{\vb{E}}
\newcommand{\B}{\vb{B}}
\renewcommand{\v}{\vb{v}}
\renewcommand{\r}{\vb{r}}
\renewcommand{\k}{\vb{k}}
\newcommand{\p}{\vb{p}}
\newcommand{\q}{\vb{q}}
\newcommand{\F}{\vb{F}}

\newcommand{\s}{\sigma}
%\newcommand{\prodint}[2]{\left\langle #1 , #2 \right\rangle}
\newcommand{\cc}[1]{\overline{#1}}
\newcommand{\Eval}[3]{\eval{\left( #1 \right)}_{#2}^{#3}}

\newcommand{\unit}[1]{\; \mathrm{#1}}

\newcommand{\n}{\medskip}
\newcommand{\e}{\quad \mathrm{e} \quad}
\newcommand{\ou}{\quad \mathrm{ou} \quad}
\newcommand{\virg}{\, , \;}
\newcommand{\ptodo}{\forall \,}
\renewcommand{\implies}{\; \Rightarrow \;}
%\newcommand{\eqname}[1]{\tag*{#1}} % Tag equation with name

\setlength{\droptitle}{-7em}

\theoremstyle{plain}
\newtheorem{theorem}{Teorema}[section]
%\newtheorem{defi}[theorem]{Definição}
\newtheorem{lemma}[theorem]{Lema}
%\newtheorem{corol}[theorem]{Corolário}
%\newtheorem{prop}[theorem]{Proposição}
%\newtheorem{example}{Exemplo}
%
%\newtheorem{inneraxiom}{Axioma}
%\newenvironment{axioma}[1]
%  {\renewcommand\theinneraxiom{#1}\inneraxiom}
%  {\endinneraxiom}
%
%\newtheorem{innerpostulado}{Postulado}
%\newenvironment{postulado}[1]
%  {\renewcommand\theinnerpostulado{#1}\innerpostulado}
%  {\endinnerpostulado}
%
%\newtheorem{innerexercise}{Exercício}
%\newenvironment{exercise}[1]
%  {\renewcommand\theinnerexercise{#1}\innerexercise}
%  {\endinnerexercise}
%
%\newtheorem{innerthm}{Teorema}
%\newenvironment{teorema}[1]
%  {\renewcommand\theinnerthm{#1}\innerthm}
%  {\endinnerthm}
%
\newtheorem{innerlema}{Lema}
\newenvironment{lema}[1]
  {\renewcommand\theinnerlema{#1}\innerlema}
  {\endinnerlema}
%
%\theoremstyle{remark}
%\newtheorem*{hint}{Dica}
%\newtheorem*{notation}{Notação}
%\newtheorem*{obs}{Observação}


\title{\Huge{\textbf{Lista 3 - Matéria Condensada 2}}}
\author{Mateus Marques}

\begin{document}

\maketitle

\section{Simetrias proibidas e quase-cristais}

(b) Definimos
$$
\boxed{ S(k) = \sum_{n} e^{-i k x_n} } = \sum_{n} e^{-i k (n a)} e^{-i k F(na)}.
$$

Como $f_k(x) = e^{-i k F(x)}$ é uma função periódica de período $b$, podemos escrevê-la como uma série de Fourier, sendo $A_q(k)$ seus coeficientes:
$$
f_k(x) = e^{-i k F(x)} = \sum_{q} A_q(k) \, e^{i \frac{2\pi q x}{b}}, \quad
\boxed{ A_q(k) = \frac{1}{b} \int_0^b e^{-\frac{2\pi i q y}{b}} e^{-i k F(y)} \dd{y}. }
$$

Assim
$$
S(k) = \sum_{n} e^{-i k (n a)} e^{-i k F(na)} = \sum_{q} A_q(k) \sum_{n} e^{-i\qty(k - \frac{2\pi q}{b})(na)}.
$$

Mas lembre que, num Bravais Lattice de $N$ sítios e sendo $\k$ arbitrário, temos (Apêndice F do Ashcroft)
$$
\sum_{\vb{R}} e^{i \k \vdot \vb{R}} = N \sum_{\vb{G}} \delta_{\k,\vb{G}},
$$
onde a soma da esquerda $\sum_{\vb{R}}$ percorre o Bravais Lattice (de $N$ sítios) e a soma da direita $\sum_{\vb{G}}$ percorre o Reciprocal Lattice. Aplicando isso para o nosso caso, temos $\vb{R} = na \vu{x}$, $\vb{G} = \frac{2\pi p}{a} \vu{x}$, com $n, p$ inteiros, de maneira que
$$
\sum_{n} e^{-i \qty(k - \frac{2 \pi q}{b}) (na)} = N \sum_{p} \delta_{\qty(k - \frac{2 \pi q}{b}), \frac{2\pi p}{a}} =
N \sum_{p} \delta_{k, G_{pq}}, \quad G_{pq} = \frac{2\pi p}{a} + \frac{2\pi q}{b}.
$$

Portanto temos que
$$
\boxed{ S(k) = N \sum_{p, q} A_q(k) \, \delta_{k, G_{pq}}. }
$$

(c) Reescrevemos a sequência $x_n$ como
$$
x_n = n + \rho [n \s] = n + \rho ( n \s - \{n\s\}) = n (1 + \rho \s) - \rho \qty{\frac{n (1 + \rho \s)}{(1+\rho\s)/\s}},
$$
onde identificamos $a = (1 + \rho \s)$, $F(x) = -\rho \qty{\frac{x}{b}}$, com $b = \frac{1+\rho\s}{\s}$. Podemos calcular diretamente os coeficientes, fazendo a mudança de variável $z = \frac{y}{b}$:
$$
A_q(k) = \frac{1}{b} \int_0^b e^{-\frac{2\pi i q y}{b}} e^{-i k F(y)} \dd{y} =
\frac{1}{b} \int_0^b e^{-\frac{2\pi i q y}{b}} e^{i k \rho \qty{\frac{x}{b}}} \dd{y} =
\int_0^1 e^{-2\pi i q z} e^{i k \rho \qty{z}} \dd{z},
$$
mas $\qty{z} = z$ para $0 \leq z < 1$, então
$$
A_q(k) = \int_0^1 e^{-i(2\pi q - k \rho) z} \dd{z} =
\frac{1}{-i(2\pi q - k\rho)} \eval{\qty[e^{-i(2\pi q - k \rho) z}]}_{z=0}^{z=1}.
$$

Definindo então $X_{q}(k) = 2\pi q - k \rho$ (só depende de $q$ e $k$), obtemos
$$
A_q(k) = \frac{1}{-i X_q(k)} \qty(e^{-i X_q(k)} - 1) =
\frac{e^{-i X_q(k)/2}}{X_q(k)/2} \cdot \frac{\qty(e^{iX_q(k)/2} - e^{-iX_q(k)/2})}{2i} =
e^{-iX_q(k) / 2} \frac{\sin(X_q(k)/2)}{X_q(k)/2}.
$$

Se agora definirmos $X_{pq} = X_q(k = G_{pq})$, vemos facilmente que $\boxed{ X_{pq} = \frac{2\pi \rho}{(1+\rho\s)}\qty(\frac{q}{\rho} - p) }$ e obtemos
$$
\boxed{ S(k) = \sum_{n} e^{-ikx_n} = N \sum_{p,q} e^{-i X_{pq}/2} \, \frac{\sin(X_{pq}/2)}{X_{pq}/2} \delta_{k, G_{pq}}. }
$$



\pagebreak

\section{Densidade de estados}

\pagebreak

\section{Elétrons de Bloch}

\pagebreak

\section{Modelos tight-binding}

\pagebreak

\section{Instabilidade de Peierls e modelo de Su-Schrieffer-Heeger}

\end{document}
