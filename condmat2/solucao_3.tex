\documentclass[a4paper,10pt]{article}
%\usepackage{mathtools}
\usepackage{amsthm}     % for definitions and theorems
\usepackage[many]{tcolorbox}    % boxes around definitions and theorems
%\usepackage{amsmath}
%\usepackage{nccmath}
\usepackage{amssymb}    % \ltimes, semi-direct product
%\usepackage{etoolbox}   % for start of Chapter
%\usepackage{amsfonts}
\usepackage{physics}    % for all Physics related
\usepackage{dsfont}     % for the identity matrix symbol \1
%\usepackage{mathrsfs}
\usepackage[notextcomp]{stix}   % font package and some symbols like filled square
%\usepackage{MnSymbol}   % symbols font package

\usepackage{titling}
\usepackage{indentfirst}

\usepackage{bm}
\usepackage[dvipsnames]{xcolor}
\usepackage{cancel}
\usepackage{enumitem}

\usepackage{xurl}
%\usepackage[colorlinks=true]{hyperref} % links have colors
\usepackage{hyperref}  % no colors

\usepackage{float}
\usepackage{graphicx}
\usepackage{subcaption}
%\usepackage{tikz}

\usepackage{ctable}     % tabelas
\renewcommand{\P}{\phantom{+}}  % empty space to indent things
\usepackage{multirow}
\usepackage{tabulary}

%%%%%%%%%%%%%%%%%%%%%%%%%%%%%%%%%%%%%%%%%%%%%%%%%%%

\newcommand{\eps}{\epsilon}
\newcommand{\vphi}{\varphi}
\newcommand{\cte}{\text{cte}}

\newcommand{\N}{{\mathbb{N}}}
\newcommand{\Z}{{\mathbb{Z}}}
%\newcommand{\Q}{{\mathbb{Q}}}
\newcommand{\C}{{\mathbb{C}}}
\renewcommand{\S}{{\hat{S}}}
%\renewcommand{\H}{\s{H}}

\renewcommand{\a}{{\vb{a}}}
\renewcommand{\b}{{\vb{b}}}
\renewcommand{\d}{{\dagger}}
\newcommand{\up}{{\uparrow}}
\newcommand{\down}{{\downarrow}}
\newcommand{\hc}{{\text{h.c.}}}

\newcommand{\ihat}{\bm{\hat{\imath}}}
\newcommand{\jhat}{\bm{\hat{\jmath}}}
\newcommand{\khat}{\bm{\hat{k}}}

\newcommand{\0}{{\vb{0}}}
\newcommand{\1}{\mathds{1}}
\newcommand{\E}{{\vb{E}}}
\newcommand{\B}{{\vb{B}}}
\renewcommand{\u}{{\vb{u}}}
\renewcommand{\v}{{\vb{v}}}
\renewcommand{\r}{{\vb{r}}}
\newcommand{\R}{{\vb{R}}}
\newcommand{\Q}{{\vb{Q}}}
\newcommand{\G}{{\vb{G}}}
\newcommand{\g}{{\vb{g}}}
\renewcommand{\k}{{\vb{k}}}
\newcommand{\K}{{\vb{K}}}
\newcommand{\p}{{\vb{p}}}
\newcommand{\q}{{\vb{q}}}
\newcommand{\F}{{\vb{F}}}
\renewcommand{\t}{{\vb{t}}}
\newcommand{\vtau}{{\bm{\tau}}}
\newcommand{\vdelta}{{\bm{\delta}}}

% COLORED SYMMETRY ELEMENTS
\newcommand{\Ct}{{\textcolor{Cyan}{C_3}}}
\newcommand{\Ctn}[1]{{\textcolor{Cyan}{C_3^{\textcolor{black}{#1}}}}}
\newcommand{\Cs}{{\textcolor{ForestGreen}{C_6}}}
\newcommand{\Csn}[1]{{\textcolor{ForestGreen}{C_6^{\textcolor{black}{#1}}}}}
\newcommand{\sd}{{\textcolor{RoyalBlue}{\sigma_d}}}
\newcommand{\sdn}[1]{{\textcolor{RoyalBlue}{\sigma_d^{\textcolor{black}{#1}}}}}
\newcommand{\sdp}{{\textcolor{RoyalBlue}{\sigma_d'}}}
\newcommand{\sdpp}{{\textcolor{RoyalBlue}{\sigma_d''}}}
\newcommand{\sv}{{\textcolor{Orange}{\sigma_v}}}
\newcommand{\svn}[1]{{\textcolor{Orange}{\sigma_v^{\textcolor{black}{#1}}}}}
\newcommand{\svp}{{\textcolor{Orange}{\sigma_v'}}}
\newcommand{\svpp}{{\textcolor{Orange}{\sigma_v''}}}

\newcommand{\GL}{{\text{GL}}}
\newcommand{\U}{{\text{U}}}

\newcommand{\s}{\sigma}
%\newcommand{\prodint}[2]{\left\langle #1 , #2 \right\rangle}
\newcommand{\cc}[1]{\overline{#1}}
\newcommand{\Eval}[3]{\eval{\left( #1 \right)}_{#2}^{#3}}
\newcommand{\sg}[2]{\{ #1 \mid #2 \}}
\renewcommand{\AA}{{\mathring{\text{A}}}}
\newcommand{\I}{{\mathbb{I}}}
\newcommand{\bP}{{\mathbb{P}}}
\newcommand{\bQ}{{\mathbb{Q}}}

\newcommand{\unit}[1]{\; \mathrm{#1}}

\newcommand{\n}{\medskip}
\newcommand{\e}{\quad \mathrm{and} \quad}
\newcommand{\ou}{\quad \mathrm{or} \quad}
\newcommand{\virg}{\, , \;}
\newcommand{\ptodo}{\forall \,}
\renewcommand{\implies}{\; \Rightarrow \;}
%\newcommand{\eqname}[1]{\tag*{#1}} % Tag equation with name

%\setlength{\droptitle}{-7em}   % título um pouco mais em cima na página
%\makeatletter
%\patchcmd{\chapter}{\if@openright\cleardoublepage\else\clearpage\fi}{}{}{}  % start 'Chapter' at the same page. needs package etoolbox
%\makeatother

%% Theorems, definitions, proofs
\theoremstyle{definition}

%%% defining my own colors %%%
\definecolor{my-blue}{HTML}{f2f4ff}
\definecolor{my-green}{HTML}{f5fcf6}    % a little better: green!5!white
\definecolor{my-cyan}{HTML}{f2fffe}
\definecolor{my-yellow}{HTML}{fffbed}
\definecolor{my-green2}{HTML}{efffdb}

%%% alternative colors %%%
\definecolor{my-pink}{HTML}{fff2f7}
\definecolor{my-teal}{HTML}{ebfffc}

\newtheorem{definition}{Definition}[section]
\tcolorboxenvironment{definition}{
  colback=my-blue,
  %colback=blue!5!white,
  boxrule=0.1pt,
  boxsep=1pt,
  left=2pt,right=2pt,top=2pt,bottom=2pt,
  oversize=2pt,
  sharp corners,
  before skip=\topsep,
  after skip=\topsep,
}

\newtheorem{theorem}{Theorem}[section]
\tcolorboxenvironment{theorem}{
  colback=my-yellow,
  %colback=yellow!22!white!95!black,
  boxrule=0.1pt,
  boxsep=1pt,
  left=2pt,right=2pt,top=2pt,bottom=2pt,
  oversize=2pt,
  sharp corners,
  before skip=\topsep,
  after skip=\topsep,
}

\newtheorem{corollary}{Corollary}[section]
\tcolorboxenvironment{corollary}{
  colback=my-green2,
  boxrule=0.1pt,
  boxsep=1pt,
  left=2pt,right=2pt,top=2pt,bottom=2pt,
  oversize=2pt,
  sharp corners,
  before skip=\topsep,
  after skip=\topsep,
}

\newtheorem{lemma}{Lemma}[section]
\tcolorboxenvironment{lemma}{
  colback=my-cyan,
  boxrule=0.1pt,
  boxsep=1pt,
  left=2pt,right=2pt,top=2pt,bottom=2pt,
  oversize=2pt,
  sharp corners,
  before skip=\topsep,
  after skip=\topsep,
}

\newtheorem{example}{Example}[section]
\tcolorboxenvironment{example}{
  %colback=my-green,
  colback=green!5!white,
  boxrule=0.1pt,
  boxsep=1pt,
  left=2pt,right=2pt,top=2pt,bottom=2pt,
  oversize=2pt,
  sharp corners,
  before skip=\topsep,
  after skip=\topsep,
}


\title{\Huge{\textbf{Lista 3 - Matéria Condensada 2}}}
\author{Mateus Marques}

\begin{document}

\maketitle

\section{Simetrias proibidas e quase-cristais}

(b) Definimos
$$
\boxed{ S(k) = \sum_{n} e^{-i k x_n} } = \sum_{n} e^{-i k (n a)} e^{-i k F(na)}.
$$

Como $f_k(x) = e^{-i k F(x)}$ é uma função periódica de período $b$, podemos escrevê-la como uma série de Fourier, sendo $A_q(k)$ seus coeficientes:
$$
f_k(x) = e^{-i k F(x)} = \sum_{q} A_q(k) \, e^{i \frac{2\pi q x}{b}}, \quad
\boxed{ A_q(k) = \frac{1}{b} \int_0^b e^{-\frac{2\pi i q y}{b}} e^{-i k F(y)} \dd{y}. }
$$

Assim
$$
S(k) = \sum_{n} e^{-i k (n a)} e^{-i k F(na)} = \sum_{q} A_q(k) \sum_{n} e^{-i\qty(k - \frac{2\pi q}{b})(na)}.
$$

Mas lembre que, num Bravais Lattice de $N$ sítios e sendo $\k$ arbitrário, temos (Apêndice F do Ashcroft)
$$
\sum_{\vb{R}} e^{i \k \vdot \vb{R}} = N \sum_{\vb{G}} \delta_{\k,\vb{G}},
$$
onde a soma da esquerda $\sum_{\vb{R}}$ percorre o Bravais Lattice (de $N$ sítios) e a soma da direita $\sum_{\vb{G}}$ percorre o Reciprocal Lattice. Aplicando isso para o nosso caso, temos $\vb{R} = na \vu{x}$, $\vb{G} = \frac{2\pi p}{a} \vu{x}$, com $n, p$ inteiros, de maneira que
$$
\sum_{n} e^{-i \qty(k - \frac{2 \pi q}{b}) (na)} = N \sum_{p} \delta_{\qty(k - \frac{2 \pi q}{b}), \frac{2\pi p}{a}} =
N \sum_{p} \delta_{k, G_{pq}}, \quad G_{pq} = \frac{2\pi p}{a} + \frac{2\pi q}{b}.
$$

Portanto temos que
$$
\boxed{ S(k) = N \sum_{p, q} A_q(k) \, \delta_{k, G_{pq}}. }
$$

(c) Reescrevemos a sequência $x_n$ como
$$
x_n = n + \rho [n \s] = n + \rho ( n \s - \{n\s\}) = n (1 + \rho \s) - \rho \qty{\frac{n (1 + \rho \s)}{(1+\rho\s)/\s}},
$$
onde identificamos $a = (1 + \rho \s)$, $F(x) = -\rho \qty{\frac{x}{b}}$, com $b = \frac{1+\rho\s}{\s}$. Podemos calcular diretamente os coeficientes, fazendo a mudança de variável $z = \frac{y}{b}$:
$$
A_q(k) = \frac{1}{b} \int_0^b e^{-\frac{2\pi i q y}{b}} e^{-i k F(y)} \dd{y} =
\frac{1}{b} \int_0^b e^{-\frac{2\pi i q y}{b}} e^{i k \rho \qty{\frac{x}{b}}} \dd{y} =
\int_0^1 e^{-2\pi i q z} e^{i k \rho \qty{z}} \dd{z},
$$
mas $\qty{z} = z$ para $0 \leq z < 1$, então
$$
A_q(k) = \int_0^1 e^{-i(2\pi q - k \rho) z} \dd{z} =
\frac{1}{-i(2\pi q - k\rho)} \eval{\qty[e^{-i(2\pi q - k \rho) z}]}_{z=0}^{z=1}.
$$

Definindo então $X_{q}(k) = 2\pi q - k \rho$ (só depende de $q$ e $k$), obtemos
$$
A_q(k) = \frac{1}{-i X_q(k)} \qty(e^{-i X_q(k)} - 1) =
\frac{e^{-i X_q(k)/2}}{X_q(k)/2} \cdot \frac{\qty(e^{iX_q(k)/2} - e^{-iX_q(k)/2})}{2i} =
e^{-iX_q(k) / 2} \frac{\sin(X_q(k)/2)}{X_q(k)/2}.
$$

Se agora definirmos $X_{pq} = X_q(k = G_{pq})$, vemos facilmente que $\boxed{ X_{pq} = \frac{2\pi \rho}{(1+\rho\s)}\qty(\frac{q}{\rho} - p) }$ e obtemos
$$
\boxed{ S(k) = \sum_{n} e^{-ikx_n} = N \sum_{p,q} e^{-i X_{pq}/2} \, \frac{\sin(X_{pq}/2)}{X_{pq}/2} \delta_{k, G_{pq}}. }
$$



\pagebreak

\section{Densidade de estados}

\pagebreak

\section{Elétrons de Bloch}

\pagebreak

\section{Modelos tight-binding}

\pagebreak

\section{Instabilidade de Peierls e modelo de Su-Schrieffer-Heeger}

\end{document}
