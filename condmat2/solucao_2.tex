\documentclass[a4paper,10pt]{article}
\usepackage[brazilian]{babel}
\usepackage[left=2.5cm,right=2.5cm,top=3cm,bottom=2.5cm]{geometry}
\usepackage{mathtools}
\usepackage{amsthm}
\usepackage{amsmath}
%\usepackage{nccmath}
\usepackage{amssymb}
\usepackage{amsfonts}
\usepackage{physics}
%\usepackage{dsfont}
%\usepackage{mathrsfs}

\usepackage{titling}
\usepackage{indentfirst}

\usepackage{bm}
\usepackage[dvipsnames]{xcolor}
\usepackage{cancel}

\usepackage{xurl}
\usepackage[colorlinks=true]{hyperref}

\usepackage{float}
\usepackage{graphicx}
%\usepackage{tikz}
\usepackage{caption}
\usepackage{subcaption}

%%%%%%%%%%%%%%%%%%%%%%%%%%%%%%%%%%%%%%%%%%%%%%%%%%%

\newcommand{\eps}{\epsilon}
\newcommand{\vphi}{\varphi}
\newcommand{\cte}{\text{cte}}

\newcommand{\N}{\mathbb{N}}
\newcommand{\Z}{\mathbb{Z}}
\newcommand{\Q}{\mathbb{Q}}
\newcommand{\R}{\vb{R}}
\newcommand{\C}{\mathbb{C}}
\renewcommand{\S}{\hat{S}}
%\renewcommand{\H}{\s{H}}

\renewcommand{\a}{\vb{a}}
\newcommand{\nn}{\hat{n}}
\renewcommand{\d}{\dagger}
\newcommand{\up}{\uparrow}
\newcommand{\down}{\downarrow}

\newcommand{\0}{\vb{0}}
%\newcommand{\1}{\mathds{1}}
\newcommand{\E}{\vb{E}}
\newcommand{\B}{\vb{B}}
\renewcommand{\v}{\vb{v}}
\renewcommand{\r}{\vb{r}}
\renewcommand{\k}{\vb{k}}
\newcommand{\p}{\vb{p}}
\newcommand{\q}{\vb{q}}
\newcommand{\F}{\vb{F}}

\newcommand{\s}{\sigma}
%\newcommand{\prodint}[2]{\left\langle #1 , #2 \right\rangle}
\newcommand{\cc}[1]{\overline{#1}}
\newcommand{\Eval}[3]{\eval{\left( #1 \right)}_{#2}^{#3}}

\newcommand{\unit}[1]{\; \mathrm{#1}}

\newcommand{\n}{\medskip}
\newcommand{\e}{\quad \mathrm{e} \quad}
\newcommand{\ou}{\quad \mathrm{ou} \quad}
\newcommand{\virg}{\, , \;}
\newcommand{\ptodo}{\forall \,}
\renewcommand{\implies}{\; \Rightarrow \;}
%\newcommand{\eqname}[1]{\tag*{#1}} % Tag equation with name

\setlength{\droptitle}{-7em}

\theoremstyle{plain}
\newtheorem{theorem}{Teorema}[section]
%\newtheorem{defi}[theorem]{Definição}
\newtheorem{lemma}[theorem]{Lema}
%\newtheorem{corol}[theorem]{Corolário}
%\newtheorem{prop}[theorem]{Proposição}
%\newtheorem{example}{Exemplo}
%
%\newtheorem{inneraxiom}{Axioma}
%\newenvironment{axioma}[1]
%  {\renewcommand\theinneraxiom{#1}\inneraxiom}
%  {\endinneraxiom}
%
%\newtheorem{innerpostulado}{Postulado}
%\newenvironment{postulado}[1]
%  {\renewcommand\theinnerpostulado{#1}\innerpostulado}
%  {\endinnerpostulado}
%
%\newtheorem{innerexercise}{Exercício}
%\newenvironment{exercise}[1]
%  {\renewcommand\theinnerexercise{#1}\innerexercise}
%  {\endinnerexercise}
%
%\newtheorem{innerthm}{Teorema}
%\newenvironment{teorema}[1]
%  {\renewcommand\theinnerthm{#1}\innerthm}
%  {\endinnerthm}
%
\newtheorem{innerlema}{Lema}
\newenvironment{lema}[1]
  {\renewcommand\theinnerlema{#1}\innerlema}
  {\endinnerlema}
%
%\theoremstyle{remark}
%\newtheorem*{hint}{Dica}
%\newtheorem*{notation}{Notação}
%\newtheorem*{obs}{Observação}


\title{\Huge{\textbf{Lista 2 - Matéria Condensada 2}}}
\author{Mateus Marques}

\begin{document}

\maketitle


\section{Gás de elétrons livres}

\url{https://en.universaldenker.org/lessons/262}

(a) Com a substituição $\sum_{\k} \to \frac{V_d}{(2\pi)^d} \int \dd[d]{k}$

Vale a seguinte propriedade da função $\delta$ de Dirac (era um dos exercícios do curso de QFT):
$$
\int_{-\infty}^{\infty} f(x) \delta(g(x)) \dd{x} =
\sum_{g(a)=0} \frac{1}{\abs{g'(a)}} \int_{-\infty}^{\infty} f(x) \delta(x-a) \dd{x}=
\sum_{g(a)=0} \frac{1}{\abs{g'(a)}} \, f(a).
$$
onde a soma é sobre todos os zeros $a$ da função $g(x)$.
$$
\rho_d(\eps) = \sum_{\k, \s} \delta(\eps - E(\k)) =
\frac{2V_d}{(2\pi)^d} \int \delta\qty(\eps - \frac{k^2}{2m}) \dd[d]{k} =
\frac{2V_d}{(2\pi)^d} \int \dd{\Omega_d} \int_0^\infty \delta\qty(\eps - \frac{k^2}{2m}) k^{d-1} \dd{k},
$$
onde $\Theta_d = \int \dd{\Omega_d}$ o ângulo sólido de acordo com a dimensão $d$, sendo $\Theta_1 = 2, \Theta_2 = 2\pi$ e $\Theta_3 = 4\pi$.

Definindo $g(k) = \eps - \frac{k^2}{2m}$, temos $\abs{g'(k)} = \abs{k} / m$ e suas raízes são $k = \pm \sqrt{2m \eps}$. Portanto
$$
\rho_d(\eps) = \frac{2V_d}{(2\pi)^d} \Theta_d \, \frac{m}{\sqrt{2m\eps}}
\qty(\sqrt{2m \eps})^{d-1} = \frac{2 m V_d \Theta_d}{(2\pi)^d} \, (2m\eps)^{\frac{d-2}{2}}.
$$

Substituindo $d = 1, 2$ e $3$:
\begin{itemize}
\item $d = 1$:
$$
\rho_1(\eps) = \frac{2 m L}{\pi} \, \frac{1}{\sqrt{2m\eps}}.
$$
\item $d = 2$:
$$
\rho_2(\eps) = \frac{m A}{\pi}.
$$
\item $d = 3$:
$$
\rho_3(\eps) = \frac{m V}{\pi^2} \sqrt{2m\eps}.
$$
\end{itemize}

(b) \url{https://eng.libretexts.org/Bookshelves/Materials_Science/Supplemental_Modules_(Materials_Science)/Electronic_Properties/Density_of_States}


$$
\rho_d(\eps) = \sum_{\k, \s} \delta(\eps - E(\k)) =
\frac{2V_d}{(2\pi)^d} \int \delta\qty(\eps - vk) \dd[d]{k} =
\frac{2V_d}{(2\pi)^d} \int \dd{\Omega_d} \int_0^\infty \delta\qty(\eps - vk) k^{d-1} \dd{k},
$$

Definindo dessa vez $g(k) = \eps - vk$, temos $\abs{g'(k)} = v$ e suas raízes são $k = \eps/v$. Portanto
$$
\rho_d(\eps) = \frac{2V_d}{(2\pi)^d} \Theta_d \, \frac{1}{v}
\qty(\frac{\eps}{v})^{d-1} = \frac{2 V_d \Theta_d}{(2\pi)^d v} \, \qty(\frac{\eps}{v})^{d-1}.
$$

Substituindo $d = 1, 2$ e $3$:
\begin{itemize}
\item $d = 1$:
$$
\rho_1(\eps) = \frac{2 L}{\pi v}.
$$
\item $d = 2$:
$$
\rho_2(\eps) = \frac{A}{\pi} \, \frac{\eps}{v^2}.
$$
\item $d = 3$:
$$
\rho_3(\eps) = \frac{V}{\pi^2 v} \qty(\frac{\eps}{v})^2.
$$
\end{itemize}

\pagebreak

\section{Expansão de Sommerfeld}

(a) Façamos a substituição de variáveis $x = \beta(\eps - \mu)$, temos
$$
I = \frac{1}{\beta} \int_{-\infty}^{\infty} \frac{G(\mu + \frac{x}{\beta})}{e^{x} + 1} \dd{x}
$$
Já que $\displaystyle{\frac{1}{e^{-x} + 1} = \frac{e^x}{e^x + 1} = 1 - \frac{1}{e^x + 1}}$, a integral de $-\infty$ até $0$ se escreve como
$$
\int_{-\infty}^0 \frac{G(\mu + \frac{x}{\beta})}{e^{x} + 1} \dd{x} =
\int_0^{\infty} \frac{G(\mu - \frac{x}{\beta})}{e^{-x} + 1} \dd{x} =
\int_0^{\infty} G\qty(\mu - \frac{x}{\beta}) \dd{x} -
\int_0^{\infty} \frac{G(\mu - \frac{x}{\beta})}{e^{x} + 1} \dd{x}.
$$
Voltando à integral $I$, temos
$$
I =
\int_0^{\infty} G\qty(\mu - \frac{x}{\beta}) \frac{\dd{x}}{\beta} +
\frac{1}{\beta}
\int_0^{\infty}\frac{G(\mu + \frac{x}{\beta}) - G(\mu - \frac{x}{\beta})}{e^x + 1}\dd{x}
$$
A primeira integral acima se reescreve facilmente como
$$
\int_0^\infty G\qty(\mu - \frac{x}{\beta}) \frac{\dd{x}}{\beta} =
\int_{-\infty}^\mu G(\eps) \dd{\eps},
$$
enquanto que para $\beta \to \infty$
$$
G\qty(\mu + \frac{x}{\beta}) - G\qty(\mu - \frac{x}{\beta}) \approx \frac{2x}{\beta} G'(\mu),
$$
de maneira que
$$
\frac{1}{\beta}
\int_0^{\infty}\frac{G(\mu + \frac{x}{\beta}) - G(\mu - \frac{x}{\beta})}{e^x + 1}\dd{x}=
\frac{2 G'(\mu)}{\beta^2}
\int_0^{\infty}\frac{x}{e^x + 1}\dd{x} =
\frac{\pi^2}{6 \beta^2} \, G'(\mu).
$$

Portanto
$$
I = \int_{-\infty}^{\mu} G(\eps) \dd{\eps} + \frac{\pi^2}{6} T^2 \eval{\dv{G}{\eps}}_{\eps = \mu}.
$$

(b) Temos
$$
n = \int_{-\infty}^{\infty} f(\eps) \rho(\eps)
$$
<++>

\pagebreak

\begin{equation} \label{eq:pertub}
(E - H_0) \ket{\psi} = \lambda V \ket{\psi}.
\end{equation}


\end{document}
