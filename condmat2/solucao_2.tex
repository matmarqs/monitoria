\documentclass[a4paper,10pt]{article}
%\usepackage{mathtools}
\usepackage{amsthm}     % for definitions and theorems
\usepackage[many]{tcolorbox}    % boxes around definitions and theorems
%\usepackage{amsmath}
%\usepackage{nccmath}
\usepackage{amssymb}    % \ltimes, semi-direct product
%\usepackage{etoolbox}   % for start of Chapter
%\usepackage{amsfonts}
\usepackage{physics}    % for all Physics related
\usepackage{dsfont}     % for the identity matrix symbol \1
%\usepackage{mathrsfs}
\usepackage[notextcomp]{stix}   % font package and some symbols like filled square
%\usepackage{MnSymbol}   % symbols font package

\usepackage{titling}
\usepackage{indentfirst}

\usepackage{bm}
\usepackage[dvipsnames]{xcolor}
\usepackage{cancel}
\usepackage{enumitem}

\usepackage{xurl}
%\usepackage[colorlinks=true]{hyperref} % links have colors
\usepackage{hyperref}  % no colors

\usepackage{float}
\usepackage{graphicx}
\usepackage{subcaption}
%\usepackage{tikz}

\usepackage{ctable}     % tabelas
\renewcommand{\P}{\phantom{+}}  % empty space to indent things
\usepackage{multirow}
\usepackage{tabulary}

%%%%%%%%%%%%%%%%%%%%%%%%%%%%%%%%%%%%%%%%%%%%%%%%%%%

\newcommand{\eps}{\epsilon}
\newcommand{\vphi}{\varphi}
\newcommand{\cte}{\text{cte}}

\newcommand{\N}{{\mathbb{N}}}
\newcommand{\Z}{{\mathbb{Z}}}
%\newcommand{\Q}{{\mathbb{Q}}}
\newcommand{\C}{{\mathbb{C}}}
\renewcommand{\S}{{\hat{S}}}
%\renewcommand{\H}{\s{H}}

\renewcommand{\a}{{\vb{a}}}
\renewcommand{\b}{{\vb{b}}}
\renewcommand{\d}{{\dagger}}
\newcommand{\up}{{\uparrow}}
\newcommand{\down}{{\downarrow}}
\newcommand{\hc}{{\text{h.c.}}}

\newcommand{\ihat}{\bm{\hat{\imath}}}
\newcommand{\jhat}{\bm{\hat{\jmath}}}
\newcommand{\khat}{\bm{\hat{k}}}

\newcommand{\0}{{\vb{0}}}
\newcommand{\1}{\mathds{1}}
\newcommand{\E}{{\vb{E}}}
\newcommand{\B}{{\vb{B}}}
\renewcommand{\u}{{\vb{u}}}
\renewcommand{\v}{{\vb{v}}}
\renewcommand{\r}{{\vb{r}}}
\newcommand{\R}{{\vb{R}}}
\newcommand{\Q}{{\vb{Q}}}
\newcommand{\G}{{\vb{G}}}
\newcommand{\g}{{\vb{g}}}
\renewcommand{\k}{{\vb{k}}}
\newcommand{\K}{{\vb{K}}}
\newcommand{\p}{{\vb{p}}}
\newcommand{\q}{{\vb{q}}}
\newcommand{\F}{{\vb{F}}}
\renewcommand{\t}{{\vb{t}}}
\newcommand{\vtau}{{\bm{\tau}}}
\newcommand{\vdelta}{{\bm{\delta}}}

% COLORED SYMMETRY ELEMENTS
\newcommand{\Ct}{{\textcolor{Cyan}{C_3}}}
\newcommand{\Ctn}[1]{{\textcolor{Cyan}{C_3^{\textcolor{black}{#1}}}}}
\newcommand{\Cs}{{\textcolor{ForestGreen}{C_6}}}
\newcommand{\Csn}[1]{{\textcolor{ForestGreen}{C_6^{\textcolor{black}{#1}}}}}
\newcommand{\sd}{{\textcolor{RoyalBlue}{\sigma_d}}}
\newcommand{\sdn}[1]{{\textcolor{RoyalBlue}{\sigma_d^{\textcolor{black}{#1}}}}}
\newcommand{\sdp}{{\textcolor{RoyalBlue}{\sigma_d'}}}
\newcommand{\sdpp}{{\textcolor{RoyalBlue}{\sigma_d''}}}
\newcommand{\sv}{{\textcolor{Orange}{\sigma_v}}}
\newcommand{\svn}[1]{{\textcolor{Orange}{\sigma_v^{\textcolor{black}{#1}}}}}
\newcommand{\svp}{{\textcolor{Orange}{\sigma_v'}}}
\newcommand{\svpp}{{\textcolor{Orange}{\sigma_v''}}}

\newcommand{\GL}{{\text{GL}}}
\newcommand{\U}{{\text{U}}}

\newcommand{\s}{\sigma}
%\newcommand{\prodint}[2]{\left\langle #1 , #2 \right\rangle}
\newcommand{\cc}[1]{\overline{#1}}
\newcommand{\Eval}[3]{\eval{\left( #1 \right)}_{#2}^{#3}}
\newcommand{\sg}[2]{\{ #1 \mid #2 \}}
\renewcommand{\AA}{{\mathring{\text{A}}}}
\newcommand{\I}{{\mathbb{I}}}
\newcommand{\bP}{{\mathbb{P}}}
\newcommand{\bQ}{{\mathbb{Q}}}

\newcommand{\unit}[1]{\; \mathrm{#1}}

\newcommand{\n}{\medskip}
\newcommand{\e}{\quad \mathrm{and} \quad}
\newcommand{\ou}{\quad \mathrm{or} \quad}
\newcommand{\virg}{\, , \;}
\newcommand{\ptodo}{\forall \,}
\renewcommand{\implies}{\; \Rightarrow \;}
%\newcommand{\eqname}[1]{\tag*{#1}} % Tag equation with name

%\setlength{\droptitle}{-7em}   % título um pouco mais em cima na página
%\makeatletter
%\patchcmd{\chapter}{\if@openright\cleardoublepage\else\clearpage\fi}{}{}{}  % start 'Chapter' at the same page. needs package etoolbox
%\makeatother

%% Theorems, definitions, proofs
\theoremstyle{definition}

%%% defining my own colors %%%
\definecolor{my-blue}{HTML}{f2f4ff}
\definecolor{my-green}{HTML}{f5fcf6}    % a little better: green!5!white
\definecolor{my-cyan}{HTML}{f2fffe}
\definecolor{my-yellow}{HTML}{fffbed}
\definecolor{my-green2}{HTML}{efffdb}

%%% alternative colors %%%
\definecolor{my-pink}{HTML}{fff2f7}
\definecolor{my-teal}{HTML}{ebfffc}

\newtheorem{definition}{Definition}[section]
\tcolorboxenvironment{definition}{
  colback=my-blue,
  %colback=blue!5!white,
  boxrule=0.1pt,
  boxsep=1pt,
  left=2pt,right=2pt,top=2pt,bottom=2pt,
  oversize=2pt,
  sharp corners,
  before skip=\topsep,
  after skip=\topsep,
}

\newtheorem{theorem}{Theorem}[section]
\tcolorboxenvironment{theorem}{
  colback=my-yellow,
  %colback=yellow!22!white!95!black,
  boxrule=0.1pt,
  boxsep=1pt,
  left=2pt,right=2pt,top=2pt,bottom=2pt,
  oversize=2pt,
  sharp corners,
  before skip=\topsep,
  after skip=\topsep,
}

\newtheorem{corollary}{Corollary}[section]
\tcolorboxenvironment{corollary}{
  colback=my-green2,
  boxrule=0.1pt,
  boxsep=1pt,
  left=2pt,right=2pt,top=2pt,bottom=2pt,
  oversize=2pt,
  sharp corners,
  before skip=\topsep,
  after skip=\topsep,
}

\newtheorem{lemma}{Lemma}[section]
\tcolorboxenvironment{lemma}{
  colback=my-cyan,
  boxrule=0.1pt,
  boxsep=1pt,
  left=2pt,right=2pt,top=2pt,bottom=2pt,
  oversize=2pt,
  sharp corners,
  before skip=\topsep,
  after skip=\topsep,
}

\newtheorem{example}{Example}[section]
\tcolorboxenvironment{example}{
  %colback=my-green,
  colback=green!5!white,
  boxrule=0.1pt,
  boxsep=1pt,
  left=2pt,right=2pt,top=2pt,bottom=2pt,
  oversize=2pt,
  sharp corners,
  before skip=\topsep,
  after skip=\topsep,
}


\title{\Huge{\textbf{Lista 2 - Matéria Condensada 2}}}
\author{Mateus Marques}

\begin{document}

\maketitle


\section{Gás de elétrons livres}

\url{https://en.universaldenker.org/lessons/262}

(a) Com a substituição $\sum_{\k} \to \frac{V_d}{(2\pi)^d} \int \dd[d]{k}$

Vale a seguinte propriedade da função $\delta$ de Dirac (era um dos exercícios do curso de QFT):
$$
\int_{-\infty}^{\infty} f(x) \delta(g(x)) \dd{x} =
\sum_{g(a)=0} \frac{1}{\abs{g'(a)}} \int_{-\infty}^{\infty} f(x) \delta(x-a) \dd{x}=
\sum_{g(a)=0} \frac{1}{\abs{g'(a)}} \, f(a).
$$
onde a soma é sobre todos os zeros $a$ da função $g(x)$.
$$
\rho_d(\eps) = \sum_{\k, \s} \delta(\eps - E(\k)) =
\frac{2V_d}{(2\pi)^d} \int \delta\qty(\eps - \frac{k^2}{2m}) \dd[d]{k} =
\frac{2V_d}{(2\pi)^d} \int \dd{\Omega_d} \int_0^\infty \delta\qty(\eps - \frac{k^2}{2m}) k^{d-1} \dd{k},
$$
onde $\Theta_d = \int \dd{\Omega_d}$ o ângulo sólido de acordo com a dimensão $d$, sendo $\Theta_1 = 2, \Theta_2 = 2\pi$ e $\Theta_3 = 4\pi$.

Definindo $g(k) = \eps - \frac{k^2}{2m}$, temos $\abs{g'(k)} = \abs{k} / m$ e suas raízes são $k = \pm \sqrt{2m \eps}$. Portanto
$$
\rho_d(\eps) = \frac{2V_d}{(2\pi)^d} \Theta_d \, \frac{m}{\sqrt{2m\eps}}
\qty(\sqrt{2m \eps})^{d-1} = \frac{2 m V_d \Theta_d}{(2\pi)^d} \, (2m\eps)^{\frac{d-2}{2}}.
$$

Substituindo $d = 1, 2$ e $3$:
\begin{itemize}
\item $d = 1$:
$$
\rho_1(\eps) = \frac{2 m L}{\pi} \, \frac{1}{\sqrt{2m\eps}}.
$$
\item $d = 2$:
$$
\rho_2(\eps) = \frac{m A}{\pi}.
$$
\item $d = 3$:
$$
\rho_3(\eps) = \frac{m V}{\pi^2} \sqrt{2m\eps}.
$$
\end{itemize}

(b) \url{https://eng.libretexts.org/Bookshelves/Materials_Science/Supplemental_Modules_(Materials_Science)/Electronic_Properties/Density_of_States}


$$
\rho_d(\eps) = \sum_{\k, \s} \delta(\eps - E(\k)) =
\frac{2V_d}{(2\pi)^d} \int \delta\qty(\eps - vk) \dd[d]{k} =
\frac{2V_d}{(2\pi)^d} \int \dd{\Omega_d} \int_0^\infty \delta\qty(\eps - vk) k^{d-1} \dd{k},
$$

Definindo dessa vez $g(k) = \eps - vk$, temos $\abs{g'(k)} = v$ e suas raízes são $k = \eps/v$. Portanto
$$
\rho_d(\eps) = \frac{2V_d}{(2\pi)^d} \Theta_d \, \frac{1}{v}
\qty(\frac{\eps}{v})^{d-1} = \frac{2 V_d \Theta_d}{(2\pi)^d v} \, \qty(\frac{\eps}{v})^{d-1}.
$$

Substituindo $d = 1, 2$ e $3$:
\begin{itemize}
\item $d = 1$:
$$
\rho_1(\eps) = \frac{2 L}{\pi v}.
$$
\item $d = 2$:
$$
\rho_2(\eps) = \frac{A}{\pi} \, \frac{\eps}{v^2}.
$$
\item $d = 3$:
$$
\rho_3(\eps) = \frac{V}{\pi^2 v} \qty(\frac{\eps}{v})^2.
$$
\end{itemize}

\pagebreak

\section{Expansão de Sommerfeld}

(a) Façamos a substituição de variáveis $x = \beta(\eps - \mu)$, temos
$$
I = \frac{1}{\beta} \int_{-\infty}^{\infty} \frac{G(\mu + \frac{x}{\beta})}{e^{x} + 1} \dd{x}
$$
Já que $\displaystyle{\frac{1}{e^{-x} + 1} = \frac{e^x}{e^x + 1} = 1 - \frac{1}{e^x + 1}}$, a integral de $-\infty$ até $0$ se escreve como
$$
\int_{-\infty}^0 \frac{G(\mu + \frac{x}{\beta})}{e^{x} + 1} \dd{x} =
\int_0^{\infty} \frac{G(\mu - \frac{x}{\beta})}{e^{-x} + 1} \dd{x} =
\int_0^{\infty} G\qty(\mu - \frac{x}{\beta}) \dd{x} -
\int_0^{\infty} \frac{G(\mu - \frac{x}{\beta})}{e^{x} + 1} \dd{x}.
$$
Voltando à integral $I$, temos
$$
I =
\int_0^{\infty} G\qty(\mu - \frac{x}{\beta}) \frac{\dd{x}}{\beta} +
\frac{1}{\beta}
\int_0^{\infty}\frac{G(\mu + \frac{x}{\beta}) - G(\mu - \frac{x}{\beta})}{e^x + 1}\dd{x}
$$
A primeira integral acima se reescreve facilmente como
$$
\int_0^\infty G\qty(\mu - \frac{x}{\beta}) \frac{\dd{x}}{\beta} =
\int_{-\infty}^\mu G(\eps) \dd{\eps},
$$
enquanto que para $\beta \to \infty$
$$
G\qty(\mu + \frac{x}{\beta}) - G\qty(\mu - \frac{x}{\beta}) \approx \frac{2x}{\beta} G'(\mu),
$$
de maneira que
$$
\frac{1}{\beta}
\int_0^{\infty}\frac{G(\mu + \frac{x}{\beta}) - G(\mu - \frac{x}{\beta})}{e^x + 1}\dd{x}=
\frac{2 G'(\mu)}{\beta^2}
\int_0^{\infty}\frac{x}{e^x + 1}\dd{x} =
\frac{\pi^2}{6 \beta^2} \, G'(\mu).
$$

Portanto
$$
I = \int_{-\infty}^{\mu} G(\eps) \dd{\eps} + \frac{\pi^2}{6} T^2 \eval{\dv{G}{\eps}}_{\eps = \mu}.
$$

(b) Para $T \to 0$, temos que $\mu \approx \eps_F$ de maneira que
$$
n = \int_{-\infty}^{\infty} f(\eps) \rho(\eps) =
\int_{-\infty}^{\eps_F} \rho(\eps) \dd{\eps} +
\int_{\eps_F}^{\mu} \rho(\eps) \dd{\eps} +
\frac{\pi^2}{6} T^2 \rho'(\eps_F),
$$
onde aproximamos $\rho'(\mu) \approx \rho'(\eps_F)$. Temos ainda que
$$
\int_{-\infty}^{\eps_F} \rho(\eps) \dd{\eps} \approx n,
$$
$$
\int_{\eps_F}^{\mu} \rho(\eps) \dd{\eps} \approx \rho(\eps_F) \, (\mu - \eps_F).
$$

Portanto concluimos que
$$
\mu \approx \eps_F - \frac{\pi^2}{6} T^2 \frac{\rho'(\eps_F)}{\rho(\eps_F)}.
$$

(c) Aproximando $\mu \approx \eps_F$ no limite $T \to 0$, temos
$$
E = \int_{-\infty}^{\infty} f(\eps) \eps \rho(\eps) \dd{\eps} \approx
\int_{-\infty}^{\eps_F} \eps \rho(\eps) \dd{\eps} +
(\mu - \eps_F) \eps_F \rho(\eps_F) +
\frac{\pi^2}{6} T^2 \eval{\dv{(\eps \rho(\eps))}{\eps}}_{\eps = \eps_F},
$$
onde aproximamos $\displaystyle{\int_{\eps_F}^{\mu} \eps \rho(\eps_F) \dd{\eps} \approx (\mu - \eps_F) \eps_F \rho(\eps_F)}$ e $\displaystyle{\eval{\dv{(\eps \rho(\eps))}{\eps}}_{\eps = \mu} \approx \eval{\dv{(\eps \rho(\eps))}{\eps}}_{\eps = \eps_F}}$.

\n

Substituindo então a aproximação que obtivemos no item (b), obtemos
$$
E = \int_{-\infty}^{\eps_F} \eps \rho(\eps) \dd{\eps}
- \cancel{ \frac{\pi^2}{6} T^2 \eps_F \rho'(\eps_F) } +
\frac{\pi^2}{6} T^2 \Big[\rho(\eps_F) + \cancel{ \eps_F \rho'(\eps_F) } \Big],
$$
de maneira que
$$
c = \pdv{E}{T} \approx \frac{\pi^2}{3} T \rho(\eps_F).
$$


\pagebreak

\section{Susceptibilidade paramagnética do gás de elétrons}

(a) No limite $\mu_B h \to 0$, temos que $\mu$ não se altera significativamente, $\mu(h) \approx \mu(h = 0) = \mu$.

Metade dos elétrons tem spin up $+1/2$ e a outra metade tem spin down $-1/2$. Devido ao campo magnético, ocorre um splitting nas energias. Os elétrons de spin up ficam com uma energia $\eps \to \eps - \mu_B h$ e os de spin down com $\eps \to \eps + \mu_B h$. Com essas considerações, temos
$$
n_+ = \frac{1}{2} \int f(\eps) \rho(\eps - \mu_B h) \dd{\eps} \e
n_- = \frac{1}{2} \int f(\eps) \rho(\eps + \mu_B h) \dd{\eps},
$$
de maneira que $h = 0 \implies n_+ + n_- = \int f(\eps) \rho(\eps) \dd{\eps} = n$ (recuperamos o caso particular). A magnetização $m$ é dada por
$$
m = - \mu_B (n_+ - n_-) =
\frac{\mu_B}{2} \int f(\eps) \Big[ \rho(\eps + \mu_B h) - \rho(\eps - \mu_B h) \Big] \dd{\eps},
$$
mas como $\mu_B h \to 0$, temos $\rho(\eps + \mu_B h) - \rho(\eps - \mu_B h) \approx 2 \mu_B h \rho'(\eps)$. Integrando por partes:
$$
m = \mu_B^2 h \int_{-\infty}^{\infty} f(\eps) \rho'(\eps) \dd{\eps} =
\mu_B^2 h \qty(\eval{f(\eps) \rho(\eps)}_{-\infty}^{\infty} - \int \rho(\eps) f'(\eps) \dd{\eps} ).
$$
Mas $\rho(\pm\infty) = 0$, portanto
$$
m = \mu_B^2 h \int \rho(\eps) \qty(- \pdv{f}{\eps}) \dd{\eps} \implies
\chi = \pdv{m}{h} = \mu_B^2 \int \rho(\eps) \qty(- \pdv{f}{\eps}) \dd{\eps}.
$$

(b) Para $T = 0$, temos
$$
m = \mu_B^2 h \int_{-\infty}^{\infty} f(\eps) \rho'(\eps) \dd{\eps} =
\mu_B^2 h \int_{-\infty}^{\eps_F} \rho'(\eps) \dd{\eps} =
\mu_B^2 h \qty[ \rho(\eps_F) - \cancelto{0}{\rho(-\infty)} ]
$$
$$
\implies \chi = \mu_B^2 \rho(\eps_F).
$$

(c) Para $T \to \infty$ ($\beta \to 0$), aproximamos $f(\eps) \approx e^{-\beta(\eps - \mu)}$ de maneira que $\pdv{f}{\eps} = -\beta f(\eps)$. Portanto
$$
\chi(T \to \infty) = \mu_B^2 \beta \int f(\eps) \rho(\eps) \dd{\eps} = n \, \frac{\mu_B^2}{k_B T}.
$$


\pagebreak


\section{Teoria de perturbação para o gás de elétrons}

(a) Em sala de aula vimos que a parte de Coulomb da hamiltoniana do Jellium se escreve como
\begin{equation} \label{eq:jellium}
\hat{V} = \frac{3 e^2}{2 N a_0 r_s}
\sum_{\substack{\k_1, \k_2, \q \neq \0 \\ \s_1 \s_2}} \frac{1}{\q^2} \,
c_{\k_1 - \q, \s_1}^\d c_{\k_2 + \q, \s_2}^\d c_{\k_2 \s_2} c_{\k_1 \s_1}
\end{equation}

A correção de primeira ordem para a energia do ground-state é dada por
$$
E_0^{(1)} = \ev{\hat{V}}{\text{GS}},
$$
onde o ground-state do sistema é
$$
\ket{\text{GS}} = \prod_{\abs{\k} \leq k_{F}, \s} c_{\k \s}^{\d} \ket{0}.
$$

Assim, teremos que calcular o elemento de matriz
$$
M_{\k_1 \k_2 \q, \s_1 \s_2} = \ev{c_{\k_1 - \q, \s_1}^\d c_{\k_2 + \q, \s_2}^\d c_{\k_2 \s_2} c_{\k_1 \s_1}}{\text{GS}}.
$$

Para isso, podemos utilizar que $\ev{c_k^\d c_l^\d c_i c_j}{n_i n_j} = (\delta_{kj} \delta_{li} - \delta_{ki} \delta_{lj}) n_i n_j$ do item 2(e) da Lista 1, onde $k = (\k_1 - \q, \s_1)$; $l = (\k_2 + \q, \s_2)$; $i = (\k_2, \s_2)$ e $j = (\k_1, \s_1)$. Assim, obtemos que

$$
M_{\k_1 \k_2 \q, \s_1 \s_2} =
\Big(
\delta_{\k_1 - \q, \k_1} \delta_{\s_1, \s_1} \delta_{\k_2 + \q, \k_2} \delta_{\s_2, \s_2} -
\delta_{\k_1 - \q, \k_2} \delta_{\s_1, \s_2} \delta_{\k_2 + \q, \k_1} \delta_{\s_1, \s_2}
\Big) \ev{\nn_{\k_2 \s_2} \nn_{\k_1 \s_1}}{\text{GS}} =
$$
$$
=
\Big(
\delta_{\q, \0} -
\delta_{\q, \k_1 - \k_2} \delta_{\s_1, \s_2}
\Big) \ev{\nn_{\k_2 \s_2} \nn_{\k_1 \s_1}}{\text{GS}}.
$$

Como a interação $\hat{V}$ se trata de uma somatória com $\q \neq 0$, o termo $\delta_{\q, \0}$ vai desaparecer. Veja também que $\ev{\nn_{\k_2 \s_2} \nn_{\k_1 \s_1}}{\text{GS}} = \theta(k_F - \abs{\k_1}) \, \theta(k_F - \abs{\k_2})$, pois essa quantidade só é diferente de zero se ambos $\k_1$ e $\k_2$ estiverem dentro do mar de Fermi.

Assim
$$
\hat{V} = - \frac{3e^2}{2N a_0 r_s} \sum_{\substack{\k_1, \k_2, \q \neq \0 \\ \s_1 \s_2}} \frac{1}{\q^2} \,
\delta_{\q, \k_1 - \k_2} \delta_{\s_1, \s_2} \theta(k_F - \abs{\k_1}) \, \theta(k_F - \abs{\k_2}).
$$
$$
= - \frac{3e^2}{2N a_0 r_s} \sum_{\s} \sum_{\q \neq \0} \frac{1}{\abs{\q}^2} \sum_{\k}
\theta(k_F - \abs{\k}) \, \theta(k_F - \abs{\k + \q}).
$$

A somatória nos spins $\sum_{\s}$ se converte em um fator $2$ e as somatórias nos momentos nós convertemos em integrais via $\sum_{\k} \to \frac{\mathcal{V}}{(2\pi)^3} \int_{\R^3} \dd[3]{\k}$. Temos que
$$
I = \sum_{\q \neq \0} \frac{1}{\abs{\q}^2} \sum_{\k}
\theta(k_F - \abs{\k}) \, \theta(k_F - \abs{\k + \q}) =
\qty[ \frac{\mathcal{V}}{(2\pi)^3} ]^2 \int_{\R^3} \frac{\dd[3]{\q}}{\abs{\q}^2} \int_{\R^3}
\theta(k_F - \abs{\k}) \, \theta(k_F - \abs{\k + \q}) \dd[3]{\k}.
$$

Fixado $\q$, temos que o termo de Heaviside $\theta(k_F - \abs{\k}) \, \theta(k_F - \abs{\k + \q})$ corresponde ao domínio de integração $D_{\q} = \{\k \in \R^3 \mid \abs{\k} < k_F\} \cap \{\k \in \R^3 \mid \abs{\k + \q} < k_F\}$, que é uma interseção de duas esferas com raio $k_F$, uma centrada em $\0$ e a outra em $-\q$. Note que esse domínio $D_{\q}$, não depende da direção de $\q$, somente do módulo $q$. Ainda mais, perceba que $D_q \neq \emptyset$ somente se $0 < q < 2 k_F$. Assim a integral $I$ se simplifica
$$
I = \qty[\frac{\mathcal{V}}{(2\pi)^3}]^2 \int_0^{2k_F} \frac{4\pi \, q^2 \dd{q}}{q^2} \int_{D_q} \dd[3]{\k} =
\qty[\frac{\mathcal{V}}{(2\pi)^3}]^2 (4\pi) \int_0^{2k_F} \text{Vol}(D_q) \dd{q}.
$$
Assim, precisamos calcular $\text{Vol}(D_q) = \int_{D_q} \dd[3]{\k}$, que o volume da interseção das duas esferas. Se fizermos o desenho geométrico (não vou fazer aqui pois dá trabalho) e utilizarmos coordenadas polares $(k, \theta_k, \vphi_k)$ é fácil ver que
$$
\text{Vol}(D_q) = 2 \int_0^{2\pi} \dd{\vphi_k} \int_{q / (2 k_F)}^{1} \dd(\cos\theta_k) \int_{q / (2 \cos\theta_k)}^{k_F} k^2 \dd{k} = \frac{4\pi}{3} k_F^3 - \pi^2 k_F^2 q + \frac{\pi}{12} q^3
$$
<++>

\pagebreak

\section{Gás de elétrons polarizado}

(a)

<++>

\pagebreak

\begin{equation} \label{eq:pertub}
(E - H_0) \ket{\psi} = \lambda V \ket{\psi}.
\end{equation}


\end{document}
